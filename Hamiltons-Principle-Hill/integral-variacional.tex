\section{La integral variacional}
Sean $x^k(k=1,...,n)$ las \textit{variables independientes}\footnote{Tambien llamadas \textit{funciones de estado} del sistema.} que describen el sistema físico y sean $\psi^\alpha (\alpha=1,...m)$ las \textit{variables dependientes}. La idea general de las ecuaciones de movimiento (EOM) es especificar las variables dependientes en términos de las independientes sujetas a valores iniciales y condiciones de borde impuestas en el problema.

Las derivadas parciales de las funciones de estado con respecto a las variables independientes las denotaremos como 
\begin{equation}
  \pdv{\psi^\alpha}{x^k}=\partial_k\psi^\alpha
\end{equation}

El supuesto general que subyacente del principio de Hamilton es que las EOM son derivables aplicando un proceso variacional sobre la acción
\begin{equation}\label{accion}
  I=\int\LLag\dd x
\end{equation}
donde $\Lag$ es la densidad Lagrangeana y supondremos que es una función de las variables dependientes, de las funciones de estado y sus primeras derivadas.
