\documentclass[a4paper,11pt]{article}
\usepackage{jheppub} % for details on the use of the package, please see the JINST-author-manual
\usepackage{lineno}
\usepackage{amsmath,amsthm,amsfonts,amssymb,amscd,physics,cancel,mathtools}
\usepackage{tcolorbox}
\usepackage{marginnote,tensor}
%~~~~~~~~~ Document setup
% \usepackage[spanish]{babel} % English formatting
\usepackage[utf8]{inputenc} % Standard encoding
% \usepackage[a4paper,left=3cm,bottom=3cm]{geometry} % Page formatting
\usepackage{indentfirst} % Indents the first paragraph
\usepackage{amsmath} % Maths type package
\usepackage{bm} % Bold font maths
\usepackage{graphicx} % Advanced graphics package
\usepackage[export]{adjustbox} 
\usepackage{pdflscape} % Make pages landscape
\usepackage{fancyhdr} % Fancy headers
% \usepackage[colorlinks=true,citecolor=blue,urlcolor=blue,linkcolor=black]{hyperref} % Link colours
%\usepackage{natbib} % Bibliography
% \usepackage{flafter} % Reference any 'float'
% \usepackage[framemethod=tikz]{mdframed} % Box off stuff
\usepackage{color} % Colour support
\usepackage{wrapfig} % Text flowing around figures
\usepackage{lipsum} % Generates meaningless text
\usepackage{xcolor}
%\usepackage{biblatex}
%\usepackage[backend=bibtex]{biblatex}
%\addbibresource{bibliography.bib}
\hypersetup{colorlinks=true, linkcolor=blue}

\newtheorem{ej}{Example}[section]
\newtheorem{sol}{Solution}[section]
\newtheorem{dem}{Proof}[section]

%\newcommand{\Lag}{L=L(q_i,\dot{q}_i,t)}
\newcommand{\Lag}{\mathcal{L}}
\newcommand{\R}{\mathcal{R}}
\newcommand{\LLag}{\mathcal{L}(x^k,\psi^\alpha,\partial_k\psi^\alpha)}
\newcommand{\LLagp}{\mathcal{L}(x'^k,\psi'^\alpha,\partial_k\psi'^\alpha)}
\newcommand{\D}{\mathcal{D}}





% \arxivnumber{1234.56789} % if you have one

\title{\boldmath Principio de Hamilton y teoremas de conservación}

% Collaborations

%% [A] If main author
%% \collaboration{\includegraphics[height=17mm]{collabroation-logo}\\[6pt]
%%  XXX collaboration}

%% or
%% [B] If "on behalf of"
%% \collaboration[c]{on behalf of XXX collaboration}


% Authors
% The "\note" macro will give a warning: "Ignoring empty anchor...", you can safely ignore it.

%% [A] simple case: 2 authors, same institution
%% \author[1]{A. Uthor\note{Corresponding author.}}
%% \author{and A. Nother Author}
%% \affiliation{Institution,\\Address, Country}

%% or, e.g.
%% [B] more complex case: 4 authors, 3 institutions, 2 footnotes
%% \author[a,b]{F. Irst,\note{Now at another university}}
%% \author[c]{S. Econd,}
%% \author[a,2]{T. Hird\note{Also at Some University.}}
%% \author[c,2]{and Fourth}
%% \affiliation[a]{Institution_1,\\Address, Country}
%% \affiliation[b]{Institution_2,\\Address, Country}
%% \affiliation[c]{Institution_3,\\Address, Country}

\author{Borja Diez}
\affiliation{Universidad Arturo Prat}
% \affiliation{Another University,\\
% different-address, Country}

% E-mail addresses: only for the corresponding author
\emailAdd{borjadiez1014@gmail.com}

\abstract{En este documento almacenaré algunas derivaciones de los cálculos realizados en \cite{RevModPhys.23.253}.}




\begin{document}
\maketitle
\flushbottom

\section{La integral variacional}
Sean $x^k(k=1,...,n)$ las \textit{variables independientes}\footnote{Tambien llamadas \textit{funciones de estado} del sistema.} que describen el sistema físico y sean $\psi^\alpha (\alpha=1,...m)$ las \textit{variables dependientes}. La idea general de las ecuaciones de movimiento (EOM) es especificar las variables dependientes en términos de las independientes sujetas a valores iniciales y condiciones de borde impuestas en el problema.

Las derivadas parciales de las funciones de estado con respecto a las variables independientes las denotaremos como 
\begin{equation}
  \pdv{\psi^\alpha}{x^k}=\partial_k\psi^\alpha
\end{equation}

El supuesto general que subyacente del principio de Hamilton es que las EOM son derivables aplicando un proceso variacional sobre la acción
\begin{equation}\label{accion}
  I=\int\LLag\dd x
\end{equation}
donde $\Lag$ es la densidad Lagrangeana y supondremos que es una función de las variables dependientes, de las funciones de estado y sus primeras derivadas.

\section{Variación funcional de la acción}
Considerems la posibilidad de un cambio tanto en la región de inegración como en las funciones de estado. En particular, consideremos una transformación infinitesimal de las variables independientes de la forma
\begin{equation}\label{3a}
  x^k\to x'^k=x^k+\delta x^k
\end{equation}
donde las cantidades $\delta x^k$ son funciones de las variables independientes arbitrariamente infinetismales. Para hacer esto más evidente, podemos escribirlas explícitamente como
\begin{equation}
  \delta x^k(x)=\lambda\xi^k(x)
\end{equation}
donde $\xi^k(x)$ es una función arbitraria y $\lambda$ es un parámetro arbitrariamente ifinitesimal.

También le asociamos una transformación infinetismal a las funciones de estado y a sus derivadas parciales,
\begin{align}
  \psi^\alpha \to \psi'^\alpha&=\psi^\alpha+\delta \psi^\alpha(x)\\
  \partial_k\psi^\alpha \to \partial_k\psi'^\alpha&=\partial¡_k\psi^\alpha+\delta (\partial_k\psi^\alpha(x))
\end{align}
las cuales pueden ser reescritas como
\begin{equation}\label{3b}
\begin{split}
  \delta \psi^\alpha(x) &=\psi'^\alpha(x')-\psi^\alpha(x)\\
  \delta (\partial_k\psi^\alpha(x)) &=\partial_k\psi'^\alpha(x')-\partial_k\psi^\alpha(x)
 \end{split}
\end{equation}

la variación funcional de la accción \eqref{accion} es definida como
\begin{align}
  \delta I&=\int_\mathcal{R'}\LLagp\dd x'-\int_\mathcal{R}\LLag \dd x\\
  &=\int_\mathcal{R'}\Lag (x^k+\delta x^k,\psi^\alpha+\delta\psi^\alpha,\partial_k\psi^\alpha+\delta(\partial_k\psi^\alpha))\dd x'-\int_\mathcal{R}\LLag\dd x \label{4}
\end{align}
Es importante notar de esta definición que \textbf{la forma funcional del integrando no es alterada}.

Es conveniente reducir la integral sobre $\mathcal{R'}$ en \eqref{4} a una integral sobre $\mathcal{R}$ mediante un cambio de variables. Esxpandiendo en serie a primer orden, tenemos
\begin{equation}\label{5}
\begin{split}
  \Lag (x^k+\delta x^k,\psi^\alpha+\delta\psi^\alpha,\partial_k\psi^\alpha+\delta(\partial_k\psi^\alpha))&=\LLag +\pdv{\Lag}{x^k}\delta x^k+\pdv{\Lag}{\psi^\alpha}\delta\psi^\alpha\\
  &~~~~+\pdv{\Lag}{(\partial_k\psi^\alpha)}\delta(\partial_k\psi^\alpha)
 \end{split}
\end{equation}
Ahora todas las cantidades del lado derecho de \eqref{5} están expresadas en téminos de coordenadas de la región $\mathcal{R}$.

La transformación del elemento de volumen de $\mathcal{R'}$ a $\mathcal{R}$ se relacionan mediante el Jacobiano de la transformación \eqref{3a},
\begin{equation}
  \dd x'=\pdv{x'}{x}\dd x
\end{equation}
De \eqref{3a} vemos que
\begin{align}
  \pdv{x'^k}{x^k}=\pdv{x^k}(x^k+\delta x^k)=1+\pdv{(\delta x^k)}{x^k}
\end{align}

Reemplazando en \eqref{4} se tiene
\begin{align}
  \nonumber\delta I&=\int_\R \dd x\left[\left(\Lag+\pdv{\Lag}{x^k}\delta x^k+\pdv{\Lag}{\psi^\alpha}\delta\psi^\alpha+\pdv{\Lag}{(\partial_k\psi^\alpha)}\delta(\partial_k\psi^\alpha)\right)\left(1+\pdv{(\delta x^k)}{x^k}\right)\right]-\int_\R \dd x\Lag\\
  &=\int_\R\dd x\left[\Lag \pdv{(\delta x^k)}{x^k}+\pdv{\Lag}{x^k}\delta x^k+\pdv{\Lag}{\psi^\alpha}\delta\psi^\alpha+\pdv{\Lag}{(\partial_k\psi^\alpha)}\delta(\partial_k\psi^\alpha)\right]\label{8}
\end{align}

Notemos que de \eqref{3b}, las funciones están definidas en puntos diferentes del espacio de las funciones independientes. Esto implica que
\begin{equation}\label{9}
  \delta (\partial_k\psi^\alpha(x))=\partial_k\psi^\alpha (x')-\partial_k\psi^\alpha (x)\neq \partial_k(\delta\psi^\alpha (x))
\end{equation}
En este punto, es conveniente definir nuevas cantidades $\delta_*\psi^\alpha$, para que \eqref{9} sea una igualdad. Así
\begin{equation}\label{10}
	\begin{split}
		\psi'^\alpha (x')&=\psi^\alpha (x')+\delta_*\psi^\alpha(x')\\
		\partial_k\psi'^\alpha (x')&=\partial_k\psi^\alpha (x')+\delta_*(\partial_k\psi^\alpha(x'))
	\end{split}
\end{equation}
de donde se desprende
\begin{equation}\label{10p}
	\begin{split}
		\delta_*\psi^\alpha(x')&=\psi'^\alpha (x')-\psi^\alpha (x')\\
		\delta_*(\partial_k\psi^\alpha(x'))&=\partial_k\psi'^\alpha (x')-\partial_k\psi^\alpha (x')
	\end{split}
\end{equation}
Así, de \eqref{10} y \eqref{10p}, se tiene
\begin{align}
  \delta_*(\partial_k\psi^\alpha(x'))&=\partial_k(\psi'^\alpha (x')-\partial_k\psi^\alpha (x')\\
  &=\partial_k(\psi'^\alpha (x')-\psi^\alpha (x'))\\
  &=\partial_k(\delta_*\psi^\alpha (x'))
\end{align}
Renombrando las coordenadas
\begin{equation}
	\delta_*(\partial_k\psi^\alpha(x))=\partial_k(\delta_*\psi^\alpha (x))
\end{equation}
Es decir, $\delta_*$ y $\partial$ conmutan.

Además, notemos que usando \eqref{3a}, se tiene
\begin{align}
  \delta \psi^\alpha (x)&=\psi'^\alpha (x')-\psi^\alpha(x)\\
  &=\psi'^\alpha (x+\delta x)-\psi^\alpha(x)\\
  &=\psi'^\alpha (x)+\partial_l\psi'^\alpha (x)\delta x^l-\psi^\alpha (x)\\
  &=\delta_*\psi^\alpha (x)+\partial_l\psi^\alpha (x)\delta x^l
\end{align}

De manera análoga,
\begin{equation}
  \delta (\partial_k\psi^\alpha(x))=\delta_*\partial_k\psi^\alpha (x)+\partial_l\partial_k\psi^\alpha (x)\delta x^l
\end{equation}

Es importante notar el énfasis en que $\partial$ no conmuta con $\delta$ pero si con $\delta_*$, por eso no es necesario considerar el paréntesis.

Reemplazando en \eqref{8},
\begin{align*}
  \delta I&=\int_\R\dd x\left[\Lag \pdv{(\delta x^k)}{x^k}+\pdv{\Lag}{x^k}\delta x^k+\pdv{\Lag}{\psi^\alpha}\delta\psi^\alpha+\pdv{\Lag}{(\partial_k\psi^\alpha)}\delta(\partial_k\psi^\alpha)\right]\\
  &=\int_\R \dd x\left[\Lag \pdv{(\delta x^k)}{x^k}+\pdv{\Lag }{x^k}\delta x^k\pdv{\Lag}{\psi^\alpha }\delta_*\psi^\alpha +\pdv{\Lag}{\psi^\alpha}\partial_k\psi^\alpha \delta x^k+\pdv{\Lag}{(\partial_k\psi^\alpha)}\delta_*\partial_k\psi^\alpha +\pdv{\Lag}{(\partial_l\psi^\alpha)}\partial_k\partial_l\psi^\alpha \delta x^k\right]
\end{align*}
donde se han renombrado algunos índices por conveniencia.

Llegado a este punto, notemos que hasta el momento hemos considerado que $\Lag=\LLag$. Es conveniente introducir el concepto de derivada parcial con respecto a las variables independientes cuando las funciones de estado y sus derivadas han sido reemplazadas como funciones de las variables independientes $x^k$. Supongamos que entonces que $\Lag$ depende a lo más de derivadas de primer orden en $\psi^\alpha$. Así,
\begin{equation}
  \Lag = \Lag (x^k,\psi^\alpha (x^k),\partial_l\psi^\alpha (x^k))
\end{equation}
Definimos la derivada parcial con respecto a las variables independientes como
\begin{equation}
	\frac{\D\Lag}{\D x^k}\equiv \pdv{\Lag}{x^k}+\pdv{\Lag}{\psi^\alpha}\pdv{\psi^\alpha}{x^k}+\pdv{\Lag}{(\partial_l\psi^\alpha)}\pdv{(\partial_l\psi^\alpha)}{(\partial x^k)}
\end{equation}
de manera que tenemos el siguiente operador
\begin{equation}
  \D _k\equiv \frac{\D}{\D x^k}=\partial _k+\partial_k\psi^\alpha\pdv{\psi^\alpha}+\partial_k\partial_l\psi^\alpha\pdv{(\partial_l\psi^\alpha)}
\end{equation}
de manera que
\begin{align*}
  \delta I&=\int_\R \dd x\left[\Lag \pdv{(\delta x^k)}{x^k}+\pdv{\Lag }{x^k}\delta x^k+\pdv{\Lag}{\psi^\alpha }\delta_*\psi^\alpha +\pdv{\Lag}{\psi^\alpha}\partial_k\psi^\alpha \delta x^k+\pdv{\Lag}{(\partial_k\psi^\alpha)}\delta_*\partial_k\psi^\alpha +\pdv{\Lag}{(\partial_l\psi^\alpha)}\partial_k\partial_l\psi^\alpha \delta x^k\right]\\
  &=\int_\R \dd x\left[\Lag \pdv{(\delta x^k)}{x^k}+\pdv{\Lag }{x^k}\delta x^k+\pdv{\Lag}{\psi^\alpha}\partial_l\psi^\alpha \delta x^l+\pdv{\Lag}{(\partial_l\psi^\alpha)}\partial_k\partial_l\psi^\alpha \delta x^k+\pdv{\Lag}{\psi^\alpha }\delta_*\psi^\alpha+\pdv{\Lag}{(\partial_k\psi^\alpha)}\delta_*\partial_k\psi^\alpha\right]\\
  &=\int_\R \dd x\left[\frac{\D}{\D x^k}(\Lag\delta x^k)+\pdv{\Lag}{\psi^\alpha }\delta_*\psi^\alpha+\pdv{\Lag}{(\partial_k\psi^\alpha)}\delta_*\partial_k\psi^\alpha\right]
\end{align*}
Integrando por partes el último término,
\begin{align}
  \pdv{\Lag}{(\partial_k\psi^\alpha)}\delta_*\partial_k\psi^\alpha&=\pdv{\Lag}{(\partial_k\psi^\alpha)}\partial_k\delta_*\psi^\alpha\\
  &=\partial_k\left(\pdv{\Lag}{(\partial_k\psi^\alpha)}\delta_*\psi^\alpha\right)-\partial_k\left(\pdv{\Lag}{(\partial_k\psi^\alpha)}\right)\delta_*\psi^\alpha\\
  &=\D_k\left(\pdv{\Lag}{(\partial_k\psi^\alpha)}\delta_*\psi^\alpha\right)-\D_k\left(\pdv{\Lag}{(\partial_k\psi^\alpha)}\right)\delta_*\psi^\alpha
\end{align}






















% Bibliography

%% [A] Recommended: using JHEP.bst file
%% \bibliographystyle{JHEP}
%% \bibliography{biblio.bib}

%% or
%% [B] Manual formatting (see below)
%% (i) We suggest to always provide author, title and journal data or doi:
%% in short all the informations that clearly identify a document.
%% (ii) please avoid comments such as "For a review'', "For some examples",
%% "and references therein" or move them in the text. In general, please leave only references in the bibliography and move all
%% accessory text in footnotes.
%% (iii) Also, please have only one work for each \bibitem.

\newpage
\bibliographystyle{JHEP}
\bibliography{biblio.bib}
\end{document}
