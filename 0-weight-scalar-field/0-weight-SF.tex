\documentclass[a4paper,10pt]{article}
\usepackage{jheppub} % for details on the use of the package, please see the JINST-author-manual
\usepackage{lineno}
\usepackage{amsmath,amsthm,amsfonts,amssymb,amscd,physics,cancel,mathtools}
\usepackage{tcolorbox}
\usepackage{marginnote,tensor}
\usepackage{tcolorbox}
%~~~~~~~~~ Document setup
% \usepackage[spanish]{babel} % English formatting
\usepackage[utf8]{inputenc} % Standard encoding
% \usepackage[a4paper,left=3cm,bottom=3cm]{geometry} % Page formatting
\usepackage{indentfirst} % Indents the first paragraph
\usepackage{amsmath} % Maths type package
\usepackage{bm} % Bold font maths
\usepackage{graphicx} % Advanced graphics package
\usepackage[export]{adjustbox} 
\usepackage{pdflscape} % Make pages landscape
\usepackage{fancyhdr} % Fancy headers
% \usepackage[colorlinks=true,citecolor=blue,urlcolor=blue,linkcolor=black]{hyperref} % Link colours
%\usepackage{natbib} % Bibliography
% \usepackage{flafter} % Reference any 'float'
% \usepackage[framemethod=tikz]{mdframed} % Box off stuff
\usepackage{color} % Colour support
\usepackage{wrapfig} % Text flowing around figures
\usepackage{lipsum} % Generates meaningless text
\usepackage{xcolor}
%\usepackage{biblatex}
%\usepackage[backend=bibtex]{biblatex}
%\addbibresource{bibliography.bib}
%\hypersetup{colorlinks=true, linkcolor=blue}

\newtheorem{ej}{Example}[section]
\newtheorem{sol}{Solution}[section]
\newtheorem{dem}{Proof}[section]
\newtheorem{prop}{Propiedad}[section]

\def\a{\alpha}
\def\b{\beta}
\def\g{\gamma}
\def\G{\Gamma}
\def\d{\delta}
%\def\D{\Delta}
%\def\e{\eta}
\def\la{\lambda}
\def\La{\Lambda}
\def\k{\kappa}
\def\m{\mu}
\def\n{\nu}
\def\r{\rho}
\def\p{\rho}
\def\o{\omega}
\def\s{\sigma}
\def\S{\Sigma}
\def\t{\tau}
\def\p{\pi}
\def\f{\phi}
\def\vf{\varphi}
\def\ep{\epsilon}
\def\th{\theta}
\def\Th{\Theta}
\def\z{\zeta}
\def\id{\mathrm{I}}
\def\M{\mathcal{M}}
\def\E{\mathcal{E}}

\newcommand{\e}{\mathrm{e}}
\newcommand{\I}{\mathrm{I}}
\newcommand{\C}{\mathcal{C}}
\newcommand{\D}{\mathrm{D}}
\newcommand{\oo}{\mathring}
\newcommand{\oomega}{\mathring{\omega}}
\newcommand{\eabc}{\epsilon_{abc}}

%-----COLORS LIST ------
\definecolor{azure(colorwheel)}{rgb}{0.0, 0.5, 1.0}
\definecolor{DarkViolet}{RGB}{148,0,211}
\definecolor{myDarkBlue}{rgb}{0,0.1,0.7}
\definecolor{DarkBlue}{RGB}{0,0,153}
\definecolor{amber}{rgb}{1.0, 0.49, 0.0}
\definecolor{amaranth}{rgb}{0.9, 0.17, 0.31}
\definecolor{nicered}{rgb}{0.7,0.1,0.1}
\definecolor{brown}{rgb}{0.5,0.1,0.1}
\definecolor{nicegreen}{rgb}{0.0,0.3,0.0}
\definecolor{tealgreen}{rgb}{0.0, 0.51, 0.5}
\def\red#1{{\color{red} #1}}
\def\green#1{{\color{green} #1}}
\def\blue#1{{\color{blue} #1}}
\def\orange#1{{\color{orange} #1}}
%----------------------
\newcommand{\mycolor}{DarkViolet}
\def\myColor#1{{\color{\mycolor} #1}}
\definecolor{tclr}{RGB}{148,0,211}
%----------------------
\newcommand{\corr}[1]{\textcolor{nicered}{#1}}
\newcommand{\nick}[1]{\textcolor{olive}{#1}}
\newcommand{\teo}[1]{\textcolor{azure(colorwheel)}{#1}}
\newcommand{\chteo}[2]{\corr{\st{#1}} \teo{(#2)}}
\newcommand{\bako}[1]{\textcolor{DarkViolet}{#1}}
\newcommand{\than}[1]{\textcolor{magenta}{#1}}

\newcommand{\rc}{\textcolor{red}}
\newcommand{\bc}{\textcolor{blue}}
\newcommand{\cc}{\textcolor{cyan}}
\newcommand{\gc}{\textcolor{green}}
\newcommand{\occ}{\textcolor{orange}}
\newcommand{\pc}{\textcolor{purple}}

%----------------------
\usepackage{hyperref}
\hypersetup{colorlinks,bookmarksopen,
	bookmarksnumbered,
	citecolor={nicered},
	linkcolor={myDarkBlue},
	urlcolor={blue},
	pdfstartview=FitH}


% \arxivnumber{1234.56789} % if you have one

\title{\boldmath Zero-weight scalar fields}

% Collaborations

%% [A] If main author
%% \collaboration{\includegraphics[height=17mm]{collabroation-logo}\\[6pt]
%%  XXX collaboration}

%% or
%% [B] If "on behalf of"
%% \collaboration[c]{on behalf of XXX collaboration}


% Authors
% The "\note" macro will give a warning: "Ignoring empty anchor...", you can safely ignore it.

%% [A] simple case: 2 authors, same institution
%% \author[1]{A. Uthor\note{Corresponding author.}}
%% \author{and A. Nother Author}
%% \affiliation{Institution,\\Address, Country}

%% or, e.g.
%% [B] more complex case: 4 authors, 3 institutions, 2 footnotes
%% \author[a,b]{F. Irst,\note{Now at another university}}
%% \author[c]{S. Econd,}
%% \author[a,2]{T. Hird\note{Also at Some University.}}
%% \author[c,2]{and Fourth}
%% \affiliation[a]{Institution_1,\\Address, Country}
%% \affiliation[b]{Institution_2,\\Address, Country}
%% \affiliation[c]{Institution_3,\\Address, Country}

\author{Borja Diez}
\affiliation{Universidad Arturo Prat}
% \affiliation{Another University,\\
% different-address, Country}

% E-mail addresses: only for the corresponding author
\emailAdd{borjadiez1014@gmail.com}

\abstract{Personal compilation of some calculations related to zero-weight conformal scalar fields.}




\begin{document}
\maketitle
%\tableofcontents
%\flushbottom

%\section{Differentiable manifolds}
\subsection{From Topological Spaces to Differentable Manifolds}
%TODO agregar imagen
It is assumed that the reader is acquainted with the notion of a topological space as a structure on which one can define a neighborhood and continuous functions. A \textbf{homeomorphism} between two topological spaces is a 1-1 map $\varphi: X\to Y$ for which both $\varphi$ and its inverse $\varphi^{-1}$ are continuous. If $\varphi$ and $\varphi^{-1}$ are continuously differentiable then $\varphi$ is called a \textbf{diffeomorphism}.

A $D$-dimensional manifold $M^D$ is a topological space that locally has the properties of a $D$-dimensional Euclidean space $\mathbb{R}^D$ : A neighborhood of a point in $M^D$
can continuously be mapped in a one-to-one way to the neighborhood of a point in
$\mathbb{R}^D$. To be more precise, introduce a \textbf{chart} $(U_\alpha,\varphi_\alpha)$ as a homeomorphism $\varphi_\alpha$ from
an open set $U_\alpha\subset M^D$ into an open set $R_\alpha\subset \mathbb{R}^D$. Two charts are compatible if the overlap maps are diffeomorphims $(\varphi_1\cdot \varphi_2\in C^\infty,\varphi_2\cdot \varphi_1^{-1}\in C^\infty)$ unless $U_1\cap U_2=\emptyset$. A set of compatible charts covering $M^D$ is called an atlas. In every chart the manifold can be equipped with a coordinate system: for $x\in M^D$ the coordinates are $x^\mu=\varphi(x)\in \mathbb{R}^D$. The naming makes it clear what one is aiming at. For instance the surface of a sphere, although not being homeomorphic to a plane, locally has enough smoothness to be mapped into an atlas. One chart is not sufficient since there will always be a point on the sphere that cannot be projected to the plane.

In manuscript will only treat finite-dimensional manifolds. One possibility of extending the notion of manifolds to infinite dimensions is to consider Banach manifolds modeled on Banach spaces. It is also assumed that we are dealing with $C^\infty$ manifolds. In certain contexts it might suffice that the charts are $C^k$-related. Also complex manifolds are investigated in mathematics and applied to modern theoretical physics (catchword: Kähler manifolds). In these the transition functions are required to be analytic.

\subsection{Tensor Bundles}
On a manifold one can erect tensor bundles as “superstructures” by starting with defining the tangent and cotangent spaces of a manifold.

\subsubsection{Tangent Bundle and Vector Fields}
We are interested in the notion of vectors on a manifold $M$ (henceforth I will mostly drop the index for the dimension of the manifold and for the Euclidean space). The idea is to introduce these as tangent vectors of curves 'through' $x\in M$: A curve through a point $x$ is a smooth mapping of an interval $I=[0,1]\subset\mathbb{R}$ to the manifold:
\begin{equation}
  C=\mathbb{I}\to M\qquad t\mapsto C(t)\qquad \mbox{with}\qquad C(0)=x
\end{equation}
The coordinates of this curve are $x^\mu(C(t))$, and the tangent vector to this curve is
\begin{equation}
  \dv{t}x^\mu(C(t))
\end{equation}
Since one can have more then one curve with $C(0)=x$, the proper definition is: A \textit{tangent vector} $x\in M$ is an equivalence class of curves in $M$, where the equivalence relation between two curves is that they are tangent at the point $x$. Another-equivalent- definition is to understand a tangent vector as a directional derivative: Consider functions $f\in \mathcal{F}M$, that is $f:M\to \mathbb{R}$. The change of $f$ along a curve is given by
\begin{equation}
  \dv{t}f(C(t)),\qquad \mbox{locally}\qquad \pdv{x^\mu}f\dv{x^\mu(C(t))}{t}
\end{equation}

In defining
\begin{equation}
  X=(X^\mu\partial_\mu)\qquad \mbox{with}\qquad X^\mu =\dv{x^\mu(C(t))}{t}
\end{equation}
we can write $\dv{t}f(C(t))=Xf$. For every point along the curve we take this expression to define the differential operator $X_x$ as the tangent vector to the manifold in $x\in M$. All tangent vectors at a point in the manifold can be shown to build a vector space $\mathfrak{X}_xM$ isomorphic to $\mathbb{R}^D$. The natural basis in $\mathfrak{X}_xM$ is the coordinate or \textbf{holonomic} basis $\{\partial_\mu\}$. But of course any other (\textbf{anholonomic}) basis $\{e_I}
\}$





























\section{Building the equation of motion}
Let us consider a scalar field with zero conformal weight that is, under an infinitesimal conformal transformation
\begin{equation}
  \d_\omega g_{\m\n }=2\omega g_{\m\n },\qquad \d_\omega\f=0
\end{equation}

Let us consider the most general second-order pseudoscalar constructed from the scalar field $\f$ and its derivatives up to second order, together with the metric tensor and its associates curvature
\begin{equation}
	\E =\sqrt{-g}E\left(\f,\nabla_\m \f ,\nabla_\m \nabla_\n \f ,g_{\m\n },R_{\m\n }^{\a\b }\right)=0
\end{equation}
The variation of this equation under infinitesimal conformal transformations reads,
\begin{align}
  \d_\omega \E&=\d_\omega(\sqrt{-g})E+\sqrt{-g}\d_\omega E\\
  &=-\frac{1}{2}\sqrt{-g}g_{\m\n }\d_\omega g^{\m\n }E+\sqrt{-g}\d_\omega E
\end{align}
From $\d_\omega g_{\m\n }=2\omega g_{\m\n }$ and $\d_\omega(g_{\m\n }g^{\m\n })=0$ implies that
\begin{equation}
  g_{\m\n }\d_\omega g^{\m\n }=-g^{\m\n }\d_\omega g_{\m\n }
\end{equation}
Furthermore, from \eqref{delta-nabla} and \eqref{delta-nabla-nabla}, we have
\begin{align}
  \d_\omega \E&=\pdv{E}{\f }\cancelto{0}{\d_\omega \f} +\pdv{E}{(\nabla_\m \f )}\cancelto{0}{\d_\omega \nabla_\m \f} +\pdv{E}{(\nabla_\m \nabla_\n \f )}\d_\omega (\nabla_\m \nabla_\n \f )+\pdv{E}{g^{\m\n }}\d_\omega g^{\m\n }+P^{\m\n }_{\a\b }\d_\omega R^{\a\b }_{\m\n }
\end{align}
where we have defined 
\begin{equation}\label{P-trnsor}
  P^{\m\n }_{\a\b }:=\pdv{E}{R_{\m\n }^{\a\b }},\qquad   P^\m_\n :=P^{\a\m }_{\a\n }
\end{equation}

So then
\begin{align*}
  \d_\omega \E&=-\frac{1}{2}\sqrt{-g}g_{\m\n }\d_\omega g^{\m\n }E+\sqrt{-g}\pdv{E}{(\nabla_\m \nabla_\n \f  )}\d_\omega (\nabla_\m \nabla_\n \f )+\sqrt{-g}\pdv{E}{g^{\m\n }}\d_\omega g^{\m\n }+\sqrt{-g}P^{\m\n }_{\a\b }\d_\omega R^{\a\b }_{\m\n }\\
  &=\sqrt{-g}\pdv{E}{(\nabla_\m \nabla_\n \f )}\left(g_{\n\b }\nabla^\r \f\nabla_\r\omega-2\nabla_{(\m }\f\nabla_{\n)}\omega\right)+\sqrt{-g}\pdv{E}{g^{\m\n }}\d_\omega g^{\m\n }+\sqrt{-g}P^{\m\n }_{\a\b }\d_\omega R^{\a\b }_{\m\n }\\& -\frac{1}{2}\sqrt{-g}g_{\m\n }\d_\omega g^{\m\n }E\\
  &=\sqrt{-g}\pdv{E}{(\nabla_\m \nabla_\n \f )}\left(g_{\n\b }\nabla^\r \f\nabla_\r\omega-2\nabla_{(\m }\f\nabla_{\n)}\omega\right)+\sqrt{-g}P^{\m\n }_{\a\b }\d_\omega R^{\a\b }_{\m\n }-2\omega g^{\m\n}\sqrt{-g}\left(\pdv{E}{g^{\m\n }}-\frac{1}{2}g_{\m\n }E\right)
\end{align*}
Note that
\begin{align}
  \pdv{(\sqrt{-g}E)}{g^{\m\n }}&=-\frac{1}{2}\sqrt{-g}g_{\m\n }E+\sqrt{-g}\pdv{E}{g^{\m\n }}\\
  &=\sqrt{-g}\left(\pdv{E}{g^{\m\n }}-\frac{1}{2}g_{\m\n }E\right) \label{d-sqrt-E}
\end{align}
By defining
\begin{align}\label{A-B}
  A^{\m\n }:=\pdv{E}{(\nabla_\m \nabla_\n\f )},\qquad B_{\m\n }=-2\left(\pdv{E}{g^{\m\n }}-\frac{1}{2}g_{\m\n }E\right)=-\frac{2}{\sqrt{-g}}\pdv{(\sqrt{-g}E)}{g^{\m\n }}
\end{align}
we obtain
\begin{align}
  \d_\omega \E&=\sqrt{-g}A^{\m\n }\left(\g_{\m\n }\nabla^\r \f \nabla_\r \omega-2\nabla_\m \f\nabla_\n \omega\right)+\sqrt{-g}P^{\m\n }\d_\omega R^{\a\b }_{\m\n }+\sqrt{-g}\omega B_{\m\n }g_{\m\n }\\
  &=\sqrt{-g}\left(A\nabla^\r \f\nabla_\r \omega-2A^{\m\n}\nabla_\m \f\nabla_\n\omega+\omega B+P^{\m\n}_{\a\b}\d_\omega R^{\a\b }_{\m\n }\right)
\end{align}
Using
\begin{equation}
  \d_\omega R^\a _{~\b\m\n }=2\nabla_{[\m }\d_\omega \G^\a _{~\n]\b }
\end{equation}
we have
\begin{align}
    \d_\omega \E&=\sqrt{-g}\left(A\nabla^\r \f\nabla_\r \omega-2A^{\m\n}\nabla_\m \f\nabla_\n\omega+\omega B+2P_\a ^{~\b\m\n }\nabla_{[\m }\d_\omega \G^{\a }_{~\n]\b }\right)\\
    &=\sqrt{-g}\left(A\nabla^\n \f\nabla_\n  \omega-2A^{\m\n}\nabla_\m \f\nabla_\n\omega+\omega B+2P_\a ^{~\b\m\n }\nabla_{[\m }\d_\omega \G^{\a }_{~\n]\b }\right)\\
    &=\sqrt{-g}\left[(A\nabla^\n \f -2A^{\m\n }\nabla_\m\f )\nabla_\n \omega+\omega B+2P_\a ^{~\b\m\n }\nabla_{[\m }\d_\omega \G^{\a }_{~\n]\b }\right]
\end{align}

Noting that
\begin{align}
  \d_\omega \G^\a_{~\n\b }&=\frac{1}{2}g^{\a\lambda}\left[\nabla_\n (\d_\omega g_{\b\lambda})+\nabla_\b (\d_\omega g_{\n\lambda})-\nabla_\lambda(\d_\omega g_{\n\b })\right]\\
  &=g^{\a\lambda}\left(g_{\b\lambda}\nabla_\n\omega +g_{\n\lambda}\nabla_\b\omega-g_{\n\b }\nabla_\lambda\omega\right)\\
  &=g^{\a\lambda}\left(g_{\b\lambda}\nabla_\n\omega +2g_{\n[\lambda}\nabla_{\b]}\omega\right)
\end{align}
we obtain
\begin{align*}
  \d_\omega\E&=\sqrt{-g}\left[(A\nabla^\n \f -2A^{\m\n }\nabla_\m\f )\nabla_\n \omega+\omega B+2P_\a ^{~\b\m\n }\nabla_{\m }g^{\a\lambda}\left(g_{\b\lambda}\nabla_\n\omega +2g_{\n[\lambda}\nabla_{\b]}\omega\right)\right]\\
  &=\sqrt{-g}\left[(A\nabla^\n \f -2A^{\m\n }\nabla_\m\f )\nabla_\n \omega+\omega B+2P ^{\lambda\b\m\n }\nabla_{\m }\left(\cancel{g_{\b\lambda}}\nabla_\n\omega +2g_{\n\lambda}\nabla_{\b}\omega\right)\right]\\
  &=\sqrt{-g}\left[(A\nabla^\n \f -2A^{\m\n }\nabla_\m\f )\nabla_\n \omega+\omega B+4P^{\lambda\b\m\n }\nabla_{\m }g_{\n\lambda}\nabla_{\b}\omega\right]\\
  &=\sqrt{-g}\left[(A\nabla^\n \f -2A^{\m\n }\nabla_\m\f )\nabla_\n \omega+\omega B-4P^{\m\n }\nabla_{\m }\nabla_{\n }\omega\right]
\end{align*}
Imposing $\d_\omega\E=0$ for all $\omega$, we obtain the following conditions,
\begin{subequations}\label{conditions}
	\begin{align}
 \label{A-condition} A\nabla^\n \f-2A^{\m\n }\nabla_\m \f&=0\\
  \label{B-condition} B&=0\\
 \label{P-condition} P^{\m\n }&=0
\end{align}
\end{subequations}
 
From \eqref{P-tensor},
\begin{equation}\label{dEdP}
    \pdv{E}{R^{\m\n }_{\a\b }}= P^{\alpha\beta}_{\mu\nu}=\hat{H}^{\alpha\beta}_{\mu\nu}+\delta^{[\alpha}_{[\mu}\hat{I}^{\beta]}_{\nu]}+J\delta^\alpha_{[\mu}\delta^\beta_{\nu]}
\end{equation}
we notice that since $\hat{H}^{\alpha\beta}_{\mu\nu}$ is the traceless part of $H^{\alpha\beta}_{\mu\nu}$,
\begin{equation}
  P^\b_\n =P^{\a\b }_{\a\n }=\delta^{[\alpha}_{[\a }\hat{I}^{\beta]}_{\nu]}+J\delta^\alpha_{[\a }\delta^\beta_{\nu]}
\end{equation}

Since the variation with respect to the Weyl tensor does not contribute to the first trace \eqref{dEdP}, the contribution of the Riemann tensor to $E$ has to be through its traceless part,
\begin{equation}
	\E =\sqrt{-g}E\left(\f,\nabla_\m \f ,\nabla_\m \nabla_\n \f ,g_{\m\n },C_{\m\n }^{\a\b }\right)=0\,.
\end{equation}

\begin{ej}
	Let us consider the following action principle
	\begin{align}
  S[\f,g]&=\int\dd^Dx\sqrt{-g}\left(-\frac{1}{2}\nabla_\m \f\nabla^\m \f \right)^{D/2}\\
  &=\int\dd^Dx\sqrt{-g}\left(-\frac{1}{2}(\nabla\f )^2\right)^{D/2}\\
  &=\int\dd^Dx\sqrt{-g}X^{D/2}
\end{align}
where we have defined $X:=-\frac{1}{2}\nabla_\m \f\nabla^\m \f$. 

Now, we must to find $E$. Varying with respect to $\f$,
\begin{align}
  \d_\f S&=-\int\dd^Dx\sqrt{-g}\frac{D}{2}\left(-\frac{1}{2}(\nabla\f )^2\right)^{\frac{D-2}{2}}\nabla^\m\f \nabla_\m \d\f \\
  &=\int\dd^Dx\sqrt{-g}\frac{D}{2}\nabla_\m \left[\left(-\frac{1}{2}(\nabla\f )^2\right)^{\frac{D-2}{2}}\nabla^\m \f \right]\d\f +\rm b.t
\end{align}
therefore,
\begin{align}
  E&=\nabla_\m \left[\left(-\frac{1}{2}(\nabla\f )^2\right)^{\frac{D-2}{2}}\nabla^\m \f \right]\\
  &=\frac{D-2}{2}\left(-\frac{1}{2}(\nabla\f )^2\right)^{\frac{D-4}{2}}(-1)\nabla^\a \f \nabla_\m \nabla_\a\f \nabla^\m \f +\left(-\frac{1}{2}(\nabla\f )^2\right)^{\frac{D-2}{2}}\Box\f 
\end{align}
which can be rewritten as
\begin{equation}
  E=X^{\frac{D-2}{2}}\Box\f -\frac{D-2}{2}X^{\frac{D-4}{2}}\nabla_\m \nabla_\n \f\nabla^\m \f\nabla^\n \f 
\end{equation}

Let's see how \eqref{A-condition} looks,
\begin{align}
  A^{\m\n }&=\pdv{(\nabla_\m \f\nabla_\n \f  )}\left[X^{\frac{D-2}{2}}g^{\m\n }\nabla_\m\f \nabla_\n\f  -\frac{D-2}{2}X^{\frac{D-4}{2}}\nabla_\m \nabla_\n \f\nabla^\m \f\nabla^\n \f \right]\\
  &=X^{\frac{D-2}{2}}g^{\m\n }-\frac{D-2}{2}X^{\frac{D-4}{2}}\nabla^\m \f \nabla^\n\f 
\end{align}
and its trace yields
\begin{align}
  A&=DX^{\frac{D-2}{2}}-\frac{D-2}{2}X^{\frac{D-4}{2}}(\nabla\f )^2\\
  &=DX^{\frac{D-2}{2}}+(D-2)X^{\frac{D-4}{2}}\left(-\frac{1}{2}(\nabla\f )^2\right)\\
  &=DX^{\frac{D-2}{2}}+(D-2)X^{\frac{D-2}{2}}\\
  &=2(D-1)X^{\frac{D-2}{2}}
\end{align}
Pluggin' into \eqref{A-condition},
\begin{align*}
  A\nabla^\n \f-2A^{\m\n }\nabla_\m \f&=2(D-1)X^{\frac{D-2}{2}}\nabla^\n\f -2X^{\frac{D-2}{2}}g^{\m\n }\nabla_\m \f +(D-2)X^{\frac{D-4}{2}}\nabla^\m\f \nabla^\n\f \nabla_\m\f \\
  &=2(D-1)X^{\frac{D-2}{2}}\nabla^\n\f-2X^{\frac{D-2}{2}}\nabla^\n\f +(D-2)X^{\frac{D-4}{2}}(\nabla\f )^2\nabla^\n\f \\
  &=\left[2(D-2)X^{\frac{D-2}{2}}-2(D-2)X^{\frac{D-4}{2}}X\right]\nabla^\n\f \\
  &=\left[2(D-2)X^{\frac{D-2}{2}}-2(D-2)X^{\frac{D-2}{2}}2\right]\nabla^\n\f \\
  &=0\qquad \checkmark
\end{align*}
\end{ej}

%--------------------


\section{Building the auxiliary metric}
We know the conditions that the most general second-order equation of motion for the zero conformal weight scalar field must satisfy. They are given by \eqref{conditions}. Now, the question is: how do we construct an auxiliary metric $\tilde{g}_{\m\n }$ such that  $\d_\omega \tilde{g}_{\m\n }=0$?

Remember that in the exponential frame for the scalar field, the conformal transformations look like
\begin{equation}
  g_{\m\n }\to \bar{g}_{\m\n }= \e^{2\omega}g_{\m\n },\qquad \f\to \bar{\f }=\f 
\end{equation}
Since the inverse metric transforms as
\begin{equation}
  g^{\m\n }\to \bar{g}^{\m\n }=\e^{-2\omega}g^{\m\n }
\end{equation}
the kinetic term for the scalar field, defined as
\begin{equation}
  X:=-\frac{1}{2}\nabla_\m \f \nabla^\m \f =-\frac{1}{2}g^{\m\n }\nabla_\m \f\nabla_\n \f 
\end{equation}
transforms as
\begin{equation}
  \bar{X}=-\frac{1}{2}\bar{g}^{\m\n }\partial_\m \bar{\f }\partial_\n \bar{\f }=-\frac{1}{2}\e^{-2\omega}g^{\m\n }\partial_\m \f\partial_\n \f =\e^{-2\omega}X
\end{equation}
Thus, the auxiliary metric is defined as
\begin{equation}
  \tilde{g}_{\m\n }=Xg_{\m\n }=-\frac{1}{2}(\nabla\f )^2g_{\m\n } \quad \implies\quad \d_\omega\tilde{g}_{\m\n }=0
\end{equation}
\textit{is conformally invariant}.

Let us consider a pseudoscalar built from the zero weight conformal scalar field and its derivatives up to second-order and the conformally invariant geometry,
\begin{equation}
  \E=\sqrt{-\tilde{g}}E(\f,\tilde{\nabla}_\m\f ,\tilde{\nabla}_\m\tilde{\nabla}_\n \f ,\tilde{g}^{\m\n },\tilde{C}_{\m\n }^{\a\b })
\end{equation}
where we used the fact that for zero-weight, the $P^{\m\n }=0$ condition implies that the only dependence on the curvature is through the Weyl tensor, which is conformally invariant.
%TODO [\rc{Update argument.}]

We notice that $\d_\omega\E=0$. Indeed
\begin{align*}
  \d_\omega\E=-\frac{1}{2}\sqrt{-\tilde{g}}\tilde{g}^{\m\n }E\cancel{\d_\omega\tilde{g}_{\m\n }}+\sqrt{-\tilde{g}}&\left(\pdv{E}{\f }\cancel{\d_\omega\f} +\pdv{E}{(\tilde{\nabla}_\m\f )}\d_\omega (\tilde{\nabla}_\m\f )+\pdv{E}{(\tilde{\nabla}_\m\tilde{\nabla}_\n\f )}\d_\omega(\tilde{\nabla}_\m\tilde{\nabla}_\n\f )\right.\\&\left. +\pdv{E}{\tilde{g}^{\m\n }}\cancel{\d_\omega\tilde{g}^{\m\n }}+\tilde{P}^{\m\n }_{\a\b}\cancel{\d_\omega\tilde{C}^{\a\b }_{\m\n }}\right)
\end{align*}
but
\begin{align}
  \d_\omega\tilde{\nabla}_\m\f&=\d_\omega\partial_\m\f =\partial_\m\d_\omega\f=0
\end{align}
and
\begin{align}
  \d_\omega(\tilde{\nabla}_\m\tilde{\nabla}_\n\f )&=\d_\omega(\tilde{\nabla}_\m\partial_\n \f )\\
 &=\d_\omega(\partial_\m\partial_\n \f -\tilde{\G}^{\lambda}_{~\m\n }\partial_\lambda\f )\\
 &=\partial_\m\partial_\n \d_\omega\f -\d_\omega\tilde{\G}^\lambda_{~\m\n }\partial_\lambda\f -\tilde{\G}^\lambda_{^\m\n }\partial_\lambda \d_\omega\f \\
 &=-\d_\omega\tilde{\G}^\lambda_{~\m\n }\partial_\lambda\f\\
 &=0
\end{align}
since $\d_\omega\tilde{\G}^\lambda_{~\m\n }=0$. Therefore,
\begin{align}
  \d_\omega\E=0\, .
\end{align}


%Since from conformal invariance for zero weight scalar field, the $P^{\m\n }=0$ condition implies that the explicit dependence of the Riemann tensor in the equation of motion is through the Weyl tensor, then,  for a equation of motion built from the conformally invariant geometry we have
%\begin{equation}\label{EE}
%  E=E(\f,\tilde{\nabla}_\m\f ,\tilde{\nabla}_\m\nabla_\n \f ,\tilde{g}^{\m\n },\tilde{C}_{\m\n }^{\a\b })=0
%\end{equation}


\section{Fréchet derivative}
Now, let's try to find an action principle from which it emerges \eqref{EE}. To do that, we will use the formalism used in \cite{Ayon-Beato:2023bzp}. Hence, first we must to introduce the concept of Fréchet derivative.

Let $P[u]=P\left(x,u^{(n)}\right)$ be a differential function, i.e. that depends in the point $x$, the function $u$ together with its derivatives $u^{(n)}$. Consider now its variation under a one-parameter family of functions. After interchanging the variation with derivatives (and without integrating by parts) we end with a differential operator acting on an arbitrary variation $\d u$ called the \textit{Fréchet derivative} of $P$,
\begin{equation}
  \d P=\dv{\varepsilon}\eval{P[u+\varepsilon\d u]}_{\varepsilon=0}:=D_P(\d u)
\end{equation}
Here, we consider the second order conformally invariant psudoscalar defined by
\begin{equation}\label{E-frechet}
  \E=\sqrt{-\tilde{g}}E(\f,\tilde{\nabla}_\m\f ,\tilde{\nabla}_\m\nabla_\n \f ,\tilde{g}^{\m\n },\tilde{C}_{\m\n }^{\a\b })
\end{equation}
which is the natural quantity that could be derived from a convariant action. The role of he dependent function $u$ is played by the zero weight conformally invariant scalar field $\f$,  and hence the Fréchet derivative of $\E$ can be calculated from
\begin{equation}
  D_\E(\d\f )=\d_\f\E\, .
\end{equation}

From \eqref{E-frechet} and using that $\d \f=0$, we have
\begin{align*}
  \d_\f \E&=-\frac{1}{2}\sqrt{-\tilde{g}}\tilde{g}_{\m\n }\d_\f \tilde{g}^{\m\n }+\sqrt{-\tilde{g}}\left(\pdv{E}{\f }\d\f \pdv{E}{(\tilde{\nabla}_\m\f )}\d_\f(\tilde{\nabla}_\m\f )+\pdv{E}{(\tilde{\nabla}_\m\tilde{\nabla}_\n\f )}\d_\f(\tilde{\nabla}_\m\tilde{\nabla}_\n \f )\right.\\&\hspace{5cm}\left.+\pdv{E}{\tilde{g}^{\m\n }}\d_\f\tilde{g}^{\m\n }+\tilde{P}^{\m\n }_{\a\b }\d_\f \tilde{C}^{\a\b }_{\m\n }\right)\\
  &=-\frac{1}{2}\sqrt{-\tilde{g}}\tilde{g}_{\m\n }\d_\f \tilde{g}^{\m\n }+\sqrt{-\tilde{g}}\left[\pdv{E}{\tilde{g}^{\m\n }}\d_\f\tilde{g}^{\m\n }+ E_\f\d\f +E^\m_\f \d_\f(\tilde{\nabla}_\m\f )+E^{\m\n }_\f \d_\f(\tilde{\nabla}_\m\tilde{\nabla}_\n\f )+\tilde{P}^{\m\n }_{\a\b }\cancel{\d_\f \tilde{C}^{\a\b }_{\m\n }}\right]\\
  &=-\frac{1}{2}\sqrt{-\tilde{g}}\tilde{g}_{\m\n }\d_\f \tilde{g}^{\m\n }+\sqrt{-\tilde{g}}\left[\pdv{E}{\tilde{g}^{\m\n }}\d_\f\tilde{g}^{\m\n }+ E_\f\d\f +E^\m_\f \d_\f(\tilde{\nabla}_\m\f )+E^{\m\n }_\f \d_\f(\tilde{\nabla}_\m\tilde{\nabla}_\n\f )\right]
\end{align*}
where we have defined
\begin{equation}
  E_\f:=\pdv{E}{\f },\qquad E^\m_\f:=\pdv{E}{(\tilde{\nabla}_\m\f )},\qquad E^{\m\n }_\f:=\pdv{E}{(\tilde{\nabla}_\m\tilde{\nabla}_\n \f )}
\end{equation}
in order to reduce some notation and we used the fact that the Weyl tensor is conformally invariant, $\tilde{C}^{\a\b }_{\m\n }=C^{\a\b }_{\m\n }$, so that $\d_\f C^{\a\b }_{\m\n }=0$. Then,
\begin{align}
  \d_\f \E &=-\frac{1}{2}\sqrt{-\tilde{g}}\tilde{g}_{\m\n }\d_\f \tilde{g}^{\m\n }+\sqrt{-\tilde{g}}\left[\pdv{E}{\tilde{g}^{\m\n }}\d_\f\tilde{g}^{\m\n }+ E_\f\d\f +E^\m_\f \d_\f(\tilde{\nabla}_\m\f )+E^{\m\n }_\f \d_\f(\tilde{\nabla}_\m\tilde{\nabla}_\n\f )\right]
\end{align}
Using \eqref{d-sqrt-E}, we can write
\begin{align}
  \d_\f \E &=\sqrt{-\tilde{g}}\left[\frac{1}{\sqrt{-\tilde{g}}}\pdv{(\sqrt{-\tilde{g}}E)}{\tilde{g}^{\m\n }}\d_\f\tilde{g}^{\m\n }+ E_\f\d\f +E^\m_\f \d_\f(\tilde{\nabla}_\m\f )+E^{\m\n }_\f \d_\f(\tilde{\nabla}_\m\tilde{\nabla}_\n\f )\right]\\
  &=\sqrt{-\tilde{g}}\left[-\frac{1}{2}E_{\m\n }+ E_\f\d\f +E^\m_\f \d_\f(\tilde{\nabla}_\m\f )+E^{\m\n }_\f \d_\f(\tilde{\nabla}_\m\tilde{\nabla}_\n\f )\right]\label{df-E}
\end{align}
where 
\begin{equation}
  E_{\m\n }:=-\frac{2}{\sqrt{-\tilde{g}}}\pdv{(\sqrt{-\tilde{g}}E)}{\tilde{g}^{\m\n }}
\end{equation}








Before proceeding, let's compute the variation w.r.t the scalar field $\f$ of some quantities:
\begin{align}
  \d_\f \tilde{g}_{\m\n }&=\d_\f (Xg_{\m\n })\\
  &=\d_\f \left(-\frac{1}{2}(\nabla\f )^2g_{\m\n }\right)\\
  &=-\nabla^\a \f \nabla_\a \d \f g_{\m\n }\\
  &=-g_{\m\n }g^{\a\b }\nabla_\a\f\nabla_\b\d\f  \\
  &=-g_{\m\n }g^{\a\b }\partial_\a\f\partial_\b\d\f\\
  &=-g_{\m\n }Xg^{\a\b }X^{-1}\partial_\a\f\partial_\b\d\f \\
  &=-\tilde{g}_{\m\n }\tilde{\nabla}^\a \f\tilde{\nabla}_\a\d\f 
\end{align}

\begin{align}
  \d_\f \tilde{g}^{\m\n }&=\d_\f\left[\left(-\frac{1}{2}(\nabla\f )^2\right)^{-1} g^{\m\n }\right]\\
  &=-\frac{1}{X^2}(-\nabla^\a\f\nabla_\a\d\f )g^{\m\n }\\
  &=\frac{1}{X^2}g^{\m\n }g^{\a\b }\partial_\a\f\partial_\b\d\f \\
  &=X^{-1}g^{\m\n }X^{-1}g^{\a\b }\partial_\a\f\partial_\b\d\f \\
  &=\tilde{g}^{\m\n }\tilde{\nabla}^\a\f\tilde{\nabla}_\a\d\f 
\end{align}

\begin{align}
  \d_\f \sqrt{-\tilde{g}}&=-\frac{1}{2}\sqrt{-\tilde{g}}\tilde{g}_{\m\n } \d_\f\tilde{g}^{\m\n }\\
  &=-\frac{1}{2}\sqrt{-\tilde{g}}\tilde{g}_{\m\n}\tilde{g}^{\m\n }\tilde{\nabla}^\a\f\tilde{\nabla}_\a\d\f \\
  &=-\frac{D}{2}\sqrt{-\tilde{g}}\tilde{\nabla}^\a\f \tilde{\nabla}_\a\d\f 
\end{align}

\begin{align}
  \d_\f (\tilde{\nabla}_\m \f)=\tilde{\nabla}_\m\d\f 
\end{align}

\begin{align}
  \d_\f (\tilde{\nabla}_\m\tilde{\nabla}_\n\f )&=\d_\f (\tilde{\nabla}_\m\partial_\n \f )\\
  &=\d_\f \left(\partial_\m \partial_\n \f -\tilde{\G }^\lambda_{~\m\n }\partial_\lambda\f \right)\\
  &=\partial_\m\partial_\n\d \f -\d \f\tilde{\G }^\lambda_{~\m\n }\partial_\lambda\f-\tilde{\G }^\lambda_{~\m\n }\partial_\lambda\d\f \\
  &=\tilde{\nabla}_\m\f\tilde{\nabla}_\n\d \f -\tilde{\nabla}_\lambda\f\d\f \tilde{\G }^\lambda_{~\m\n }\\
  &=\tilde{\nabla}_\m\f\tilde{\nabla}_\n\d \f -\tilde{\nabla}_\lambda\f \d_\f \tilde{\G }^\lambda_{~\m\n }\\
  &=\tilde{\nabla}_\m\f\tilde{\nabla}_\n\d \f-\tilde{\nabla}_\lambda\f\frac{1}{2}\tilde{g}^{\lambda\r }\left(\tilde{\nabla}_\m\d_\f \tilde{g}_{\n\r }+\tilde{\nabla}_\n \d_\f \tilde{g}_{\m\r }-\tilde{\nabla}_\r \d_\f \tilde{g}_{\m\n }\right)\\
  &=\tilde{\nabla}_\m\f\tilde{\nabla}_\n\d \f-\frac{1}{2}\tilde{\nabla}^\r\f \left(2\tilde{\nabla}_{(\m }\d_\f \tilde{g}_{\n)\r }-\tilde{\nabla}_\r \d_\f \tilde{g}_{\m\n }\right)
\end{align}


Therefore,
\begin{align}
  E^{\m\n }_\f \d_\f (\tilde{\nabla}_\m\tilde{\nabla}_\n\f )&=E^{\m\n }_\f \left[\tilde{\nabla}_\m\f\tilde{\nabla}_\n\d \f-\frac{1}{2}\tilde{\nabla}^\r\f \left(2\tilde{\nabla}_{(\m }\d_\f \tilde{g}_{\n)\r }-\tilde{\nabla}_\r \d_\f \tilde{g}_{\m\n }\right)\right]\\
  &=E^{\m\n }_\f \left[\tilde{\nabla}_\m\f\tilde{\nabla}_\n\d \f+\tilde{\nabla}^\r\f\tilde{\nabla}_\m(\tilde{g}_{\n\r }\tilde{\nabla}^\a\f\tilde{\nabla}_\a\d\f )-\frac{1}{2}\tilde{\nabla}^\r\f\tilde{\nabla}_\r (\tilde{g}_{\m\n }\tilde{\nabla}^\a\f\tilde{\nabla}_\a\d\f )\right]\\
  &=E^{\m\n }_\f \left[\tilde{\nabla}_\m\f\tilde{\nabla}_\n\d\f +\tilde{\nabla}_\n\f \tilde{\nabla}_\m(\tilde{\nabla}^\a\f\tilde{\nabla}_\a\d\f )-\frac{1}{2}\tilde{g}_{\m\n }\tilde{\nabla}^\r\f\tilde{\nabla}_\r(\tilde{\nabla}^\a\f\tilde{\nabla}_\a\d\f )\right]
\end{align}

So \eqref{df-E} becomes

%[\rc{Complete!!}]

\begin{align*}
  \d_\f\E &=\sqrt{-\tilde{g}}\left\{E_\f\d\f +E^\m_\f \tilde{\nabla}_\m\d\f -\frac{1}{2}E_{\m\n }\tilde{g}^{\m\n }\tilde{\nabla}^\a \tilde{\nabla}_\a \d\f \right.\\
  &  \quad \left. + E^{\m\n }_\f \left[\tilde{\nabla}_\m \tilde{\nabla}_\n \d\f +\tilde{\nabla}_\n\f\tilde{\nabla}_\m (\tilde{\nabla}^\a\tilde{\nabla}_\a\d\f )-\frac{1}{2}\tilde{g}_{\m\n }\tilde{\nabla}^\r\f\tilde{\nabla}_\r (\tilde{\nabla}^\a\f\tilde{\nabla}_\a\d\f )\right]\right\} \\
  &=\sqrt{-\tilde{g}}\left\{E_\f\d_\f+\left(E^\m_\f\tilde{\nabla}_\m-\frac{1}{2}E_{\r\tau }\tilde{g}^{\r\tau }\tilde{\nabla}^\m\f \right)\tilde{\nabla}_\m\d\f +E^{\m\n }_\b\tilde{\nabla}_\m\tilde{\nabla}_\n \d\f +E^{\m\n }_\f\tilde{\nabla}_\n \f(\tilde{\nabla}_\m \tilde{\nabla}^\a\f )\tilde{\nabla}_\a\d\f \right.\\
  &\left.\quad + E^{\m\n }_\f\tilde{\nabla}_\n\f\tilde{\nabla}^\a \f(\tilde{\nabla}_\m\tilde{\nabla}_\a\d\f )-\frac{1}{2}E^{\m\n}_\f \tilde{g}_{\m\n }\tilde{\nabla}^\r\f(\tilde{\nabla}_\r\tilde{\nabla}^\a \f )\tilde{\nabla}_\a\d\f  -\frac{1}{2}E^{\m\n }_\f\tilde{g}_{\m\n }\tilde{\nabla}^\r\f\tilde{\nabla}^\a\f(\tilde{\nabla}_\r\tilde{\nabla}_\a\d\f )\right\}
\end{align*}

At this point it is convenient to introduce the following notation
\begin{equation}
  \tr \tilde{E}:=\tilde{g}^{\m\n }E_{\m\n },\qquad \tr\tilde{E}_\f :=\tilde{g}_{\m\n }\tilde{E}^{\m\n }_\f 
\end{equation}
Therefore,
\begin{align*}
   \d_\f\E &=\sqrt{-\tilde{g}}\left\{E_\f\d_\f+\left(E^\m_\f\tilde{\nabla}_\m-\frac{1}{2}\tr\tilde{E }\, \tilde{\nabla}^\m\f \right)\tilde{\nabla}_\m\d\f +E^{\m\n }_\f \tilde{\nabla}_\m\tilde{\nabla}_\n \d\f +E^{\m\n }_\f\tilde{\nabla}_\n \f(\tilde{\nabla}_\m \tilde{\nabla}^\a\f )\tilde{\nabla}_\a\d\f \right.\\
  &\left.\quad + E^{\m\n }_\f\tilde{\nabla}_\n\f\tilde{\nabla}^\a \f(\tilde{\nabla}_\m\tilde{\nabla}_\a\d\f )-\frac{1}{2}E^{\m\n}_\f \tilde{g}_{\m\n }\tilde{\nabla}^\r\f(\tilde{\nabla}_\r\tilde{\nabla}^\a \f )\tilde{\nabla}_\a\d\f  -\frac{1}{2}\tr \tilde{E}_\f \tilde{\nabla}^\r\f\tilde{\nabla}^\a\f(\tilde{\nabla}_\r\tilde{\nabla}_\a\d\f )\right\}\\
  &=\sqrt{-\tilde{g}}\left\{E_\f\d_\f+\left(E^\m_\f\tilde{\nabla}_\m-\frac{1}{2}\tr\tilde{E }\, \tilde{\nabla}^\m\f \right)\tilde{\nabla}_\m\d\f +E^{\m\n }_\f \tilde{\nabla}_\m\tilde{\nabla}_\n \d\f +E^{\m\n }_\f\tilde{\nabla}_\n \f(\tilde{\nabla}_\m \tilde{\nabla}^\a\f )\tilde{\nabla}_\a\d\f \right.\\
  &\left.\quad + E^{\m\n }_\f\tilde{\nabla}_\n\f\tilde{\nabla}^\a \f(\tilde{\nabla}_\m\tilde{\nabla}_\a\d\f )-\frac{1}{2}E^{\m\n}_\f \tilde{g}_{\m\n }\tilde{\nabla}^\r\f(\tilde{\nabla}_\r\tilde{\nabla}^\a \f )\tilde{\nabla}_\a\d\f \right.\\
   &\left.\quad  -\frac{1}{2}\tr\tilde{E}_\f \left[\tilde{\nabla}^\r\f (\tilde{\nabla}_\r\tilde{\nabla}^\a\f )\tilde{\nabla}_\a\d\f +\tilde{\nabla}^\r\f\tilde{\nabla}^\a\f (\tilde{\nabla}_\r\tilde{\nabla}_\a\d\f )\right]\right\}\\
   &=\sqrt{-\tilde{g}}\left\{E_\f\d\f +\left(E^\m_\f-\frac{1}{2}\tr \tilde{E}\tilde{\nabla}^\m\f \right)\tilde{\nabla}_\m\d\f +\left[E^{\m\n }_\d\tilde{\nabla}_\n\f (\tilde{\nabla}_\m\tilde{\nabla}^\a\f )-\frac{1}{2}\tr \tilde{E}_\f \tilde{\nabla}^\r\f (\tilde{\nabla}_\r\tilde{\nabla}^\a\f )\right]\tilde{\nabla}_\d\f\right. \\
   &\left. \quad +E^{\m\n }_\f\tilde{\nabla}_\m\tilde{\nabla}_\n \d\f +E^{\m\n }_\f\tilde{\nabla}_\n\f\tilde{\nabla}^\a\f (\tilde{\nabla}_\m\tilde{\nabla}_\a\d\f )-\frac{1}{2}\tr \tilde{E}_\f \tilde{\nabla}^\r\f\tilde{\nabla}^\a\f (\tilde{\nabla}_\r\tilde{\nabla}_\a\d\f )\right\}\\
    &=\sqrt{-\tilde{g}}\left\{E_\f\d\f +\left[E^\m_\f \tilde{\nabla}_\m-\frac{1}{2}\tr\tilde{E}\tilde{\nabla}^\m\f +\left(E^{\a\b }_\f\tilde{\nabla}_\f-\frac{1}{2}\tr \tilde{E}_\b\f \tilde{\nabla}^\a\f \right)\tilde{\nabla}_\a\tilde{\nabla}^\m\f \right]\tilde{\nabla}_\m\d\f\right.\\
    &\quad \left.+\left[E^{\m\n }_\f+\left(E^{\m\a }_\f \tilde{\nabla}_\n\f-\frac{1}{2}\tr\tilde{E}_\f \tilde{\nabla}^\m\f \right)\tilde{\nabla}^\n\f \right]\tilde{\nabla}_\m\tilde{\nabla}_\n\d\f  \right\}
\end{align*}
Therefore, the Fréchet derivative is given by the following operator

\begin{equation}\label{DE}
  D_\E =\sqrt{-\tilde{g}}\left[E_\f +H^\m\tilde{\nabla}_\m +H^{(\m\n )}\tilde{\nabla}_\m\tilde{\nabla}_\n \right]
\end{equation}
where
\begin{align}
  H^\m &:=E^\m_\f -\frac{1}{2}\tr\tilde{E}\tilde{\nabla}^\m\f +\left(E^{\a\b }_\f\tilde{\nabla}_\b \f -\frac{1}{2}\tr \tilde{E}_\f\tilde{\nabla}^\a\f \right)\tilde{\nabla}_\a\tilde{\nabla}^\m\f \\
  H^{\m\n }&:=E^{\m\n }_\f+\left(E^{\m\a }_\f \tilde{\nabla}_\a \f-\frac{1}{2}\tr\tilde{E}_\f\tilde{\nabla}^\m\f \right)\tilde{\nabla}^\n\f 
\end{align}
%[\rc{Check possible typos!}]



At this point let us remember that the adjoint of a differential operator $O$, denoted bu $O^\dagger$  satisfies
\begin{equation}
	\int\dd^Dx AO(B)=\int\dd ^DxBO^\dagger(A)
\end{equation}
for every pair of differential functions $A$ and $B$, with equality achieved up to boundary terms.
In order to ensure that the equations arise from an action principle, we need $D_\E $ to be self-adjoint

\begin{align}
  \int\dd^DxAD_\E (B)&=\int\dd^Dx\sqrt{-\tilde{g}}\left[A\left(E_\f +H^\m\tilde{\nabla}_\m +H^{(\m\n )}\tilde{\nabla}_\m\tilde{\nabla}_\n\right)B\right]\\
  &=\int\dd^Dx\sqrt{-\tilde{g}}\left[AE_\f B +AH^\m \tilde{\nabla}_\m B+AH^{\m\n }\tilde{\nabla}_\m\tilde{\nabla}_\n B\right]\\
  &=\int\dd^Dx\sqrt{-\tilde{g}}\left[BE_\f A-B\tilde{\nabla}_\m (H^\m A)-\tilde{\nabla}_\m (AH^{\m\n })\tilde{\nabla}_\n B\right]+\rm b.t\\
  &=\int\dd^Dx\sqrt{-\tilde{g}}\left[BE_\f A-B\tilde{\nabla}_\m (H^\m A)+B\tilde{\nabla}_\m\tilde{\nabla}_\n (H^{\m\n }A)\right]+\rm b.t
\end{align}
Then, 
\begin{align}
  D^\dagger_\E (A)&=\sqrt{-\tilde{g}}\left[E_\f A-\tilde{\nabla}_\m (H^\m A)+\tilde{\nabla}_\m\tilde{\nabla}_\n (H^{\m\n }A)\right]\\
  &=\sqrt{-\tilde{g}}\left[E_\f A-A\tilde{\nabla}_\m H^\m -H^\m \tilde{\nabla}_\m A+\tilde{\nabla}_\n (A\tilde{\nabla}_\m H^{\m\n }+H^{\m\n }\tilde{\nabla}_\m A)\right]
\end{align}
But, in order to make appear $D_\E$ from \eqref{DE}, we can add an smart zero,
\begin{align*}
  D^\dagger_\E (A)&=\sqrt{-\tilde{g}}\left[E_\f A-\tilde{\nabla}_\m (H^\m A)+\tilde{\nabla}_\m\tilde{\nabla}_\n (H^{\m\n }A)\right]+2H^\m \tilde{\nabla}_\m A-2H^\m \tilde{\nabla}_\m A\\
  &=\sqrt{-\tilde{g}}\left[E_\f A-A\tilde{\nabla}_\m H^\m +H^\m \tilde{\nabla}_\m A+\tilde{\nabla}_\n A\tilde{\nabla}_\m H^{\m\n }+A\tilde{\nabla}_\m \tilde{\nabla}_\n H^{\m\n }\right. \\
  &\left.\quad +\tilde{\nabla}_\n H^{\m\n }\tilde{\nabla}_\m A +H^{\m\n }\tilde{\nabla}_\m \tilde{\nabla}_\n A -2H^\m \tilde{\nabla}_\m A\right]\\
  &=D_\E (A)+\sqrt{-\tilde{g}}\left[-A\tilde{\nabla}_\m H^\m +\tilde{\nabla}_\n A\tilde{\nabla}_\m H^{\m\n }+A\tilde{\nabla}_\m \tilde{\nabla}_\n H^{\m\n }+\tilde{\nabla}_\n H^{\m\n }\tilde{\nabla}_\m A-2H^\m \tilde{\nabla}_\m A\right]\\
  &=D_\E (A)+\sqrt{-\tilde{g}}\left[-A\tilde{\nabla}_\m H^\m +2\tilde{\nabla}_\m H^{\m\n }\tilde{\nabla}_\n A+A\tilde{\nabla}_\m \tilde{\nabla}_\n H^{\m\n }-2H^\m \tilde{\nabla}_\m A\right]\\
  &=D_\E (A)+\sqrt{-\tilde{g}}\left[2J^\m \tilde{\nabla}_\m A+\tilde{\nabla}_\m J^\m A\right]
\end{align*}
where we have defined
\begin{equation}
  J^\m :=\tilde{\nabla}_\n H^{(\n\m) }-H^\m 
\end{equation}

Thus, for $D_\E $ to be self-adjoint, it must be fulfilled that
\begin{equation}
  D^\dagger _\E (A)=D_\E (A) \Leftrightarrow J^\m =0
\end{equation}
and the Helmholtz condition is reduced to
\begin{tcolorbox}
\begin{equation}\label{Hemholtz-condition}
\boxed{  \tilde{\nabla}_\n H^{(\n\m) }=H^\m }
\end{equation}
where 
\begin{align}
  H^\m &:=\left[E^\m_\f -\frac{1}{2}\tr\tilde{E}\tilde{\nabla}^\m\f +\left(E^{\a\b }_\f\tilde{\nabla}_\b \f -\frac{1}{2}\tr \tilde{E}_\f\tilde{\nabla}^\a\f \right)\tilde{\nabla}_\a\tilde{\nabla}^\m\f \right]\\
  H^{\m\n }&:=E^{\m\n }_\f+\left(E^{\m\a }_\f \tilde{\nabla}_\a \f-\frac{1}{2}\tr\tilde{E}_\f\tilde{\nabla}^\m\f \right)\tilde{\nabla}^\n\f 
\end{align}
\end{tcolorbox}

Helmholtz conditions imply that a symmetric second rank tensor depending up to second order in the scalar field $\f$ and auxiliary metric $\tilde{g}_{\m\n }$ has a second order divergence. Horndeski characterized the most general tensor with these properties \cite{Horndeski:1974wa}. Consequently, $H^{\m\n }$ belongs to the Horndeski family built for the auxiliary metric $\tilde{g}_{\m\n }$.

\subsection{Quantities written in the auxiliary frame}
As an example, let's consider again the action principle
\begin{equation}\label{S-auxiliar-frame}
  S[g_{\m\n },\f ]=\int\dd^Dx\sqrt{-g}X^{D/2}
\end{equation}

Its equation is rewritten in the auxiliary frame as
\begin{align}
  \E&=\sqrt{-g}E\\
  &=\sqrt{-g}\nabla_\m \left(X^{\frac{D-2}{2}}\nabla^\m \f \right)\\
  &=\partial_\m \left(\underbrace{\sqrt{-g}X^\frac{D}{2}}_{\sqrt{-\tilde{g}}}\underbrace{X^{-1}g^{\m\n }}_{\tilde{g}^{\m\n }}\partial_\n \f \right)\\
  &=\partial_\m (\sqrt{-\tilde{g}}\tilde{g}^{\m\n }\partial_\n \f )\\
  &=\sqrt{-\tilde{g}}\tilde{g}^{\m\n }\tn_\m\tn_\n\f \\
  &=\sqrt{-\tilde{g}}\tilde{\Box }\f 
\end{align}
which implies that
\begin{equation}
  E=\tilde{\Box }\f =\tilde{g}^{\m\n }\tn_\m \tn_\n\f 
\end{equation}
Thus
\begin{equation}
  E^{\m\n }_\f :=\pdv{E}{(\tn_\m\tn_\n\f )}=\tilde{g}^{\m\n }
\end{equation}
So then
\begin{align}
  H^{\m\n }&:=E^{\m\n }_\f+\left(E^{\a\m  }_\f \tilde{\nabla}_\a \f-\frac{1}{2}\tr\tilde{E}_\f\tilde{\nabla}^\m\f \right)\tilde{\nabla}^\n\f \notag \\
  &=\tilde{g}^{\m\n } +\left(\tilde{g}^{\a\m }\tn_\a\f -\frac{1}{2}\tilde{g}_{\lambda\r }\tilde{E}^{\lambda\r }_\f \tn^\m\f \right)\tn^\n\f \notag  \\
  &=\tilde{g}^{\m\n }+\left(\tn^\m\f-\frac{1}{2}\tilde{g}_{\lambda\r }\tilde{g}^{\lambda\r }\tn^\m\f \right)\tn^\n\f \notag \\
  &=\tilde{g}^{\m\n }\left(\tn^\m\f -\frac{D}{2}\tn^\m\f \right)\tn^\n\f \notag  \\
  &=\tilde{g}^{\m\n }-\frac{(D-2)}{2}\tn^\m\f\tn^\n\f \label{Hmn-auxiliar-frame}
\end{align}

Furthermore,
\begin{align}
  E_{\m\n }&:=-\frac{2}{\sqrt{-\tilde{g}}}\pdv{(\sqrt{-\tilde{g}}E)}{\tilde{g}^{\m\n }}\\
  &=-\frac{2}{\sqrt{-\tilde{g}}}\left(-\frac{1}{2}\sqrt{-\tilde{g}}\tilde{g}_{\m\n }E+\sqrt{-\tilde{g}}\pdv{E}{\tilde{g}^{\m\n }}\right)\\
  &=-2\left(-\frac{1}{2}\tilde{g}_{\m\n }E+\tn_\m\tn_\n\f \right)\\
  &=-2\left(-\frac{1}{2}\tilde{g}_{\m\n }\tilde{\Box }\f +\tn_\m\tn_\n\f \right)\\
  &=\tilde{g}_{\m\n }\tilde{\Box }\f-2\tn_\m\tn_\n\f 
\end{align}
which implies
\begin{equation}
  \tr\tilde{E}:=\tilde{g}^{\m\n }E_{\m\n }=D\tilde{\Box}\f-2\tilde{\Box }\f=(D-2)\tilde{\Box}\f 
\end{equation}
It is also clear to see that
\begin{equation}
  E^\m_\f :=\pdv{E}{(\tn_\m \f )}=0
\end{equation}
So then
\begin{align}
  H^\m &:=\left[E^\m_\f -\frac{1}{2}\tr\tilde{E}\tilde{\nabla}^\m\f +\left(E^{\a\b }_\f\tilde{\nabla}_\b \f -\frac{1}{2}\tr \tilde{E}_\f\tilde{\nabla}^\a\f \right)\tilde{\nabla}_\a\tilde{\nabla}^\m\f \right]\\
  &=-\frac{1}{2}(D-2)\tilde{\Box}\f\tn^\m\f+\left(\tilde{g}^{\a\b }\tn_\b\f -\frac{1}{2}\tilde{g}_{\lambda\r }\tilde{E}^{\lambda\r }_\f \tn^\a\f \right)\tn_\a\tn_\m\f \\
  &=-\frac{(D-2)}{2}\tilde{\Box }\f\tn^\m\f +\left(\tn^\a\f-\frac{1}{2}\tilde{g}_{\lambda\r }\tilde{g}^{\lambda\r }\tn^{\a}\f \right)\tn_\a\tn^\m\f \\
  &=-\frac{(D-2)}{2}\tilde{\Box\f }\tn^\m \f +\left(\tn^{\a }\f-\frac{D}{2}\tn^{\a }\f \right)\tn_\a\tn^\m\f \\\
  &=-\frac{(D-2)}{2}\tilde{\Box}\f \tn^\m\f -\frac{(D-2)}{2}\tn^{\a }\f\tn_\a\tn^\m\f \\
  &=-\frac{(D-2)}{2}\left(\tilde{\Box}\f\tn^\m\f +\tn^{\a }\f\tn_\a\tn^\m \f \right)
\end{align}
Finally, we have
\begin{align}
  \tn_\m H^{\m\n }&=\tn_\m\left( \tilde{g}^{\m\n }-\frac{(D-2)}{2}\tn^\m\f\tn^\n\f \right)\\
  &=-\frac{(D-2)}{2}\tn_\m \left(\tn^\m\f\tn^\n \f \right)\\
  &=-\frac{(D-2)}{2}\left(\tilde{\Box}\f \tn^\n\f+\tn^\m\f\tn_\m\tn^\n\f \right)\\
  &=H^\n 
\end{align}

That is to say, the equations of motion coming from \eqref{S-auxiliar-frame}, writen in the auxiliar frame, satisfy the Hemholtz conditions \eqref{Hemholtz-condition}.


\section{Horndeski theorem}
Horndeski theorem says that in a space of dimension four, the most general symmetric contravariant tensor density of the form
\begin{equation}
  A^{ab}=A^{ab}(g_{ij},\partial_hg_{ij},\partial_{h}\partial_k g_{ij},\f,\partial_h\f,\partial_h\partial_k\f )
\end{equation}
which is such that $\nabla_aA^{ab }$ is at most of second-order in the derivatives of both $g_{ij}$ and $\f$ is given by
\begin{equation}\label{Horndeski}
\begin{split}
  A^{ab}&=\sqrt{-g}\left\{K_1\d^{acde}_{fhjk}g^{fb}\nabla^h\nabla_c\f R^{jk}_{de}+K_2\d^{acd}_{efh}g^{eb}R^{fh}_{cd}\right.\\
  &\left.\qquad +K_3\d^{acde}_{fhjk}g^{fb}\nabla_c\f\nabla^h\f R^{jk}_{de}+K_4\d^{acde}_{fhjk}g^{fb}\nabla_h\nabla_c\f \nabla^j\nabla_d\f \nabla^k\nabla_e\f \right.\\
  &\left.\qquad +K_5\d^{acd}_{efh}g^{eb}\nabla^f\nabla_c\f \nabla^h\nabla_d\f +K_6\d^{acde}_{fhjk}g^{fb}\nabla_c\f \nabla^h\f \nabla^j\nabla_f\f \nabla^k\nabla_e\f \right.\\
  &\left.\qquad +K_7\d^{ac}_{de}g^{db}\nabla^{e}\nabla_c\f+K_8\d^{acd}_{efh}g^{eb}\nabla_c\f \nabla^f\f \nabla^h\nabla_c\nabla_d\f +K_9g^{ab}+K_{10}\nabla^{a}\f\nabla^b\f  \right\}
\end{split}
\end{equation}
%TODO check possible typos.
where $K_i $ are arbitrary differentiable functions of $\f$ and $\partial_i\f$ \cite{Horndeski:1974wa}.

Note that the dependence on the scalar curvature only appears in the first three terms of \eqref{Horndeski} with all indices contracted with the Kronecker delta, then, from its irreducible decomposition
\begin{equation}\label{R-CSR}
  R^{ab}_{mn}=C^{ab}_{mn}+2\d^{[a}_{[m}S^{b]}_{n]}+\frac{1}{6}\d^{a}_{[m}\d^b_{n]}R
\end{equation}
 only the terms proportional to the trace of the Ricci tensor survive.
 
 Now, using the identities
 \begin{equation}
  \d^{\m_1...\m_p}_{\n_1...\n_p}\d^{\n_1}_{\m_1}\cdots \d^{\n_k}_{\m_k}=\frac{(D-p+k)!}{(D-p)!}\d^{\m_{k+1}...\m_p}_{\n_{k+1}...\n_p}
\end{equation}
and
\begin{equation}
  {\displaystyle \delta _{\nu _{1}\dots \nu _{p}}^{\mu _{1}\dots \mu _{p}}=p!\delta _{[\nu _{1}}^{\mu _{1}}\dots \delta _{\nu _{p}]}^{\mu _{p}}=p!\delta _{\nu _{1}}^{[\mu _{1}}\dots \delta _{\nu _{p}}^{\mu _{p}]}}
\end{equation}
 let's see how the first three terms of \eqref{Horndeski} look like:
 
\begin{align}
  K_1\d^{acde}_{fhjk}g^{fb}\nabla^h\nabla_c\f R^{jk}_{de}&=\frac{K_1}{6}\d^{acde}_{fhjk}g^{fb}\nabla^h\nabla_c\f \d^j_d\d^k_eR\\
  &=\frac{K_1}{6}\frac{(4-4+2)!}{(4-4)!}\d^{ac}_{fh}g^{fb}\nabla^h\nabla_c\f R\\
  &=\frac{K_1}{3}\d^{ac}_{fh}g^{fb}\nabla^h\nabla_c\f R\label{K1}
\end{align}

\begin{align}
	K_2\d^{acd}_{efh}g^{eb}R^{fh}_{cd}&=\frac{K_2}{6}\d^{acd}_{efh}g^{eb}\d^c_e\d^h_dR\\
	&=\frac{K_2}{6}\frac{(4-3+2)!}{(4-3)!}\d^{a}_eg^{eb}R\\
	&=K_2g^{ab}R\label{K2}
\end{align}

\begin{align}
  K_3\d^{acde}_{fhjk}g^{fb}\nabla_c\f\nabla^h\f R^{jk}_{de}&=\frac{K_3}{6}\d^{acde}_{fhjk}g^{fb}\nabla_c\f\nabla^h\f \d^j_d\d^k_eR\\
  &=\frac{K_3}{6}\frac{(4-4+2)!}{(4-4)!}\d^{ac}_{fh}g^{fb}\nabla_c\f\nabla^h\f R\\
  &=\frac{K_3}{3}\d^{ec}_{fh}g^{fb}\nabla_c\f\nabla^h\f R\label{K3}
\end{align}

Adding these terms, we have
\begin{align*}
  [\eqref{K1}+\eqref{K2}+\eqref{K3}]&=\frac{1}{3}\d^{ac}_{fh}g^{fb}\left(K_1\nabla^h\nabla_c\f +K_3\nabla_c\f\nabla^h\f \right)R+K_2g^{ab}R\\
  &=\frac{1}{3}\d^{a}_f\d^{c}_{h}g^{fb}(K_1\nabla^h\nabla_c\f +K_3\nabla_c\f\nabla^h\f ) -\frac{1}{3}\d^{a}_h\d^{c}_{f}g^{fb}(K_1\nabla^h\nabla_c\f +K_3\nabla_c\f\nabla^h\f )\\
  &\quad +K_2g^{ab}R\\
  &=\frac{1}{3}g^{ab}(K_1\Box\f +K_3\nabla_c\f \nabla^c\f )R -\frac{1}{3}g^{cb}(K_1\nabla^{a}\nabla_c\f +K_3\nabla_c \f \nabla^{a}\f )R+K_2g^{ab}R\\
  &=\frac{1}{3}g^{ab}(K_1\Box\f +K_3\nabla_c\f\nabla^c\f )R -\frac{1}{3}(K_1\nabla^{a}\nabla^b\f +K_3\nabla^{a}\f\nabla^b\f )R +K_2g^{ab}R\\
  &=\frac{1}{3}g^{ab}(K_1\Box\f -2K_3X)R -\frac{1}{3}(K_1\nabla^{a}\nabla^b\f +K_3\nabla^{a}\f\nabla^b\f )R +K_2g^{ab}R\\
  &=\frac{1}{3}g^{ab}R(K_1\Box\f+3K_2-2K_3X)-\frac{1}{3}(K_1\nabla^{a}\nabla^b\f +K_3\nabla^{a}\f\nabla^b\f )R
\end{align*}
Since conformal invariance only allows those second order tensors independent $S^{a}_b$ and $R$ and here are all the terms that depend on the curvature in \eqref{Horndeski}, we have
\begin{align}
  0=\pdv{R}\left(\frac{A^{ab}}{\sqrt{-g}}\right)&=\frac{1}{3}g^{ab}(K_1\Box\f+3K_2-2K_3X)-\frac{1}{3}(K_1\nabla^{a}\nabla^b\f +K_3\nabla^{a}\f\nabla^b\f )
\end{align}
Taking the trace,
\begin{equation}\label{trace-AR}
\begin{split}
 0= g_{ab}\pdv{R}\left(\frac{A^{ab}}{\sqrt{-g}}\right)&=\frac{4}{3}(K_1\Box\f+3K_2-2K_3X)-\frac{1}{3}(K_1\Box\f -2K_3X)\\
  &=K_1\Box\f +4K_2-2K_3X
\end{split}
\end{equation}
Since $K_i=K_i(\f,\partial_a\f )$ are independent of second scalar field derivatives we have
\begin{equation}
  K_1=0\qquad \text{and}\qquad K_2=\frac{1}{2}K_3X
\end{equation}
Plugging into \eqref{trace-AR},
\begin{align}
   0= g_{ab}\pdv{R}\left(\frac{A^{ab}}{\sqrt{-g}}\right)&=\frac{1}{3}g^{ab}\left(\frac{3}{2}K_3X-2K_3X\right)-\frac{1}{3}K_3(\nabla^{a}\f\nabla^b\f )\\
   &=-\frac{1}{6}g^{ab}K_3X-\frac{1}{3}K_3\nabla^{a}\f\nabla^b\f \\
   &=-\frac{K_3}{3}\left(\nabla^{a}\f\nabla^b\f +\frac{1}{2}g^{ab}X\right) \label{0-AR}
\end{align}
Now, using the fact that given any scalar field there always exists a vector field $Y^{a}$ for which \cite{Horndeski:1974wa}
\begin{equation}
  Y^{a}\nabla_a\f =0\qquad \text{and}\qquad Y^{a}Y_a\neq 0
\end{equation}
we can multiply \eqref{0-AR} by $Y_aY_b$,
\begin{align}
  0&=-\frac{K_3}{3}\left(\cancel{Y_a\nabla^{a}\f Y_b\nabla^b\f} +\frac{1}{2}Y_aY^{a}g^{ab}X\right) \\
  &=-\frac{K_3}{3}\frac{1}{2}Y^{a}Y_a X
\end{align}
but, $X\neq 0$, so then $K_3=0$. In summary we have
\begin{equation}\label{K1230}
  K_1=K_2=K_3=0\,.
\end{equation}

Let's consider now the associated divergence, calculated by Horndeski as \cite{Horndeski:1974wa}
\begin{equation}\label{nabla-A}
\begin{split}
  \frac{\nabla_b A^{ab}}{\sqrt{-g}}&=K_1'\d^{acde}_{fhjk}\nabla^f\f \nabla^h\nabla_c\f R^{jk}_{de}+2\dot{K}_2\d^{acd}_{efh}\nabla_p\f \nabla^{e}\nabla^p\f R^{fh}_{cd}\\
  &\quad +K_3\d^{acde}_{fhjk}\nabla^h\f \nabla^f\nabla_c\f R^{jk}_{de}+K_5\d^{acd}_{efh}\nabla^m\f R^{fe}_{mc}\nabla^h\nabla_d\f \\
  &\quad +2\dot{K}_1\d^{acde}_{fhjk}\nabla_p\f \nabla^f\nabla^p\f \nabla^h\nabla_c\f R^{jk}_{de}+\frac{3}{2}K_4\d^{acde}_{fhjk}\nabla^mR^{hf}_{mc}\nabla^j\nabla_d\f \nabla^k\nabla_e\f \\
  &\quad +\frac{1}{2}K_1\d^{acde}_{fhjk}\nabla^m\f R^{hf}_{mc}R^{jk}_{de}+K_2'\d^{acd}_{efh}\nabla^{e}\f R^{fh}_{cd}+\frac{1}{2}K_7\d^{ac}_{de}\nabla^m\f R^{ed}_{mc}\\
  &\quad +\frac{1}{2}K_8\d^{acd}_{efh}\nabla_c\f \nabla^f\f \nabla^m\f R^{he}_{md}+2\dot{K}_3\d^{acde}_{fhjk}\nabla_p\f \nabla^f\nabla^p\f \nabla_c\f R^{jk}_{de}\\
  &\quad +K_6\d^{acde}_{fhjk}\nabla_c\f \nabla^h\f \nabla^m\f R^{jf}_{md}\nabla^k\nabla_e\f +K_4'\d^{acde}_{fhjk}\nabla^f\f\nabla^h\nabla_c\f \nabla^j\nabla_d\f \nabla^k\nabla_e\f \\
  &\quad +2\dot{K}_5\d^{acd}_{efh}\nabla_p\f \nabla^{e}\nabla^p\f \nabla^f\nabla_c\f \nabla^h\nabla_d\f +K_6\d^{acde}_{fhjk}\nabla^h\f \nabla^f\nabla_c\f \nabla^j\nabla_d\f \nabla^k\nabla_e\f \\
  &\quad +2\dot{K}_8\d^{acd}_{efh}\nabla_p\f \nabla^{e}\nabla^p\f \nabla_c\f\nabla^f\f\nabla^h\nabla_d\f +2\dot{K}_6\d^{acde}_{fhjk}\nabla_p\f \nabla^f\nabla^p\f \nabla_c\f \nabla^h\f \nabla^j\nabla_d\f\nabla^k\nabla_e\f \\
  &\quad +2\dot{K}_4\d^{acde}_{fhjk}\nabla_p\f \nabla^f\nabla^p\f \nabla^h\nabla_c\f \nabla^j\nabla_d\f \nabla^k\nabla_e\f +K_5'\d^{acd}_{efh}\nabla^{e}\f \nabla^f\nabla_c\f \nabla^h\nabla_d\f \\
  &\quad +2\dot{K}_7\d^{ac}_{de}\nabla_p\f \nabla^d\nabla^p\f \nabla^{e}\nabla_c\f +K_8\d^{acd}_{efh}\nabla^f\f\nabla^{e}\nabla_c\f \nabla^h\nabla_d\f +(2\dot{K}_9+K_{10})\nabla_b\f \nabla^b\nabla^{a}\f \\
  &\quad+ K_7'\d^{ac}_{de}\nabla^d\f \nabla^{e}\nabla_c\f +\nabla^{a}\f (K_9'+\r K_{10}'+2\dot{K}_{10}\nabla^b\f \nabla^c\f \nabla_c\nabla_b\f +K_{10}\nabla^c\nabla_c\f )
\end{split}
\end{equation}
where a prime denotes a partial derivative with respect to $\f$ and a dot denotes a partial derivative with respect to $\r$. Pluggin \eqref{K1230} into \eqref{nabla-A} we obtain
\begin{align*}
  \frac{\nabla_b A^{ab}}{\sqrt{-g}}&=K_5\d^{acd}_{efh}\nabla^m\f R^{fe}_{mc}\nabla^h\nabla_d\f +\frac{3}{2}K_4\d^{acde}_{fhjk}\nabla^mR^{hf}_{mc}\nabla^j\nabla_d\f \nabla^k\nabla_e\f +\frac{1}{2}K_7\d^{ac}_{de}\nabla^m\f R^{ed}_{mc}\\
  &\quad +\frac{1}{2}K_8\d^{acd}_{efh}\nabla_c\f \nabla^f\f \nabla^m\f R^{he}_{md}+K_6\d^{acde}_{fhjk}\nabla_c\f \nabla^h\f \nabla^m\f R^{jf}_{md}\nabla^k\nabla_e\f\\&\quad  +K_4'\d^{acde}_{fhjk}\nabla^f\f\nabla^h\nabla_c\f \nabla^j\nabla_d\f \nabla^k\nabla_e\f +2\dot{K}_5\d^{acd}_{efh}\nabla_p\f \nabla^{e}\nabla^p\f \nabla^f\nabla_c\f \nabla^h\nabla_d\f \\&\quad +K_6\d^{acde}_{fhjk}\nabla^h\f \nabla^f\nabla_c\f \nabla^j\nabla_d\f \nabla^k\nabla_e\f +2\dot{K}_8\d^{acd}_{efh}\nabla_p\f \nabla^{e}\nabla^p\f \nabla_c\f\nabla^f\f\nabla^h\nabla_d\f \\ &\quad+2\dot{K}_6\d^{acde}_{fhjk}\nabla_p\f \nabla^f\nabla^p\f \nabla_c\f \nabla^h\f \nabla^j\nabla_d\f\nabla^k\nabla_e\f \\
  &\quad +2\dot{K}_4\d^{acde}_{fhjk}\nabla_p\f \nabla^f\nabla^p\f \nabla^h\nabla_c\f \nabla^j\nabla_d\f \nabla^k\nabla_e\f +K_5'\d^{acd}_{efh}\nabla^{e}\f \nabla^f\nabla_c\f \nabla^h\nabla_d\f \\
  &\quad +2\dot{K}_7\d^{ac}_{de}\nabla_p\f \nabla^d\nabla^p\f \nabla^{e}\nabla_c\f +K_8\d^{acd}_{efh}\nabla^f\f\nabla^{e}\nabla_c\f \nabla^h\nabla_d\f +(2\dot{K}_9+K_{10})\nabla_b\f \nabla^b\nabla^{a}\f \\
  &\quad+ K_7'\d^{ac}_{de}\nabla^d\f \nabla^{e}\nabla_c\f +\nabla^{a}\f (K_9'+\r K_{10}'+2\dot{K}_{10}\nabla^b\f \nabla^c\f \nabla_c\nabla_b\f +K_{10}\nabla^c\nabla_c\f )
\end{align*}
The scalar curvature dependence would be through those terms with the Riemann tensor contracted totally with the Kronecker delta
\begin{align*}
  \frac{6}{R}\frac{\nabla_b A^{ab}}{\sqrt{-g}}&=K_5\d^{acd}_{efh}\d^f_m\d^{e}_c\nabla^m\f \nabla^h\nabla_d\f +\frac{3}{2}K_4\d^{acde}_{fhjk}\d^h_m\d^f_c\nabla^m\nabla^j\nabla_d\f \nabla^k\nabla_e\f +\frac{1}{2}K_7\d^{ac}_{de}\d^{e}_m\d^d_c\nabla^m\f \\
  &\quad +\frac{1}{2}K_8\d^{acd}_{efh}\d^h_m\d^{e}_d\nabla_c\f \nabla^f\f \nabla^m\f +K_6\d^{acde}_{fhjk}\d^j_m\d^f_d\nabla_c\f \nabla^h\f \nabla^m\f\nabla^k\nabla_e\f\\
  &=-2K_5\d^{ad}_{fh}\nabla^f\f\nabla^h\nabla_d\f-\frac{3}{2}K_4\d^{ade}_{hjk}\nabla^h\f\nabla^j\nabla_d\f\nabla^k\nabla_e\f -\frac{3}{2}K_7\nabla^{a}\f \\
  &\quad +K_8\d^{ac}_{fh}\nabla_c\f \nabla^f\f\nabla^h\f +K_6\d^{ace}_{hjk}\nabla_c\f \nabla^h\f\nabla^j\f\nabla^k\nabla_e\f 
\end{align*}
Note that due to symmetry in the covariant derivative indices contracted with the antisymmetric Kronecker delta, the last two terms vanish. Therefore,
\begin{align*}
  \frac{6}{R}\frac{\nabla_b A^{ab}}{\sqrt{-g}}&=-2K_5\d^{ad}_{fh}\nabla^f\f\nabla^h\nabla_d\f-\frac{3}{2}K_4\d^{ade}_{hjk}\nabla^h\f\nabla^j\nabla_d\f\nabla^k\nabla_e\f -\frac{3}{2}K_7\nabla^{a}\f \\
  &=-2K_5\left(\d^{a}_f\d^d_h-\d^d_f\d^{a}_h\right)\nabla^f\f\nabla^h\nabla_d\f-\frac{3}{2}K_7\nabla^{a}\f  \\
  &\quad -\frac{3}{2}K_43!\left(\d^{a}_h\d^d_j\d^{e}_k+\d^{a}_j\d^d_k\d^{e}_h+\d^{a}_k\d^d_h\d^{e}_j-\d^{a}_j\d^d_h\d^{e}_k-\d^{a}_h\d^d_k\d^{e}_j-\d^{a}_j\d^d_k\d^{e}_h\right)\nabla^h\f\nabla^j\nabla_d\f\nabla^k\nabla_e\f \\
  &=-2K_5 (\nabla^{a}\Box\f -\nabla^d\f\nabla^{a}\nabla_d\f )-\frac{3}{2}K_7\nabla^{a}\f \\
  &\quad -9K_4\left[\nabla^{a}(\Box\f )^2+\nabla^{e}\f\nabla^{a}\nabla_d \f\nabla^d\nabla_e\f +\nabla^d\f\nabla^{e}\nabla_d\f \nabla^{a}\nabla_e\f -\nabla^d\f\nabla^{a}\nabla_d\f \Box\f \right.\\
  &\left.\qquad \qquad -\nabla^{a}\f \nabla^{e}\nabla_d\f\nabla^d\nabla_e\f -\nabla^d\f\nabla^{a}\nabla_d\f\Box\f  \right]
\end{align*}
Introducing the following notation,
\begin{equation}
  \f^{a}:=\nabla^{a}\f,\quad \f^{ab}:=\nabla^{a}\nabla^b \f ,\quad X^{a}:=\nabla^{a}X=-\nabla^b \nabla^{a}\nabla_b\f =-\f^b \nabla^{a}\f_b 
\end{equation}
we have
\begin{align*}
  \frac{6}{R}\frac{\nabla_b A^{ab}}{\sqrt{-g}}&=-2K_5(\f^{a}\Box\f +X^{a})-\frac{3}{2}K_7\f^{a}\\
  &\quad -9K_4\left[\f^{a}(\Box\f )^2-\nabla^{a}\f_dX^d-X^{e}\nabla^{a}\f_e+X^{a}\Box\f -\f^{a}\f_{ed}\f^{ed}+X^{a}\Box\f \right]\\
  &=-2K_5(\f^{a}\Box\f +X^{a})-\frac{3}{2}K_7\f^{a}\\
  &\quad -9K_4\left[\f^{a}(\Box\f )^2-\f^{a}\f_{ed}\f^{ed}+2X^{a}\Box\f -2X^d\nabla^{a}\f_d \right]
\end{align*}

The partial derivative with respect to the scalar curvature is given by
\begin{align}
  \pdv{R}\left(\frac{6\nabla_b A^{ab}}{\sqrt{-g}}\right)&=-2K_5(\f^{a}\Box\f +X^{a})-\frac{3}{2}K_7\f^{a}\\
  &\quad -9K_4\left[\f^{a}(\Box\f )^2-\f^{a}\f_{ed}\f^{ed}+2X^{a}\Box\f -2X^d\nabla^{a}\f_d \right]
\end{align}
[\rc{Check factors}] %TODO Check factors

The above must vanish for any value of the second derivative. Since the terms with the same degree have a common coefficient, such coefficients must vanish independently. Hence
\begin{equation}
  K_4=K_5=K_7=0
\end{equation}
In this way, the divergence of $A^{ab}$ is reduced to
\begin{align*}
    \frac{\nabla_b A^{ab}}{\sqrt{-g}}&= \frac{1}{2}K_8\d^{acd}_{efh}\nabla_c\f \nabla^f\f \nabla^m\f R^{he}_{md}+K_6\d^{acde}_{fhjk}\nabla_c\f \nabla^h\f \nabla^m\f R^{jf}_{md}\nabla^k\nabla_e\f\\&\quad +K_6\d^{acde}_{fhjk}\nabla^h\f \nabla^f\nabla_c\f \nabla^j\nabla_d\f \nabla^k\nabla_e\f +2\dot{K}_8\d^{acd}_{efh}\nabla_p\f \nabla^{e}\nabla^p\f \nabla_c\f\nabla^f\f\nabla^h\nabla_d\f \\ &\quad+2\dot{K}_6\d^{acde}_{fhjk}\nabla_p\f \nabla^f\nabla^p\f \nabla_c\f \nabla^h\f \nabla^j\nabla_d\f\nabla^k\nabla_e\f \\
  &\quad+K_8\d^{acd}_{efh}\nabla^f\f\nabla^{e}\nabla_c\f \nabla^h\nabla_d\f +(2\dot{K}_9+K_{10})\nabla_b\f \nabla^b\nabla^{a}\f \\
  &\quad+\nabla^{a}\f (K_9'+\r K_{10}'+2\dot{K}_{10}\nabla^b\f \nabla^c\f \nabla_c\nabla_b\f +K_{10}\nabla^c\nabla_c\f )
\end{align*}

Let's see the dependence of the traceless Ricci tensor $S^{a}_b$ in the expression above. Using \eqref{R-CSR} we have
\begin{align*}
  \frac{\nabla_bA^{ab}}{\sqrt{-g}}&=\frac{1}{2}K_8\d^{acd}_{efh}\f_c\f^f\f^m R^{he}_{md}+K_6\d^{acde}_{fhjk}\f_c\f^h\f^m\f^k_e R^{jf}_{md}\\
  &=\frac{1}{2}2K_8\d^{acd}_{efh}\d^{[h}_{[m}S^{e]}_{d]}\f_c\f^f\f^m +2K_6\d^{acde}_{fhjk}\d^{[j}_{[m}S^{f]}_{d]}\f_c\f^h\f^m\f^k_e\\
  &=\frac{K_8}{2}\d^{acd}_{efh}\left(\d^h_mS^{e}_d-\d^h_dS^{e}_m\right)\f_c\f^f\f^m+K_6\d^{acde}_{fhjk}\left(\d^j_mS^f_d-\d^j_dS^f_m\right)\f_c\f^h\f^m\f^k_e\\
  &=\frac{K_8}{2}\left(\underbrace{\d^{acd}_{efh}S^{e}_d\f_c\f^{(f}\f^{h)}}_{=0}-2\d^{ac}_{ef}S^{e}_m\f_c\f^f\f^m\right)+K_6\left(\underbrace{\d^{acde}_{fhjk}S^f_d\f_c\f^{(h}\f^{j)}\f^k_e}_{=0}-\d^{ace}_{fhk}S^f_m\f_c\f^h\f^m\f^k_e\right)\\
  &=-K_8\d^{ac}_{ef}S^{e}_m\f_c\f^f\f^m-K_6\d^{ace}_{fhk}S^f_m\f_c\f^h\f^m\f^k_e\\
  &=-2K_8\d^{a}_{[e}\d^{c}_{f]}S^{e}_m\f_c\f^f\f^m-6K_6\d^{a}_{[f}\d^{c}_{h}\d^{e}_{k]}S^f_m\f_c\f^h\f^m\f^k_e
\end{align*}
Since both terms have different degrees in the second derivative, they must vanish independently, that is 
\begin{equation}
  K_6=K_8=0
\end{equation}
In summary, if one demands that only curvature couplings occur through the Weyl tensor, then there is no nonminimal coupling at all, and the second-order tensor becomes, in fact, of first-order,

\begin{equation}
\begin{split}
  A^{ab}&=\sqrt{-g}\left\{K_9g^{ab}+K_{10}\nabla^{a}\f\nabla^b\f  \right\}
\end{split}
\end{equation}
with second-order divergence given by
\begin{align}
   \nabla_b A^{ab}&= \sqrt{-g}\left\{(2\dot{K}_9+K_{10})\nabla_b\f \nabla^b\nabla^{a}\f\right. \\
  &\quad\left.+\nabla^{a}\f (K_9'+\r K_{10}'+2\dot{K}_{10}\nabla^b\f \nabla^c\f \nabla_c\nabla_b\f +K_{10}\nabla^c\nabla_c\f )\right\}
\end{align}

Another restriction in the allowed terms is given by the fact that
\begin{align}
  \tilde{X}=-\frac{1}{2}\tilde{g}^{\m\n }\tn_\m\f \tn_\n\f =-\frac{1}{2}\frac{1}{X}g^{\m\n }\nabla_\m\f \nabla_\n\f=\frac{1}{X}X=1
\end{align}


In this way, the symmetric second rank tensor depending up to second order in the scalar field $\f$ and auxiliary metric $\tilde{g}^{\m\n}$ such that a second order divergence is given by
\begin{align}
  H^{\m\n}=\tilde{K}_9\tilde{g}^{\m\n }+\tilde{K}_{10}\tn^\m\f\tn^\n\f 
\end{align}

Taking its divergence,
\begin{align}
  \nabla_\n H^{(\m\n)}&=\partial_\f\tilde{K}_9\tn^\m\f +\partial_\f\tilde{K}_{10}\tn^\m\f \tn^\n\f \tn_\n\f +\tilde{K}_{10}\tn^\n\tn^\m\f\tn^\n\f +\tilde{K}_{10}\tn^\m\f\tilde{\Box}\f \\
  &=\partial_\f \tilde{K}_9\tn^\m\f +\tilde{K}_{10}\tilde{\Box\f }\tn^\m\f -2\partial_\f\tilde{K}_{10}\tilde{X}\tn^\m\f -\tilde{K}_{10}\tn^\m\f 
\end{align}
but $\tilde{X}=1$, so then
\begin{align}
  H^\m = \nabla_\n H^{(\m\n)}&=\partial\f\tilde{K}_9+\tilde{K}_{10}\tilde{\Box}\f\tn^\m\f  -2\partial_\f \tilde{K}_{10}\tn^\m\f \\
   &=(\partial_\f\tilde{K}_9+\tilde{K}_{10}\Box\f -2\partial_\f \tilde{K}_{10})\tn^\m\f 
\end{align}

In summary:
\begin{tcolorbox}
	\begin{align}
  H^{\m\n}&=\tilde{K}_9\tilde{g}^{\m\n }+\tilde{K}_{10}\tn^\m\f\tn^\n\f \\
  H^\m &=(\partial_\f\tilde{K}_9+\tilde{K}_{10}\Box\f -2\partial_\f \tilde{K}_{10})\tn^\m\f \\
  K_i&=\eval{K_i(\f,\tilde{X})}_{\tilde{X}=1}
\end{align}
\end{tcolorbox}




We can decompose $\pdv*{E}{(\tn_\n\tn_\n\f )}$ in its trace and traceless part,
\begin{align}
  \tn_\m\tn_\n\f =\TL_{\m\n }+\frac{1}{D}\tilde{g}^{\m\n }\tilde{\Box}\f 
\end{align}
where the $\TL_{\m\n }$ is the traceless part,
\begin{equation}
  \TL^\m_\n =\tn^\m\tn_\n\f -\frac{1}{D}\d^\m_\n \tilde{\Box}\f 
\end{equation}
Therefore, we have
\begin{align}
  E^{\m\n }_\f =\pdv{E}{(\tn_\m\tn_\n\f )}=\pdv{E}{(\TL^\a_\b) }\pdv{(\TL^\a_\b )}{(\tn_\m\tn_\n\f )}+\pdv{E}{(\tilde{\Box}\f )}\pdv{(\tilde{\Box}\f )}{(\tn_\m\tn_\n\f )}
\end{align}
but
\begin{align}
  \TL_{\a\b }&=\tn_\a\tn_\b \f-\frac{1}{D}\tilde{g}_{\a\b }\tilde{\Box}\f \\
  \implies \TL^\a_\b &=\tn^\a\tn_\b\f -\frac{1}{D}\d^\a_\b\tilde{\Box}\f \\
  &=\tilde{g}^{\m\a }\tn_\m\tn_\b\f  -\frac{1}{D}\d^\a_\b \tilde{g}^{\m\n }\tn_m\tn_\n\f \\
  \implies \pdv{(\TL^\a_\b )}{(\tn_\m\tn_\n\f )}&=\tilde{g}^{\m\a }\d^\n_\b -\frac{1}{D}\d^\a_\b \tilde{g}^{\m\n }
\end{align}
and
\begin{equation}
  \pdv{(\tilde{\Box}\f )}{(\tn_\m\tn_\n\f )}=\pdv{(\tn_\m\tn_\n\f )}\left(\tilde{g}^{\m\n }\tn_\m\tn_\n\f \right)=\tilde{g}^{\m\n }
\end{equation}
Therefore
\begin{align}
  E^{\m\n }_\f &=\pdv{E}{(\TL^\a_\b )}\left(\tilde{g}^{\m\a }\d^\n_\b -\frac{1}{D}\d^\a_\b \tilde{g}^{\m\n }\right)+\pdv{E}{(\tilde{\Box}\f )}\tilde{g}^{\m\n }\\
  &=\A ^{\m\n }-\frac{1}{D}\A \tilde{g}^{\m\n }+\pdv{E}{(\tilde{\Box}\f )}\tilde{g}^{\m\n }\\
  &=\hat{\A }^{\m\n }+\pdv{E}{(\tilde{\Box}\f )}\tilde{g}^{\m\n }
\end{align}
where we have defined $\A^{\m\n } :=\pdv{E}{(\TL_{\m\n })}$ and $\hat{\A}$ is its traceless part.

Now, let's decompose the two-rank Helmholtz conditions in its trace and traceless part,
\begin{align}
  \tilde{g}_{\m\n }H^{\m\n }&=\tilde{g}_{\m\n }\left[E^{\m\n }_\f+\left(E^{\m\a }_\f \tilde{\nabla}_\a \f-\frac{1}{2}\tr\tilde{E}_\f\tilde{\nabla}^\m\f \right)\tilde{\nabla}^\n\f \right]\\
  &=\tr \tilde{E}_\f +E^{\a\m }_\f \tn_\a\tn_\m\f -\frac{1}{2}\tr \tilde{E}_\f \tn^\m\f \tn_\m\f \\
  &=\tr\tilde{E}_\f +E^{\a\m }_\f\tn_\a\f\tn_\m\f +\tr \tilde{E}_\f \tilde{X}\\
  &=2\tr\tilde{E}_\f +E^{\a\m }_\f\tn_\a\f\tn_\m\f\\
  &=(2\tilde{g}_{\a\m }+\tn_\a\f\tn_\m\f )E^{\a\m }_\f \\
  &=(2\tilde{g}_{\a\m }+\tn_\a\f\tn_\m\f )\left(\hat{\A }^{\m\n }+\pdv{E}{(\tilde{\Box}\f )}\tilde{g}^{\m\n }\right)\\
  &=\cancel{2\tilde{g}_{\a\m }\hat{\A}^{\a\m }}+8\pdv{E}{(\tilde{\Box}\f )}+\hat{\A }^{\a\m }\tn_\a\f\tn_\m\f +\pdv{E}{(\tilde{\Box}\f )}\left(\tn\f \right)^2\\
  &=\pdv{E}{(\tilde{\Box}\f )}\left(8-2\tilde{X}\right)+\hat{\A}^{\a\m }\tn_\a\f\tn_\m\f \\
  &=6\pdv{E}{(\tilde{\Box}\f )}+\hat{\A }^{\m\n}\tn_\m\f\tn_\n\f \\
  &=6\pdv{E}{(\tilde{\Box}\f )}+\A^{\m\n}\tn_\m\f\tn_\n\f-\frac{1}{4}\A (\tn\f )^2\\
  &=6\pdv{E}{(\tilde{\Box}\f )}+\A^{\m\n}\tn_\m\f\tn_\n\f+\frac{1}{2}\A \tilde{X}\\
  &=6\pdv{E}{(\tilde{\Box}\f )}+\left(\tn_\m\f\tn_\n\f +\frac{1}{2}\tilde{g}_{\m\n }\right)\pdv{E}{(\TL_{\m\n })}
\end{align}
but also
\begin{align}
  \tilde{g}_{\m\n }H^{\m\n }&=\tilde{g}_{\m\n }(\tilde{K}_9\tilde{g}^{\m\n }+\tilde{K}_{10}\tn^\m\f\tn^\n\f)\\
  &=4\tilde{K}_9+\tilde{K}_{10}(\tn\f )^2\\
  &=4\tilde{K}_9-2\tilde{K}_{10}\tilde{X}\\
  &=2(2\tilde{K}_9-\tilde{K}_{10})
\end{align}
Thus
\begin{align}
  6\pdv{E}{(\tilde{\Box}\f )}+\left(\tn_\m\f\tn_\n\f +\frac{1}{2}\tilde{g}_{\m\n }\right)\pdv{E}{(\TL_{\m\n })}&=2(2\tilde{K}_9-\tilde{K}_{10})
\end{align}












































































\clearpage
\appendix
\section{Some useful calculations}
\subsection{Variation of the Christoffel symbols}
The Christoffel symbols in terms of the metric are given by
\begin{equation*}
    \tensor{\Gamma}{^\lambda_{\mu\beta}}=\frac{1}{2}g^{\lambda\rho}\left(\partial_\mu g_{\beta\rho}+\partial_\beta g_{\mu\rho}-\partial_\rho g_{\mu\beta}\right)
\end{equation*}
Varying both sides, we have
\begin{align*}
    \delta\tensor{\Gamma}{^\lambda_{\mu\beta}}&=\frac{1}{2}\delta g^{\lambda\rho}\left(\partial_\mu g_{\beta\rho}+\partial_\beta g_{\mu\rho}-\partial_\rho g_{\mu\beta}\right)+\frac{1}{2}g^{\lambda\rho}\left(\partial_\mu \delta g_{\beta\rho}+\partial_\beta \delta g_{\mu\rho}-\partial_\rho \delta g_{\mu\beta}\right)\\
    &=-\frac{1}{2}g^{\lambda\sigma}g^{\rho\tau}(\delta g_{\sigma\tau})\left(\partial_\mu g_{\beta\rho}+\partial_\beta g_{\mu\rho}-\partial_\rho g_{\mu\beta}\right)+\frac{1}{2}g^{\lambda\rho}\left(\partial_\mu\delta g_{\beta\rho}+\partial_\beta \delta g_{\mu\rho}-\partial_\rho\delta g_{\mu\beta}\right)\\
    &=-g^{\lambda\sigma}(\delta g_{\sigma\tau})\tensor{\Gamma}{^\tau_{\mu\beta}}+\frac{1}{2}g^{\lambda\rho}\left(\partial_\mu\delta g_{\beta\rho}+\partial_\beta \delta g_{\mu\rho}-\partial_\rho\delta g_{\mu\beta}\right)
\end{align*}
Changing the dumb indice $\sigma$ by $\rho$,
\begin{align}
    \nonumber\delta\tensor{\Gamma}{^\lambda_{\mu\beta}}&=-g^{\lambda\rho}(\delta g_{\rho\tau})\tensor{\Gamma}{^\tau_{\mu\beta}}+\frac{1}{2}g^{\lambda\rho}\left(\partial_\mu\delta g_{\beta\rho}+\partial_\beta \delta g_{\mu\rho}-\partial_\rho\delta g_{\mu\beta}\right)\\
   \nonumber &=\frac{1}{2} g^{\lambda\rho}\left(\partial_\mu \delta g_{\beta\rho}+\partial_\beta \delta g_{\mu\rho}-\partial_\rho\delta g_{\mu\beta}-2\delta g_{\rho\tau}\tensor{\Gamma}{^\tau_{\mu\beta}}\right)\\
   \nonumber &=\frac{1}{2}g^{\lambda\rho}\left(\partial_\mu \delta g_{\beta\rho}-\tensor{\Gamma}{^\tau_{\mu\beta}}\delta g_{\rho\tau}-\tensor{\Gamma}{^\tau_{\rho\mu}}\delta g_{\tau\beta}+\partial_\beta \delta g_{\mu\rho}-\tensor{\Gamma}{^\tau_{\mu\beta}}\delta g_{\rho\tau}-\tensor{\Gamma}{^\tau_{\rho\beta}}\delta g_{\tau\mu}\right.\\\nonumber &\left.-\partial_\rho \delta g{\mu\beta}+\tensor{\Gamma}{^\tau_{\mu\rho}}\delta g_{\tau\beta}+\tensor{\Gamma}{^\tau_{\beta\rho}}\delta g_{\mu\tau}\right)\\
    &=\frac{1}{2}g^{\lambda\rho}\left(\nabla_\mu \delta g_{\beta\rho}+\nabla_\beta \delta g_{\mu\rho}-\nabla_\rho \delta g_{\mu\beta}\right)\label{delta Gamma}
\end{align}


\subsection{Variation of the Riemann tensor}
The Riemann tensor is given by
\begin{equation*}
    \tensor{R}{^\rho_{\lambda\mu\nu}}=\partial_\mu\tensor{\Gamma}{^\rho_{\lambda\nu}}-\partial_\nu\tensor{\Gamma}{^\rho_{\lambda\mu}}+\tensor{\Gamma}{^\rho_{\tau\mu}}\tensor{\Gamma}{^\tau_{\lambda\nu}}-\tensor{\Gamma}{^\rho_{\tau\nu}}\tensor{\Gamma}{^\tau_{\lambda\mu}}
\end{equation*}
Varying both sides,
\begin{align*}
    \delta  \tensor{R}{^\rho_{\lambda\mu\nu}}&=\partial_\mu\delta\tensor{\Gamma}{^\rho_{\lambda\nu}}-\partial_\nu\delta\tensor{\Gamma}{^\rho_{\lambda\mu}}+\delta\tensor{\Gamma}{^\rho_{\tau\mu}}\tensor{\Gamma}{^\tau_{\lambda\nu}}+\tensor{\Gamma}{^\rho_{\tau\mu}}\delta\tensor{\Gamma}{^\tau_{\lambda\nu}}-\delta\tensor{\Gamma}{^\rho_{\tau\nu}}\tensor{\Gamma}{^\tau_{\lambda\mu}}-\tensor{\Gamma}{^\rho_{\tau\nu}}\delta\tensor{\Gamma}{^\tau_{\lambda\mu}}\\
    &=\partial_\mu\delta\tensor{\Gamma}{^\rho_{\nu\lambda}}+\tensor{\Gamma}{^\rho_{\tau\mu}}\delta\tensor{\Gamma}{^\tau_{\nu\lambda}}-\tensor{\Gamma}{^\tau_{\mu\lambda}}\delta\tensor{\Gamma}{^\rho_{\tau\nu}}-\partial_\nu\delta\tensor{\Gamma}{^\rho_{\mu\lambda}}+\tensor{\Gamma}{^\tau_{\nu\lambda}}\delta\tensor{\Gamma}{^\rho_{\tau\mu}}-\tensor{\Gamma}{^\rho_{\tau\nu}}\delta\tensor{\Gamma}{^\tau_{\mu\lambda}}
\end{align*}
Adding a convenient zero of the form $\tensor{\Gamma}{^\tau_{\mu\nu}}\delta\tensor{\Gamma}{^\rho_{\tau\lambda}}-\tensor{\Gamma}{^\tau_{\mu\nu}}\delta\tensor{\Gamma}{^\rho_{\tau\lambda}}$, and using the fact that $\delta\tensor{\Gamma}{^\lambda_{\mu\nu}}$ is a tensor, we have
\begin{equation}
    \delta\tensor{R}{^\rho_{\lambda\mu\nu}}=\nabla_\mu\delta\tensor{\Gamma}{^\rho_{\nu\lambda}}-\nabla_\nu\delta\tensor{\Gamma}{^\rho_{\mu\lambda}}=2\nabla_{[\mu}\delta\tensor{\Gamma}{^\rho_{\nu]\lambda}}\label{delta Riemann}
\end{equation}




\subsection{Variation of derivatives of $\f$ w.r.t $\omega$}
Let's compute the infinitesimal variations of the covariant derivatives of the scalar field. First, let us calculate $\d_\omega \nabla_\m \f $:
\begin{align}
  \d_\omega\nabla_\m \f&=\nabla_\m \d_\omega\f=0 \label{delta-nabla}
\end{align}
Now, let's compute $\d_\omega (\nabla_\m \nabla_\n \f )$:
\begin{align}
  \d_\omega (\nabla_\m \nabla_\n \f )&=\d_\omega \nabla_\m (\partial_\n \f )\\
  &=\d_\omega(\partial_\m \partial_\n \f -\G^\lambda_{~\n\m }\partial_\lambda \f )\\
  &=\partial_\m \partial_\n \d_\omega\f -\d_\omega\G^\lambda_{~\n\m }\partial_\lambda\f -\G^\lambda_{~\n\m }\d_\omega\partial_\lambda\f \\
  &=\partial_\m \partial_\n \d_\omega\f -\d_\omega \G^\lambda_{~\n\m }\partial_\lambda\f -\G^\lambda_{~\n\m }\partial_\lambda \d_\omega \f \\
  &=-\partial_\lambda\f \d_\omega \G^\lambda_{~\n\m }
\end{align}
Using that the variation of the Christoffel connection is
\begin{equation}
  \d \G^\lambda_{~\m\b }=\frac{1}{2}g^{\lambda\r }(\nabla_\m \d g_{\b\r }+\nabla_\b \d g_{\m\r }-\nabla_\r \d g_{\m\b })
\end{equation}
we have
\begin{equation}
  \d_\omega(\nabla_\m \nabla_\n\f )=-\partial_\lambda\f \frac{1}{2}g^{\lambda\r }(\nabla_\n \d_\omega g_{\m\r }+\nabla_\m \d_\omega g_{\n\r }-\nabla_\r\d_\omega g_{\n\b })
\end{equation}
but $\d_\omega g_{\n\m }=2\omega g_{\n\m }$, so then
\begin{equation}
   \d_\omega(\nabla_\m \nabla_\n\f )=-\partial_\lambda\f g^{\lambda\r }\left[\nabla_\n (\omega g_{\m\r })+\nabla_\m (\omega g_{\n\r })-\nabla_\r (\omega g_{\n\b })\right]
\end{equation}
Using the metric compatibility condition $\nabla_\m g_{\a\b }=0$ and $\nabla_\a\f =\partial_\a\f  $, we obtain
\begin{align}
   \d_\omega(\nabla_\m \nabla_\n\f )&=-\partial_\lambda\f g^{\lambda\r }(g_{\m\r }\nabla_\n \omega+g_{\n\r }\nabla_\m \omega-g_{\n\b }\nabla_\r \omega)\\
   &=-\partial^\r \f (g_{\m\r }\nabla_\n \omega+g_{\n\r }\nabla_\m \omega-g_{\n\b }\nabla_\r \omega)\\
   &=-\nabla_\m \f\nabla_\n \omega-\nabla_\n \f \nabla_\m \omega +g_{\n\b }\nabla^\r \f\nabla_\r\omega\\
   &=-2\nabla_{(\m }\f\nabla_{\n)}\omega+g_{\n\b }\nabla^\r \f\nabla_\r\omega \label{delta-nabla-nabla}
\end{align}


\subsection{Variation of $E$ w.r.t Riemann tensor}
In order to see what \eqref{P-condition} implies, let us split the dependence of the Riemann tensor in terms of its traceless part $C_{\m\n }^{\a\b }$, the traceless part of the Ricci tensor $S^\a_\b $, and the scalar curvature $R$. So we have
\begin{equation*}
    E\left(g^{\mu\nu},R^{\alpha\beta}_{\mu\nu}\right)=E\left(g^{\mu\nu},C^{\alpha\beta}_{\mu\nu},S^\alpha_\beta,R\right)
\end{equation*}
The variation w.r.t the Riemann tensor yields
\begin{align}
    \nonumber\delta_{\rm Riem}E&=P^{\alpha\beta}_{\mu\nu}\delta R^{\mu\nu}_{\alpha\beta}\\
    &=H^{\alpha\beta}_{\mu\nu}\delta_{\rm Riem}C^{\mu\nu}_{\alpha\beta}+I^\alpha_\beta\delta_{\rm Riem}S^\beta_\alpha+J\delta_{\rm Riem}R\label{var R}
\end{align}
where
\begin{equation*}
    H^{\alpha\beta}_{\mu\nu}\equiv\pdv{E}{C^{\mu\nu}_{\alpha\beta}},\qquad I^\alpha_\beta\equiv \pdv{E}{S^\beta_\alpha}\qquad \text{y} \qquad  J\equiv\pdv{E}{R}
\end{equation*}
Since $P_{\a\b }^{\m\n }$ has the same algebraic symmetries as the Riemann tensor, its traceless part is given by
\begin{equation}
    \hat{P}^{\mu\nu}_{\alpha\beta}=P^{\mu\nu}_{\alpha\beta}-\frac{4}{D-2}\delta^{[\mu}_{[\alpha}P^{\nu]}_{\beta]}+\frac{2}{(D-2)(D-1)}P\delta^\mu_{[\alpha}\delta^\nu_{\beta]}
\end{equation}




Let us note that
\begin{align}
   \nonumber J\delta_{\rm Riem}R&=J\delta_{\rm Riem}\left(R^{\alpha\beta}_{\mu\nu}\delta^\mu_\alpha\delta^\nu_\beta\right)\\
    &=J\delta^\mu_\alpha\delta^\nu_\beta R^{\alpha\beta}_{\mu\nu}\label{delta R}
\end{align}

Writing $S^\alpha_\beta$ in terms of the Riemann,
\begin{align*}
    S^\beta_\nu&=R^\beta_\nu-\frac{1}{D}R\delta^\beta_\nu\\
    &=R^{\alpha\beta}_{\mu\nu}\delta^\mu_\alpha-\frac{1}{D}\delta^\beta_\nu Rs^{\alpha\gamma}_{\mu\lambda}\delta^\mu_\alpha\delta^\lambda_\gamma
\end{align*}
then,
\begin{align*}
    \delta_{\rm Riem}\tilde{S}^\beta_\nu&=\delta^\mu_\alpha\delta\tilde{R}^{\alpha\beta}_{\mu\nu}-\frac{1}{D}\delta^\beta_\nu\delta\tilde{R}^{\alpha\gamma}_{\mu\lambda}\delta^\mu_\alpha \delta^\lambda_\gamma
\end{align*}
Hence,
\begin{align}
   \nonumber I^\nu_\beta\delta_{\rm Riem}S^\beta_\nu&=I^\nu_\beta\delta^\mu_\alpha\delta R^{\alpha\beta}_{\mu\nu}-\frac{1}{D}I^\nu_\beta\delta^\beta_\nu\delta^\mu_\alpha\delta^\lambda_\gamma\delta R^{\alpha\gamma}_{\mu\lambda}\\
    \nonumber&=I^\nu_\beta\delta^\mu_\alpha\delta\tilde{R}^{\alpha\beta}_{\mu\nu}-\frac{1}{D}I\delta^\mu_\alpha\delta R^{\alpha\gamma}_{\mu\lambda}\\
    \nonumber&=\delta^\mu_\alpha\delta R^{\alpha\beta}_{\mu\nu}\left(I^\nu_\beta-\frac{1}{D}I\delta^\nu_\beta\right)\\
    &=\delta^\mu_\alpha\hat{I} \delta R^{\alpha\beta}_{\mu\nu}\label{dela S}
\end{align}

Finally, let us write the Weyl tensor in terms of the Riemann,
\begin{align*}
    \tilde{C}^{\alpha\beta}_{\mu\nu}&=\tilde{R}^{\alpha\beta}_{\mu\nu}-\frac{4}{D-2}\delta^{[\alpha}_{[\mu}\tilde{R}^{\beta]}_{\nu]}+\frac{2}{(D-1)(D-2)}\tilde{R}\delta^{[\alpha}_{[\mu}\delta^{\beta]}_{\nu]}\\
    &=\tilde{R}^{\alpha\beta}_{\mu\nu}-\frac{4}{D-2}\delta^\lambda_\gamma \delta^{[\alpha}_{[\mu}\tilde{R}^{\beta]\gamma}_{\nu]\lambda}+\frac{2}{(D-1)(D-2)}\delta^{[\alpha}_{[\mu}\delta^{\beta]}_{\nu]}\delta^\tau_\rho\delta^\lambda_\sigma\tilde{R}^{\rho\sigma}_{\tau\lambda}
\end{align*}
Varying with respect to $R^{\alpha\beta}_{\mu\nu}$,
\begin{align*}
    \delta_{\rm Riem}C^{\alpha\beta}_{\mu\nu}&=\delta R^{\alpha\beta}_{\mu\nu}-\frac{4}{D-2}\delta^\lambda_\gamma\delta^{[\alpha}_{[\mu}\delta R^{\beta]\gamma}_{\nu]\lambda}+\frac{2}{(D-1)(D-2)}\delta^\alpha_{[\mu}\delta^\beta_{\nu]}\delta^\tau_\rho\delta^\lambda_\sigma \delta R^{\rho\sigma}_{\tau\lambda}
\end{align*}
Then,
\begin{align}
   \nonumber H^{\mu\nu}_{\alpha\beta}\delta_{\rm Riem}C^{\alpha\beta}_{\mu\nu}&=H^{\mu\nu}_{\alpha\beta}\left[\delta R^{\alpha\beta}_{\mu\nu}-\frac{4}{D-2}\delta^\lambda_\gamma\delta^{[\alpha}_{[\mu}\delta R^{\beta]\gamma}_{\nu]\lambda}+\frac{2}{(D-1)(D-2)}\delta^\alpha_{[\mu}\delta^\beta_{\nu]}\delta^\tau_\rho\delta^\lambda_\sigma \delta R^{\rho\sigma}_{\tau\lambda}\right]\\
    \nonumber&=H^{\mu\nu}_{\alpha\beta}\delta R^{\alpha\beta}_{\mu\nu}-\frac{4}{D-2}H^{\lambda\nu}_{\gamma\beta}\delta^\mu_\alpha\delta^\gamma_\lambda\delta R^{\beta\alpha}_{\nu\mu}+\frac{2}{(D-1)(D-2)}H^{\tau\lambda}_{\rho\sigma}\delta^\rho_\tau\delta^\sigma_\lambda \delta^\mu_\alpha\delta^\nu_\beta\delta R^{\alpha\beta}_{\mu\nu}\\
    \nonumber&=\delta R^{\alpha\beta}_{\mu\nu}\left[H^{\mu\nu}_{\alpha\beta}-\frac{4}{D-2}H^\nu_\beta \delta^\mu_\alpha +\frac{2}{(D-1)(D-2)}H\right]\\
    &=\hat{H}^{\mu\nu}_{\alpha\beta}\delta R^{\alpha\beta}_{\mu\nu}\label{delta C}
\end{align}
where the indices have been renamed in a convenient way and has been used the fact that $H^{\alpha\beta}_{\mu\nu}$ has the same algebraic symmetries as the Riemann tensor. 

In this way, plugging (\ref{delta R}), (\ref{dela S}) and (\ref{delta C}) into (\ref{var R}), we obtain
\begin{align*}
    P^{\alpha\beta}_{\mu\nu}\delta R^{\mu\nu}_{\alpha\beta}&=\hat{H}^{\alpha\beta}_{\mu\nu}\delta R^{\mu\nu}_{\alpha\beta}+\delta^\alpha_\mu\hat{I}^\beta_\nu\delta R^{\mu\nu}_{\alpha\beta}+J\delta^\alpha_\mu\delta^\beta_\nu\delta R^{\mu\nu}_{\alpha\beta}
\end{align*}
Hence,
\begin{equation}\label{P-tensor}
    P^{\alpha\beta}_{\mu\nu}=\hat{H}^{\alpha\beta}_{\mu\nu}+\delta^{[\alpha}_{[\mu}\hat{I}^{\beta]}_{\nu]}+J\delta^\alpha_{[\mu}\delta^\beta_{\nu]}
\end{equation}








% Bibliography

%% [A] Recommended: using JHEP.bst file
%% \bibliographystyle{JHEP}
%% \bibliography{biblio.bib}

%% or
%% [B] Manual formatting (see below)
%% (i) We suggest to always provide author, title and journal data or doi:
%% in short all the informations that clearly identify a document.
%% (ii) please avoid comments such as "For a review'', "For some examples",
%% "and references therein" or move them in the text. In general, please leave only references in the bibliography and move all
%% accessory text in footnotes.
%% (iii) Also, please have only one work for each \bibitem.



\newpage
\bibliographystyle{JHEP}
\bibliography{biblio.bib}
\end{document}
