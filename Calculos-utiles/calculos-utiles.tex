\documentclass[a4paper,11pt]{article}
\usepackage{jheppub} % for details on the use of the package, please see the JINST-author-manual
\usepackage{lineno}
\usepackage{amsmath,amsthm,amsfonts,amssymb,amscd,physics,cancel,mathtools}
\usepackage{tcolorbox}
\usepackage{marginnote,tensor}
%~~~~~~~~~ Document setup
\usepackage[spanish]{babel} % English formatting
\usepackage[utf8]{inputenc} % Standard encoding
%\usepackage[a4paper,left=3cm,bottom=3cm]{geometry} % Page formatting
\usepackage{indentfirst} % Indents the first paragraph
\usepackage{amsmath} % Maths type package
\usepackage{bm} % Bold font maths
\usepackage{graphicx} % Advanced graphics package
\usepackage[export]{adjustbox} 
\usepackage{pdflscape} % Make pages landscape
\usepackage{fancyhdr} % Fancy headers
%\usepackage[colorlinks=true,citecolor=blue,urlcolor=blue,linkcolor=black]{hyperref} % Link colours
%\usepackage{natbib} % Bibliography
\usepackage{flafter} % Reference any 'float'
\usepackage[framemethod=tikz]{mdframed} % Box off stuff
\usepackage{color} % Colour support
\usepackage{wrapfig} % Text flowing around figures
\usepackage{lipsum} % Generates meaningless text
% \usepackage{biblatex}
% \addbibresource{sample.bib}

\def\a{\alpha}
\def\b{\beta}
\def\g{\gamma}
\def\G{\Gamma}
\def\d{\delta}
%\def\D{\Delta}
%\def\e{\eta}
\def\la{\lambda}
\def\La{\Lambda}
\def\k{\kappa}
\def\m{\mu}
\def\n{\nu}
%\def\r{\rho}
\def\p{\rho}
\def\o{\omega}
\def\s{\sigma}
\def\S{\Sigma}
\def\t{\tau}
\def\p{\pi}
\def\f{\phi}
\def\vf{\varphi}
\def\ep{\epsilon}
\def\th{\theta}
\def\Th{\Theta}
\def\z{\zeta}


\theoremstyle{definition}
\newtheorem{ej}{Ejemplo}[section]
\newtheorem{prop}{Propiedad}[section]
\newtheorem{teo}{Teorema}[section]
% \linenumbers



%\arxivnumber{1234.56789} % if you have one

\title{\boldmath Cálculos útiles}

% Collaborations

%% [A] If main author
%% \collaboration{\includegraphics[height=17mm]{collabroation-logo}\\[6pt]
%%  XXX collaboration}

%% or
%% [B] If "on behalf of"
%% \collaboration[c]{on behalf of XXX collaboration}


% Authors
% The "\note" macro will give a warning: "Ignoring empty anchor...", you can safely ignore it.

%% [A] simple case: 2 authors, same institution
%% \author[1]{A. Uthor\note{Corresponding author.}}
%% \author{and A. Nother Author}
%% \affiliation{Institution,\\Address, Country}

%% or, e.g.
%% [B] more complex case: 4 authors, 3 institutions, 2 footnotes
%% \author[a,b]{F. Irst,\note{Now at another university}}
%% \author[c]{S. Econd,}
%% \author[a,2]{T. Hird\note{Also at Some University.}}
%% \author[c,2]{and Fourth}
%% \affiliation[a]{Institution_1,\\Address, Country}
%% \affiliation[b]{Institution_2,\\Address, Country}
%% \affiliation[c]{Institution_3,\\Address, Country}

\author{Borja Diez B.}
\affiliation{Universidad Arturo Pratt, Iquique, Chile}
%\affiliation{Another University,\\
%different-address, Country}

% E-mail addresses: only for the corresponding author
\emailAdd{borjadiez1014@gmail.com}

% \abstract{En estas notas iré recopilando los resultados más importantes de mis cálculos de mi tesis de magister. En particular, iré recopilando los polinomios de Wheeler encontrados en diferenter ordenes en la teoría de Lovelock para métricas tipo Eguchi-Hanson con diferentes base manifolds con topología no trivial.}



\begin{document}
\maketitle
\flushbottom

\section{Transformación de coordenadas $\mathbb{CP}^2$}
La métrica de $\mathbb{CP}^2$ escrita en las coordenadas $x^\m=\{r,\th,\f,\psi\}$ luce como
\begin{equation}\label{CP2.1}
  \dd s^2=\frac{\dd r^2}{(1+\frac{\Lambda}{6}r^2)^2}+\frac{\dfrac{r^2}{4}}{(1+\frac{\Lambda}{6}r^2)^2}\sigma_3^2+\frac{\dfrac{r^2}{4}}{(1+\frac{\Lambda}{6}r^2)}(\sigma_1^2+\sigma_2^2)
\end{equation}
donde $0\leq r<\infty,0\leq\th\leq\p ,0\leq \f<2\p $ y $0\leq\psi\leq 4\p $, y las $\s_i$ son las formas invariantes izquierdas de Maurier-Cartan de $SU(2)$, definidas como
\begin{align}
  \s_1&=\cos\psi\dd\th+\sin\th\sin\psi \dd\f\\
  \s_2&=-\sin\psi\dd\th+\sin\th\cos\psi \dd\f \\
  \s_3&=\dd\psi+\cos\th\dd\f 
\end{align}
las cuales satisfacen $\dd\s_i+\frac{1}{2}\epsilon_{ijk}\s^{j}\wedge\s^k=0$. Además es directo notar que $\s_1^2+\s_2^2=\dd\Omega_2^2$ es el elemento de línea de las $2$-esfera unitaria. Luego, \eqref{CP2.1} puede ser escrita como
\begin{equation}\label{CP2-r}
  \dd s^2=\frac{\dd r^2}{(1+\frac{\Lambda}{6}r^2)^2}+\frac{\dfrac{r^2}{4}}{(1+\frac{\Lambda}{6}r^2)^2}(\dd\psi+\cos\th\dd\f )^2+\frac{\dfrac{r^2}{4}}{(1+\frac{\Lambda}{6}r^2)}(\dd\th^2+\sin^2\th\dd\phi^2)
\end{equation}
La métrica \eqref{CP2.1} es solución de las ecuaciones de Einstein con constante cosmológica en vacío
\begin{equation}
  R_{\m\n}-\frac{1}{2}Rg_{\m\n}+\Lambda g_{\m\n}=0
\end{equation}
%sujetas a $\Lambda=6$.
% Se puede hacer que $\Lambda$ sea arbitrario colocando un factor de $\Lambda/6$ delante de los $r^2$ en los denominadores. (Para asegurar que esté bien lo hice así primero).


Además,  esta métrica se puede expresar en términos de coordenadas periódicas $x'^\m=\{\th_1,\th_2,\f_1,\f_2\}$ como \footnote{Ver Ref. \cite{PhysRevD.106.084055} y Ref. \cite{Hoxha:2000jf}}
\begin{equation}\label{CP2-angle}
  \dd s^2=\frac{6}{\Lambda}\left[\dd\th_2^2+\sin^2\th_2\cos^2\th_2(\dd\f_2+\sin^2\th_1\dd\f_1)^2+\sin^2\th_2(\dd\th_1^2+\sin^2\th_1\cos^2\th_1\dd\f_1^2)\right]
\end{equation}
donde $0\leq\th_i\leq\p/2$ y $0\leq\f_i\leq 2\p $. Escrita en estas coordenadas es evidente que $\mathbb{CP}^2$ es una variedad compacta.

La transformación de coordenadas que relaciona la métrica de $\mathbb{CP}^2$ escrita como \eqref{CP2-r} y \eqref{CP2-angle}
% (\textcolor{purple}{up to el factor global $6$})
 viene dada por
\begin{equation}
\left\{
\begin{array}{ll}
  r&=\sqrt{\dfrac{6}{\Lambda}}\tan\th_2 \\
  \th&=2\th_1\\
  \f&=-\f_1\\
  \psi&=2\f_2+\f_1
  \end{array}
  \right.
\end{equation}
Es directo ver que
\begin{align}
  \dd r&=\sqrt{\dfrac{6}{\Lambda}}(1+\tan^2\th_2)\dd\th_2\\
  \dd\th&=2\dd\th_1\\
  \dd\f&=-\dd\f_1\\
  \dd\psi&=2\dd\f_2+\dd\f_1
\end{align}

Para ver esto, expresemos cada término de \eqref{CP2-r} en términos de las coordenadas $x'^\m $. El primer término queda
\begin{align}
  \frac{\dd r^2}{(1+\frac{\Lambda}{6}r^2)^2}&=\frac{\frac{6}{\Lambda}(1+\tan^2\th_2)^2\dd\th_2^2}{(1+\tan^2\th_2)^2}=\frac{6}{\Lambda}\dd\th_2^2\label{1}
\end{align}

El segundo queda
\begin{align}
  \frac{\dfrac{r^2}{4}}{(1+\frac{\Lambda}{6}r^2)^2}(\dd\psi+\cos\th\dd\f )^2&=\frac{1}{4}\frac{6}{\Lambda}\frac{\tan^2\th_2}{(1+\tan^2\th_2)^2}\left[2\dd\f_2+\dd\f_1-\cos(2\th_1)\dd\f_1\right]^2
\end{align}
pero
\begin{equation}
  \frac{\tan^2\th_2}{(1+\tan^2\th_2)^2}=\sin^2\th_2\cos^2\th_2
\end{equation}
luego,
\begin{align}
   \frac{\dfrac{r^2}{4}}{(1+\frac{\Lambda}{6}r^2)^2}(\dd\psi+\cos\th\dd\f )^2&=\frac{6}{\Lambda}\frac{1}{4}\sin^2\th_2\cos^2\th_2\left[2\dd\f_2+\dd\f_1-\cos(2\th_1)\dd\f_1\right]^2\\
   &=\frac{6}{\Lambda}\sin^2\th_2\cos^2\th_2\left[\frac{1}{2}(2\dd\f_2+\dd\f_1-\cos(2\th_1)\dd\f_1)\right]^2\\
   &=\frac{6}{\Lambda}\sin^2\th_2\cos^2\th_2\left[\dd\f_2+\frac{1}{2}\dd\f_1-\frac{1}{2}\cos(2\th_1)\dd\f_1\right]^2\\
   &=\frac{6}{\Lambda}\sin^2\th_2\cos^2\th_2\left[\dd\f_2+\frac{1}{2}\dd\f_1-\frac{1}{2}\cos^2(\th_1)\dd\f_1+\frac{1}{2}\sin^2(\th_1)\dd\f_1\right]^2\\
   &=\frac{6}{\Lambda}\sin^2\th_2\cos^2\th_2\left[\dd\f_2+\sin^2(\th_1)\dd\f_1\right]^2\label{2}
\end{align}
donde se usó que $\cos(2\a )=\cos^2\a -\sin^2\a $.

Finalmente el tercer término queda
\begin{align}
  \frac{\dfrac{r^2}{4}}{(1+\frac{\Lambda	}{6}r^2)}(\dd\th^2+\sin^2\th\dd\phi^2)&=\frac{6}{\Lambda}\frac{1}{4}\frac{\tan^2\th_2}{(1+\tan^2\th_2)}(4\dd\th_1^2+\sin^2(2\th_1)\dd\f_1^2)\\
  &=\frac{6}{\Lambda}\frac{1}{4}\sin^2\th_2(4\dd\th_1^2+4\sin^2\th_1\cos^2\dd\f_1^2)\\
  &=\frac{6}{\Lambda}\sin^2\th_2(\dd\th_1^2+\sin^2\th_1\cos^2\dd\f_1^2)\label{3}
\end{align}
donde se usó, 
\begin{equation}
  \frac{\tan^2\th_2}{1+\tan^2\th_2}=\sin^2\th_2,\qquad \sin(2\a )=2\sin\a\cos\a 
\end{equation}
Finalmente de \eqref{1}, \eqref{2} y \eqref{3} se tiene en efecto
\begin{align}
  \dd s^2&=\frac{\dd r^2}{(1+\frac{\Lambda}{6}r^2)^2}+\frac{\dfrac{r^2}{4}}{(1+\frac{\Lambda}{6}r^2)^2}(\dd\psi+\cos\th\dd\f )^2+\frac{\dfrac{r^2}{4}}{(1+\frac{\Lambda}{6}r^2)}(\dd\th^2+\sin^2\th\dd\phi^2)\\
  &=\frac{6}{\Lambda}[\dd\th_2^2+\sin^2\th_2\cos^2\th_2(\dd\f_2+\sin^2\th_1\dd\f_1)^2+\sin^2\th_2(\dd\th_1^2+\sin^2\th_1\cos^2\th_1\dd\f_1^2)]\label{no-6}
\end{align}

%Notar el factor global de $6$ que no está presente en \eqref{no-6} si se compara con \eqref{CP2-angle}.

Debido a los rangos de las coordenadas $x'^\m=\{\th_1,\th_2,\f_1,\f_2\}$, se tiene que el rango de las coordenadas $x^\m$ es $0\leq r<\infty,0\leq\th\leq\p ,0\leq \f<2\p $ y \textcolor{purple}{$0\leq\psi\leq 6\p $} \footnote{El rango de la coordenada $\psi$ no coincide con el que debería tener.} .



\begin{align}
  \sigma_1&=-4\cos\th_1\sin\th_1(\sin\f_2\cos\f_2\cos\f_1+\cos^2\th_2\sin\f_1)\dd\f_1\\
  &~~~~ + 2(\cos^2\f_2\cos\f_1-2\sin\f_2\cos\f_2\sin\f_1-\cos\f_1)\dd\th_1\\
  \sigma_2&=-2\cos\th_1\sin\th_1(2\cos^2\f_2\cos\f_1-2\sin\f_2\cos\f_2\sin\f_1-\cos\f_1)\dd\f_1\\
  &~~~~ -2(2\sin\f_2\cos\f_2\cos\f_1+2\cos^2\f_2\sin\f_1-\sin\f_1)\dd \th_1\\
  \sigma3&=2\dd\f_2+2\sin^2\th_1\dd\f_1
\end{align}















\newpage
\bibliographystyle{JHEP}
\bibliography{biblio.bib}
\end{document}
