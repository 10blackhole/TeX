\section{Conexión de Lorentz}
La parte sin torsión de la conexión se obtiene como sigue. Dado que $\dd e^{a}$ es una 2-forma, deben existir coeficientes $\Omega^{a}_{~ bc}(x)$ tales que podamos escribir
\begin{equation}\label{izau}
  \dd e^{a}=-\frac{1}{2}\Omega^{a}_{~bc} e^{b}\wedge e^{c}
\end{equation}
en donde claramente $\Omega^{a}_{~bc}$ es antisimétrico en los índices contraídos con los vielbeins
\begin{equation}\label{Omega-anti}
  \Omega^{a}_{~bc}=-\Omega^{a}_{~cb}
\end{equation}
La condición de torsión nula
\begin{equation}
  T^{a}=\dd e^{a}+\omega^{a}_{~b}\wedge e^b=0
\end{equation}
puede reescribirse como
%TODO Cambiar
\begin{align}
  \frac{1}{2}\Omega^{a}_{~bc}e^b\wedge e^c+\omega^{a}_{~b}\wedge e^b&=0\\
  \frac{1}{2}\Omega^{a}_{~bc}e^b\wedge e^c+\omega^{a}_{~bc}e^c\wedge e^b&=0\\
  \frac{1}{2}\Omega^{a}_{~bc}e^b\wedge e^c-\omega^{a}_{~bc}e^b\wedge e^c&=0
\end{align}
es decir,
\begin{equation}
  \omega^{a}_{~bc}e^b\wedge e^c=\frac{1}{2}\Omega^{a}_{~bc}e^b\wedge e^c
\end{equation}

Usando \eqref{Omega-anti}, y expandiendo el producto cuña, se tiene
\begin{align}
  \frac{1}{2}\omega^{a}_{~bc}(e^b\otimes e^c-e^c\otimes e^b)&=\frac{1}{4}\Omega^{a}_{~bc}(e^b\otimes e^c-e^c\otimes e^b)\\
  (\omega^{a}_{~bc}-\omega^{a}_{~cb})e^b\otimes e^c&=\Omega^{a}_{~bc}e^b\wedge e^c
\end{align}
luego
\begin{equation}
  \omega_{abc}-\omega_{acb}=\Omega_{abc}
\end{equation}
Considerando permutaciones cíclicas de esta última ecuación,
\begin{align}
  \omega_{abc}-\omega_{acb}&=\Omega_{abc}\\
  \omega_{bca}-\omega_{bac}&=\Omega_{bca}\\
  \omega_{cab}-\omega_{cba}&=\Omega_{cab}
\end{align}
Sumando estas 3 ecuaciones en la forma $(++-)$, obtenemos
\begin{align}
   \omega_{abc}-\omega_{acb}+\omega_{bca}-\omega_{bac}-\omega_{cab}+\omega_{cba}&=\Omega_{abc}+\Omega_{bca}-\Omega_{cab}
\end{align}
Usando el hecho de que escritas de esta manera, $\omega_{abc}$ son antisimétricas en dos primeros 2 índices, i.e, $\omega_{abc}=-\omega_{bac}$, se tiene
\begin{equation}
  2\omega_{abc}=\Omega_{abc}+\Omega_{bca}-\Omega_{cab}
\end{equation}
Despejando,
\begin{equation}
  \omega_{abc}=\frac{1}{2}(\Omega_{abc}+\Omega_{bca}-\Omega_{cab})
\end{equation}
Finalmente, se tiene
\begin{equation}\label{izau2}
  \boxed{\omega_{ab}=\frac{1}{2}(\C_{bac}+\C_{acb}-\C_{cba})e^c}
\end{equation}

\subsection{Curvatura de Lorentz}
Notemos que al separar la conexión como ${\omega}^{ab}=\oo\omega^{ab}+\kappa^{ab}$ induce tambien una separación en la curvatura. En efecto
\begin{align}
  R^{ab}&=\dd\omega^{ab}+\omega^{a}_{~c}\wedge\omega^{cb}\\
  &=\dd (\oo\omega^{ab}+\kappa^{ab})+(\oo\omega^{a}_{~c}+\kappa^{a}_{~c})\wedge (\oo\omega^{cb}+\kappa^{cb})\\
  &=\dd \oo\omega^{ab}+\dd \kappa^{ab}+\oo\omega^{a}_{~c}\wedge \oo\omega^{cb}+\oo\omega^{a}_{~c}\wedge \kappa^{cb}+ \kappa^{a}_{~c}\wedge \oo\omega^{cb}+\kappa^{a}_{~c}\wedge \kappa^{cb}\\
  &=\dd \oo\omega^{ab}+\oo\omega^{a}_{~c}\wedge \oo\omega^{cb}+\dd \kappa^{ab}+\oo\omega^{a}_{~c}\wedge \kappa^{cb}+\oo{\omega}^b_{~c}\wedge \kappa^{ac}+\kappa^{a}_{~c}\wedge \kappa^{cb}\\
  &=\oo{R}^{ab}+\oo{\D}\kappa^{ab}+\kappa^{a}_{~c}\wedge \kappa^{cb}
\end{align}
Así
\begin{equation}\label{R-decomposed}
  \boxed{R^{ab}=\oo{R}^{ab}+\oo{\D}\kappa^{ab}+\kappa^{a}_{~c}\wedge \kappa^{cb}}
\end{equation}



















\section{Transformación conforme}
Consideremos una transformación conforme de la forma
\begin{equation}
  g_{\mu\nu}\to \bar{g}_{\mu\nu}=\mathrm{e}^{2\sigma(x)}g_{\mu\nu}
\end{equation}
donde $\sigma(x)$ es una función arbitraria de las coordenadas del espacio-tiempo. De aquí es fácil ver que bajo esta transformación, el vielbein transforma como
\begin{equation}
  e^{a}\to \bar{e}^{a}=\e^{\sigma(x)}e^{a}
\end{equation}
De \eqref{izau} podemos escribir\begin{equation}
	\dd e^{a}=\frac{1}{2}\mathcal{C}^{a}_{~bc} e^{b}\wedge e^{c}
\end{equation}
donde los coeficientes $\C_{abc}$ son los parámetros de estructura y se relacionan con la parte sin torsión de la conexión de Lorentz según \eqref{izau2}, i.e.,
\begin{equation}
 	\omega_{ab}=\frac{1}{2}(\C_{bac}+\C_{acb}-\C_{cba})e^c
\end{equation}

Se define el operador $\I_a:\Omega^p\to \Omega^{p-1}$ según \cite{Izaurieta:2020kuy}
\begin{equation}
  \I _a=-*(e_a\wedge *
\end{equation}
Calculemos $\dd\bar{e}^{a}$,
\begin{align}
  \dd \bar{e}^{a}&=\dd (\e ^\sigma e^{a})\\
  &=\e ^\sigma (\dd \sigma e^{a}+\dd e^{a}) \label{de}
\end{align}
Notemos que podemos escribir la 1-forma $\dd\sigma$ como
\begin{equation}
  \dd\sigma = \xi_a e^{a},\qquad \text{donde } \xi_a=\I_a\dd\sigma
\end{equation}
En efecto, si escribimos $\dd\sigma=\alpha_f e^{f}$, se tiene
\begin{align}
  \I_a\dd\sigma&=-*(e_a\wedge *\dd\sigma)\\
  &=-*\left(e_a\wedge \alpha_f\frac{1}{3!}\epsilon^{fbcd}e^b\wedge e^c\wedge e^d\right)\\
  &=-*\left(\alpha_f\frac{1}{3!}\epsilon^{fbcd}e^{a}\wedge e^b\wedge e^c\wedge e^d\right)\\
  &=-\alpha_f\frac{1}{3!}\epsilon^{fbcd}\epsilon_{abcd}\\
  &=\alpha_f\frac{1}{3!}3!\\
  &=\alpha_f
\end{align}
luego, $\dd \sigma = \I_a\dd \sigma e^{a}=\xi_a e^{a}$. Reemplazando en \eqref{de},
\begin{align}
  \dd\bar{e}^{a}&=\e^\sigma (\dd\sigma e^{a}+\dd e^{a})\\
  &=\e^\sigma\left(\xi_me^m\wedge e^{a}-\frac{1}{2}\C^{a}_{~bc}e^b\wedge e^c\right)\\
  &=e^{-\sigma}\left(\xi_m\bar{e}^m\wedge \bar{e}^{a}-\frac{1}{2}\C^{a}_{~bc}\bar{e}^b\wedge\bar{e}^c\right)\\
  &=\e^{-\sigma}\left(\xi_m\delta^m_{[b}\delta^{a}_{c]}-\frac{1}{2}\C^{a}_{~bc}\right)\bar{e}^b\wedge \bar{e}^c\\
  &=\e^{-\sigma}\left(\frac{1}{2}\xi_m\delta^{ma}_{bc}-\frac{1}{2}\C^{a}_{~bc}\right)\bar{e}^b\wedge \bar{e}^c\\
  &=\e^{-\sigma}\left(-\frac{1}{2}\xi_m\delta^{am}_{bc}-\frac{1}{2}\C^{a}_{~bc}\right)\bar{e}^b\wedge \bar{e}^c\\
  &=-\frac{1}{2}\bar{\C}^{a}_{~bc}\bar{e}^b\wedge \bar{e}^c
\end{align}
Así,
\begin{equation}
  \bar{\C}^{a}_{~bc}=\e^{-\sigma} \left(\C^{a}_{~bc}+\xi_m\delta^{am}_{bc}\right)
\end{equation}
Notemos que
\begin{align}
  \bar{\C}_{abc}&=\e^{-\sigma}\left(\C_{abc}+\xi_m\eta_{ab}\delta^m_c-\xi_m\delta^m_b\eta_{ac}\right)\\
  &=\e^{-\sigma}(\C_{abc}+\xi_c\eta_{ab}-\xi_b\eta_{ac})
\end{align}
De manera que
\begin{subequations}\label{bar-C}
\begin{align}
  \bar{\C}_{bac}&=\e^{-\sigma}(\C_{bac}+\xi_c\eta_{ba}-\xi_a\eta_{bc})\\
  \bar{\C}_{acb}&=\e^{-\sigma}(\C_{acb}+\xi_a\eta_{cb}-\xi_b\eta_{ca})\\
  \bar{\C}_{cba}&=\e^{-\sigma}(\C_{cba}+\xi_a\eta_{cb}-\xi_b\eta_{ca})
\end{align}
\end{subequations}
De \eqref{izau2} es fácil ver que
\begin{equation}\label{bar-omega}
	\bar{\omega}_{ab}=\frac{1}{2}(\bar{\C}_{bac}+\bar{\C}_{acb}-\bar{\C}_{cba})\bar{e}^c
\end{equation}
Reemplazando \eqref{bar-C} en \eqref{bar-omega} se tiene
\begin{align}
  \bar{\omega}_{ab}&=\frac{1}{2}\e^{-\sigma}(\C_{bac}+\xi_c\eta_{ba}-\xi_a\eta_{bc}+\C_{acb}+\xi_a\eta_{cb}-\xi_b\eta_{ca}-\C_{cba}-\xi_a\eta_{cb}+\xi_b\eta_{ca})\bar{e}^c\\
  &=\frac{1}{2}\e^c(\C_{bac}+\C_{acb}-\C_{cba})+\frac{1}{2}e^c(\xi_c\eta_{ba}-\xi_a\eta_{bc}+\xi_a\eta_{cb}-\xi_b\eta_{ca}-\xi_a\eta_{cb}+\xi_b\eta_{ca})\\
  &=\omega_{ab}+\frac{1}{2}(\cancel{\xi_c\eta_{ba}e^c}-\xi_ae_b+\xi_be_a-\cancel{\xi_c\eta_{ab}e^c}-\xi_ae_b+\xi_be_a)\\
  &=\omega_{ab}+e_a\xi_b-e_b\xi_a
\end{align}
Así,
\begin{equation}\label{bar-omega-1}
  \bar{\omega}_{ab}=\omega_{ab}+\theta_{ab}
\end{equation}

donde hemos definido $\theta_{ab}=e_a\xi_b-e_b\xi_a$.

Una importante pregunta que concierne a las transformaciones conforme en el formalismo de primer orden está relacionada con el tipo de transformación de la parte sin torsión de la conexión. 

Primero veamos el el caso \textbf{canónico} en el cual la estructura torsional es invariante bajo las transformaciones conforme, es decir,
\begin{equation}
  \kappa_{ab}\to \bar{\kappa}_{ab}=\kappa_{ab}
\end{equation}
Esto implica que la conexión de spin transforma según \eqref{bar-omega-1},
\begin{equation}
  \bar{\omega}_{ab}=\omega_{ab}+\theta_{ab}
\end{equation}

Para ver cómo transforma la torsión en este caso, usamos el hecho que
\begin{equation}
  T^{a}=\kappa^{a}_{~b}\wedge e^b
\end{equation}
luego,
\begin{align}
  \bar{T}^{a}&=\bar{\kappa}^{a}_{~b}\wedge\bar{e}^b\\
  &=\kappa^{a}_{~b}\wedge (\e^\sigma e^b)\\
  &=\e^\sigma T^{a}
\end{align}
es decir,
\begin{equation}
 \boxed{\bar{T}^{a} =\e^\sigma T^{a}}
\end{equation}

Veamos como transforma la curvatura $R^{ab}$. De \eqref{R-decomposed},
\begin{align}
	\bar{R}^{ab}&=\oo{\bar{R}}^{ab}+\oo{\bar{\D}}\bar{\kappa}^{ab}+\bar{\kappa}^{a}_{~c}\wedge \bar{\kappa}^{cb}\\
	&=\dd\oo{\bar{\omega}}^{ab}+\oo{\bar{\omega}}^{a}_{~c}\wedge\oo{\bar{\omega}}^{cb}+\dd\bar{\kappa}^{ab}+\oo{\bar{\omega}}^{a}_{~c}\wedge \bar{\kappa}^{cb}+\oo{\bar{\omega}}^{b}_{~c}\wedge \bar{\kappa}^{ac}+\bar{\kappa}^{a}_{~c}\wedge\bar{\kappa}^{cb}\\
	&=\dd(\oomega^{ab}+\theta^{ab})+(\oomega^{a}_{~c}+\theta^{a}_{~c})\wedge (\oomega^{cb}+\theta^{cb})+\dd \kappa^{ab}+(\oomega^{a}_{~c}+\theta^{a}_{~c})\wedge \kappa^{cb}\\&~~~+(\oomega^b_{~c}+\theta^b_{~c})\wedge \kappa^{ac}+\kappa^{a}_{~c}\wedge\kappa^{cb}\\
	&=\dd\oomega^{ab}+\dd\theta^{ab}+\oomega^{a}_{~c}\wedge\oomega^{cb}+\oomega^{a}_{~c}\wedge \theta^{cb}+\theta^{a}_{~c}\wedge \oomega^{ab}+\theta^{a}_{~c}\wedge \theta^{cb}+\dd\kappa^{ab}\\&~~~~ \oomega^{a}_{~c}\wedge\kappa^{cb}+\theta^{a}_{~c}\wedge\kappa^{cb}+\oomega^b_{~c}\wedge\kappa^{ac}+\theta^b_{~c}\wedge\kappa^{ac}+\kappa^{a}_{^c}\wedge\kappa^{cb}\\
	&=\dd\oomega^{ab}+\oomega^{a}_{~c}\wedge\oomega^{cb}+\dd\kappa^{ab}+\oomega^{a}_{~c}\wedge\kappa^{cb}+\oomega^b_{~c}\wedge\kappa^{ac}+\kappa^{a}_{~c}\wedge\kappa^{cb}\\&~~~~
	+\dd\theta^{ab}+\oomega^{a}_{~c}\wedge \theta^{cb}+\kappa^{a}_{~c}\wedge \theta^{cb}+\oomega^b_{~c}\wedge\theta^{ac}+\kappa^b_{~c}\wedge\theta^{ac}+\theta^{a}_{~c}\wedge\theta^{cb}\\
	&=R^{ab}+\omega^{a}_{~c}\wedge \theta^{cb}+\omega^b_{~c}\wedge\theta^{ac}\\
	&=R^{ab}+\D\theta^{ab}+\theta^{a}_{~c}\wedge\theta^{cb}
\end{align}
es decir,
\begin{equation}
  \boxed{\bar{R}^{ab}=R^{ab}+\D\theta^{ab}+\theta^{a}_{~c}\wedge\theta^{cb}}
\end{equation}

Por otro lado tenemos el caso \textbf{exótico}, donde la conexión de spin queda invariante bajo las transformaciones conformes, 
\begin{equation}
  \omega_{ab}\to \bar{\omega}_{ab}=\omega_{ab}
\end{equation}
Veamos como transforma la contorsión $\kappa_{ab}$,
\begin{align}
  \bar{\omega}_{ab}&=\omega_{ab}\\
  \oo{\bar{\omega}}_{ab}+\bar{\kappa}_{ab}&=\oomega_{ab}+\kappa_{ab}\\
  \oomega_{ab}+\theta_{ab}+\bar{\kappa}_{ab}&=\oomega_{ab}+\kappa_{ab}\\
  \theta_{ab}+\bar{\kappa}_{ab}&=\kappa_{ab}
\end{align}
luego,
\begin{equation}
\boxed{\bar{\kappa}_{ab}=\kappa_{ab}-\theta_{ab}}
\end{equation}
Veamos como transforma la curvatura en este caso
\begin{align*}
  \bar{R}^{ab}&=\oo{\bar{R}}^{ab}+\oo{\bar{\D}}\bar{\kappa}^{ab}+\bar{\kappa}^{a}_{~c}\wedge \bar{\kappa}^{cb}\\
  &=\dd\oo{\bar{\omega}}^{ab}+\oo{\bar{\omega}}^{a}_{~c}\wedge\oo{\bar{\omega}}^{cb}+\dd\bar{\kappa}^{ab}+\oo{\bar{\omega}}^{a}_{~c}\wedge \bar{\kappa}^{cb}+\oo{\bar{\omega}}^{b}_{~c}\wedge \bar{\kappa}^{ac}+\bar{\kappa}^{a}_{~c}\wedge\bar{\kappa}^{cb}\\
  &=\dd\oo{\bar{\omega}}^{ab}+\cancel{\dd\theta^{ab}}+(\oomega^{a}_{~c}+\theta^{a}_{~c})\wedge (\oomega^{cb}+\theta^{cb})+\dd\kappa^{a}-\cancel{\dd\theta^{ab}}\\&
  ~~~~+(\oomega^{a}_{~c}+\theta^{a}_{~c})\wedge (\kappa^{cb}-\theta^{cb})+(\oomega^b_{~c}+\theta^b_{~c})\wedge (\kappa^{ac}-\theta^{ac})+(\kappa^{a}_{~c}-\theta^{a}_{~c})\wedge (\kappa^{cb}-\theta^{cb})\\
  &=\dd\oomega^{ab}+\oomega^{a}_{~c}\wedge\oomega^{cb}+\oomega^{a}_{~c}\wedge\theta^{cb}+\theta^{a}_{~c}\wedge\oomega^{cb}+\theta^{a}_{~c}\wedge\theta^{cb}+\dd\kappa^{ab}+\oomega^{a}_{~c}\wedge\kappa^{cb}-\oomega^{a}_{~c}\wedge\theta^{cb}\\
  &~~~~+\theta^{a}_{~c}\wedge\kappa^{cb}-\theta^{a}_{~c}\wedge\theta^{cb}+\oomega^b_{~c}\wedge\kappa^{ac}-\oomega^b_{~c}\wedge\theta^{ac}+\theta^b_{~c}\wedge\kappa^{ac}-\theta^b_{~c}\wedge\theta^{ac}+\kappa^{a}_{~c}\wedge\kappa^{cb}\\
  &~~~-\kappa^{a}_{~c}\wedge\theta^{cb}-\theta^{a}_{~c}\wedge \kappa^{cb}+\theta^{a}_{~c}\wedge\theta^{cb}\\
  &=\dd\oomega^{ab}+\oomega^{a}_{~c}\wedge \oomega^{cb}+\dd\kappa^{ab}+\oomega^{a}_{~c}\wedge\kappa^{cb}+\oomega^b_{~c}\wedge\kappa^{ac}+\kappa^{a}_{~c}\wedge\kappa^{cb}\\
  &=R^{ab}
\end{align*}
donde es fácil ver que los demás términos se cancelan entre sí. Finalmente
\begin{equation}
   \boxed{\bar{R}^{ab}=R^{ab}}
\end{equation}


Veamos ahora como transforma la torsión,
\begin{equation}\label{theta e}
\begin{split}
  \bar{T}^{a}&=\bar{\kappa}^{a}_b\wedge\bar{e}^b\\
  &=(\kappa^{a}_{~b}-\theta^{a}_{~b})\wedge\e^\sigma e^b\\
  &=\e^\sigma (\kappa^{a}_{~b}-\theta^{a}_{~b})\wedge e^b\\
  &=\e^\sigma (\kappa^{a}_{~b}\wedge e^b-\theta^{a}_{~b}\wedge e^b)\\
  &=\e^\sigma (T^{a}-\theta^{a}_{~b}\wedge e^b)
\end{split}
\end{equation}
Notemos que
\begin{equation}\label{theta e2}
\begin{split}
  \theta^{a}_{~b}\wedge e^b&=(e^{a}\xi_b-e_b\xi^{a})\wedge e^b\\
  &=e^{a}\xi_b\wedge e^b\\
  &=e^{a}\wedge\dd\sigma
\end{split}
\end{equation}
Luego,
\begin{equation}
   \boxed{\bar{T}^{a}=e^\sigma (T^{a}-e^{a}\wedge\dd\sigma)}
\end{equation}

Sin embargo, podemos considerar un rango más amplio de transformaciones interpolando entre ambos casos. Definimos el parámetro real $\lambda\in \mathbb{R}$, de manera que
\begin{subequations}\label{lambda}
\begin{align}
  \bar{e}^{a}&=\e ^\sigma e^{a}\label{e lambda}\\
  \bar{\omega}^{ab}&=\omega^{ab}+\lambda\theta^{ab}\label{omega lambda}\\
  \bar{\kappa}^{ab}&=\kappa^{ab}+(\lambda-1)\theta^{ab}\label{kappa lambda}
\end{align}
\end{subequations}
así, el límite $\lambda=1$ corresponde al caso canónico y $\lambda=0$ al exótico. Veamos como queda la torsión y la curvatura bajo las transformaciones \eqref{lambda}. Para la torsión tenemos
\begin{align}
  \bar{T}^{a}&=\bar{\kappa}^{a}_{~b}\wedge\bar{e}^b\\
  &=\e^\sigma\left[\kappa^{a}_{~b}\wedge e^b +(\lambda-1)\theta^{a}_{~b}\wedge e^b\right]\\
  &=\e^\sigma\left[T^{a}+(\lambda-1)e^{a}\wedge \dd\sigma\right]
\end{align}
donde para pasar a la última línea se usó \eqref{theta e}. Expandiendo a primer orden la exponencial, tenemos
\begin{align}
  \bar{T}^{a}&=T^{a}+(\lambda-1)e^{a}\wedge \dd\sigma+\sigma T^{a}+\sigma (\lambda-1)e^{a}\wedge \dd\sigma
\end{align}
luego, la transformación infinitesimal de la torsión es
\begin{equation}
  \boxed{\delta T^{a}=\sigma T^{a}+(\lambda-1)e^{a}\wedge\dd\sigma}
\end{equation}
ya que el último término es de segundo orden en $\sigma$.

Veamos ahora cómo queda la curvatura bajo \eqref{lambda},
\begin{align*}
  \bar{R}^{ab}&=\oo{\bar{R}}^{ab}+\oo{\bar{\D}}\bar{\kappa}^{ab}+\bar{\kappa}^{a}_{~c}\wedge \bar{\kappa}^{cb}\\
  &=\dd\oo{\bar{\omega}}^{ab}+\oo{\bar{\omega}}^{a}_{~c}\wedge\oo{\bar{\omega}}^{cb}+\dd\bar{\kappa}^{ab}+\oo{\bar{\omega}}^{a}_{~c}\wedge \bar{\kappa}^{cb}+\oo{\bar{\omega}}^{b}_{~c}\wedge \bar{\kappa}^{ac}+\bar{\kappa}^{a}_{~c}\wedge\bar{\kappa}^{cb}\\
  &=\dd\oomega^{ab}+\dd\theta^{ab}+\oomega^{a}_{~c}\wedge \oomega^{cb}+\oomega^{a}_{~c}\wedge \theta^{cb}+\theta^{a}_{~c}\wedge \oomega^{cb}+\theta^{a}_{~c}\wedge \theta^{cb}+\dd\kappa^{ab}+(\lambda-1)\dd\theta^{ab}+\oomega^{a}_{~c}\wedge\kappa^{cb}\\&~~~~+(\lambda-1)\oomega^{a}_{~c}\wedge\theta^{cb}+\theta^{a}_{~c}\wedge\kappa^{cb}+(\lambda-1)\theta^{a}_{~c}\wedge\theta^{cb}+\oomega^b_{~c}\wedge\kappa^{ac}+(\lambda-1)\oomega^b_{~c}\wedge\theta^{ac}+\theta^c_{~c}\wedge\kappa^{ac}\\
  &~~~~+(\lambda-1)\theta^b_{~c}\wedge\theta^{ac}+\kappa^{a}_{~c}\wedge\kappa^{cb}+(\lambda-1)\kappa^{a}_{~c}\wedge\theta^{cb}+(\lambda-1)\theta^{a}_{~c}\wedge\kappa^{cb}+(\lambda-1)^2\theta^{a}_{~c}\wedge\theta^{cb}\\
  &=R^{ab}+\D\theta^{ab}+(\lambda-1)\D\theta^{ab}+(\lambda^2-2\lambda+1)\theta^{a}_{~c}\wedge\theta^{cb}+(2\lambda-1)\theta^{a}_{~c}\wedge\theta^{cb}\\
  &=R^{ab}+\lambda\D\theta^{ab}+\lambda^2\theta^{a}_{~c}\wedge\theta^{cb}
 \end{align*}
 donde se usó \eqref{bar-omega-1}. Así
 \begin{equation}
  \boxed{\delta R^{ab}=\lambda\D\theta^{ab}++\lambda^2\theta^{a}_{~c}\wedge\theta^{cb}}
\end{equation}
ya que el último término es de segundo orden en $\sigma$.

Además, el la transformación del campo escalar tal que sea invariante conforme en $d$-dimensones es \cite{Martinez:1996gn}
\begin{equation}
  \phi\to \bar{\phi}=\exp\left[\sigma\left(\frac{2-d}{2}\right)\right]\phi
\end{equation}


En resúmen, tenemos
\begin{tcolorbox}
\begin{subequations}\label{infini}
\begin{align}
    \delta e^{a}&=\sigma e^{a}\\
    \delta \phi&=\left(\frac{2-d}{2}\right)\sigma\phi\\
    \delta\omega^{a}&=\lambda \theta^{ab}\\
    \delta R^{ab}&=\lambda D\theta^{ab}\\
    \delta T^{a}&=\sigma T^{a}+(\lambda-1)e^{a}\wedge\dd\sigma
\end{align}
\end{subequations}
\end{tcolorbox}


\subsection{Campos auxiliares}
Definimos los siguientes campos auxiliares
\begin{align}
    \Tilde{e}^{a}&=\phi^\frac{2}{d-2} e^{a}\\
    \Tilde{\omega}^{ab}&=\omega^{ab}+\lambda\Sigma ^{ab}\\
    \Tilde{\kappa}^{ab}&=\kappa^{ab}+(\lambda-1)\Sigma ^{ab}
\end{align}
donde 
\begin{equation}
  \Sigma^{ab}=\frac{2}{(d-2)}\frac{1}{\phi}\chi^{ab}
\end{equation}
y
\begin{equation}
  \chi^{ab}\equiv e^{a}z^b-z^{a}e^b,\qquad z^{a}=\I^{a}\dd\phi
\end{equation}



Veamos cómo es su variacion infinitesimal bajo transformaciones conformes infinitesimales usando \eqref{infini}. Para $\tilde{e}^{a}$, tenemos
\begin{align}
  \delta \tilde{e}^{a}&=\frac{2}{d-2}\phi^\frac{4-d}{d-2}\left(\frac{2-d}{2}\right)\sigma\phi e^{a}+\phi^\frac{2}{d-2}\sigma e^{a}\\
  &=-\phi^\frac{2}{d-2}\sigma e^{a}+\phi^\frac{2}{d-2}\sigma e^{a}\\
  &=0
\end{align}
Luego, $\tilde{e}^{a}$ es invariante conforme.

Veamos ahora cómo varia $\tilde{\omega}^{ab}$,
\begin{align}
  \delta\tilde{\omega}^{ab}&=\delta \omega^{ab}+\lambda\delta\Sigma^{ab}\\
  &=\lambda\theta^{ab}+\lambda\delta\Sigma^{ab}\label{d omega}
\end{align}

Calculemos $\delta\Sigma^{ab}$,
\begin{align}
  \delta\Sigma^{ab}&=-\frac{2}{(d-2)}\frac{1}{\phi^2}\delta\phi \chi^{ab}+\frac{2}{(d-2)}\frac{1}{\phi}\delta\chi^{ab}\\
  &=-\frac{2}{(d-2)}\frac{1}{\phi^2}\left(\frac{2-d}{d}\right)\sigma\phi\chi^{ab}+\frac{2}{(d-2)}\frac{1}{\phi}\delta\chi^{ab}\\
  &=\frac{\sigma}{\phi}\chi^{ab}+\frac{2}{(d-2)}\frac{1}{\phi}\delta\chi^{ab}\label{d sigma}
\end{align}

Calculemos $\delta\chi^{ab}$,
\begin{align}
  \delta\chi^{ab}&=\delta e^{a}z^b+e^{a}\delta z^b-\delta z^{a}e^b-z^{a}\delta e^b\\
  &=\sigma (e^{a}z^b-z^{a}e^b)+e^{a}\delta z^b-\delta z^{a}e^b\\
  &=\sigma\chi^{ab}+e^{a}\delta z^b-\delta z^{a}e^b \label{delta chi}
\end{align}
Calculemos $\delta z^{a}$. Para ello, notemos que (\textcolor{blue}{Por demostrar!})
\begin{equation}
  \delta_e z^{a}=-\I ^n(\delta e^{a})z_n
\end{equation}
luego,
\begin{align}
  \delta z^{a}&=\delta_a z^{a}+\delta_\phi z^{a}\\
  &=-\I ^n(\delta e^{a})z_n+\I^{a}\dd (\delta \phi)\\
  &=-\I^n (\sigma e^{a})z_n+\left(\frac{2-d}{2}\right)\I^{a}\dd (\sigma \phi)\\
  &=-\sigma \I^n (e^{a})z_n+\left(\frac{2-d}{2}\right)(\phi\dd\sigma+\sigma\dd\phi)\\
  &=-\sigma \eta^{na}z_n+\left(\frac{2-d}{2}\right)(\phi \xi^{a}+\sigma z^{a})\\
  &=-\frac{d}{2}\sigma z^{a}+\left(\frac{2-d}{2}\right)\phi \xi^{a}
\end{align}
Reemplazando en \eqref{delta chi}
\begin{align}
  \delta\chi^{ab}&=\sigma\chi^{ab}+e^{a}\left[-\frac{d}{2}\sigma z^{b}+\left(\frac{2-d}{2}\right)\phi \xi^{b}\right]-\left[-\frac{d}{2}\sigma z^{a}+\left(\frac{2-d}{2}\right)\phi \xi^{a}\right]e^b\\
  &=\sigma \chi^{ab}-\sigma\frac{d}{2}\chi^{ab}+\left(\frac{2-d}{2}\right)\phi\theta^{ab}\\
  &=\sigma\left(\frac{2-d}{2}\right)\chi^{ab}+\left(\frac{2-d}{2}\right)\phi\theta^{ab}
\end{align}
Reemplazando en \eqref{d sigma},
\begin{equation}
\begin{split}
  \delta\Sigma^{ab}&=\frac{\sigma}{\phi}\chi^{ab}+\frac{2}{(d-2)}\frac{1}{\phi}\delta\chi^{ab}\label{d sigma}\\
  &=\frac{\sigma}{\phi}\chi^{ab}+\left(\frac{2}{d-2}\right)\frac{1}{\phi}\left[\sigma\left(\frac{2-d}{2}\right)  \chi^{ab}+\left(\frac{2-d}{2}\right)\phi\theta^{ab}\right]\\
  &=\frac{\sigma}{\phi}\chi^{ab}-\frac{\sigma}{\phi}\chi^{ab}-\theta^{ab}\\
  &=-\theta^{ab}
\end{split}
\end{equation}
Así, de \eqref{d omega},
\begin{align}
  \delta\tilde{\omega}^{ab}&=\lambda\theta^{ab}+\lambda\delta\Sigma^{ab}\label{d omega}\\
  &=\lambda\theta^{ab}-\lambda\theta^{ab}\\
  &=0
\end{align}
es decir, $\tilde{\omega}^{ab}$ es invariante conforme.

Finalmente calaculemos la variación de $\tilde{\kappa}^{ab}$,
\begin{align}
  \delta \bar{\kappa}^{ab}&=\delta \kappa^{ab}+(\lambda-1)\delta \Sigma^{ab}\\
  &=\delta \kappa^{ab}-(\lambda-1) \theta^{ab}\\
  &=(\lambda-1) \theta^{ab}-(\lambda-1) \theta^{ab}\\
  &=0
\end{align}
donde usamos \eqref{d sigma}.

Veamos ahora cómo queda la torsión y la curvartura definidas a partir de estos campos auxiliares. Para la torsión tenemos
\begin{align}
  \Tilde{T}^{a}&=\tilde{\kappa}^{a}_{~b}\wedge \tilde{e}^b\\
  &=\left[\kappa^{a}_{~b}+(\lambda-1)\Sigma^{a}_{~b}\right]\wedge \left[\phi^\frac{2}{d-2} e^b\right]\\
  &=\phi^\frac{2}{d-2}\left[\kappa^{a}_{~b}\wedge e^b+(\lambda-1)\Sigma^{a}_{~b}\wedge e^b\right]\\
  &=\phi^\frac{2}{d-2}\left[T^{a}+(\lambda-1)\Sigma^{a}_{~b}\wedge e^b\right]
\end{align}
pero
\begin{align}
  \Sigma^{a}_{~b}\wedge e^b&=\frac{2}{(d-2)}\frac{1}{\phi}\chi^{a}_{~b}\wedge e^b\\
  &=\frac{2}{(d-2)}\frac{1}{\phi}\left(e^{a}z_{b}-z^{a}e_b\right)\wedge e^b\\
  &=\frac{2}{(d-2)}\frac{1}{\phi}e^{a}\wedge \dd \phi
\end{align}
luego
\begin{align}
   \Tilde{T}^{a}&=\phi^\frac{2}{d-2}\left[T^{a}+(\lambda-1)\Sigma^{a}_{~b}\wedge e^b\right]\\
   &=\phi^\frac{2}{d-2}\left[T^{a}+(\lambda-1)\frac{2}{(d-2)}\frac{1}{\phi}e^{a}\wedge \dd \phi\right]\\
   &=\phi^\frac{2}{d-2}T^{a}+\frac{2}{(d-2)}\phi^\frac{4-d}{d-2} (\lambda-1)e^{a}\wedge\dd\phi
\end{align}

Calculemos la curvatura construida a partir de los campos auxiliares
\begin{align}
  \Tilde{R}^{ab}&=\dd\tilde{\omega}^{ab}+\tilde{\omega}^{a}_{~c}\wedge\tilde{\omega}^{cb}\\
  &=\dd (\omega^{ab}+\lambda \Sigma^{ab})+\left(\omega^{a}_{~c}+\lambda\Sigma^{a}_{~c}\right)\wedge \left(\omega^{cb}+\lambda\Sigma^{cb}\right)\\
  &=\dd\omega^{ab}+\lambda \dd\Sigma^{ab} +\omega^{a}_{~c}\wedge \omega^{cb}+\lambda\omega^{a}_{~c}\wedge\Sigma^{cb}+\lambda \Sigma^{a}_{~c}\wedge\omega^{cb}+\lambda^2\Sigma^{a}_{~c}\wedge\Sigma^{cb}\\
  &=R^{ab}+\lambda (\dd\Sigma^{ab}+\omega^{a}_{~c}\wedge \Sigma^{cb}+\omega^b_{~c}\wedge\Sigma^{ac})+\lambda^2\Sigma^{a}_{~c}\wedge\Sigma^{cb}\\
  &=R^{ab}+\lambda \D \Sigma^{ab}+\lambda^2\Sigma^{a}_{~c}\wedge\Sigma^{cb}
\end{align}

Debido a que estas cantidades están construidas a partir de objetos que son conformalmente invariantes, ellas también lo son.

En resumen, tenemos los siguientes campos auxiliares conformalmente invariantes
\begin{tcolorbox}
\begin{align}
    \Tilde{e}^{a}&=\phi^\frac{2}{d-2} e^{a}\\
    \Tilde{\omega}^{ab}&=\omega^{ab}+\lambda\Sigma ^{ab}\\
    \Tilde{\kappa}^{ab}&=\kappa^{ab}+(\lambda-1)\Sigma ^{ab}\\
    \tilde{T}^{a}&=\phi^\frac{2}{d-2}T^{a}+\frac{2}{(d-2)}\phi^\frac{4-d}{d-2} (\lambda-1)e^{a}\wedge\dd\phi\\
    \Tilde{R}^{ab}&=R^{ab}+\lambda \D \Sigma^{ab}+\lambda^2\Sigma^{a}_{~c}\wedge\Sigma^{cb}
\end{align}
donde
\begin{align}
  \Sigma^{ab}=\frac{2}{(d-2)}\frac{1}{\phi}\left(e^{a}z^b-z^{a}e^b\right),\qquad z^{a}=\I^{a}\dd\phi
\end{align}
\end{tcolorbox}

\subsection{Invariantes en $2+1$}
En $(2+1)$-dimensiones estas cantidades se reducen a
\begin{align}
    \Tilde{e}^{a}&=\phi^2 e^{a}\\
    \Tilde{\omega}^{ab}&=\omega^{ab}+\lambda\Sigma ^{ab}\\
    \Tilde{\kappa}^{ab}&=\kappa^{ab}+(\lambda-1)\Sigma ^{ab}\\
    \tilde{T}^{a}&=\phi^2T^{a}+2\phi (\lambda-1)e^{a}\wedge\dd\phi\\
    \Tilde{R}^{ab}&=R^{ab}+\lambda \D \Sigma^{ab}+\lambda^2\Sigma^{a}_{~c}\wedge\Sigma^{cb}
\end{align}
donde
\begin{align}
  \Sigma^{ab}=\frac{2}{\phi}\left(e^{a}z^b-z^{a}e^b\right)=\frac{4}{\phi}e^{[a}z^{b]} ,\qquad z^{a}=\I^{a}\dd\phi
\end{align}


Calculemos los invariantes que nos podemos construir a partir de estas cantidades en $(2+1)$. 

Primero
\begin{align}
  \epsilon_{abc}\tilde{R}^{ab}\wedge \tilde{e}^c&=\epsilon_{abc}\left(R^{ab}+\lambda \D \Sigma^{ab}+\lambda^2 \Sigma^{a}_{~d}\wedge \Sigma^{db}\right)\wedge\left(\phi^2e^c\right)\\
  &=\phi^2\epsilon_{abc}\left(R^{ab}\wedge e^c+\lambda \D \Sigma^{ab}\wedge e^c+\lambda^2 \Sigma^{a}_{~d}\wedge \Sigma^{db}\wedge e^c\right)\label{inv}
%  &=\phi^2\epsilon_{abc}R^{ab}\wedge e^c-2\lambda \epsilon_{abc}\dd\phi\wedge\chi^{ab}\wedge e^c+2\lambda\phi\epsilon_{abc}\D \chi^{ab}\wedge e^c+4\lambda^2\epsilon_{abc}\chi^{a}_{~d}\wedge\chi^{db}\wedge e^c
\end{align}
Trabajemos un poco más los últimos dos términos\footnote{De aquí en adelante para simplificar notación, consideraremos el \textit{wedge product} implícito.}. Calculemos $\eabc \D\S^{ab}e^c$,
\begin{align}
  \eabc \D\S^{ab}e^c&=-\frac{4}{\phi^2}\eabc\dd\phi e^{a}z^be^c+\frac{2}{\phi}\eabc \D(2e^{a}z^b)e^c\\
  &=-\frac{4}{\phi^2}\eabc\dd\phi e^{a}z^be^c+\frac{4}{\phi}\eabc T^{a}z^be^c-\frac{4}{\phi}\eabc e^{a}\D z^be^c
\end{align}
Luego,
\begin{equation}\label{2do termino}
  \lambda\phi^2\eabc\D \S^{ab}e^c=-4\lambda\dd\phi e^{a}z^be^c+4\lambda\phi\eabc T^{a}z^be^c-4\lambda\eabc e^{a}\D z^be^c
\end{equation}

Ahora calculemos $\eabc\S^{a}_{~d}\S^{db}e^c$,
\begin{align}
  \eabc\S^{a}_{~d}\S^{db}e^c&=\frac{4}{\phi}\eabc(e^{a}z_d-z^{a}e_d)(e^dz^b-z^de^b)e^c\\
  &=\frac{8}{\phi^2}\eabc e^{a}\dd\f z^be^c-\frac{4}{\f^2}\eabc z^2e^{a}e^be^c
\end{align}
Sin embargo, estos dos términos se pueden relacionar. Para evidenciar esto, escribámoslos en la base coordenada
\begin{align}
  \eabc z^2e^{a}e^be^c&=\eabc z^2e^{a}_\m e^b_\n e^c_\lambda \dd x^\m\wedge\dd x^\n\wedge\dd x^\lambda\\
  &=\eabc e^{d}_\r \partial^\r\phi e^\t_d\partial_t\phi e^{a}_\m e^b_\n e^c_\lambda \dd x^\m\wedge\dd x^\n\wedge\dd x^\lambda\\
  &=\det(e)\ep_{\m\n\lambda}\ep^{\m\n\lambda}\delta ^\t_\r \partial^\r\phi\partial_\t\phi\dd^3x\\
  &=6\det(e)\partial_\m\phi\partial^\m\phi\dd^3x\label{z2eee}
\end{align}
de manera similar,
\begin{align}
  \eabc e^{a}\dd\phi z^be^c&=\eabc e^{a}_\m\partial_\n \phi z^be^c_\lambda \dd x^\m \wedge\dd x^\n\wedge\dd x^\lambda\\
  &=2\det(e)\partial_\m\phi\partial^\m\phi\dd^3x\\
  &=\frac{1}{3}\eabc z^2e^{a}e^be^c
\end{align}
Así
\begin{equation}
  \eabc\Sigma^{a}_{~d}\S^{db}e^c=-\frac{4}{3}\frac{1}{\f^2}\eabc z^2e^{a}e^be^c
\end{equation}
De esta manera \eqref{inv} queda
\begin{align*}
    \epsilon_{abc}\tilde{R}^{ab} \tilde{e}^c&=\phi^2\epsilon_{abc}\left(R^{ab} e^c+\lambda \D \Sigma^{ab} e^c+\lambda^2 \Sigma^{a}_{~d} \Sigma^{db} e^c\right)\\
    &=\f^2\eabc R^{ab}e^c-4\eabc\lambda\dd\f e^{a}z^be^c+4\lambda \f\eabc(T^{a}z^be^c-e^{a}\D z^be^c)-\frac{4\lambda^2}{3}\eabc z^2e^{a}e^be^c\\
    &=\f^2\eabc R^{ab}e^c+\frac{4\lambda}{3}\eabc z^2e^{a}e^be^c+4\lambda \f\eabc(T^{a}z^be^c-e^{a}\D z^be^c)-\frac{4\lambda^2}{3}\eabc z^2e^{a}e^be^c\\
    &=\f^2\eabc R^{ab}e^c+\frac{4\lambda}{3}\eabc z^2e^{a}e^be^c(1-\lambda)+4\lambda \f\eabc(T^{a}z^be^c-e^{a}\D z^be^c)
\end{align*}
Notemos que dado que estas cantidades construiremos una acción, podemos manipular los términos de manera de ir armando términos de borde y hacer que la acción sea quasi-invariante bajo transformaciones conformes. En particular notemos que
\begin{align}
  \f\eabc e^{a}\D z^be^c&=2\det(e)\phi\nabla_\m\partial^\m \f\dd^3x\\
  &=-2\det(e)\nabla_\m\f\partial^\m\f \dd^3x+\rm b.t.\\
 &=-\frac{1}{3}\eabc z^2e^{a}e^be^c+\rm b.t.
\end{align}
Luego,
\begin{align}
  \epsilon_{abc}\tilde{R}^{ab} \tilde{e}^c&=\phi^2\epsilon_{abc}\left(R^{ab} e^c+\lambda \D \Sigma^{ab} e^c+\lambda^2 \Sigma^{a}_{~d} \Sigma^{db} e^c\right)\\
  &=\f^2\eabc R^{ab}e^c+\frac{4\lambda}{3}(2-\lambda)\eabc z^2e^{a}e^be^c+4\lambda \f\eabc T^{a}z^be^c + \rm b.t.\\
  &=\f^2\eabc R^{ab}e^c+\frac{8\lambda}{3}\left(1-\frac{\lambda}{2}\right)\eabc z^2e^{a}e^be^c+4\lambda \f\eabc T^{a}z^be^c + \rm b.t.
\end{align}
Este término es el análogo en $(2+1)$ que el que aparecee en la acción de \cite{Aviles:2024muk}.

%Notemos que podemos armar el término de borde de otra manera. En efecto,
%\begin{align}
%  \eabc \phi^2\D \left(\frac{4}{\phi}e^{a}z^b\right)e^c=\eabc \D(4\phi e^{a}z^be^c)-8\eabc\dd\phi e^{a}z^be^c-4\eabc \phi T^{a}z^be^c
%\end{align}



















Otro invariante que nos podemos construir es
\begin{align}
  \tilde{T}^{a}\wedge\tilde{e}_a&=\left(\phi^2T^{a}+2\phi (\lambda-1)e^{a}\wedge\dd\phi\right)\wedge \left(\phi^2 e_{a}\right)\\
  &=\phi^4 T^{a}\wedge e_a
\end{align}

Otro más
\begin{align}
  \epsilon_{abc}\tilde{e}^{a}\wedge \tilde{e}^b\wedge \tilde{e}^c&=\phi^6\epsilon_{abc}e^{a}\wedge e^b\wedge e^c
\end{align}

El otro invariante que podemos construir en $(2+1)$ es el Chern-Simons a partir de los campos auxiliares . Calculemos
\begin{align}\label{CS}
  \tilde{\omega}^{ab}\dd\tilde{\omega}_{ab}+\frac{2}{3}\tilde{\omega}^{a}_{~b}\tilde{\omega}^{b}_{~c}\tilde{\omega}^{c}_{~a}
\end{align}

Veamos primero cómo queda $\tilde{\omega}^{ab}\dd\tilde{\omega}_{ab}$. Para ello, recordemos que
\begin{equation}
  \Sigma^{ab}=\frac{4}{\phi}e^{[a}z^{b]},\qquad \dd\Sigma^{ab}=-\frac{4}{\phi^2}\dd\phi e^{[a}z^{b]}+\frac{4}{\phi}\dd e^{[a}z^{b]}-\frac{4}{\phi}e^{[a}\dd z^{b]}
\end{equation}
Así,
\begin{align*}
  \tilde{\omega}_{ab}\dd \tilde{\omega}^{ab}&=(\omega_{ab}+\lambda\Sigma_{ab})\wedge \dd (\omega^{ab}+\lambda\Sigma^{ab})\\
  &=\omega_{ab}\dd\omega^{ab}+\lambda \omega_{ab}\left(-\frac{4}{\phi^2}\dd\phi e^{a}z^b+\frac{4}{\phi}\dd e^{a}z^b-\frac{4}{\phi}e^{a}\dd z^b\right)+\frac{4\lambda}{\phi}e^{a}z^b\dd\omega_{ab}+\frac{16\lambda^2}{\phi^2}z^2e_a\dd e^{a}
\end{align*}

Calculemos ahora $\frac{2}{3}\tilde{\omega}^{ab}\tilde{\omega}_{bc}\tilde{\omega}^{cd}\eta_{ca}$. Primero
\begin{align}
  \tilde{\omega}^{ab}\tilde{\omega}_{bc}&=\omega^{ab}\omega_{bc}+\frac{4\lambda}{\phi}\omega^{ab}e_{[b}z_{c]}+\frac{4\lambda}{\phi}e^{[a}z^{b]}\omega_{bc}+\frac{4\lambda^2}{\phi^2}(e^{a}\dd\phi z_c-z^2e^{a}e_c+z^{a}\dd\phi e_c)
\end{align}

Ahora $\tilde{\omega}^{ab}\tilde{\omega}_{bc}\tilde{\omega}^{cd}$,
\begin{align}
  \tilde{\omega}^{ab}\tilde{\omega}_{bc}\tilde{\omega}^{cd}&=\omega^{ab}\omega_{bc}\omega^{cd}+\frac{4\lambda}{\phi}\omega^{ab}\omega_{bc}e^{[c}z^{d]}+\frac{4\lambda}{\phi}\omega^{ab}e_{[b}z_{c]}\omega^{cd}+\frac{16\lambda^2}{\phi^2}\omega^{ab}e_{[b}z_{c]}e^{[c}z^{d]}\\&+\frac{4\lambda}{\phi}e^{[a}z^{b]}\omega_{bc}\omega^{cd}+\frac{16\lambda^2}{\phi^2}e^{[a}z^{b]}\omega_{bc}e^{[c}z^{d]}-\frac{4\lambda^2}{\phi^2}z^2e^{a}e_c\omega^{cd}-\frac{16\lambda^3}{\phi^3}z^2e{a}e_ce^{[c}z^{d]}\\
  &+\frac{4\lambda^2}{\phi^2}z^{a}\dd\phi e_c\omega^{cd}+\frac{16\lambda^3}{\phi^3}z^{a}\dd\phi e_ce^{[c}z^{d]}
\end{align}
Bajando el último índice con una $\eta_{da}$,
\begin{align*}
  \tilde{\omega}^{ab}\tilde{\omega}_{bc}\tilde{\omega}^{c}_{~a}&=\omega^{ab}\omega_{bc}\omega^{c}_{~a}+\frac{2\lambda}{\phi}\omega^{ab}\omega_{bc}(e^cz_a-z^ce_a)+\frac{2\lambda}{\phi}\omega^{ab}(e_bz_c-z_be_c)\omega^c_{~a}+\frac{2\lambda}{\phi}(e^{a}z^b-z^{a}e^b)\omega_{bc}\omega^c_{~a}\\
  &+\frac{4\lambda^2}{\phi^2}\omega^{ab}(2z_ae_b\dd\phi+z^2e_ae_b)+\frac{4\lambda^2}{\phi^2}\omega_{bc}(2z^be^c\dd\phi+z^2e^be^c)+\frac{4\lambda^2}{\phi^2}\omega^c_{~a}(2z_ce^{a}\dd\phi+z^2e_ce^{a})\\
  &=\omega^{ab}\omega_{bc}\omega^{c}_{~a}+\frac{6\lambda}{\phi}\omega^{ab}\omega_{bc}(e^cz_a-z^ce_a)+\frac{12\lambda^2}{\phi^2}\omega^{ab}(2z_ae_b\dd\phi+z^2e_ae_b)
%  &+\frac{4\lambda^2}{\phi^2}\omega^{ab}(e_b\dd\phi z_a-z^2e_be_a+z_b\dd\phi e_a)\\&+\frac{4\lambda^2}{\phi^2}(e^{a}z^b\omega_{bc}e^cz_a-e^{a}z^b\omega_{bc}z^ce_a-z^{a}e^b\omega_{bc}e^cz_a+z^{a}e^b\omega_{bc}z^ce_a)+\frac{4\lambda^2}{\phi^2}e^{a}\dd\phi z_c\omega^{c}_{~a}\\
%  &+\frac{8\lambda^2}{\phi^3}e^{a}\dd\phi z_c(e^cz_a-z^ce_a)-\frac{4\lambda^2}{\phi^2}z^2e^{a}e_c\omega^{c}_{~a}-\frac{16\lambda^3}{\phi^3}z^2e^{a}e_c(e^cz_a-z^ce_a)+\frac{4\lambda^2}{\phi^2}z^{a}\dd\phi e_c\omega^{c}_{~a}\\
%  &+\frac{16\lambda^3}{\phi^3}z^{a}\dd\phi e_c(e^cz_a-z^ce_a)
\end{align*}
Luego, 
\begin{align*}
    \tilde{\omega}^{ab}\dd\tilde{\omega}_{ab}&+\frac{2}{3}\tilde{\omega}^{a}_{~b}\tilde{\omega}^{b}_{~c}\tilde{\omega}^{c}_{~a}=\omega_{ab}\dd\omega^{ab}+\lambda \omega_{ab}\left(-\frac{4}{\phi^2}\dd\phi e^{a}z^b+\frac{4}{\phi}\dd e^{a}z^b-\frac{4}{\phi}e^{a}\dd z^b\right)+\frac{4\lambda}{\phi}e^{a}z^b\dd\omega_{ab}+\frac{16\lambda^2}{\phi^2}z^2e_a\dd e^{a}\\
    &+\frac{2}{3}\omega^{ab}\omega_{bc}\omega^{c}_{~a}+\frac{4\lambda}{\phi}\omega^{ab}\omega_{bc}(e^cz_a-z^ce_a)+\frac{8\lambda^2}{\phi^2}\omega^{ab}(2z_ae_b\dd\phi+z^2e_ae_b)
%    &+\frac{2}{3}\omega^{ab}\omega_{bc}\omega^{c}_{~a}+\frac{4\lambda}{3\phi}\omega^{ab}\omega_{bc}(e^cz_a-z^ce_a)+\frac{4\lambda}{3\phi}\omega^{ab}(e_bz_c-z_be_c)\omega^c_{~a}+\frac{4\lambda}{3\phi}(e^{a}z^b-z^{a}e^b)\omega_{bc}\omega^c_{~a}\\
%    &+\frac{16\lambda^2}{3\phi^2}\omega^{ab}(2z_ae_b\dd\phi+z^2e_ae_b)+\frac{16\lambda^2}{3\phi^2}\omega_{bc}(2z^be^c\dd\phi+z^2e^be^c)+\frac{16\lambda^2}{3\phi^2}\omega^c_{~a}(2z_ce^{a}\dd\phi+z^2e_ce^{a})
\end{align*}

































































































