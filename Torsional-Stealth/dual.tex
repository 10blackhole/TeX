\section{Bichos en el dual del grupo}
Una propiedad interesante que aparece en $(2+1)$ es que podemos simplificar un poco lo cálculos trabajando con la conexión de Lorentz (y los demás objetos) en el dual del grupo. 

La convención a seguir será
\begin{align}
  \epsilon_{012}&=-\epsilon^{012}=1\\
  \epsilon_{abc}\epsilon^{def}&=-\delta_{abc}^{def}\\
  \omega_{ab}&=\epsilon_{abc}\omega^c\label{omega dual}
\end{align}
Multiplicando \eqref{omega dual} por $\epsilon^{abf}$, tenemos
\begin{align}
  \epsilon^{abf}\omega_{ab}&=\epsilon^{abf}\epsilon_{abc}\omega^c\\
  \epsilon^{abf}\omega_{ab}&=-\delta_{abc}^{abf}\omega^c\\\
  \epsilon^{abf}\omega_{ab}&=-2!\delta^f_c\omega^c\\
  \epsilon^{abf}\omega_{ab}&=-2\omega^f
\end{align}
Así, 
\begin{equation}
  \omega^{f}=-\frac{1}{2}\epsilon^{abf}\omega_{ab}
\end{equation}
o de manera equivalente
\begin{equation}
  \boxed{\omega_a=\frac{1}{2}\epsilon_{abc}\omega^{bc}}
\end{equation}

Veamos como queda la torsión escrita en términos de estas conexiones duales,
\begin{align}
  T^{a}&=\dd e^{a}+\omega^{a}_{~b}\wedge e^b\\
  &=\dd e^{a}+\omega^{ab}\wedge e_b\\
  &=\dd e^{a}-\epsilon^{abc}\omega_c\wedge e_b\\
  &=\dd e^{a}-\epsilon^{acb}\omega_b\wedge e_c\\
  &=\dd e^{a}+\epsilon^{abc} \omega_b\wedge e_c
\end{align}
Es decir,
\begin{equation}
	\boxed{T^{a}=\dd e^{a}+\epsilon^{abc} \omega_b\wedge e_c}
\end{equation}

Veamos ahora cómo queda la curvatura de Lorentz en el dual de grupo $R_{a}=\frac{1}{2}\epsilon_{abc}R^{bc}$. Notemos que
\begin{align}
  R^{bc}&=\dd\omega^{bc}+\omega^b_{~d}\wedge \omega^{dc}\\
  &=\dd (-\epsilon^{bcf}\omega_f)+\eta^{bg}\omega_{gd}\wedge \omega^{bc}\\
  &=-\epsilon^{bcf}\dd\omega_f-\eta^{bg}\epsilon_{gdm}\omega^m\wedge (-\epsilon^{dcl}\omega_l)\\
  &=-\epsilon^{bcf}\dd\omega_f-\eta^{bg}\epsilon_{gdm}\epsilon^{dcl}\omega^m\wedge \omega_l\\
  &=-\epsilon^{bcf}\dd\omega_f+\eta^{bg}\epsilon_{dgm}\epsilon^{dcl}\omega^m\wedge \omega_l\\
  &=-\epsilon^{bcf}\dd\omega_f+\eta^{bg}\delta_{gm}^{cl}\omega^m\wedge\omega_l\\
  &=-\epsilon^{bcf}\dd\omega_f+\cancel{\eta^{bg}\delta_g^c\delta_m^l\omega^m\wedge \omega_l}-\eta^{bg}\delta_g^l\delta_m^c\omega^m\wedge\omega_l\\
  &=-\epsilon^{bcf}\dd\omega_f-\omega^c\wedge\omega^b\\
  &=-\epsilon^{bcf}\dd\omega_f+\omega^b\wedge\omega^c
\end{align}
Luego,
\begin{align}
  \epsilon_{abc}R^{bc}&=\epsilon_{abc}\left(-\epsilon^{bcf}\dd\omega_f+\omega^b\wedge\omega^c\right)\\
  &=-\epsilon_{abc}\epsilon^{bcf}\dd\omega_f+\epsilon_{abc}\omega^b\wedge \omega^c\\
  &=-\epsilon_{abc}\epsilon^{fbc}\dd\omega_f+\epsilon_{abc}\omega^b\wedge \omega^c\\
  &=\delta_{abc}^{bfc}\dd\omega_f+\epsilon_{abc}\omega^b\wedge\omega^c\\
  &=2\dd\omega_a+\epsilon_{abc}\omega^b\wedge\omega^c
\end{align}
Así,
\begin{align}
    R_{a}&=\frac{1}{2}\epsilon_{abc}R^{bc}\\
    &=\dd\omega_a+\frac{1}{2}\epsilon_{abc}\omega^b\wedge \omega^c
\end{align}
\begin{equation}
\boxed{R^{a}=\dd\omega^{a}+\frac{1}{2}\epsilon^{abc}\omega_b\wedge \omega_c}
\end{equation}

