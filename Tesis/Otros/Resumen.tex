\documentclass[../Main.tex]{subfiles}
\begin{document}

En esta tesis, estudiamos el método de renormalización conforme aplicado en teorías con grados de libertad más allá de los métricos. Específicamente, revisamos este método en presencia de un campo escalar. Para ello, a modo de revisión, repasamos el principio de acción de Relatividad General y las ecuaciones de Einstein, además de revisitar las condiciones para que esta teoría tenga un principio variacional bien definido cuando imponemos condiciones de borde tipo Dirichlet. Luego, examinamos diversos métodos para calcular cargas conservadas en espacios asintóticamente planos. En espacios asintóticamente anti-de Sitter, estudiamos dos esquemas de renormalización los cuales son relevantes para este trabajo. Con el fin de motivar el método aquí empleado, veremos que la teoría de Gravedad Conforme es finita para espacios que son asintóticamente anti-de Sitter, tal como se demostró en \cite{Grumiller_2014}; un principio guía que nos da una pista de cómo la simetría conforme estaría relacionada con la renormalización en espacio-tiempos con dicha asíntota. La base de esta construcción es la extensión de un tensor covariante bajo rescalamientos de Weyl compuesto de la métrica y del campo escalar propuesto en la \cite{Oliva:2011np}. Esta extensión hace que el peso conforme de dicho tensor sea igual al del tensor de Weyl. Extendemos esta realización revisando teorías tensor-escalar que gozan de simetría conforme, acopladas con la acción de Einstein-AdS escrita en la forma de MacDowell-Mansouri. A pesar de que el sector de Einstein-AdS rompe la simetría conforme, mostramos que la teoría completa aún puede ser renormalizada si el campo escalar tiene un decaimiento adecuado cuando se consideran soluciones asintóticamente anti-de Sitter. Finalmente, estudiamos las soluciones tipo agujero negro, calculando su temperatura de Hawking y la acción Euclídea on-shell, mostrando explícitamente que esta última es finita para espacios asintóticamente anti-de Sitter.


\par\vspace*{\fill} % Mueve las palabras clave al final de la página
% \textbf{\textit{Keywords --}} Template, NHH, master thesis, LaTeX %Agregar todas las palabras claves asosciadas con la tesis aquí.

%-----------Si se desea poner el Abstract Des-comentar lo siguiente-----------
\newpage
\addcontentsline{toc}{chapter}{Abstract} %Agrega esta sección al índice
\section*{Abstract}

In this thesis, we investigate the method of conformal renormalization applied to theories with degrees of freedom beyond the metric ones. Specifically, we examine this method in the presence of a scalar field. To do this, as part of a review, we revisit the action principle of General Relativity and Einstein's equations, in addition to re-examining the conditions for this theory to have a well-defined variational principle when Dirichlet boundary conditions are imposed. We then explore various methods for calculating conserved charges in asymptotically flat spaces. In asymptotically anti-de Sitter spaces, we study two renormalization schemes that are relevant to this work. To motivate the method used here, we observe that Conformal Gravity is finite for spaces that are asymptotically anti-de Sitter, as demonstrated in \cite{Grumiller_2014}. This guiding principle gives us a clue as to how conformal symmetry may be related to renormalization in spacetimes with such asymptotics. The basis of this construction is the extension of a covariant tensor under Weyl rescalings composed of the metric and the scalar field as proposed in \cite{Oliva:2011np}. This extension ensures that the conformal weight of this tensor is equal to that of the Weyl tensor. We extend this realization by considering tensor-scalar theories with conformal symmetry, coupled with the Einstein-AdS action written in the MacDowell-Mansouri form. Despite the fact that the Einstein-AdS sector breaks conformal symmetry, we show that the entire theory can still be renormalized if the scalar field has an appropriate decay when considering asymptotically anti-de Sitter solutions. Finally, we study black hole-type solutions, calculating their Hawking temperature and the Euclidean on-shell action, explicitly demonstrating that the latter is finite for asymptotically anti-de Sitter spaces.

\par\vspace*{\fill} % Mueve las palabras clave al final de la página
\textbf{\textit{Keywords --}} General Relativity, Master Thesis, Black Holes, Renormalization % Agregar las palabras claves en inglés

\biblio %Se necesita para referenciar cuando se compilan subarchivos individuales - NO SACAR
\end{document}