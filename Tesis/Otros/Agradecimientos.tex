\documentclass[../Main.tex]{subfiles}
\begin{document}

\vspace*{2\baselineskip}

Sinceramente creo que esta es la sección mas difícil de escribir, así que tratare de hacerlo de la mejor forma posible y no olvidar a nadie en el camino. En primer lugar, deseo expresar mi profundo agradecimiento a mi familia,  quienes han sido mi fuente inagotable de amor, apoyo y aliento a lo largo de esta travesía académica. A mis padres, Myriam Barrientos, mejor conocida como la Gordita y Elio Bunsego aka Gordin, quienes me han inculcado valores de perseverancia y dedicación, y me han brindado el respaldo incondicional en cada paso que he dado. A mi hermana, Nathalia por ser mi compañía y preocupación constante y por su confianza en mis capacidades. A Bastian Ayala, mi pareja, eres una inspiración constante para mi y un pilar fundamental en mi vida. Sin su constante apoyo emocional, realmente esta meta y tantas otras no habría sido posible. 

Asimismo, quisiera agradecer a mis amigos  y compañeros quienes han sido mi red de apoyo y aliento durante los momentos de estrés y desafíos. La lista es enorme, pero quisiera ser enfática también en esas personas que se tomaron el tiempo de enviarme memes y canciones, Gabriel Arenas, Camilo Nuñez, les debo el soundtrack de este logro.  


No puedo dejar de reconocer a mis profesores, cuya sabiduría, conocimientos han sido invaluables en el desarrollo de mi tesis. Agradezco especialmente a Cristóbal Corral, Nelson Merino, Ignacio Araya y Giorgos Anastasiou por su dedicación y paciencia al dirigir este proyecto. Sus comentarios y sugerencias críticas han sido fundamentales para enriquecer mi trabajo y llevarlo a un nivel superior. Quisiera mencionar que siento una gratitud inmensa a la familia que se ha formado en Iquique, como primera experiencia de postgrado, Carlangas, Luchito, Luis, los profesores por ustedes fue como sentirse en casa. También me gustaría agradecer a Rodrigo Aros, quien ha sido una guía fundamental desde mi pregrado y sigue creyendo en mi como desde el primer día en que acepto ser mi tutor en la licenciatura en física. 

Además, quiero dar las gracias a las personas que me brindaron su apoyo en los cálculos e interpretaciones de este trabajo. Agradezco a Nicolás Cáceres, Gonzalo Barriga, Francisco Colipí, Rodrigo Olea, Leonardo Sanhueza y Kristiansen Lara, por su asesoramiento y su disposición para responder mis preguntas y su habilidad para simplificar conceptos complejos fueron esenciales para llevar a cabo esta investigación.

Finalmente, deseo agradecer a todas las personas que, de una forma u otra, contribuyeron a este logro. A aquellos que he conocido en escuelas, conferencias y workshops, es lindo saber que la comunidad de la física teórica se esta formando con gente tan dedicada, amable y unida, donde todos se ayudan a crecer como académicos y personas. 


\vspace*{3\baselineskip}





\end{document}