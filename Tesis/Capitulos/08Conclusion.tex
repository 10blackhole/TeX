\documentclass[../Main.tex]{subfiles}
\begin{document}
En esta tesis se estudió cómo la renormalización de las teorías de gravedad escalar-tensor no mínimamente acopladas está dictada por la restauración de la simetría conforme on-shell en el volumen. Consideramos el caso de un campo escalar acoplado conformemente con un potencial cuártico, cuya teoría produce un término de frontera al realizar una transformación de Weyl. Entonces, la simetría conforme se restaura escribiendo el término cinético del escalar de manera no canónica, de tal manera que la acción pueda escribirse en términos del operador Yamabe. La acción resultante es invariante de Weyl, excepto por la configuración escalar constante, donde la transformación conforme se vuelve singular, lo que se evidencia en el hecho de que la acción es divergente para espacio-tiempos AlAdS. La restauración de la simetría de reescalamiento local se logra definiendo el tensor $\Sigma^{\mu \nu}_{\rho \lambda}$, en términos de la métrica y los grados de libertad escalares, de tal manera que sea covariante conforme y tenga el mismo peso conforme que el tensor de Weyl. Entonces, $\Sigma$ al cuadrado es un invariante conformal local de la teoría, que se puede usar para definir la acción de tal manera que la teoría se renormalice y se restaure la invariancia conforme completa, lo que implica que la variación de Weyl se anula exactamente y no hay puntos singulares en la transformación. Se demostró que la acción resultante se renormaliza on-shell, de tal manera que cualquier configuración de campo que satisfaga las EOM de la teoría tenga una acción finita al considerar espacio-tiempos AlAdS con condiciones asintóticas AdS débiles. Además, se observa que el campo escalar no se puede eliminar con una transformación conforme sin modificar el comportamiento de la frontera fuera de la condición AlAdS, lo que cambiaría el estado físico.

También se estudió la teoría que considera la gravedad Einstein-AdS renormalizada escrita en forma de MacDowell-Mansouri en presencia del campo escalar acoplado conformemente renormalizado. En ese caso, la teoría no tiene invariancia conforme debido al sector de Einstein-Hilbert. Sin embargo, para espacios Einstein y campos escalares constantes, se vuelve invariante conforme on-shell. En este caso, la teoría se renormaliza para espaciotiempos Einstein, ya que la acción completa se vuelve proporcional a Weyl al cuadrado. Otras configuraciones que también tienen una acción finita son aquellas con métricas AlAdS tales que sus grados de libertad no Einstein, codificados en el tensor de Ricci sin traza, disminuyen lo suficientemente rápido hacia la frontera conforme. El agujero negro MSTZ~\cite{Martinez:2002ru,Martinez:2005di} pertenece a esta categoría y se calculó el valor de la acción on-shell y la correspondiente temperatura del agujero negro.

Luego, se realizó el mismo análisis en gravedad conforme más campos escalares acoplados conformemente renormalizados, lo cual corresponde a la acción localmente conforme invariante más general construida a partir de contracciones antisimétricas cuadráticas de tensores covariantes de Weyl. De hecho, se demostró explícitamente que esta teoría está renormalizada para espacios de Bach planos, que por virtud de la EOM también son configuraciones stealths. Para este tipo de espacios, la acción está renormalizada para espacios de AlAdS ya que es proporcional a la acción CG. Como ejemplo particular, se consideraron configuraciones stealths sobre la métrica de Riegert y se demostró que su acción Euclidiana on-shell es finita.

En ambas teorías, existen puntos interesantes en el espacio de parámetros donde la acción se puede volver trivial para ciertos tipos de configuraciones métricas y escalares. En particular, para la teoría que incluye gravedad conforme, la acción se anula en el punto crítico de $\alpha = \zeta$ para configuraciones de Bach planas. De manera análoga, la teoría que incluye un sector de Einstein-AdS tiene una acción trivial en el punto crítico de $\alpha = \zeta$ para espacios de Einstein, que corresponden a soluciones a la EOM con un valor constante del campo escalar.

El hecho de que la acción sea trivial implica que todas las cargas asintóticas se anulan de manera idéntica, así como el potencial termodinámico que es proporcional a la acción en la sección Euclidiana. Esto sugiere una novedosa noción de criticidad en teorías escalar-tensoriales, que es diferente de la definición estándar formulada de manera perturbativa en términos de un desacoplamiento de los modos masivos de la métrica del espectro~\cite{Lu:2011zk}. Esta noción de criticidad termodinámica corresponde a puntos en el espacio de la teoría donde el estado fundamental de la teoría se agranda, de tal manera que la configuración máximamente simétrica tiene la misma energía libre que toda una clase de soluciones. De esta manera, forman un espacio de móduli de configuraciones de vacío, que permitirían transiciones espontáneas entre ellas sin costo de energía libre. Esta idea ya fue discutida en el caso de teorías de gravedad pura, para la gravedad de Einstein-AdS en 4D y 6D en las Refs.~\cite{Anastasiou:2016jix,Anastasiou:2017rjf,Anastasiou:2021tlv}. Aunque este punto es muy interesante, su análisis completo requiere un estudio cuidadoso de la termodinámica.



\biblio %Se necesita para referenciar cuando se compilan subarchivos individuales - NO SACAR
\end{document}