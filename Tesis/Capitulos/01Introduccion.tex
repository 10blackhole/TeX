\documentclass[../Main.tex]{subfiles}
\begin{document}

La renormalización de las teorías gravitacionales que admiten soluciones asintóticamente localmente anti-de Sitter (AlAdS) se ha convertido en un ingrediente crucial en la termodinámica de agujeros negros y en la correspondencia anti-de Sitter/teorías de campo conforme (AdS/CFT). La prescripción estándar ---denominada renormalización holográfica (HR)--- consiste en agregar el término de Gibbons-Hawking-York (GHY) para fijar el principio variacional de Dirichlet para la métrica inducida en la foliación radial y luego introducir contratérminos de borde intrínsecos para anular las divergencias que aparecen al evaluar la acción en soluciones con comportamiento AlAdS. Esto hace que la acción Euclídea on-shell y las cargas asintóticas sean finitas~\cite{Henningson:1998gx,Balasubramanian:1999re,Chamblin:1999tk,Emparan:1999pm,Chamblin:1999hg,Nojiri:1999mh,deHaro:2000vlm,Bianchi:2001kw,Skenderis:2002wp}, lo que permite definir el funcional generador para correladores de la CFT dual y obtener los datos holográficos de dicha teoría~\cite{Gubser:1998bc,Witten:1998qj}. 

Una observación interesante, que se hizo en Ref.~\cite{Papadimitriou:2005ii}, es que imponer la condición de Dirichlet para la métrica inducida en el borde de AdS está mal definido debido a que el elemento de volumen del espacio de AdS diverge. Además, fijar la condición de Dirichlet para la fuente holográfica asintóticamente del borde no requiere fijar la condición de Dirichlet para la métrica intrínseca, ya que tanto la curvatura intrínseca como la extrínseca admiten una expansión de de Fefferman-Graham (FG ) en la fuente holográfica~\cite{AST_1985__S131__95_0,Graham:1999jg}. Esto condujo al desarrollo de la prescripción de Kounterterms en Ref.~\cite{Olea:2006vd}, donde la renormalización se logra mediante la adición de un término de borde adecuado, que depende de las curvaturas tanto extrínseca como intrínseca del borde en una forma cerrada. Incluso para dimensiones pares en el volumen, el Kounterterm es la forma de Chern, que es el término de borde que aparece en el teorema de Euler. Entonces, la renormalización también se puede lograr agregando la densidad de Euler en el volumen con un acoplamiento fijo, de modo que cancela la divergencia de la configuración maximalmente simétrica (AdS global). En el caso de dimensiones de volumen impares, existe un procedimiento similar donde el término de frontera es el término de contacto de la forma de transgresión del grupo AdS~\cite{Mora:2006ka}. En ese caso, la segunda conexión de gauge describe una variedad de producto que comparte el mismo borde que la variedad dinámica.

En particular, como se muestra en Ref.~\cite{Miskovic:2009bm}, la acción de Einstein-AdS parcialmente renormalizada con Kounterterms en 4D se puede escribir en la forma de McDowell-Mansouri para el grupo de AdS~\cite{Stelle:1976gc,MacDowell:1977jt}. En el caso de los espacio-tiempos tipo Einstein-AdS, este última se puede escribir en términos del tensor de Weyl al cuadrado y la acción on-shell se convierte en la de Conformal Gravity (CG), que es la única invariante conforme local en 4D. Esta reescritura de la teoría de Einstein encajándola en CG es consistente a nivel de las ecuación de movimiento (EOM), ya que todos los espacio-tiempos tipo Einstein pertenecen al espacio de solución de la teoría, cuya EOM está dada por la condición de que el tensor de Bach es igual a cero. Dado que el conjunto de soluciones de CG contiene a los espacios tipo Einstein, la acción correspondiente se puede separar explícitamente en una parte de MacDowell-Mansouri más términos que se anulan para los espacios tipo Einstein~\cite{Anastasiou:2016jix}.

Para las variedades con asintótica de AdS tenue, se demostró que la acción de CG es finita off-shell, incluso para espacios-tiempos que no tienen un tensor de Bach que se anula~\cite{Grumiller:2013mxa}. Por lo tanto, la finitud de la acción de McDowell-Mansouri para la gravedad de Einstein-AdS se sigue de inmediato, ya que las dos acciones son equivalentes para los espacios tipo Einstein. Este fue el primer ejemplo en el que la incorporación de una teoría de gravedad en otra con invariancia conforme local en el volumen permitió obtener la forma renormalizada de la acción. Más tarde, en la Ref.~\cite{Anastasiou:2020mik}, se generalizó el mismo procedimiento para la gravedad de Einstein-AdS en 6D, siendo esta inmersa en la única acción de CG en 6D que admite espacios tipo Einstein como soluciones, construida en Ref.~\cite {Lu:2011zk}. Es importante enfatizar que, aunque la incorporación de Einstein-AdS en CG en 4D da el mismo principio de acción que los Kounterterms, esto no es cierto en 6D. Como se discutió en Ref.~\cite{Anastasiou:2020zwc}, en el caso 6D, la acción topológicamente renormalizada cancela todas las divergencias solo para espacios AlAdS con un borde conformemente plano. En el caso genérico, la última prescripción no cancela una divergencia del borde que depende del cuadrado de Weyl de la variedad del borde. Sin embargo, la inmersión en la teoría de CG en 6D reproduce correctamente todos los términos necesarios para lograr la renormalización, de modo que la acción obtenida es totalmente equivalente a la dada por HR, hasta el orden normalizable. Así, es esta prescripción la que da la renormalización correcta, generalizando los Kounterterms más allá del requisito de que la 
 variedad sea conformalmente plana en el borde.


Más allá de las teorías puramente métricas, se han considerado enfoques de renormalización para casos con grados de libertad adicionales, e.g. campos escalares. De hecho, en el caso de renormalizacion holográfica, los contratérminos para espacios AlAdS en las teorías escalar-tensor se han discutido en Refs.~\cite{Nojiri:1998dh,Padilla:2012ze,Caldarelli:2016nni,Liu:2017kml,Li:2018rgn,Agurto-Sepulveda:2022vvf}. Por lo tanto, una pregunta natural es si el uso de la simetría conforme local en el volumen se puede generalizar para estas teorías, y de esa forma determinar los términos de renormalización. Aquí, abordamos este tema y construimos acciones de gravedad renormalizadas que poseen un sector escalar-tensor acoplado conforme, cuyas soluciones han sido estudiadas en la literatura. Por lo tanto, esto constituye la primera aplicación de la idea de \emph{Renormalización Conforme} a las teorías escalar-tensor de la gravedad.


\biblio %Se necesita para referenciar cuando se compilan subarchivos individuales - NO SACAR
\end{document}