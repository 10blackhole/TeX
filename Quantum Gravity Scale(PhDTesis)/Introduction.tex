\thispagestyle{simple}

The quest for a unified theory of fundamental interactions remains one of the most ambitious endeavors in theoretical physics. At the heart of this pursuit lies the challenge of finding a fully-fledged theory of quantum gravitational interactions, which of course reduces to General Relativity \cite{EinsteinGR} for low enough energies --- equivalently curvatures, but also crucially resolves the pathologies found therein, such as the endemic presence of spacetime singularities (e.g., black holes). Over the last fifty years or so, remarkable progress has been made in understanding the quantum nature of spacetime, thanks in part to the (still ongoing) development of string theory, here understood as the richer set of ideas which incorporate the quantum physics of extended objects (e.g, strings, D-branes, etc.), as well as make manifest certain non-trivial features of quantum gravity such as the concept of holography \cite{Maldacena:1997re,Witten:1998qj}. However, despite the enormous successes of string theory in uncovering new phenomena both in gravity as well as in (supersymmetric) field theory, the challenge remains to reproduce the physics observed in particle accelerators, where gravity plays no important role. This has led to various puzzles over the years, since it seems that one can get a priori as many as $10^{272000}$ inequivalent 4d vacua directly from string theory constructions \cite{Taylor:2015xtz}, thus suggesting a poor predictive power of the theory. In this regard, a key role has been played by the realization that in fact, the amount of consistent vacua in string theory (more generally in quantum gravity) comprises a set of measure zero within the complete set of possibilities that are allowed by field theory arguments such as gauge anomaly cancellation, etc. This has led to the idea of the Swampland program \cite{Vafa:2005ui}, which aims to find what are the quantum gravity consistency conditions that arise purely in gravitational theories and must be satisfied by any effective field theory weakly coupled to Einstein gravity. The present thesis aims to explore and extend these recent developments, focusing on a particular quantity that seems to play an starring role within this story: The quantum gravity cut-off.

\subsubsection*{Current status of high energy physics}

High energy physics has undergone significant advancements over the last decades, fueled by both experimental discoveries and theoretical developments. The discovery of the Higgs boson \cite{ATLAS:2012yve, CMS:2012qbp} at the Large Hadron Collider (LHC) in 2012 provided the final piece of the Standard Model, confirming our understanding of particle physics up to energies of approximately 14 TeV. Despite this triumph, the Standard Model leaves several profound questions unanswered, such as the nature of dark matter/energy \cite{Planck:2015fie}, the origin of neutrino masses \cite{Super-Kamiokande:1998uiq,SNO:2002tuh,KamLAND:2002uet}, and the hierarchy/cosmological constant problems \cite{Weinberg:1988cp}.

In parallel, the detection of gravitational waves by the LIGO and Virgo collaborations \cite{LIGO} has opened a new observational window into the cosmos. These ripples in spacetime, predicted by the classical theory of gravity \cite{GW1,GW2}, have provided crucial insights and precision tests concerning the dynamics of strongly-curved spacetimes, such as black holes and neutron stars. The first direct observation of a binary black hole merger in 2015 and the subsequent detection of numerous gravitational wave events  provide us with new experimental tools that can potentially revolutionize our understanding of strong-field gravity and compact astrophysical objects. Furthermore, the observation of gravitational waves has not only confirmed general relativity in extreme curvature regimes but has also spurred a deeper investigation into the nature of black holes. The Event Horizon Telescope's image of the M87 black hole's shadow in 2019 \cite{EventHorizonTelescope:2019dse} provided a direct visual confirmation of black hole horizons. These observations challenge us to understand at a deeper level how the classical descriptions of these objects connect with the quantum theory, especially in relation with outstanding questions such as the origin of black hole entropy \cite{Bekenstein:1972tm,Hawking:1975vcx} or the information paradox \cite{hawking2015information}.

String theory, positing that fundamental particles are not point-like but rather one-dimensional strings, remains the leading candidate for a quantum theory of gravity. It naturally incorporates a spin-2 massless particle mediating long-range gravitational interactions, and moreover unifies it with other fundamental forces within a consistent quantum framework. Hence, it predicts a rich spectrum of particles and suggests the existence of extra dimensions, which could have profound implications for our understanding of the Universe at the smallest distance scales in case they are experimentally confirmed.

%The Swampland program has emerged as a significant paradigm within string theory and quantum gravity. It seeks to delineate the landscape of consistent low-energy effective theories that can arise from string theory, distinguishing them from those that cannot, known as the "swampland." This approach has led to various conjectures, such as the Distance Conjecture and the Weak Gravity Conjecture, which impose constraints on the field space and gauge interactions in effective field theories.

\subsubsection*{A golden era for quantum gravity}

Importantly, it seems that we are living now in a particularly auspicious time for studying the (hard) problem of quantum gravity. The convergence of theoretical advancements and experimental discoveries mentioned before has created a fertile ground for new insights. The ability to observe and measure gravitational waves and black hole phenomena provides empirical data that can inform and constrain theoretical models. On the other hand, the rapidly growing set of knowledge gathered from the string theory point of view offers a robust framework for understanding the quantum aspects of gravity.

This thesis aims to delve into the role of the maximum energy cut-off that effective field theory descriptions of gravity at low energies can have, with a particular emphasis on its connections to the Swampland program. The former must be understood as the energy scale beyond which local effective field theory breaks down due to the appearance of purely quantum gravitational effects, such as the presence of strings or extra dimensions. Therefore, by examining various recents developments in the theoretical front, we seek to understand the limits of effective field theories weakly coulpled to gravity and explore the universal properties that this quantum gravity cut-off can exhibit. Through this investigation, we hope that we can contribute to the ongoing efforts to uncover the fundamental nature of spacetime and gravity.

In summary, the current era in high energy physics is marked by a synergy between theory and experiment that is driving forward our understanding of the Universe at its most fundamental level. This thesis is positioned within this vibrant landscape, aiming to humbly contribute to some of the most pressing questions in the field.

\subsubsection*{Plan of the Thesis}

To be concrete, in this thesis we will consider different string theory compactifications preserving 32, 16 or 8 supercharges in dimensions ranging from ten to four. In particular, the rest of Part \ref{partI} includes a detailed review on the basic aspects and ingredients that the aforementioned string theory constructions present, placing special emphasis on the two-derivative dynamics as well as the massive (non-perturbative) content of the theories. We also draw some deep connections between the latter that are captured by the phenomenon of (string) dualities, which will be used at many instances in the thesis. To finish, we briefly introduce the Swampland program, focusing on the conjectures that will play a major role on the rest of this work, namely the Distance  \cite{Ooguri:2006in} and Weak Gravity conjectures \cite{Heidenreich:2015nta, Heidenreich:2016aqi, Montero:2016tif, Andriolo:2018lvp}. 

The bulk of the results reported in this thesis are contained in Parts \ref{part:QGscale}-\ref{part:pattern}. Hence, in Chapter \ref{ch:SpeciesIntro} we introduce and discuss in detail the concept of the quantum gravity cut-off. We first explain what are the basic expectations coming from the non-renormalizable character of General Relativity, so as to later confront this intuition with several ideas that are believed to play a fundamental role in quantum gravity, such as the holographic principle. This leads us instead to propose this quantity to be given by a seemingly different energy scale, usually denoted as the species cut-off \cite{Han:2004wt, Dvali:2007hz, Dvali:2007wp}. Interestingly, we review and extend several perturbative and non-perturbative arguments pointing toward the species scale as encapsulating the minimum length-scale describable by any effective field theory weakly coupled to gravity. Moreover, this turns out to be in agreement with our familiar intuition based on theories of extra dimensions and string theory itself, where the UV cut-off is given either by the higher-dimensional Planck mass or the fundamental string scale. Many of the discussions presented in this part of the thesis build on material already existing in the literature, whilst the new contributions are based on the publications \cite{Castellano:2022bvr,Castellano:2023aum}.

Part \ref{part:StringTheoryTests} is devoted to a careful study and application of the ideas introduced in Part \ref{part:QGscale} within the context of string theory. In particular, in Chapter \ref{ch:Higherdimops} and using a large set of the string compactifications described in Chapter \ref{ch:reviewstringtheory}, we test whether the species scale indeed arises as the ultra-violet cut-off in gravity. This is signalled by the appearance of the latter as the energy scale controlling the EFT expansion of higher-dimensional and higher-curvature operators in the low energy EFT. Crucially, we find perfect agreement with the expectations based on our discussion from Part \ref{part:QGscale}. In addition, we also investigate in Chapter \ref{ch:Emergence} the precise role of the species cut-off within certain conjectural criteria proposed in the Swampland program. More precisely, we analyze how the Emergence mechanism \cite{Harlow:2015lma, Grimm:2018ohb, Heidenreich:2018kpg, Palti:2019pca} is realized in string theory, which hinges on the precise identification of the gravitational cut-off in the EFT with the species scale. The material presented in this part of the thesis is based on earlier publications by the author \cite{Castellano:2022bvr,Castellano:2023aum}.

In Part \ref{part:pattern} of the thesis we turn our attention to finding universal constraints and patterns concerning the species cut-off close to infinite distance boundaries in field space. In particular, in Chapter \ref{ch:bounds} we introduce and discuss certain lower bound on the exponential decay rate that the species scale seems to satisfy along any infinite distance trajectory in moduli space. This non-trivial constraint can be reformulated as a convex hull condition and indeed exhibits lots of geometric structure which is deeply rooted in the duality properties of the theories under consideration. Later on, in Chapter \ref{ch:pattern} we follow up on these ideas and present another seemingly universal pattern that relates the aforementioned decay rates of the species and the (lightest) tower mass scales. This latter property is seen to be satisfied in all up to now explored supersymmetric vacua in string theory, being moreover intimately related (although not completely equivalent) to the Emergent String Conjecture \cite{Lee:2019wij}. The material of these chapters builds on earlier results by the author contained in references \cite{Calderon-Infante:2023ler, Castellano:2023jjt, PhysRevLett.132.181601}.

Finally, in Part \ref{part:Conclussions} we draw some general conclusions that follow naturally from the work presented here, whereas in Part \ref{part:appendices} several technical details which are relevant for the analysis in the bulk of the thesis are presented.


\selectlanguage{spanish}
\chapter*{Introducción}	\thispagestyle{simple}


La búsqueda de una teoría unificada de las interacciones fundamentales sigue siendo uno de los esfuerzos más ambiciosos dentro del marco de la física teórica. En el corazón de esta búsqueda se hallaría el desafío de encontrar una teoría completa de las interacciones gravitacionales a nivel cuántico, que por supuesto se reduzca a la ya conocida Relatividad General \cite{EinsteinGR} para energías suficientemente bajas, pero que también resuelva crucialmente toda patología presente en la misma, como la inevitable presencia de singularidades en el espacio-tiempo (por ejemplo, agujeros negros). Durante los últimos cincuenta años, se ha logrado un progreso notable en la comprensión de la naturaleza cuántica del espacio-tiempo, gracias en parte al desarrollo (aún en curso) de la teoría de cuerdas, entendida globalmente como el conjunto de ideas que incorporan la física de objetos extendidos (por ejemplo cuerdas, D-branas, etc.); así como manifestar ciertas características no triviales de la gravedad cuántica, como el concepto de holografía \cite{Maldacena:1997re, Witten:1998qj}. Sin embargo, a pesar del enorme éxito de la teoría de cuerdas al descubrir nuevos fenómenos tanto en gravedad como en teoría de campos, todavía perdura el importante desafío de reproducir de forma teórica la física observada en los aceleradores de partículas, donde la gravedad no juega ningún papel importante. Esto habría conducido a varios enigmas importantes a lo largo de los años, ya que pareciera que uno puediera obtener a priori hasta $10^{272000}$ vacíos cuatridimensionales inequivalentes directamente de teoría de cuerdas \cite{Taylor:2015xtz}, lo que sugiere un poder predictivo casi nulo de la teoría. En este sentido, un papel crucial lo ha jugado la realización de que, de hecho, la cantidad de vacíos consistentes que teoría de cuerdas puede proporcionar comprende en realidad un conjunto de medida cero dentro de la completitud de posibilidades que serían permitidas por argumentos puramente de teoría de campos, como la cancelación de anomalías gauge, etc. Esto ha conducido a proponer la interesante idea del programa de la Ciénaga \cite{Vafa:2005ui}, que tendría como objetivo encontrar cuáles son las condiciones de consistencia que surgen de tener gravedad acoplada a nuestras teorías, y que por tanto deben ser satisfechas por cualquier teoría de campos efectiva débilmente acoplada a la misma. La presente tesis tiene como objetivo principal explorar y extender estos desarrollos, centrándose en una cantidad particular que parece jugar un papel protagonista dentro de esta historia: la escala de gravedad cuántica.

\subsubsection*{Estado actual de la física de altas energías}

La física de altas energías ha experimentado avances significativos en las últimas décadas, impulsados a la par por descubrimientos experimentales así como desarrollos teóricos. El descubrimiento del bosón de Higgs \cite{ATLAS:2012yve, CMS:2012qbp} en el Gran Colisionador de Hadrones en 2012 proporcionó la pieza final que confirmaba el Modelo Estándar, completando así nuestra comprensión de la física de partículas hasta energías de aproximadamente 14 TeV. A pesar de este triunfo, el Modelo Estándar dejaría varias preguntas profundas sin responder, como la naturaleza de la materia/energía oscura \cite{Planck:2015fie}, el origen de las masa de los neutrinos \cite{Super-Kamiokande:1998uiq, SNO:2002tuh, KamLAND:2002uet} o el problema de la jerarquía/constante cosmológica \cite{Weinberg:1988cp}, entre otros.

En paralelo, la detección de ondas gravitacionales por las colaboraciones LIGO y Virgo \cite{LIGO} habría abierto una nueva ventana de observación hacia el cosmos. Estas ondulaciones en el espacio-tiempo, predichas por la teoría clásica de la gravedad de Einstein \cite{GW1, GW2}, han proporcionado asimismo conocimientos cruciales y pruebas de alta precisión sobre la dinámica de espacio-tiempos fuertemente curvados, por ejemplo en presencia de agujeros negros o estrellas de neutrones. De hecho, la observación directa de fusiones de agujeros negros y la detección subsiguiente de numerosos eventos de ondas gravitacionales nos proporcionan nuevas herramientas experimentales que pueden potencialmente revolucionar nuestra comprensión de la gravedad en presencia campos fuertes. Además, la observación de ondas gravitacionales no solo ha confirmado la relatividad general en regímenes de curvatura extrema, sino que también ha impulsado una investigación más profunda sobre la naturaleza de los agujeros negros. La imagen del horizonte del agujero negro situado en el centro de la galaxia M87, que fue tomada por el Telescopio de Horizonte de Sucesos en 2019 \cite{EventHorizonTelescope:2019dse}, proporcionó una confirmación visual directa de la existencia de los horizontes predichos por la teoría. Estas observaciones nos desafían a comprender a un nivel más profundo las descripciones clásicas de estos objetos en relación con la teoría cuántica, especialmente teniendo en cuenta ciertas preguntas abiertas como el origen de la entropía de los agujeros negros \cite{Bekenstein:1972tm, Hawking:1975vcx} o la paradoja de la información \cite{hawking2015information}.

La teoría de cuerdas, la cual postula que las partículas fundamentales no serían objetos puntuales sino filamentos vibrantes con estructura unidimensional, sigue siendo además la principal candidata para proporcionar una teoría unificada de la gravedad con el resto de interacciones. Esta teoría incorpora por tanto de forma natural una partícula sin masa y de espín 2 que mediaría las interacciones gravitatorias de largo alcance, proporcionando un entendimiento cuántico de la dinámica gravitacional. Además, la teoría de cuerdas predice un espectro rico de partículas y sugiere la existencia de dimensiones extra, lo que podría tener, en caso de confirmarse de forma experimental, consecuencias de enorme impacto para nuestra comprensión del universo.

\subsubsection*{Una era dorada para la gravedad cuántica}

En efecto, vivimos ahora en un momento particularmente excitante para abordar el difícil problema de la gravedad cuántica. La convergencia tanto de avances teóricos como de descubrimientos experimentales habría creado un terreno fértil para el desarrollo de nuevo conocimiento. La capacidad de observar y medir ondas gravitacionales así como fenómenos de agujeros negros proporciona datos empíricos directos que pueden informar y restringir los modelos teóricos que podamos construir. Por otro lado, el conocimiento teórico rápidamente creciente proporcionado por la teoría de cuerdas ofrece un marco robusto para entender los aspectos cuánticos más sutiles de la gravedad.

Esta tesis tiene como objetivo profundizar en el papel de la escala de gravedad cuántica dentro de las descripciones de teorías de campos efectivas que incorporan gravedad a bajas energías, con énfasis en sus conexiones con el programa de la Ciénaga. Esta escala debe entenderse además como la escala de energía más allá de la cual toda teoría de campos efectiva local quedaría invalidada debido a la aparición de efectos puramente de gravedad cuántica, como la presencia de cuerdas o dimensiones adicionales. Por lo tanto, al examinar varios desarrollos recientes en el frente teórico, buscamos comprender los límites de las teorías de campos efectivas débilmente acopladas a la gravedad y explorar las propiedades universales que dicha escala pueda exhibir. A través de esta investigación, esperamos contribuir a los esfuerzos aún en curso para descubrir la naturaleza fundamental del espacio-tiempo y la gravedad.

En resumen, la era actual en que se ve inmersa la física de altas energías está marcada por una sinergia entre teoría y experimento, lo que impulsaría de forma considerable nuestra comprensión del universo a su nivel más fundamental. Esta tesis se posiciona dentro de este vibrante paisaje, con el objetivo de contribuir humildemente a algunas de las preguntas más apremiantes en el campo.

\subsubsection*{Organización de la tesis}

Concretamente, en esta tesis consideraremos diferentes compactificaciones de la teoría de cuerdas que preservan 32, 16 u 8 supercargas en dimensiones que van desde diez hasta cuatro. En particular, el resto de la Parte \ref{partI} incluye una revisión detallada sobre los aspectos básicos y los ingredientes que presentan las construcciones de teoría de cuerdas antes mencionadas, poniendo un énfasis especial en la dinámica así como en el contenido masivo (no perturbativo) de estas teorías. También establecemos algunas conexiones entre estas construcciones que son capturadas por el fascinante fenómeno de dualidad, las cuales se utilizarán en varias ocasiones a lo largo de la tesis. Para finalizar, introducimos brevemente el programa de la Ciénaga, centrándonos en las conjeturas que juegan un papel principal en el resto de este trabajo, a saber, las conjeturas de la Distancia \cite{Ooguri:2006in} y de la Gravedad Débil \cite{Heidenreich:2015nta, Heidenreich:2016aqi, Montero:2016tif, Andriolo:2018lvp}.

La mayor parte de los resultados reportados en esta tesis se encuentran en las Partes \ref{part:QGscale}-\ref{part:pattern}. Así, en el Capítulo \ref{ch:SpeciesIntro} introducimos y discutimos en detalle el concepto de escala de gravedad cuántica. Primero explicamos cuáles son las expectativas básicas provenientes del carácter no renormalizable de la Relatividad General, para luego confrontar esta intuición con varias ideas que se cree juegan un papel fundamental en gravedad cuántica, como el principio holográfico. Esto nos lleva a proponer que esta cantidad esté dada por una escala de energía aparentemente diferente, usualmente denominada como escala de especies \cite{Han:2004wt, Dvali:2007hz, Dvali:2007wp}. En consecuencia, revisamos y extendemos varios argumentos perturbativos y no perturbativos que apuntan a la escala de especies como aquella que encapsularía la longitud mínima describible por cualquier teoría de campos efectiva débilmente acoplada a la gravedad. Además, esto resulta estar en concordancia con nuestra intuición física basada en teorías de dimensiones extra y en la propia teoría de cuerdas, donde el corte ultravioleta estaría dado por la masa de Planck de dimensiones superiores o por la escala de cuerdas fundamental. Muchas de las discusiones presentadas en esta parte se apoyan en material ya existente en la literatura, mientras que las nuevas contribuciones se basan en las publicaciones \cite{Castellano:2022bvr, Castellano:2023aum}.

La Parte \ref{part:StringTheoryTests} está dedicada a un estudio y aplicación cuidadosos de las ideas introducidas en la Parte \ref{part:QGscale} dentro del contexto de la teoría de cuerdas. En particular, utilizando un gran conjunto de compactificaciones de teoría de cuerdas descritas en el Capítulo \ref{ch:reviewstringtheory}, probamos en el Capítulo \ref{ch:Higherdimops} si la escala de especies realmente surge como el corte ultravioleta en gravedad. Esto quedaría patente con la aparición de esta última como la escala de energía que controla la expansión efectiva de operadores de dimensiones superiores dentro de la descripcióne efectiva a bajas energías. Asimismo, encontramos un acuerdo perfecto con las expectativas basadas en nuestra discusión de la Parte \ref{part:QGscale}. Además, investigamos en el Capítulo \ref{ch:Emergence} el papel preciso de la escala de especies dentro de ciertos criterios conjeturales propuestos en el programa de la Ciénaga. Más concretamente, analizamos cómo funcionaría el mecanismo de Emergencia \cite{Harlow:2015lma, Grimm:2018ohb, Heidenreich:2018kpg, Palti:2019pca} en construcciones de teoría de cuerdas (al menos en su versión más débil), que depende de la identificación precisa del corte ultravioleta en la teoría efectiva con la escala de especies. El material presentado en esta parte de la tesis se basa en publicaciones anteriores del autor \cite{Castellano:2022bvr, Castellano:2023aum}.

En la Parte \ref{part:pattern} de la tesis, dirigimos nuestra atención a encontrar restricciones y patrones universales relacionados con la escala de especies cerca de las fronteras a distancia infinita en el espacio de campos. En particular, en el Capítulo \ref{ch:bounds} introducimos y discutimos un límite inferior en la tasa de decaimiento exponencial que parece satisfacer la escala de especies a lo largo de cualquier trayectoria de distancia infinita en el espacio de módulos. Esta restricción no trivial puede reformularse como una condición de envoltura convexa y, de hecho, exhibe mucha estructura geométrica que estaría profundamente enraizada en las propiedades de dualidad de las teorías bajo consideración. Posteriormente, en el Capítulo \ref{ch:pattern} proseguimos con estas ideas y presentamos otro patrón aparentemente universal que relaciona las tasas de decaimiento antes mencionadas de la escala de especies y la propia asociada a la torre más ligera en la teoría. Esta última propiedad se ha observado en todos los vacíos supersimétricos explorados hasta la fecha en teoría de cuerdas, estando además íntimamente relacionada con la Conjetura de la Cuerda Emergente \cite{Lee:2019wij}. El material de estos dos capítulos se basa en resultados anteriores del autor contenidos en las referencias \cite{Calderon-Infante:2023ler, Castellano:2023jjt, PhysRevLett.132.181601}.

Finalmente, en la Parte \ref{part:Conclussions} extraemos algunas conclusiones generales que surgen naturalmente del trabajo aquí presentado, mientras que en la Parte \ref{part:appendices} se presentan varios detalles técnicos que serían relevantes para el análisis en la mayor parte de la tesis.






\selectlanguage{british}