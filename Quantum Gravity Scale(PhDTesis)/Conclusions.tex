\thispagestyle{simple}

The main theme of this thesis has been the investigation of the regime of validity associated with any effective field theory weakly coupled to Einstein gravity. This is encapsulated in the concept of the \emph{quantum gravity cut-off} $\LQG$, defined as the energy scale beyond which quantum-gravitational effects can no longer be neglected, thereby invalidating the original effective field theory description. The significance of this scale is twofold: on one hand, it is intimately linked with physical phenomena associated with the underlying UV completion of gravity, such as extra dimensions or fundamental vibrating strings. On the other hand, from a modern perspective, it is also crucial for studying how different effective descriptions of gravitational interactions at low energies deal with non-trivial infra-red constraints, such as the non-existence of exact global symmetries in the theory. The rigorous study of these issues constitutes the central quest of the Swampland program \cite{Vafa:2005ui}, for which a detailed understanding of the cut-off scale $\LQG$ could thus have a significant impact.

%The main theme of this thesis has been the investigation of the regime of validity associated to any effective field theory weakly coupled to Einstein gravity. This can be encapsulated in the \emph{quantum gravity cut-off} $\LQG$, namely the energy scale beyond which quantum-gravitational effects can no longer be neglected, invalidating the original effective field theory description. The significance of this scale is twofold: On the one hand, it is intimately linked with physical phenomena associated to the underlying UV completion of gravity, i.e. extra dimensions, fundamental vibrating strings, etc. On the other hand, from a modern perspective, it is also a crucial element so as to study how different effective descriptions of gravitational interactions at low energies deal with non-trivial infra-red constraints, such as the non-existence of exact global symmetries in the theory. The rigorous study of these issues constitutes the central quest of the Swampland program \cite{Vafa:2005ui}, for which a detailed understanding of the cut-off scale $\LQG$ could thus have an enormous impact.

After reviewing and introducing in Part \ref{partI} the main theoretical tools employed in this thesis, we turned in Part \ref{part:QGscale} to the core ideas discussed in this work. Based on our experience from other non-renormalizable field theories, we concluded that the most natural candidate for $\LQG$ should be the Planck scale $\Mpd$. This is precisely the energy scale associated with the gravitational coupling constant itself, namely Newton's constant $G_N$, and it signals the point where strong back-reaction effects are to be expected, giving rise to extreme phenomena in gravity such as the formation of black holes. However, as originally discussed in \cite{Han:2004wt, Dvali:2007hz, Dvali:2007wp}, this naive picture seems to fail in the presence of a large number of light degrees of freedom $N$. Indeed, using different theoretical arguments invoking black hole physics, non-perturbative (i.e. holographic) considerations as well as perturbation theory of the graviton state, one can argue for a \emph{species scale} $\LSP$ instead as the relevant energy cut-off in semi-classical gravity. Crucially, this scale is sensitive to the aforementioned number of degrees of freedom, is bounded from above by $\Mpd$ — coinciding with the latter when $ N= \mathcal{O}(1)$ — and can be parametrically lowered (when measured in Planck units) in the presence of a large number of species. In Chapter \ref{ch:SpeciesIntro}, we also investigated the behavior of $\LSP$ close to infinite distance and weak coupling regions within the EFT, where this number $N$ grows exponentially, as per the Distance \cite{Ooguri:2006in} and Weak Gravity conjectures \cite{Heidenreich:2015nta, Heidenreich:2016aqi, Montero:2016tif, Andriolo:2018lvp}. There, it was found precise agreement with our expectations arising from Kaluza-Klein theories of gravity with extra dimensions and string perturbation theory, where $\LSP$ is given either by the higher-dimensional Planck mass or the fundamental string scale, respectively. Furthermore, a completely general bottom-up algorithm was provided to compute the species scale in the presence of several towers of states becoming light.

%Therefore, after reviewing and introducing in Part \ref{partI} the main theoretical tools employed in this thesis, we turned in Part \ref{part:QGscale} to the core ideas discussed in this work. Hence, based on our experience from other non-renormalizable field theories, we concluded that the most natural candidate for $\LQG$ should be the Planck scale $\Mpd$. This is precisely the energy scale associated to the gravitational coupling constant itself, namely Newton's constant $G_N$, and it signals the point where strong back-reaction effects are to be expected, giving rise to extreme phenomena in gravity such as the formation of black holes. However, as originally discussed in \cite{Han:2004wt, Dvali:2007hz, Dvali:2007wp}, this naive picture seems to fail in the presence of a large number of light degrees of freedom $N$ in our theory. Indeed, using different theoretical arguments invoking black hole physics, non-perturbative (i.e. holographic) considerations as well as perturbation theory of the graviton state, one can argue for a \emph{species scale} $\LSP$ instead as the relevant energy cut-off in semi-classical gravity. Crucially, this scale is sensitive to the aforementioned number of degrees of freedom, is bounded from above by $\Mpd$ --- coinciding with the latter when $ N= \mathcal{O}(1)$ --- and can be parametrically lowered (when measured in Planck units) in the presence of a large number of species. In Chapter \ref{ch:SpeciesIntro} we also investigated the behaviour of $\LSP$ close to infinite distance and weak coupling regions within the EFT, where this number $N$ grows exponentially, as per the Distance \cite{Ooguri:2006in} and Weak Gravity conjectures \cite{Heidenreich:2015nta,Heidenreich:2016aqi,Montero:2016tif,Andriolo:2018lvp}. There it was found precise agreement with our expectations arising from Kaluza-Klein theories of gravity with extra dimensions and string perturbation theory, where $\LSP$ is given either by the higher-dimensional Planck mass or the fundamental string scale, respectively. Furthermore, a completely general bottom-up algorithm was provided so as to compute the species scale in the presence of several towers of states becoming light.

Subsequently, in Part \ref{part:StringTheoryTests}, we tested the general picture advocated in Chapter \ref{ch:SpeciesIntro} using various concrete string theory constructions. More specifically, in Chapter \ref{ch:Higherdimops} we studied the behavior of certain supersymmetric operators arising in diverse string theory compactifications, which involve curvature invariants with mass dimension greater than two. This allowed us to elucidate the energy scale suppressing these operators with respect to the tree-level kinematics of gravity, i.e. the Einstein-Hilbert term. Interestingly, it was found that the corresponding cut-off turns out being the aforementioned species scale, whose physical interpretation changes depending on the asymptotic corner of the theory that we probe. However, several relevant observations were made. First, it was shown, upon studying further BPS terms in the supergravity action, that the EFT expansion in terms of a unique clear-cut scale $\LSP$ only arises close to infinite distance boundaries, where the classical dimension of the operators provides a good approximation and the quantum corrections (e.g., anomalous dimensions) are rendered parametrically small. Second, we noticed that along decompactification limits, the scale suppressing the higher-curvature operators was sometimes given by the Kaluza-Klein scale instead of the quantum gravity cut-off. This occurs when the finite threshold contributions to the operator under study dominate over the ‘bare’ suppression by $\LSP$, which happens precisely when the latter is irrelevant (in the Wilsonian sense) already in the higher-dimensional theory. Crucially, for energies well above the Kaluza-Klein scale, the non-local threshold effects induced by the tower get washed away (which usually requires a non-trivial resummation procedure), and the only surviving suppression corresponds to that controlled by $\LSP$, as expected.

%Subsequently, in Part \ref{part:StringTheoryTests} we proceeded to test the general picture advocated in Chapter \ref{ch:SpeciesIntro} using various concrete string theory constructions. Consequently, in Chapter \ref{ch:Higherdimops} we studied the behaviour of certain supersymmetric operators arising in diverse string theory compactifications, which involve curvature invariants having dimension greater than two. This allowed us to elucidate what is the energy scale suppressing these operators with respect to the tree-level kinematics of gravity, i.e. the Einstein-Hilbert term. Interestingly, it was found that the corresponding cut-off turns out to be the aforementioned species scale, whose physical interpretation changes depending on the asymptotic corner of the theory that we probe. However, various relevant observations were made. First, it was shown, upon studying further BPS terms in the supergravity action, that the EFT expansion in terms of a clear-cut scale $\LSP$ only arises close to infinite distance boundaries, where the classical dimension of the operators provides a good approximation and the quantum corrections (e.g., anomalous dimensions) are rendered parametrically small. Second, we noticed that along decompactification limits, the scale suppressing the higher-curvature operators was sometimes given by the Kaluza-Klein scale instead of the quantum gravity cut-off. This occurs when the finite threshold contributions to the operator under study dominate over the `bare' suppresion by $\LSP$, which happens precisely when the latter is irrelevant (in the Wilsonian sense) already in the higher-dimensional theory. Crucially, for energies well above the Kaluza-Klein scale, the non-local threshold effects induced by the tower get washed away (which usually requires from a non-trivial ressummation procedure), and the only surviving suppression corresponds to that controlled by $\LSP$, as expected.

In Chapter \ref{ch:Emergence}, we investigated the precise role of the species scale within an intriguing conjecture in quantum gravity usually referred to as the \emph{Emergence Proposal} \cite{Harlow:2015lma, Grimm:2018ohb, Heidenreich:2018kpg, Palti:2019pca}. This conjecture holds that the kinematics of the low energy degrees of freedom entering the EFT (including the graviton itself), emerge upon integrating out the dual infinite number of heavy modes in the UV complete theory. Our aim was to elucidate whether this proposal is respected — in its weakest versions — by the numerous string theory constructions available in the literature. Interestingly, we found that in this regard it seems crucial to identify the physical cut-off of the EFT with the species scale, thereby ensuring that the proposal is verified at leading order, regardless of the nature of the infinite distance limit, the number of non-compact dimensions and the amount of supersymmetry preserved. 

%On another note, in Chapter \ref{ch:Emergence} we investigated the precise role of the species scale within an intriguing conjecture in quantum gravity usually referred to as the \emph{Emergence Proposal} \cite{Harlow:2015lma, Grimm:2018ohb, Heidenreich:2018kpg,Palti:2019pca}. This conjecture holds that the kinematics of the low energy degrees of freedom entering the EFT (including the graviton itself), emerge upon integrating out the dual infinite number of heavy modes in the UV complete theory. Our aim was to elucidate whether this proposal is respected --- in its weakest versions --- by the numerous amount of string theory constructions that are available in the literature....

Finally, in Part \ref{part:pattern}, we focused on the investigation of universal constraints exhibited by $\LSP$ close to the infinite distance boundaries in field space. As a result, we were able to motivate a lower bound on the exponential decay rate $\lambda_{\rm sp}$ of the species cut-off within these regimes, which presents various interesting features singling it out from other potential candidates, such as its preservation under dimensional reduction, and the explicit non-trivial verification in maximally supersymmetric theories. Furthermore, inspired by certain hidden symmetries exhibited by the convex hull diagrams constructed from the decay rates of the towers and species cut-offs within particular string theory examples, we proposed and analyzed in detail in Chapter \ref{ch:pattern} a certain asymptotic equality relating the variation of both quantities over field space. This relation, which we dubbed \emph{the pattern}, has strong implications for the asymptotic behavior of the decay rates as well as for the possible duality phases that can be glued together within a given quantum gravity theory. Accordingly, we thoroughly investigated using a range of string theory constructions in various dimensions and with different amounts of supersymmetry, how this pattern is non-trivially satisfied at each possible infinite distance boundary of moduli space, paying special attention to the global structures emerging at infinity. This also allowed us to extract a minimal set of requirements which, when imposed as bottom-up constraints, directly imply the fulfillment of the pattern. This is significant for various reasons. First, it would be intriguing to understand the physics behind this constraint, potentially using black hole or unitarity arguments. Second, given the intimate connection with the Emergent String Conjecture \cite{Lee:2019wij}, it opens a pathway to argue for the latter from a bottom-up perspective, something that has been lacking since its conception.

%Finally, in Part \ref{part:pattern} we devoted our attention to the investigation of universal constraints exhibited by $\LSP$ close to the infinite distance boundaries in field space. As a result, we were able to motivate some lower bound on the exponential decay rate $\lambda_{\rm sp}$ of the species cut-off within these regimes, which presents various interesting features singling it out from other potential candidates, such as its preservation under dimensional reduction, and the explicit non-trivial verification in maximally supersymmetric theories. Furthermore, inspired by certain hidden symmetries exhibited by the convex hull diagrams constructed from the decay rates of the towers and species cut-offs in particular string theory examples, we proposed and studied in Chapter \ref{ch:pattern} a certain asymptotic equality relating the variation of both quantities over field space. This relation, which we dubbed \emph{the pattern}, has both strong implications for the asymptotic behaviour the decay rates as well as for the possible duality phases that can be glued together within a given quantum gravity theory. Accordingly, we thoroughly investigated using a range of string theory constructions living in various dimensions and with different amounts of supersymmetry preserved, how this pattern is non-trivially satisfied at each possible infinite distance boundary of moduli space, paying special attention to the global structures emerging at infinity. This moreover allowed us to extract a minimal set of requirements which, when imposed as bottom-up constraints, directly imply the fulfillment of the pattern. This is interesting for various reasons. First, since it would be interesting to understand what is the physics behind this constraint, using perhaps black hole or unitarity arguments. Secondly, given the intimate connection with the Emergent String Conjecture \cite{Lee:2019wij}, it open up a way to argue for the latter from the bottom up perspective, something that has been lacking since its conception.

Overall, the work presented in this thesis provides new insights into both the nature and importance of the quantum gravity scale, which is linked to the holographic principle and can thus be motivated and understood regardless of explicit UV completions of gravity, such as string theory. While it ultimately agrees with the behavior expected from a top-down perspective, the fact that it can already be detected in the low energy realm suggests that it may be intimately related to non-trivial infra-red constraints that gravitational EFTs must feature. Therefore, a sharp and complete understanding of this scale, as well as any possible universal constraint exhibited by the latter, provides a useful and fruitful approach to understanding quantum gravity from a low energy point of view, as well as to deduce possible phenomenological consequences that are not apparent from the field theory perspective. We thus believe that the present work can serve to motivate further investigations on this exciting topic, uncovering new connections and consequences that may teach us valuable lessons about Nature.

%Overall, the work presented in this thesis provides new insights on both the nature and importance of the quantum gravity scale, which is linked to the holographic principle and thus can be motivated/understood regardless of explicit UV completions of gravity --- such as string theory. Of course, it ultimately agrees with the behaviour expected from the top-down perspective, but the fact that it can already been detected in the low energy realm suggests that it may be intimately related with non-trivial infra-red constraints that gravitational EFTs must feature. Therefore, a sharp and complete understanding of this scale, as well as any possible universal constraint exhibited by the latter, provides a useful and fruitful way to understand quantum gravity as seen from the low energy point of view as well as to deduce possible phenomenological consequences which are not apparent from the field theory prism. We thus believe that the present work can serve to motivate further investigations on this exciting topic so as to uncover new surprising connections and consequences that hopefully may teach us valuable lessons about Nature. 






\chapter*{Comentarios Finales}
\selectlanguage{spanish}\thispagestyle{simple}

El tema principal de esta tesis ha consistido en la investigación del régimen de validez asociado a cualquier teoría de campos efectiva acoplada débilmente a la gravedad de Einstein. Este último quedaría encapsulado en el concepto de la \emph{escala de gravedad cuántica} $\LQG$, definida como la escala de energía más allá de la cual los efectos cuántico-gravitacionales se vuelven significativos y no pueden ignorarse de forma sistemática, invalidando así la descripción original en términos de una teoría de campos efectiva. La importancia de esta escala es a su vez doble. Por un lado, resulta estar íntimamente ligada a fenómenos físicos asociados a la compleción en el ultravioleta de la gravedad, que incluiría elementos nuevos como la existencia de dimensiones extra o de objetos extendidos en la teoría. Por otro lado, desde una perspectiva quizá más moderna, también resulta ser crucial para estudiar cómo diferentes descripciones efectivas de las interacciones gravitacionales implementan las restricciones infrarrojas que gravedad cuántica implicaría a bajas energías, como la no existencia de simetrías globales exactas en la teoría. El estudio riguroso de estos temas constituye la búsqueda central del programa de la Ciénaga \cite{Vafa:2005ui}, para el cual una comprensión detallada de la escala de gravedad cuántica $\LQG$ podría tener un impacto considerable.

En consecuencia, después de introducir en la Parte \ref{partI} las principales herramientas teóricas empleadas en esta tesis, procedemos en la Parte \ref{part:QGscale} a abordar las ideas centrales discutidas en este trabajo. Así, basándonos en nuestra experiencia con otras teorías de campos no renormalizables, obtenemos nuestro primer candidato para $\LQG$, que identificamos con la masa de Planck $\Mpd$. Esta cantidad se corresponde precisamente con la escala de energías asociada a la constante de acoplamiento gravitacional, es decir, la constante de Newton $G_N$, y marca el punto de inflexión donde los efectos gravitacionales se vuelven dominantes, dando lugar a fenómenos extremos como la formación de agujeros negros. Sin embargo, como se discutió originalmente en \cite{Han:2004wt, Dvali:2007hz, Dvali:2007wp}, este dibujo plausible parece fallar cuando nuestra teoría bajo estudio presenta un gran número de grados de libertad $N$. De hecho, utilizando diferentes argumentos teóricos que involucran la física de agujeros negros, consideraciones no perturbativas así como teoría de perturbaciones del cuanto de la gravitación, puede argumentarse en favor de la denominada \emph{escala de especies} $\LSP$ como la escala de energía relevante en gravedad semi-clásica. Crucialmente, esta escala resulta ser sensible al mencionado número de grados de libertad, presenta una cota superior dada por $\Mpd$ --- coincidiendo con esta última cuando $N= \mathcal{O}(1)$ --- y además puede disminuir paramétricamente respecto a la escala de Planck en presencia de un gran número de especies. Asimismo, en el Capítulo \ref{ch:SpeciesIntro} investigamos el comportamiento exhibido por $\LSP$ cerca de puntos a distancias infinita así como regiones de acoplamiento débil dentro de la teoría efectiva, donde este número $N$ crece exponencialmente, según las conjeturas de Distancia \cite{Ooguri:2006in} y Gravedad Débil \cite{Heidenreich:2015nta, Heidenreich:2016aqi, Montero:2016tif, Andriolo:2018lvp}. Allí, se recupera una concordancia precisa con nuestras expectativas derivadas de las teorías de dimensiones extra (o de Kaluza-Klein) así como con la teoría perturbativa de supercuerdas, donde $\LSP$ es dado por la masa de Planck de mayor dimensión o la escala de la cuerda fundamental, respectivamente. Además, se propuso un algoritmo completamente general para el cálculo de la escala de especies en presencia de varias torres de estados que se vuelven ligeras.

Posteriormente, en la Parte \ref{part:StringTheoryTests}, tratamos de corroborar las ideas centrales defendidas en el Capítulo \ref{ch:SpeciesIntro} utilizando varias construcciones concretas dentro de teoría de cuerdas. Específicamente, en el Capítulo \ref{ch:Higherdimops}, estudiamos el comportamiento de ciertos operadores supersimétricos que aparecen en diversas compactificaciones de teoría de cuerdas, los cuales involucran invariantes de la curvatura cuya dimensión clásica es mayor que dos. Esto nos permitió dilucidar cuál es la escala de energía que suprime dichos operadores con respecto al término cinético de la gravedad. Curiosamente, se encontró que la energía así hallada se corresponde con la antes mencionada escala de especies, cuya interpretación física cambia dependiendo de la frontera asintótica de la teoría que investiguemos. Sin embargo, como consecuencia de este estudio se hicieron varias observaciones de carácter relevante. En primer lugar, se mostró al estudiar ciertos términos adicionales en la acción de supergravedad, que la expansión efectiva de la teoría en términos de una única escala $\LSP$ solo surge cerca de los límites a distancia infinita, donde la dimensión clásica de los operadores proporciona una buena aproximación y las correcciones cuánticas quedan paramétricamente suprimidas. En segundo lugar, notamos que a lo largo de los límites de descompactificación, la escala que suprime estos operadores a veces es dada por la escala de Kaluza-Klein en lugar de la propia de gravedad cuántica. Esto ocurre precisamente cuando las contribuciones cuánticas finitas inducidas por la torre ligera dominan sobre la supresión ‘neta’ dada por $\LSP$. Es crucial para la consistencia de la teoría, no obstante, que para energías muy superiores que la escala de Kaluza-Klein, dichos efectos cuánticos se desvanezcan, dejando pues como única supresión la correspondiente a $\LSP$.

En el Capítulo \ref{ch:Emergence}, investigamos el papel que la escala de especies juega dentro de la conocida como la \emph{Propuesta de Emergencia} \cite{Harlow:2015lma, Grimm:2018ohb, Heidenreich:2018kpg, Palti:2019pca}. Dicha conjetura sostiene que la cinemática de los grados de libertad a baja energía que ingresan en la teoría efectiva (incluyendo el propio cuanto de la gravitación), emerge al integrar un número infinito de modos pesados en la teoría completa. Nuestro objetivo fue dilucidar si esta propuesta es respetada --- en sus versiones más débiles --- por la gran cantidad de construcciones de teoría de cuerdas disponibles en la literatura. Curiosamente, encontramos que, en este sentido, parece crucial identificar la escala de gravedad cuántica con la escala de especies, asegurando así que la propuesta se verifique en todos los ejemplos estudiados en este trabajo, independientemente de la naturaleza del límite bajo consideración, el número de dimensiones no compactas así como la cantidad de supersimetría preservada.

Finalmente, en la Parte \ref{part:pattern}, nos enfocamos en la investigación de aquellas restricciones universales exhibidas por $\LSP$ cerca de los límites a distancia infinita en el espacio de módulos de la teoría. Como resultado, pudimos motivar un límite inferior sobre la tasa de decaimiento exponencial $\lambda_{\rm sp}$ que la escala de especies satisface dentro de estos regímenes. Además, inspirados por ciertas simetrías exhibidas por los diagramas convexos construidos a partir de las tasas de decaimiento tanto de las torres como de la escala de especies (usando ejemplos particulares en teoría de cuerdas), propusimos y estudiamos en el Capítulo \ref{ch:pattern} una sorprendente igualdad asintótica que relaciona la variación de ambas cantidades a lo largo del espacio de módulos. Esta relación, que denotamos como \emph{el patrón}, tendría fuertes implicaciones sobre el comportamiento asintótico de las tasas de decaimiento antes mencionadas, así como para las posibles fases de dualidad que puedan aparecer dentro de una teoría de gravedad cuántica dada. En consecuencia, utilizando una gran variedad de construcciones derivadas de teoría de cuerdas en diversas dimensiones y con diferente cantidad de supersimetría, investigamos a fondo cómo este patrón se verifica de manera no trivial en cada posible límite a distancia infinita del espacio de módulos, prestando especial atención a las complejas estructuras globales que emergen en el infinito. Esto también nos permitió extraer un conjunto mínimo de requisitos que, una vez impuestos como restricciones en la teoría, implican automáticamente el cumplimiento del patrón. Esto último resulta ser significativo por varias razones. Primero, sería interesante entender la física detrás de esta restricción, potencialmente utilizando argumentos de agujeros negros o de unitariedad. Segundo, dada la conexión íntima con la Conjetura de la Cuerda Emergente \cite{Lee:2019wij}, el patrón abriría una nueva vía para argumentar en favor de esta última desde una perspectiva infrarroja.

En general, el trabajo presentado en esta tesis proporciona nuevas ideas sobre la naturaleza e importancia de la escala de gravedad cuántica, la cual estaría vinculada al principio holográfico y, por tanto, puede ser motivada y entendida independientemente de la teoría subyacente de gravedad cuántica. Asimismo, si bien en última instancia esta escala de especies concuerda con el comportamiento esperado desde una perspectiva ultravioleta, el hecho de que pueda detectarse ya en el ámbito de bajas energías sugiere que puede estar íntimamente relacionada con restricciones infrarrojas no triviales que las teorías efectivas gravitacionales deben presentar. Por lo tanto, una comprensión precisa y completa de dicha escala, así como cualquier posible ligadura que esta haya de satisfacer, proporcionaría un enfoque útil para entender la gravedad cuántica desde un punto de vista efectivo, así como para deducir posibles consecuencias fenomenológicas que puedan derivar de forma no trivial de esta última. 