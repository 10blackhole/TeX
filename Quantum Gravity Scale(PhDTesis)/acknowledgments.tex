\selectlanguage{spanish}
%\clearpage
%\begin{flushright}
   % \thispagestyle{empty}
    %\vspace*{30mm}
  %\textit{Frasecita} \\ \textit{motivadora.}
    %\vspace*{\fill}
%\end{flushright}
%\clearpage
\thispagestyle{empty}
\chapter*{Agradecimientos}

Esta tesis no podría comenzar de otra manera que no fuera dando las gracias. En primer lugar, gracias Luis por haberme acogido desde el minuto uno, ya fuera para realizar una beca de colaboración, visitar el ift durante los veranos de grado, o para hacer y descubrir nueva física juntos. De ti he aprendido qué preguntas hacer, cómo abordar los problemas, superando cualquier obstáculo que pudiera surgir por el camino, y sobre todo disfrutar haciendo investigación. Gracias por haberme hecho sentir partícipe de todos y cada uno de los proyectos, ideas y discusiones, por animarme a desarrollar mis propias inquietudes, y por compartir conmigo esa intuición y emoción mágicas que posees y desprendes. Me siento realmente un afortunado de haber podido trabajar y aprender de ti durante los últimos 4 años, y espero poder seguir haciéndolo en el futuro.

Asimismo, quisiera dar mi más sincero agradecimiento a quien considero ha sido mi otro mentor desde que un día cualquiera decidiera acercarme al final de una clase de problemas a preguntar y pedir alguna referencia. Gracias Álvaro por haber dedicado una infinidad de tiempo a explicarme los entresijos de las cuerdas, las compactificaciones y el Swampland. Por haberme invitado a visitarte en París, por tener siempre dos minutos para discutir cualquier cosilla, por tu enorme honestidad y también amistad. En resumen, por ser sin duda uno de los mejores físicos y amigos que he tenido el placer de descubrir. Poder conocer y colaborar con gente como tú hace que la experiencia (a veces dura) en la Academia merezca mucho más la pena. También te estoy agradecido por tus comentarios sobre la versión preliminar de la tesis, que no han hecho sino mejorarla.

Por supuesto, una mención especial al resto de geniales colaboradores de los que he podido disfrutar durante estos años: Anamaría, Fernando, Irene, José, Lorenzo, Luca y Nacho. En particular, quisiera agradecer a Irene por ofrecer apoyo continuo y por haberme dado la inmensa oportunidad de visitar y trabajar en el CERN durante tres fantásticos meses; ha sido realmente un sueño cumplido. También a José, por compartir muchas de sus ideas conmigo, por tomarse en serio las mías, y por sus agudas críticas y comentarios sobre la tesis. De todos vosotros me llevo bonitos recuerdos y muchísimo aprendizaje en cada proyecto. ¡Gracias!

Asimismo, quisiera darle las gracias a todos los miembros del grupo de cuerdas del ift. A Fernando por organizar los mejores workshops, por ayudarnos a navegar la bucrocracia y por toda la física que nos enseñas. A Ángel, por compartir toda tu sabiduría durante los seminarios, por tu curso de teoría de cuerdas y por liderar el mejor canal de física de YouTube que pueda existir. Y gracias por supuesto a Miguel, por animarnos a ser más críticos, a participar y hacer preguntas, por instaurar la cosmo-pizza, por tu buen rollo y por haberme hecho sentir siempre arropado, sobre todo en los momentos más difíciles.

Igualmente agradecido estoy por los grandes amigos y compañeros que hacen que el día a día en el instituto sea enriqucedor y mucho más divertido. En particular, quisiera agradecerle a Max por su amistad y por haber sido mi otro gran apoyo durante el primer año de doctorado, cuando el Covid y la falta de presencialidad hacían aún más difícil la entrada al mundo de la investigación. Gracias a Matilda, por ser mi `partner in crime' en este negocio, por su alegría contagiosa todos los días del año y por enseñarme tantas cosas, dentro y fuera de la física. Y por supuesto a Luca, a quien admiro profundamente como científico pero sobre todo como amigo. (Pd: aún tenemos unos cuantos memes pendientes.) Mención especial también para David, con quien compartí apartamento y aventuras durante los tres meses de estancia en Ginebra. También quisiera acordarme aquí de mis compis del 413: Luca, Manu y Pablo. Echaré muchísimo de menos los ``bueeenoooos díiiiiaas'' de Pablo, las comidas a horas intempestivas de Manu y los ``koffings'' con Luca cada tarde. No quisiera olvidarme por supuesto del resto de la horda del ift con los que he compartido muchos momentos inolvidables durante estos cuatro años: Alexa, Alexander, Andriana, Bernardo, Bruno, Camilo, Christian, David Alonso, David Pereñíguez, Edu, Fer, Florent, Gonzalisto, Gonchi, Jesús, Joan, Jonathan, Matteo, Michelangelo, Mikel, Naredo, Pau, Roberta, Sergio, Víctor y Thibaut. Gracias también a Lars Aalsma, por toda la ayuda, esfuerzo y dedicación prestados durante el tercer y cuarto años de doctorado.  

Quisiera darme el lujo en este momento de mencionar y agradecer como se merece a Pablo García Abia, ya que de tu mano tuve mi primer contacto directo con la investigación. Fue un enorme placer poder aprender de ti, de observar en vivo y en directo cómo se resuelven los problemas reales en física, tanto teóricos como experimentales. Y sobre todo, gracias por haber mantenido el contacto conmigo todos estos años, por poder ser ahora no sólo colegas de profesión sino también amigos. Has sido y sigues siendo mi modelo a seguir en este mundillo. ¡Gracias por tanto!

Fuera del ámbito de la física, son muchas las personas cuyo cariño y apoyo han resultado imprescindibles para poder estar hoy aquí escribiendo estos agradecimientos. En primer lugar, gracias a mis colegas de toda la vida: Arturo, Carlota, Dani, Mayte y Piña. Estoy tremendamente orgulloso de todos y cada uno de vosotros, de haber crecido a vuestro lado, de disfrutar y celebrarnos cada finde con una cervecita, de sentir vuestro aliento y apoyo en cada paso que doy; en resumen, de poder considerarme vuestro amigo. Gracias también a los grandes amigos que la carrera me trajo: Ángela, David, Diego, Pablo y Sami. Sois todos una fuente de inspiración para mi, y aunque no nos veamos tanto como quisiera, os guardo siempre en un rinconcito importante de mi cora. Quisiera dedicar una mención especial a la Residencia de Estudiantes por el fantástico año en que pude disfrutar de vivir en un edificio histórico rodeado de gente brillante y a la que admiro sobremanera. En especial a los becarios: Adri, Alba (historia), Alba (mates), Alberto (mi tocayo), Alicia (bio), Alicia (bailarina), Ana, Guille, Indira, Juan, Laura, Lucía, Mónica y Rocío. ¡Sois un equipazo!

Por supuesto, un lugar especial dentro de estos agradecimientos queda reservado a mi familia. Gracias a mis padres, por quererme y educarme desde siempre, por animarme a perseguir mis sueños, ya fuera en España o en el extranjero. Gran parte de la persona que soy hoy refleja todo el trabajo, cariño y esfuerzo que emanan de vosotros. En especial quisiera agradecerle a mi madre por haber compartido conmigo las alegrías y tristezas de este trabajo, por haberme acompañado desde que era niño a charlas de divulgación, e incluso haber conseguido ponerme en contacto con investigadores de la talla de Pablo. Gracias a mi hermano, por ser como es, cariñoso, inteligente, trabajador y un cachondo. Estoy muy orgulloso de ti y de todos tus logros. Y gracias al resto de mi familia, a mis tíos, abuelos, primos (y primitas :), por siempre estar ahí e incluso interesarse por mi trabajo, aunque suene a lengua extraterrestre. A todos vosotros os quiero y os adoro muchísimo.

Finalmente, y precisamente por ello aún más importante si cabe, gracias por absolutamente todo Teresa. Has sido mi gran descubrimiento, mi gran apoyo durante estos cuatro años, tanto en los momentos de celebración como en aquellos en que el mundo se venía abajo. Eres la alegría de mis días y la ilusión de mi vida. Gracias por ser mi mejor amiga, mi compi de aventuras, mi pinche de cocina y mi otaku favorita. Gracias por animarme y empujarme en cada decisión, por acompañarme a vivir la aventura americana, por quererme y hacerme crecer. Te quiero como los patos, Teresa, ¡pato' la vida!




  
  