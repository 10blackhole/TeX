\thispagestyle{empty}
\chapter*{Abstract}
\thispagestyle{empty}

In this thesis we investigate the role of the quantum gravity cut-off in effective descriptions of gravity at low energies, also in connection with the Swampland program. The focus is placed on understanding in a model-independent way what is the maximum regime of validity of generic effective field theories weakly-coupled to Einstein gravity, as well as characterizing any possible universal behaviour exhibited by the aforementioned cut-off. 

After reviewing some background material on string theory and the Swampland program, we then discuss in great generality the energy scale that supposedly captures the point where quantum-gravitational effects cannot be neglected. Based on various kinds of arguments, which are perturbative and non-perturbative in nature, we arrive at the species scale as the most natural candidate for the latter. This should be understood as the energy cut-off controlling generically the EFT expansion in gravity, therefore signalling the maximum energies/curvatures that can be reliably accommodated by the semi-classical effective description.

Later on, we proceed to check systematically the above picture in consistent theories arising from string compactifications, finding a non-trivial agreement with the former. In addition, we study various formal applications of the concept of the species scale in quantum gravity, including the conjectured phenomenon of Emergence, which posits that all kinematics in the low energy field theory arise from integrating out the massive dual degrees of freedom up to the quantum gravity scale. Indeed, we find that a naive field-theoretic analysis requires from the identification of the quantum gravity cut-off with the species scale, so as to be able to recover the singular behaviour exhibited by the different kinetic functions in the theory, when approaching various kinds of infinite distance limits in field space.

Finally, we perform a thorough analysis and characterization of the species cut-off close to infinite distance boundaries in moduli space, where certain universal properties emerge. In particular, we are able to motivate and provide non-trivial evidence for a lower bound on the exponential decay rate of the species scale, which forces the quantum gravity cut-off to fall off at infinity at least exponentially with the canonical distance defined therein. Relatedly, we are able to uncover some intriguing pattern relating the variation of the species cut-off and the characteristic mass of the lightest tower in the theory for any infinite distance limit. This is moreover satisfied in all up to now explored string theory constructions with at least eight supercharges, even though a purely bottom-up argument for the latter is still missing.




\newpage
\thispagestyle{empty}

\chapter*{Resumen}
\selectlanguage{spanish}\thispagestyle{empty}

En esta tesis se investiga el papel que la escala de gravedad cuántica juega en las descripciones efectivas de gravedad a bajas energías, así como su conexión con el programa de la Ciénaga. El enfoque principal se centra en comprender de manera independiente de cualquier modelo particular de gravedad cuántica, cuál es el régimen máximo de validez de las teorías de campo efectivo acopladas débilmente a la gravedad de Einstein, así como en caracterizar cualquier posible comportamiento universal que esta pueda exhibir.

Una vez sentadas las bases y explicado el material introductorio necesario para la comprensión de esta tesis, procedemos a discutir de forma general la escala de energías precisa que capturaría el punto donde los efectos cuántico-gravitacionales no pueden ser ignorados. Así, basándonos en varios argumentos teóricos (perturbativos y no perturbativos), concluimos que dicha escala se corresponde con la conocida como escala de especies. De esta manera, sería la escala de especies la que controlaría de manera genérica la expansión de toda teoría efectiva de campos incluyendo la gravedad, señalando así las energías/curvaturas máximas que pueden ser descritas por la misma.

A continuación, tratamos de verificar sistemáticamente las ideas presentadas anteriormente usando teorías consistentes que surgen de compactificaciones de la teoría de cuerdas como laboratorio, confirmando así nuestras expectativas. Asimismo, estudiamos varias aplicaciones formales del concepto de la escala de especies dentro de gravedad cuántica, incluyendo el conjeturado fenómeno de Emergencia, que postula que toda la cinemática en la teoría de campos efectiva (incluyendo la propia interacción gravitacional) surgiría de integrar los grados de libertad duales masivos hasta la escala de gravedad cuántica. De hecho, encontramos que un análisis a primer orden usando el formalismo de teoría de campos e identificando dicha escala con la propia de especies, nos permite recuperar de forma no trivial el comportamiento singular exhibido por las diferentes funciones cinéticas que la teoría presenta cuando probamos límites a distancia infinita en el espacio de módulos.

Finalmente, realizamos un análisis exhaustivo así como una caracterización de la escala de especies cerca de los límites a distancia infinita en el espacio de módulos, donde pudieran emerger ciertas propiedades universales. En particular, somos capaces de motivar y proporcionar evidencia significativa acerca de la existencia de un límite inferior en la tasa de decaimiento exponencial de la escala de especies, que obligaría a la misma a decrecer al menos exponencialmente a lo largo de dichos límites. Asimismo, se discute un patrón interesante que relaciona la variación de la escala de especies y la masa característica de la torre más ligera en la teoría, para cualquier límite de distancia infinita. Esto, además, parece cumplirse en todas las construcciones consistentes de teoría de cuerdas exploradas hasta la fecha con al menos ocho supercargas. No obstante, no somos capaces de proporcionar un argumento puramente desde la perspectiva infrarroja. 

\newpage

%\thispagestyle{empty}
\selectlanguage{british}
 
 