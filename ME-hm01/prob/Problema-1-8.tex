\section{Problema 1.8}
\begin{tcolorbox}
\begin{problema}
	Comenzando con\begin{equation}
  T\dd S=\dd E+P\dd V-\m\dd Q,\quad \text{y}\quad  G\equiv E+PV-TS
\end{equation}
\begin{enumerate}
	\item Muestre que
	\begin{equation}
  S=-\eval{\pdv{G}{T}}_{N,P},\qquad V=\eval{\pdv{G}{P}}_{N,T}
\end{equation}
\item Comenzando con
 \begin{equation}\label{18-dS}
  \d S(P,N,T)=\pdv{S}{P}\d P+\pdv{S}{N}\d N+\pdv{S}{T}\d T
\end{equation}
Muestre que los calores específicos,
\begin{equation}
  C_P\equiv T\eval{\pdv{S}{T}}_{N,P},\qquad C_V\equiv T\eval{\pdv{S}{T}}_{N,V}
\end{equation}
satisfacen la relación:
\begin{equation}
  C_P=C_V-T\left(\eval{\pdv{V}{T}}_{P,N}\right)^2\left(\eval{\pdv{V}{P}}_{T,N}\right)^{-1}
\end{equation}
Notar que la compresibilidad, $\equiv -\pdv*{V}{P}$, es positiva (a menos que el sistema sea inestable), luego $C_P>C_V$.
\end{enumerate}
\end{problema}
\end{tcolorbox}

\begin{sol}
\
\begin{enumerate}
\item 
La energía libre de Gibbs viene dada por
	\begin{equation}
  G= E+PV-TS
\end{equation}
o en su forma diferencial
\begin{equation}\label{3.dG}
  \dd G=\dd E+V\dd P+P\dd V-S\dd T-T\dd S
\end{equation}
de la expresión de la primera y la segunda ley de la termodinámica combinadas, donde consideraremos $Q=N$, tenemos
\begin{equation}
  T\dd S=\dd E+P\dd V-\m\dd N
\end{equation}
reemplazando en \eqref{3.dG},
\begin{align}
  \dd G&=\cancel{\dd E}+V\dd P+\cancel{P\dd V}-S\dd T-\cancel{\dd E}-\cancel{P\dd V}+\m\dd N\\
  &=V\dd P-S\dd T+\m\dd N
\end{align}
pero además,
\begin{equation}
  \dd G=\eval{\pdv{G}{P}}_{T,N}\dd P+\eval{\pdv{G}{T}}_{P,N}\dd T+\eval{\pdv{G}{N}}_{T,P}\dd N
\end{equation}
comparando estas dos última expresiones, se tiene
que 
\begin{equation}\label{18-GTGP}
  \boxed{\eval{\pdv{G}{T}}_{P,N}=-S,\qquad \eval{\pdv{G}{P}}_{T,N}=V}
\end{equation}


\item
De \eqref{18-dS} podemos calcular $\eval{\dd S}_N$,
\begin{align}
  \d S(P,N,T)&=\eval{\pdv{S}{P}}_{T,N}\d P+\cancelto{0}{\eval{\pdv{S}{N}\d N}_{T,P}}+\eval{\pdv{S}{T}}_{P,N}\d T\\
  &=\eval{\pdv{S}{P}}_{T,N}\d P+\eval{\pdv{S}{T}}_{P,N}\d T \label{18-dd S}
\end{align}
Queremos encontrar una expresión para $C_V$. Notemos que a volumen y número de partículas constantes se tiene
\begin{align}
  0=\eval{\d V}_N&=\eval{\pdv{V}{P}}_{T,N}\d P+\eval{\pdv{V}{T}}_{P,N}\d T+\cancelto{0}{\eval{\pdv{V}{N}}_{T,P}\d N}\\
  &=\eval{\pdv{V}{P}}_{T,N}\d P+\eval{\pdv{V}{T}}_{P,N}\d T
\end{align}
\begin{equation}
  \implies \eval{\pdv{V}{P}}_{T,N}\d P=-\eval{\pdv{V}{T}}_{P,N}\d T
\end{equation}
\begin{equation}
  \implies \d P=-\eval{\pdv{V}{T}}_{P,N}\left(\eval{\pdv{V}{P}}_{T,N}\right)^{-1}\d T
\end{equation}
Reemplazando en \eqref{18-dd S},
\begin{align}
  \eval{\d S}_{N,V}&=\eval{\pdv{S}{P}}_{T,N}\left[-\eval{\pdv{V}{T}}_{P,N}\left(\eval{\pdv{V}{P}}_{T,N}\right)^{-1}\d T\right]+\eval{\pdv{S}{T}}_{P,N}\d T\\
  &=-\eval{\pdv{S}{P}}_{T,N}\eval{\pdv{V}{T}}_{P,N}\left(\eval{\pdv{V}{P}}_{T,N}\right)^{-1}\d T+\eval{\pdv{S}{T}}_{P,N}\d T\\
  &=\left[-\eval{\pdv{S}{P}}_{T,N}\eval{\pdv{V}{T}}_{P,N}\left(\eval{\pdv{V}{P}}_{T,N}\right)^{-1}+\eval{\pdv{S}{T}}_{P,N}\right]\d T
\end{align}
Luego,
\begin{align}
  \eval{\pdv{S}{T}}_{N,V}&=-\eval{\pdv{S}{P}}_{T,N}\eval{\pdv{V}{T}}_{P,N}\left(\eval{\pdv{V}{P}}_{T,N}\right)^{-1}+\eval{\pdv{S}{T}}_{P,N}
\end{align}
Multiplicando a ambos lados por $T$,
\begin{align}\label{18-casi}
  T\eval{\pdv{S}{T}}_{N,V}&=-T\eval{\pdv{S}{P}}_{T,N}\eval{\pdv{V}{T}}_{P,N}\left(\eval{\pdv{V}{P}}_{T,N}\right)^{-1}+T\eval{\pdv{S}{T}}_{P,N}\\
  C_V&=-T\eval{\pdv{S}{P}}_{T,N}\eval{\pdv{V}{T}}_{P,N}\left(\eval{\pdv{V}{P}}_{T,N}\right)^{-1}+C_P
\end{align}
Además, podemos obtener una relación de Maxwell utilizando \eqref{18-GTGP}. Usando el hecho que
\begin{align}
  \pdv{G}{T}{P}&=\pdv{G}{P}{T}\\
  \left(\pdv{T}\left(\pdv{G}{P}\right)_{T,N}\right)_{P,N}&=\left(\pdv{P}\left(\pdv{G}{T}\right)_{P,N}\right)_{T,N}\\
  \eval{\pdv{V}{T}}_{P,N}&=-\eval{\pdv{S}{P}}_{N,T}
\end{align}
Reemplazando en \eqref{18-casi}, tenemos
\begin{align}
  C_V&=-T\left(-\eval{\pdv{V}{T}}_{P,N}\eval{\pdv{V}{T}}_{P,N}\right)\left(\eval{\pdv{V}{P}}_{T,N}\right)^{-1}+C_P\\
  &=T\left(\eval{\pdv{V}{T}}_{P,N}\right)^2\left(\eval{\pdv{V}{P}}_{T,N}\right)^{-1}+C_P
\end{align}
Finalemte,
\begin{equation}
\boxed{  C_P=C_V-T\left(\eval{\pdv{V}{T}}_{P,N}\right)^2\left(\eval{\pdv{V}{P}}_{T,N}\right)^{-1}}
\end{equation}
















\end{enumerate}
\end{sol}