\section{Problema 2.5}
\begin{tcolorbox}
\begin{problema}
	Derive la expresión correspondiente para la presión para un gas de   Bosones/Fermiones no-interactuante en $2$-dimensiones. Note que en $2$-dimensiones, $P$ describe el trabajo realizado al expandirse por unidad de área, $\dd W=P\dd A$.
\end{problema}
\end{tcolorbox}

\begin{sol}
	Veamos primero el caso bosónico\footnote{Varios de los cálculos hechos en este problema son análogos a los realizados en el Problema \ref{prob:2.7}, por lo tanto recomiendo ver ese antes que este. }.
	
	Notemos que debido a que estamos considerando un gas en $2$ dimensiones, \eqref{Omega zgc} queda,
\begin{equation}\label{Omega zgc}
  \Omega_{\rm GC}=-PA=-\k T\ln(\zgc )
\end{equation}
\begin{equation}
  \implies \frac{PA}{\k T}=\ln(\zgc)
\end{equation}
Considerando los signos $(-)$ en \eqref{zgc pm}, tenemos
\begin{align}
	\frac{PA}{\k T}&=-\frac{N_sA}{(2\p\hbar)^2}\int \dd^2p\ln\left(1-e^{-\b(\epsilon(p)-\m )}\right)\\
	&=-\frac{2\p N_sA}{(2\p\hbar)^2}\int_0^\infty\dd p p\ln\left(1-e^{-\b(\epsilon(p)-\m )}\right)\\
	&=-\frac{2\p N_sA}{(2\p\hbar)^2}\left[\eval{\frac{p^2}{2}\ln\left(1-e^{-\b(\epsilon(p)-\m )}\right)}_{p=0}^{p=\infty} -\int_0^\infty\frac{p^2}{2}\dd p\dv{p}\ln\left(1-e^{-\b(\epsilon(p)-\m )}\right)\right]\\
	&=\frac{2\p N_sA}{(2\p\hbar)^2}\left[\int_0^\infty\frac{p^2}{2}\dd p\dv{p}\ln\left(1-e^{-\b(\epsilon(p)-\m )}\right)\right]\\
	&=\frac{2\p N_s}{(2\p\hbar)^2}\int_0^\infty\frac{p^2}{2}\dd p\frac{-(-\b  )e^{-\b(\epsilon(p)-\m )}}{1-e^{-\b(\epsilon(p)-\m )}}\dv{\epsilon}{p}\\
	&=\frac{2\p N_s A\b }{(2\p\hbar)^2}\int_0^\infty\frac{p^2}{2}\dd p\frac{e^{-\b(\epsilon(p)-\m )} }{1-e^{-\b(\epsilon(p)-\m )}}\frac{pc^2}{\epsilon}\\
	&=\frac{2\p N_s A\b c^2}{(2\p\hbar)^2}\int_0^\infty\dd p\frac{p^3}{2\epsilon}\frac{e^{-\b(\epsilon(p)-\m )}}{1-e^{-\b(\epsilon(p)-\m )}}\\
	&=\frac{N_s A\b c^2}{(2\p\hbar)^2}\int\dd^2p \frac{p^2}{2\epsilon}\frac{e^{-\b(\epsilon(p)-\m )}}{1-e^{-\b(\epsilon(p)-\m )}},\qquad \b=\frac{1}{\k T}\\
	&=\frac{N_s A c^2}{\k T(2\p\hbar)^2}\int\dd^2p \frac{p^2}{2\epsilon}\frac{e^{-\b(\epsilon(p)-\m )}}{1-e^{-\b(\epsilon(p)-\m )}}
\end{align}
Donde se ha usado el mismo argumento que en \eqref{int-partes} para eliminar unos de los términos de la integración por partes. Además para expresar $\dv*{\epsilon}{p}$ se ha usado \eqref{dedp}.

Multiplicando a ambos lados por $\k T/A$, obtenemos
\begin{equation}
 \boxed{ P_{\rm Bosones}=\frac{N_s  c^2}{(2\p\hbar)^2}\int\dd^2p \frac{p^2}{2\epsilon}\frac{e^{-\b(\epsilon(p)-\m )}}{1-e^{-\b(\epsilon(p)-\m )}}}
\end{equation}

Para el caso fermionico el cálculo es similar, solo que ahora debemos considerar el signo positvo en \eqref{zgc pm},
\begin{align}
	\frac{PA}{\k T}&=\frac{N_sA}{(2\p\hbar)^2}\int \dd^2p\ln\left(1+e^{-\b(\epsilon(p)-\m )}\right)\\
	&=\frac{2\p N_sA}{(2\p\hbar)^2}\int_0^\infty\dd p p\ln\left(1+e^{-\b(\epsilon(p)-\m )}\right)\\
	&=\frac{2\p N_sA}{(2\p\hbar)^2}\left[\eval{\frac{p^2}{2}\ln\left(1+e^{-\b(\epsilon(p)-\m )}\right)}_{p=0}^{p=\infty} -\int_0^\infty\frac{p^2}{2}\dd p\dv{p}\ln\left(1+e^{-\b(\epsilon(p)-\m )}\right)\right]\\
	&=-\frac{2\p N_sA}{(2\p\hbar)^2}\left[\int_0^\infty\frac{p^2}{2}\dd p\dv{p}\ln\left(1+e^{-\b(\epsilon(p)-\m )}\right)\right]\\
	&=-\frac{2\p N_s}{(2\p\hbar)^2}\int_0^\infty\frac{p^2}{2}\dd p\frac{(-\b  )e^{-\b(\epsilon(p)-\m )}}{1+e^{-\b(\epsilon(p)-\m )}}\dv{\epsilon}{p}\\
	&=\frac{2\p N_s A\b }{(2\p\hbar)^2}\int_0^\infty\frac{p^2}{2}\dd p\frac{e^{-\b(\epsilon(p)-\m )} }{1+e^{-\b(\epsilon(p)-\m )}}\frac{pc^2}{\epsilon}\\
	&=\frac{2\p N_s A\b c^2}{(2\p\hbar)^2}\int_0^\infty\dd p\frac{p^3}{2\epsilon}\frac{e^{-\b(\epsilon(p)-\m )}}{1+e^{-\b(\epsilon(p)-\m )}}\\
	&=\frac{N_s A\b c^2}{(2\p\hbar)^2}\int\dd^2p \frac{p^2}{2\epsilon}\frac{e^{-\b(\epsilon(p)-\m )}}{1+e^{-\b(\epsilon(p)-\m )}},\qquad \b=\frac{1}{\k T}\\
	&=\frac{N_s A c^2}{\k T(2\p\hbar)^2}\int\dd^2p \frac{p^2}{2\epsilon}\frac{e^{-\b(\epsilon(p)-\m )}}{1+e^{-\b(\epsilon(p)-\m )}}
\end{align}
\begin{equation}
  \implies \boxed{P_{\rm Fermiones}=\frac{N_s  c^2}{(2\p\hbar)^2}\int\dd^2p \frac{p^2}{2\epsilon}\frac{e^{-\b(\epsilon(p)-\m )}}{1+e^{-\b(\epsilon(p)-\m )}}}
\end{equation}

Así, podemos considerar ambos casos,
\begin{equation}
 \boxed{ P=\frac{N_s  c^2}{(2\p\hbar)^2}\int\dd^2p \frac{p^2}{2\epsilon}\frac{e^{-\b(\epsilon(p)-\m )}}{1\pm e^{-\b(\epsilon(p)-\m )}}}
\end{equation}
donde el signo $(-)$ es para bosones y el $(+)$ para los fermiones.
\end{sol}