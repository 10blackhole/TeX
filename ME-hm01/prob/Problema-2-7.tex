\section{Problema 2.7}\label{prob:2.7}
\begin{tcolorbox}
\begin{problema}
	Considere un gas de bosones $3$-dimensional sin masa con degeneración de spin $N_s$. Asumiendo que el potencial químico es nulo ($\m=0$), encuentre los coeficientes $A$ y $B$ para las expresiones para la densidad de presión y densidad de energía,
	\begin{equation}
  P=AN_sT^4,\qquad \left(\frac{E}{V}\right)=BN_sT^4
\end{equation}
\end{problema}
\end{tcolorbox}

\begin{sol}
Por completitud, derivemos la expresión para la presión $P$ para un gas de bosones no interactuantes mostrada en el libro, asumiendo una relación de dispersión relativista $\epsilon(p)=\sqrt{p^2c^2+m^2c^4}$

En clases vimos que para el caso de un gas de partículas no interactuantes en $D$-dimensiones con momentum $p$ y carga $q$ las cuales son consideradas independientes una de otra, el logaritmo de la función partición del ensamble gran canónico viene dado por
\begin{equation}\label{zgc pm}
  \ln(\zgc)=\frac{N_sV}{(2\p\hbar)^D}\int\dd^Dp(\mp )\ln(1\mp e^{-\b (\epsilon(p)-\m )})
\end{equation}
donde el signo $-$ es para bosones y el $+$ para fermiones. Ademas $N_s$ es la degeneración de spin, la cual en la mayoría de los casos se considera $(2s+1)$ pero acá no es relevante.

Así, para un gas de bosones en $3$-dimensiones sin masa, asumiendo $\m=0$ se tiene
\begin{equation}
    \ln(\zgc)=\frac{N_sV}{(2\p\hbar)^3}\int\dd^3p\ln\left(\frac{1}{1-e^{-\b(\epsilon(p)-\m )}}\right)
\end{equation}


Sabemos que el potencial gran canónico se relaciona con $\zgc$ según,
\begin{equation}\label{Omega zgc}
  \Omega_{\rm GC}=-PV=-\k T\ln(\zgc )
\end{equation}
%es decir,
%\begin{align}
%  \frac{PV}{\k T}&=\ln(\zgc )\\
%  &=\sum_p\ln(1+e^{-\b(\ep_p-\m )}+e^{-2\b(\ep_p-\m )}+\cdots)\\
%  &=\sum_p\ln\left[\left(e^{-\b(\ep_p-\m )}\right)^p\right]\\
%  &=\sum_p\ln\left[\frac{1}{1-e^{-\b (\epsilon_p-\m )}}\right]
%\end{align}
%donde usamos la serie geométrica.
%Podemos aproximar la sumatoria al continuo, como se vio en clases y como lo indica el libro,
%\begin{equation}
%  \sum_p\to \frac{N_s}{(2\p\hbar)^3}V\int\dd^3 p
%\end{equation}
%donde $N_s$ es la degeneración de spin, la cual en la mayoría de los casos se considera $(2s+1)$ pero acá no es relevante. Luego,
entonces,
\begin{align}
  \frac{PV}{\k T}&=\ln(\zgc)\\
  &=\frac{N_sV}{(2\p\hbar)^3}\int\dd^3p\ln\left(\frac{1}{1-e^{-\b (\epsilon(p)-\m )}}\right)\\
  &=\frac{4\p N_sV}{(2\p\hbar)^3}\int_0^\infty\dd pp^2\ln\left(\frac{1}{1-e^{-\b (\epsilon(p)-\m )}}\right)\\
  &=-\frac{4\p N_sV}{(2\p\hbar)^3}\int_0^\infty\dd pp^2\ln\left(1-e^{-\b(\epsilon(p)-\m )}\right)\\
  &=-\frac{4\p N_sV}{(2\p\hbar)^3}\left[\eval{\frac{p^3}{3}\ln\left(1-e^{-\b(\epsilon(p)-\m )}\right)}_{p=0}^{p=\infty}-\int_0^\infty \frac{p^3}{3}\dd p\dv{p}\ln\left(1-e^{-\b(\epsilon(p)-\m )}\right)\right]
\end{align}
donde en el último paso se ha integrado por partes y se ha usado el hecho de que debido a que $\epsilon(p)$ sólo depende de la magnitud de $p$, podemos usar la fórmula vista en clases
\begin{equation}
  \dd^Dp=\Omega p^{D-1}\dd p
\end{equation}
donde $\Omega$ es el elemento de ángulo sólido en $D$ dimensiones.


 Notemos que
\begin{equation}\label{int-partes}
  \eval{\frac{p^3}{3}\ln\left(1-e^{-\b(\epsilon(p)-\m )}\right)}_{p=0}^{p=\infty}=0
\end{equation}
Esto debido a que cuando se evalúa en el límite inferior $p=0$ se anula directamente y cuando se evalúa en el límite superior, y asumimos la relación de dispersión relativista $\epsilon=\sqrt{p^2c^2+m^2c^4}$, la exponencial decae mucho más rápido que $p^3$, luego,
\begin{equation}
  \lim_{p\to \infty}\frac{p^3}{3}\ln\left(1-e^{-\b(\sqrt{p^2c^2+m^2c^4}-\m )}\right)=\frac{p^3}{3}\ln(1)=0
\end{equation}
Así, tenemos
\begin{align}
   \frac{PV}{\k T}&=\frac{4\p N_sV}{(2\p\hbar)^3}\int_0^\infty \frac{p^3}{3}\dd p\dv{p}\ln\left(1-e^{-\b(\epsilon(p)-\m )}\right)\\
   &=\frac{4\p N_sV}{3(2\p\hbar)^3}\int_0^\infty \dd pp^3\frac{e^{-\b (\epsilon(p)-\m )}}{1-e^{-\b (\epsilon(p)-\m )}}\b \dv{\ep(p)}{p}
\end{align}
pero
\begin{equation}\label{dedp}
  \epsilon=\sqrt{p^2c^2+m^2c^4}\implies \dv{\epsilon}{p}=\frac{1}{2}\frac{2pc^2}{\sqrt{p^2c^2+m^2c^4}}=\frac{pc^2}{\epsilon}
\end{equation}
reemplazando en lo anterior,
\begin{align}
  \frac{PV}{\k T}&=\frac{4\p N_sV}{3(2\p\hbar)^3}\int_0^\infty \dd pp^3\frac{e^{-\b (\epsilon(p)-\m )}}{1-e^{-\b (\epsilon(p)-\m )}}\b \dv{\ep(p)}{p},\qquad \b=\frac{1}{\k T}\\
  &=\frac{4\p N_sV}{3\k T(2\p\hbar)^3}\int_0^\infty\dd p p^4\frac{c^2}{\epsilon}\frac{e^{-\b (\epsilon(p)-\m )}}{1-e^{-\b (\epsilon(p)-\m )}}\\
  &=\frac{N_sV}{\k T(2\p\hbar)^3}\int\dd^3p\frac{p^2c^2}{3\epsilon}\frac{e^{-\b (\epsilon(p)-\m )}}{1-e^{-\b (\epsilon(p)-\m )}}
\end{align}
multiplicando por $\k T/V$ a ambos lados, obtenemos
\begin{equation}\label{27-P}
  \boxed{P=\frac{N_s}{(2\p\hbar)^3}\int\dd^3p\frac{p^2c^2}{3\epsilon}\frac{e^{-\b (\epsilon(p)-\m )}}{1-e^{-\b (\epsilon(p)-\m )}}}
\end{equation}
la cual corresponde a la expresión de partida del libro pero ya no en unidades naturales.



%
%\begin{equation}
%  P=\frac{N_s}{(2\p\hbar)^3}\int\dd^3p\frac{p^2}{3\epsilon}f(p),\qquad f(p)=\frac{e^{-\b\epsilon(p)}}{1-e^{-\b\epsilon(p)}}
%\end{equation}
Dado que estamos considerando un gas de bosones sin masa, la relación de dispersión relativista queda
\begin{equation}
  \epsilon=\sqrt{p^2c^2}=pc
\end{equation}
Luego, de \eqref{27-P}
\begin{align}
  P&=\frac{N_s}{(2\p\hbar)^3}\int\dd^3p\frac{p^2c^2}{3\epsilon}\frac{e^{-\b pc}}{1-e^{-\b pc}}\\
  &=\frac{N_s}{(2\p\hbar)^3}\int\dd^3p\frac{p^2c^2}{3}\frac{1}{pc}\frac{e^{-\b pc}}{1-e^{-\b pc}}\\
  &=\frac{N_sc}{3(2\p\hbar)^3}\int\dd^3p p\frac{e^{-\b pc}}{1-e^{-\b pc}}\\
  &=\frac{4\p N_sc}{3(2\p\hbar)^3}\int_0^\infty p^2\dd p p\frac{e^{-\b pc}}{1-e^{-\b pc}}\\
  &=\frac{4\p N_sc}{3(2\p\hbar)^3}\int_0^\infty \dd pp^3\frac{e^{-\b pc}}{1-e^{-\b pc}}\quad /\cdot\frac{e^{\b pc}}{e^{\b pc}}\label{mult-ee}\\
  &=\frac{4\p N_sc}{3c(2\p\hbar)^3}\int_0^\infty \dd pp^3\frac{1}{e^{\b pc}-1}
\end{align}
Haciendo el siguiente cambio de variables
\begin{equation}\label{27-cambio-varable}
  u=\b pc,\quad p=\frac{1}{\b c}u ,\quad  \dd p =\frac{1}{\b c}\dd u
\end{equation}
donde es claro ver que los límites de integración no cambian, se tiene
\begin{align}
  P&=\frac{4\p N_sc}{3(2\p\hbar)^3}\int_0^\infty \left(\frac{1}{\b c}\right)^4\dd u\frac{u^3}{e^{u}-1}\\
  &=\frac{4\p N_sc}{3(2\p\hbar)^3}\left(\frac{1}{\b c}\right)^4\int_0^\infty \dd u\frac{u^3}{e^{u}-1}
\end{align}
La integral que queda por hacer se puede calcular usando algún software de cálculo analítico, por ejemplo Maple, el cual arroja
\begin{equation}\label{27-integral}
  \int_0^\infty \dd u\frac{u^3}{e^{u}-1}=\frac{\p^4}{15}
\end{equation}
Finalmente,
\begin{align}
  P &=\frac{4\p N_sc}{3(2\p\hbar)^3}\left(\frac{1}{\b c}\right)^4\frac{\p^4}{15},\qquad \b=\frac{1}{\k T}\\
  &=\frac{4\p N_sc}{3(2\p\hbar)^3}\left(\frac{\k T}{c}\right)^4\frac{\p^4}{15}\\
  &=\frac{\p^2N_s\k^4}{90c^3\hbar^3}T^4
\end{align}
Luego, el coeficiente $A$ pedido viene dado por
\begin{equation}
\boxed{  A=\frac{\p^2\k^4}{90c^3\hbar^3}}
\end{equation}

Para calcular $E/V$ usamos la relación derivada en clases,
\begin{equation}\label{J/E}
  \frac{J}{V}=N_s\int\frac{\dd^3p}{(2\p\hbar)^3}j(p)f(\epsilon(p))
\end{equation}
donde $N_s$ es la degeneración de spin, $j(p)$ es una cantidad física para una partícula con magnitud de momentum $p$ y 
\begin{equation}
  f(\epsilon(p))=\frac{e^{-\b(\epsilon(p))}}{1-e^{-\b(\epsilon(p))}}
\end{equation}
para el caso de bosones con potencial químico nulo. Así, se tiene
\begin{align}
  \frac{E}{V}&=\frac{N_s}{(2\p\hbar)^3}\int\dd^3p\epsilon(p)f(\epsilon(p)),\qquad \epsilon=pc\\
  &=\frac{N_s}{(2\p\hbar)^3}\int\dd^3p pc \frac{e^{-\b pc}}{1-e^{-\b pc}}\\
  &=\frac{4\p cN_s}{(2\p\hbar)^3}\int_0^\infty p^2\dd p p \frac{e^{-\b pc}}{1-e^{-\b pc}}\qquad /\cdot \frac{e^{\b pc}}{e^{\b pc}-1}\\
  &=\frac{4\p cN_s}{(2\p\hbar)^3}\int_0^\infty\dd pp^3\frac{1}{e^{\b pc}-1}
\end{align}
Haciendo el mismo cambio de variable \eqref{27-cambio-varable}, se obtiene
\begin{align}
  \frac{E}{V}&=\frac{4\p cN_s}{(2\p\hbar)^3}\int_0^\infty \left(\frac{1}{\b c}\right)\dd u\left(\frac{1}{\b c}\right)^3u^3\frac{1}{e^{u}-1}\\
  &=\frac{4\p cN_s}{(2\p\hbar)^3}\left(\frac{1}{\b c}\right)^4\underbrace{\int_0^\infty \dd u\frac{u^3}{e^{u}-1}}_{\p^4/15},\qquad \b=\frac{1}{\k T}\\
  &=\frac{4\p cN_s}{(2\p\hbar)^3}\left(\frac{\k T}{c}\right)^4\frac{\p^4}{15}\\
  &=\frac{\p^2N_s\k^4}{30\hbar^3c^3}T^4
\end{align}
donde se usó \eqref{27-integral}. Así, el coeficiente $B$ viene dado por
\begin{equation}
\boxed{  B=\frac{\p^2\k^4}{30\hbar^3c^3}}
\end{equation}



















\end{sol}