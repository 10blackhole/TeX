\section{Classical Field Theory}
Quantum field theory is just quantum mechanics with an infinite number of oscillators

\subsection{Hamiltonians and Lagrangians}
A classical field theory is just a mechanical system with a continuous set of degrees of freedom. Field theories can be defined in terms of either a Hamiltonian or a Lagrangian, which we often write as integrals over all space of Hamiltonian or Lagrangian densities:
\begin{equation}
	H=\int\dd^3x\mathcal{H},\quad L=\int\dd^3x\mathcal{L}.
\end{equation}

Formally, the \textit{Hamiltonian} (density) is a functional of fields and their conjugate momenta $\mathcal{H}[\f,\p ]$. The \textit{Lagrangian} (density) is the Legendre transform of the Hamiltonian (density). Formally, it is defined as
\begin{equation}
  \mathcal{L}[\f,\dot{\f }]=\p [\f,\dot{\f }]\dot{\f }-\mathcal{H}[\f,\p[\f\dot{\f }]],
\end{equation}
where $\dot{\f }=\partial_t\f $ and $\p [\f,\dot{\f }]$ is implicitly defined by $\pdv*{\mathcal{H}[\f,\p ]}{\p }=\dot{\f }$. The inverse transform is
\begin{equation}
  \mathcal{H}[\f,\p ]=\p \dot{\f }[\f,\p ]-\mathcal{L}[\f,\dot{\f }[\f,\p ]],
\end{equation}
where $\dot{\f }[\f,\p ]$ is implicitly defined by $\pdv*{\mathcal{L}[\f,\dot{\f }]}{\dot{\f }}=\p $.



\subsection{Euler-Lagrange equations}
In quantum field theory, we will almost exclusively use Lagrangians. The simplest reason for this is that Lagrangians are manifestly Lorentz invariant. Dynamics for a Lagrangian system are determined by the principle of least action. The \textit{action} is the integral over time of the Lagrangian:
\begin{equation}
  S=\int\dd tL=\int\dd^4x\lag(x).
\end{equation}
Say we have a Lagrangian $\lag[\f,\partial_\m\f ]$ that is a functional only of a field $\f$ and its first derivatives. Now we imagine varying $\f\to\f+\d\f $ where $\d\f $ can be any field. Then, 
\begin{align}
  \d S&=\int\dd^4x\left[\pdv{\lag}{\f }+\pdv{\lag}{(\partial_\m\f )}\d(\partial_\m\f )\right] \nonumber\\
  &=\int\dd^4x\left\{\left[\pdv{\lag}{\f}-\partial_\m \pdv{\lag}{(\partial_\m\f )}\right]\d\f +\partial_\m \left[\pdv{\lag}{(\partial_\m \f )}\d\f \right]\right\}. \label{3.13}
\end{align}
We have integred by parts in order to factorize by $\d\f$. The last term is a total derivative and therefore its integral only depends on the field values at spatial and temporal infinity. We will always make the physical assumption that our fields vanish on these asymptotic boundaries, which lets us drop such total derivatives from Lagrangians.

In classical field theory, just as in classical mechanics, the equations of motion are deter- mined by the principle of least action: when the action is evaluated on fields that satisfy the equations of motion, it should be insensitive to small variations of those fields, $\frac{\d S}{\d\f }=0$. If this holds for all variations, then \eqref{3.13} implies
\begin{equation}\label{3.15}
  \pdv{\lag}{\f}-\partial_\m \pdv{\lag}{(\partial_\m\f )}=0.
\end{equation}
The are the celebrated \textit{Euler-Lagrange equations}. They give the \textit{equations of motion} following from a Lagrangian.

\begin{ej}
	If our action is
	\begin{equation}
  S=\int\dd^4x\left[\frac{1}{2}(\partial_\m\f)(\partial^\m\f)  -V[\f ]\right],
\end{equation}
then
\begin{equation}
  \pdv{\lag}{\f}=-V'[\f],\quad \pdv{\lag}{(\partial_\m\f )}=\partial^\m\f \to \partial_\m \pdv{\lag}{(\partial_\m\f )}=\Box \f .
\end{equation}
So then, the equations of motion \eqref{3.15} are
\begin{equation}
  \Box \f+V'[\f]=0.
\end{equation}
where $\Box:=\partial^2$ is the d'Alambertian. In particular, if $\lag =\frac{1}{2}(\partial\f )^2-\frac{1}{2}m^2\f^2$, the equations of motion are
\begin{equation}
  (\Box +m^2)\f =0.
\end{equation}
This is known as the \textit{Klein-Gordon equation} which describes the equations of motion for a free scalar field.
\end{ej}
Why do we restrict to Lagrangians of the form $\lag[\f,\partial_\m\f]$? First of all, this is the form that all "classical" Lagrangians had. If only first derivatives are involved, boundary conditions can be specified by initial positions and velocities only, in accordance with Newton's laws. In the quantum theory, if kinetic terms have too many derivatives, for example $\lag=\f\Box^2\f $, there will generally be disastrous consequences. For example, there may be states with negative energy or negative norm, permitting the vacuum to decay. But interactions with multiple derivatives may occur. Actually, they must occur due to quantum effects in all but the simplest renormalizable field theories; for example, they are generic in all effective field theories.




\subsection{Noether's theorem}
It may happen that a Lagrangian is invariant under some special type of variation $\f\to \f+\d\f  $. For exmaple, a Lagrangian for a complex field $\f $ is
\begin{equation}
  \lag = |\partial_\m \f |^2-m^2|\f |^2.
\end{equation}
This Lagrangian is invariant under $\f\to e^{-i\a }\f$ for any $\a\in\mathbb{R}$. This transformation is a \textit{symmetry} of the Lagrangian. There are two independents degrees of fredoom in a complex field $\f$, which we can take as $\f=\f_1+i\f_2 $ or more conveniently $\f$ and $\f^*$. Then the Lagrangian is
\begin{equation}\label{3.19}
  \lag =(\partial_\m\f )(\partial^\m \f^*)-m^2\f\f^*,
\end{equation}
and the symmetry transformations are
\begin{equation}\label{sim}
  \f\to e^{-i\a }\f,\quad \f^*\to e^{i\a }\f^*.
\end{equation}
We note that 
\begin{equation}
  \pdv{\lag}{\f }=-m^2\f^*,\quad \pdv{\lag}{(\partial_\m\f )}=\partial^\m \f^* \to \partial_\m \left(\pdv{\lag}{(\partial_\m\f )}\right)=\Box \f^* ,
\end{equation}
\begin{equation}
  \pdv{\lag}{\f^* }=-m^2\f,\quad \pdv{\lag}{(\partial_\m\f^* )}=\partial^\m \f \to \partial_\m \left(\pdv{\lag}{(\partial_\m\f^* )}\right)=\Box \f.
\end{equation}
Thus, the equation of motion for this Lagrangian are given by
\begin{equation}\label{eom-phi}
  (\Box +m^2)\f=0,\qquad (\Box +m^2)\f^*=0.
\end{equation}
When there is such a symmetry that depends on some parameter $\a$ that can be taken small (that is, the symmetry is \textit{continuous}), we find, that
\begin{equation}\label{3.22}
  0=\frac{\d \lag}{\d\a }=\sum_n\left\{\left[\pdv{\lag}{\f_n}-\partial_\m \pdv{\lag}{(\partial_\m \f_n )}\right]\frac{\d\f_n}{\d\a }+\partial_\m \left[\pdv{\lag}{(\partial_\m \f_n)}\frac{\d\f_n}{\d\a }\right]\right\},
\end{equation}
where $\f_n$ may be whatever set of fields the Lagrangian depends on. In constrast with \eqref{3.13}, this equation holds even for fields configurations $\f_n $ for which the action is not extremal (i.e. for $\f_n$ that do not satisfy the EOM), since the variation corresponds to a symmetry.

When the EOM \textit{are} satisfied, the \eqref{3.22} reduces to $\partial_\m J^\m=0$, where
\begin{equation}\label{3.23}
	J^\m =\sum_n\pdv{\lag}{(\partial_\m\f_n )}\frac{\d\f_n }{\d\a }.
\end{equation}
This is known as a \textit{Noether current}.

\begin{ej}
	Let us consider the Lagrangian in \eqref{3.19}
	\begin{equation}
   \lag =(\partial_\m\f )(\partial^\m \f^*)-m^2\f\f^*.
\end{equation}
Since it is invariant under the continuous symmetries \eqref{sim} parametrized with $\a$, which can be as small as we want, we can expand in Taylor series up-to first order in $\a$
\begin{equation}
  e^{-i\a }\approx 1-i\a ,\quad e^{i\a }\approx 1+i\a,
\end{equation}
so that,
\begin{equation}
  \frac{\d\f }{\d\a }=-i\f ,\quad \frac{\d\f^*}{\d\a }=i\f^*.
\end{equation}
By plugging in \eqref{3.23} we obtain
\begin{equation}
  J^\m =\pdv{\lag}{(\partial_\m\f )}\frac{\d\f }{\d\a }+\pdv{\lag}{(\partial_\m\f^* )}\frac{\d\f^* }{\d\a }=-i(\f\partial^\m \f^*-\f^*\partial\m \f ).
\end{equation}
We can check that
\begin{equation}
  \partial_\m J^\m =-i[\f\Box \f^*-\f^*\Box\f ],
\end{equation}
which vanishes when the EOM \eqref{eom-phi} are satisfied.
\end{ej}

A vector field $J^\m $ that satisfies $\partial_\m J^\m =0$ is called a \textit{conserved current}, because the total charge $Q$, defined as
\begin{equation}
  Q:=\int \dd^3x J^0,
\end{equation}
satisfies 
\begin{equation}
  \partial_tQ=\int\dd^3x\partial_t J^0=\int\dd^3x\nabla\cdot\vb*{J}=0,
\end{equation}
where in the last step we have assumed $\vb*{J}$ vanishes at the spatial boundary, since, by assumption, nothing is leaving our experiment. Thus, the total charge does not change with time, and is conserved.\footnote{Remeber that $\partial_\m J^\m =\partial_t J^0 - \partial_i J^{i}$.}

We have just proved a very general and important theorem known as \textit{Noether's theorem.}

\begin{teo}{\textbf{Noether's theorem}}

	If a Lagrangian has a continuous symmetry then there exists a current asso- ciated with that symmetry that is conserved when the EOM are satisfied.
\end{teo}

Recall that we needed to assume the symmetry was continuous so that small variations $\frac{\d\lag }{\d\a }$ could be taken. So, Noether's theorem does not apply to discrete symmetries.

Important points about this theorem are:
\begin{itemize}
	\item The symmetry must be continuous, otherwise $\d \a $ has no meaning.
	\item The current is conserved \textit{on-shell}, that is, when the EOM are satisfied
	\item It works for \textit{global symmetries}, parametrized by numbers $\a $, not only for \textit{local (gauge)} \textit{symmetries} parametrized by functions $\a(x)$.
\end{itemize}

\subsubsection{Energy-momentum tensor}
There is a very important case of Noether's theorem that applies to a global symmetry of the action, not the Lagrangian. This is the symmetry under (global) space-time translations. In general relativity this symmetry is promoted to a local symmetry -- diffeomorphism invariance -- but all one needs to get a conserved current is a global symmetry. The current in this case is the energy-momentum tensor, $T_{\m\n }$.

Space-time translation invariance says that physics at a point $x$ should be the same as physics at any other point $y$. We have to be careful distinguishing this symmetry which acts on fields from a trivial symmetry under relabeling our coordinates. Acting on fields, it says that if we replace the value of the field $\f(x)$ with its value at a different point $\f(y)$, we will not be able to tell the difference. Scalar fields then transform as $f(x)\to \f(x+\xi )$. For infinitesimal $\xi^\m $, this is
\begin{equation}
  \f(x)\to \f(x+\xi )=\f(x)+\xi^\n \partial_\n \f(x)+\cdots,
\end{equation}
where the $\cdot$ are higher order in the infinitesimal transformation $\xi^\n $. To be clear, we are considering variations where we replace the field $\f(x)$ with a linear combination of the field and its derivatives evaluated at the same point $x$. The point $x$ does not change. Our coordinates do not change. A theory with a global translation symmetry is invariant under this replacement.

This transformation law,
\begin{equation}\label{3.30}
  \frac{\d\f }{\d\xi^\n }=\partial_\n \f ,
\end{equation}
applies for any field, whether tensor or spinor or anything else. It is also applies to the Lagrangian itself, which is a scalar:
\begin{equation}\label{3.31}
  \frac{\d\lag }{\d\xi^\n }=\partial_\n \lag  .
\end{equation}
Proceeding as before, using the equations of motion, the variation of the Lagrangian is
\begin{equation}
  \frac{\d \lag [\f_n ,\partial_\m \f_n]}{\d\xi^\n }=\cancel{(EOM)}+\partial_\m \left(\pdv{\lag}{(\partial_\m \f_n )}\frac{\d\f_n}{\d \xi^\n }\right).
\end{equation}
Equating this with \eqref{3.31} and using \eqref{3.30} we find
\begin{equation}
  \partial_\n \lag=\partial_\m \left(\pdv{\lag}{(\partial_\m \f_n )}\partial_\n \f_n \right)
\end{equation}
or equivalently
\begin{equation}
	\partial_\m \underbrace{\left(\pdv{\lag}{(\partial_\m \f_n )}\partial_\n \f_n -\delta^\m _\n \lag\right)}_{T^\m _{~\n }}=0.
\end{equation}
The four symmetries have produced four Noether currents, one for each $\n $:
\begin{equation}
  T^\m _{~\n }:=\pdv{\lag}{(\partial_\m \f_n )}\partial_\n \f_n -\delta^\m _\n \lag,
\end{equation}
all of which are conserved: $\partial_\m T^\m _{~\n }=0$. The four conserved quantities are energy and momentum. $T_{\m\n }$ is called the \textit{energy-momentum tensor}.

An important component of the energy-momentum tensor is the energy density:
\begin{equation}
  \mathcal{E}=T^{0}_{~0}=\pdv{\lag}{\dot{\f }_n}\dot{\f }_n-\lag,
\end{equation}
where $\dot{\f }_n=\partial_t\f_n$.




















































