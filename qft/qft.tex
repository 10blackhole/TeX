\documentclass[11pt,letterpaper]{article}
%\pdfoutput=1
\usepackage{amsmath,amssymb,amsfonts,amsthm}
%\usepackage{tcolorbox}
\usepackage[T1]{fontenc}
\usepackage{palatino}
\def\author{Borja Diez}
\def\title{Notes on Quantum Field Theory} 
\usepackage{graphicx}
\usepackage[colorlinks=true, pdfstartview=FitV, linkcolor=black, citecolor=black, urlcolor=blue]{hyperref}
\usepackage{cancel}
\usepackage{mdframed,framed}
\usepackage[usenames,dvipsnames]{xcolor}
\usepackage{mathpazo}
\usepackage{layout}
\usepackage[nosort]{cite}
\usepackage{physics}

\theoremstyle{definition}
\newtheorem{ej}{Example}[section]
\newtheorem{sol}{Solution}[section]
\newtheorem{dem}{Proof}[section]
\newtheorem{prop}{Propiedad}[section]
\newtheorem{teo}{Theorem}[section]

\def\a{\alpha}
\def\b{\beta}
\def\g{\gamma}
\def\G{\Gamma}
\def\d{\delta}
%\def\D{\Delta}
%\def\e{\eta}
\def\la{\lambda}
\def\La{\Lambda}
\def\k{\kappa}
\def\m{\mu}
\def\n{\nu}
\def\r{\rho}
\def\p{\rho}
\def\o{\omega}
\def\s{\sigma}
\def\S{\Sigma}
\def\t{\tau}
\def\p{\pi}
\def\f{\phi}
\def\vf{\varphi}
\def\ep{\epsilon}
\def\th{\theta}
\def\Th{\Theta}
\def\z{\zeta}
\def\id{\mathrm{I}}
\def\M{\mathcal{M}}
\def\E{\mathcal{E}}
\def\tn{\tilde{\nabla}}
\def\TL{\text{TL}}
\def\A{\mathbb{A}}
\def\i{\mathrm{i}}
\def\M{\mathscr{M}}
\def\LL{\mathscr{L}}
\def\lag{\mathcal{L}}


%-----COLORS LIST ------
\definecolor{azure(colorwheel)}{rgb}{0.0, 0.5, 1.0}
\definecolor{DarkViolet}{RGB}{148,0,211}
\definecolor{myDarkBlue}{rgb}{0,0.1,0.7}
\definecolor{DarkBlue}{RGB}{0,0,153}
\definecolor{amber}{rgb}{1.0, 0.49, 0.0}
\definecolor{amaranth}{rgb}{0.9, 0.17, 0.31}
\definecolor{nicered}{rgb}{0.7,0.1,0.1}
\definecolor{brown}{rgb}{0.5,0.1,0.1}
\definecolor{nicegreen}{rgb}{0.0,0.3,0.0}
\definecolor{tealgreen}{rgb}{0.0, 0.51, 0.5}
\def\red#1{{\color{red} #1}}
\def\green#1{{\color{green} #1}}
\def\blue#1{{\color{blue} #1}}
\def\orange#1{{\color{orange} #1}}
%----------------------
\newcommand{\mycolor}{DarkViolet}
\def\myColor#1{{\color{\mycolor} #1}}
\definecolor{tclr}{RGB}{148,0,211}
%----------------------
\newcommand{\corr}[1]{\textcolor{nicered}{#1}}
\newcommand{\nick}[1]{\textcolor{olive}{#1}}
%\newcommand{\teo}[1]{\textcolor{azure(colorwheel)}{#1}}
\newcommand{\chteo}[2]{\corr{\st{#1}} \teo{(#2)}}
\newcommand{\bako}[1]{\textcolor{DarkViolet}{#1}}
\newcommand{\than}[1]{\textcolor{magenta}{#1}}

\newcommand{\rc}{\textcolor{red}}
\newcommand{\bc}{\textcolor{blue}}
\newcommand{\cc}{\textcolor{cyan}}
\newcommand{\gc}{\textcolor{green}}
\newcommand{\occ}{\textcolor{orange}}
\newcommand{\pc}{\textcolor{purple}}

%----------------------
\usepackage{hyperref}
\hypersetup{colorlinks,bookmarksopen,
	bookmarksnumbered,
	citecolor={nicered},
	linkcolor={black},
	urlcolor={blue},
	pdfstartview=FitH}




% set up roman fonts in text and in mathmode
%\renewcommand{\familydefault}{\rmdefault}
%\usepackage{sfmath}

% format table of contents
\usepackage{tocloft}
\setlength\cftparskip{5pt}
\setlength\cftbeforesecskip{5pt}
\setlength\cftaftertoctitleskip{2pt}

% generic pages setups
\textwidth = 6.5 in
\textheight = 9 in
\oddsidemargin = -0 in
%\evensidemargin = -0.25 in
\topmargin = -0. in
\headheight = 0.0 in
\headsep = 0.0 in
%\parskip = 0.2in
%\parindent = 0.2in
\marginparsep = 0 in
\footskip = 0.5 in

% display references as hyperlinks
\newcommand{\displayref}[2]{\href[pdfnewwindow]{#1}{{\color{BrickRed}[#2]}}}


\begin{document}

{\centering
 \vspace*{1cm}
\textbf{\LARGE{\title{}}}
\vspace{0.5cm}
\begin{center}
\author{}\\
\vspace{0.5cm}
\textit{Instituto de Ciencias Exactas y Naturales, Universidad Arturo Prat,\\
Avenida Playa Brava 3256, 1111346, Iquique, Chile.\\
Facultad de Ciencias, Universidad Arturo Prat,\\
Avenida Arturo Prat Chac\'on 2120, 1110939, Iquique, Chile.}
\end{center}
\vspace{1cm}
\begin{abstract}
\vspace{0.5cm}
These notes are based on \cite{Schwartz:2014sze} and are for personal study purposes only.
\end{abstract}}

\vspace{5cm}

\begin{center}
\textit{Please write to} borjadiez1014@gmail.com \textit{for corrections, typos, and literature suggestions.}
\end{center}


\thispagestyle{empty}

\newpage
\tableofcontents
\thispagestyle{empty}
\newpage
\clearpage
\pagenumbering{arabic} 

%\layout

%\tableofcontents
%\flushbottom

%\section{Differentiable manifolds}
\subsection{From Topological Spaces to Differentable Manifolds}
%TODO agregar imagen
It is assumed that the reader is acquainted with the notion of a topological space as a structure on which one can define a neighborhood and continuous functions. A \textbf{homeomorphism} between two topological spaces is a 1-1 map $\varphi: X\to Y$ for which both $\varphi$ and its inverse $\varphi^{-1}$ are continuous. If $\varphi$ and $\varphi^{-1}$ are continuously differentiable then $\varphi$ is called a \textbf{diffeomorphism}.

A $D$-dimensional manifold $M^D$ is a topological space that locally has the properties of a $D$-dimensional Euclidean space $\mathbb{R}^D$ : A neighborhood of a point in $M^D$
can continuously be mapped in a one-to-one way to the neighborhood of a point in
$\mathbb{R}^D$. To be more precise, introduce a \textbf{chart} $(U_\alpha,\varphi_\alpha)$ as a homeomorphism $\varphi_\alpha$ from
an open set $U_\alpha\subset M^D$ into an open set $R_\alpha\subset \mathbb{R}^D$. Two charts are compatible if the overlap maps are diffeomorphims $(\varphi_1\cdot \varphi_2\in C^\infty,\varphi_2\cdot \varphi_1^{-1}\in C^\infty)$ unless $U_1\cap U_2=\emptyset$. A set of compatible charts covering $M^D$ is called an atlas. In every chart the manifold can be equipped with a coordinate system: for $x\in M^D$ the coordinates are $x^\mu=\varphi(x)\in \mathbb{R}^D$. The naming makes it clear what one is aiming at. For instance the surface of a sphere, although not being homeomorphic to a plane, locally has enough smoothness to be mapped into an atlas. One chart is not sufficient since there will always be a point on the sphere that cannot be projected to the plane.

In manuscript will only treat finite-dimensional manifolds. One possibility of extending the notion of manifolds to infinite dimensions is to consider Banach manifolds modeled on Banach spaces. It is also assumed that we are dealing with $C^\infty$ manifolds. In certain contexts it might suffice that the charts are $C^k$-related. Also complex manifolds are investigated in mathematics and applied to modern theoretical physics (catchword: Kähler manifolds). In these the transition functions are required to be analytic.

\subsection{Tensor Bundles}
On a manifold one can erect tensor bundles as “superstructures” by starting with defining the tangent and cotangent spaces of a manifold.

\subsubsection{Tangent Bundle and Vector Fields}
We are interested in the notion of vectors on a manifold $M$ (henceforth I will mostly drop the index for the dimension of the manifold and for the Euclidean space). The idea is to introduce these as tangent vectors of curves 'through' $x\in M$: A curve through a point $x$ is a smooth mapping of an interval $I=[0,1]\subset\mathbb{R}$ to the manifold:
\begin{equation}
  C=\mathbb{I}\to M\qquad t\mapsto C(t)\qquad \mbox{with}\qquad C(0)=x
\end{equation}
The coordinates of this curve are $x^\mu(C(t))$, and the tangent vector to this curve is
\begin{equation}
  \dv{t}x^\mu(C(t))
\end{equation}
Since one can have more then one curve with $C(0)=x$, the proper definition is: A \textit{tangent vector} $x\in M$ is an equivalence class of curves in $M$, where the equivalence relation between two curves is that they are tangent at the point $x$. Another-equivalent- definition is to understand a tangent vector as a directional derivative: Consider functions $f\in \mathcal{F}M$, that is $f:M\to \mathbb{R}$. The change of $f$ along a curve is given by
\begin{equation}
  \dv{t}f(C(t)),\qquad \mbox{locally}\qquad \pdv{x^\mu}f\dv{x^\mu(C(t))}{t}
\end{equation}

In defining
\begin{equation}
  X=(X^\mu\partial_\mu)\qquad \mbox{with}\qquad X^\mu =\dv{x^\mu(C(t))}{t}
\end{equation}
we can write $\dv{t}f(C(t))=Xf$. For every point along the curve we take this expression to define the differential operator $X_x$ as the tangent vector to the manifold in $x\in M$. All tangent vectors at a point in the manifold can be shown to build a vector space $\mathfrak{X}_xM$ isomorphic to $\mathbb{R}^D$. The natural basis in $\mathfrak{X}_xM$ is the coordinate or \textbf{holonomic} basis $\{\partial_\mu\}$. But of course any other (\textbf{anholonomic}) basis $\{e_I}
\}$





























\section{Classical Field Theory}
Quantum field theory is just quantum mechanics with an infinite number of oscillators

\subsection{Hamiltonians and Lagrangians}
A classical field theory is just a mechanical system with a continuous set of degrees of freedom. Field theories can be defined in terms of either a Hamiltonian or a Lagrangian, which we often write as integrals over all space of Hamiltonian or Lagrangian densities:
\begin{equation}
	H=\int\dd^3x\mathcal{H},\quad L=\int\dd^3x\mathcal{L}.
\end{equation}

Formally, the \textit{Hamiltonian} (density) is a functional of fields and their conjugate momenta $\mathcal{H}[\f,\p ]$. The \textit{Lagrangian} (density) is the Legendre transform of the Hamiltonian (density). Formally, it is defined as
\begin{equation}
  \mathcal{L}[\f,\dot{\f }]=\p [\f,\dot{\f }]\dot{\f }-\mathcal{H}[\f,\p[\f\dot{\f }]],
\end{equation}
where $\dot{\f }=\partial_t\f $ and $\p [\f,\dot{\f }]$ is implicitly defined by $\pdv*{\mathcal{H}[\f,\p ]}{\p }=\dot{\f }$. The inverse transform is
\begin{equation}
  \mathcal{H}[\f,\p ]=\p \dot{\f }[\f,\p ]-\mathcal{L}[\f,\dot{\f }[\f,\p ]],
\end{equation}
where $\dot{\f }[\f,\p ]$ is implicitly defined by $\pdv*{\mathcal{L}[\f,\dot{\f }]}{\dot{\f }}=\p $.



\subsection{Euler-Lagrange equations}
In quantum field theory, we will almost exclusively use Lagrangians. The simplest reason for this is that Lagrangians are manifestly Lorentz invariant. Dynamics for a Lagrangian system are determined by the principle of least action. The \textit{action} is the integral over time of the Lagrangian:
\begin{equation}
  S=\int\dd tL=\int\dd^4x\lag(x).
\end{equation}
Say we have a Lagrangian $\lag[\f,\partial_\m\f ]$ that is a functional only of a field $\f$ and its first derivatives. Now we imagine varying $\f\to\f+\d\f $ where $\d\f $ can be any field. Then, 
\begin{align}
  \d S&=\int\dd^4x\left[\pdv{\lag}{\f }+\pdv{\lag}{(\partial_\m\f )}\d(\partial_\m\f )\right] \nonumber\\
  &=\int\dd^4x\left\{\left[\pdv{\lag}{\f}-\partial_\m \pdv{\lag}{(\partial_\m\f )}\right]\d\f +\partial_\m \left[\pdv{\lag}{(\partial_\m \f )}\d\f \right]\right\}. \label{3.13}
\end{align}
We have integred by parts in order to factorize by $\d\f$. The last term is a total derivative and therefore its integral only depends on the field values at spatial and temporal infinity. We will always make the physical assumption that our fields vanish on these asymptotic boundaries, which lets us drop such total derivatives from Lagrangians.

In classical field theory, just as in classical mechanics, the equations of motion are deter- mined by the principle of least action: when the action is evaluated on fields that satisfy the equations of motion, it should be insensitive to small variations of those fields, $\frac{\d S}{\d\f }=0$. If this holds for all variations, then \eqref{3.13} implies
\begin{equation}\label{3.15}
  \pdv{\lag}{\f}-\partial_\m \pdv{\lag}{(\partial_\m\f )}=0.
\end{equation}
The are the celebrated \textit{Euler-Lagrange equations}. They give the \textit{equations of motion} following from a Lagrangian.

\begin{ej}
	If our action is
	\begin{equation}
  S=\int\dd^4x\left[\frac{1}{2}(\partial_\m\f)(\partial^\m\f)  -V[\f ]\right],
\end{equation}
then
\begin{equation}
  \pdv{\lag}{\f}=-V'[\f],\quad \pdv{\lag}{(\partial_\m\f )}=\partial^\m\f \to \partial_\m \pdv{\lag}{(\partial_\m\f )}=\Box \f .
\end{equation}
So then, the equations of motion \eqref{3.15} are
\begin{equation}
  \Box \f+V'[\f]=0.
\end{equation}
where $\Box:=\partial^2$ is the d'Alambertian. In particular, if $\lag =\frac{1}{2}(\partial\f )^2-\frac{1}{2}m^2\f^2$, the equations of motion are
\begin{equation}
  (\Box +m^2)\f =0.
\end{equation}
This is known as the \textit{Klein-Gordon equation} which describes the equations of motion for a free scalar field.
\end{ej}
Why do we restrict to Lagrangians of the form $\lag[\f,\partial_\m\f]$? First of all, this is the form that all "classical" Lagrangians had. If only first derivatives are involved, boundary conditions can be specified by initial positions and velocities only, in accordance with Newton's laws. In the quantum theory, if kinetic terms have too many derivatives, for example $\lag=\f\Box^2\f $, there will generally be disastrous consequences. For example, there may be states with negative energy or negative norm, permitting the vacuum to decay. But interactions with multiple derivatives may occur. Actually, they must occur due to quantum effects in all but the simplest renormalizable field theories; for example, they are generic in all effective field theories.




\subsection{Noether's theorem}
It may happen that a Lagrangian is invariant under some special type of variation $\f\to \f+\d\f  $. For exmaple, a Lagrangian for a complex field $\f $ is
\begin{equation}
  \lag = |\partial_\m \f |^2-m^2|\f |^2.
\end{equation}
This Lagrangian is invariant under $\f\to e^{-i\a }\f$ for any $\a\in\mathbb{R}$. This transformation is a \textit{symmetry} of the Lagrangian. There are two independents degrees of fredoom in a complex field $\f$, which we can take as $\f=\f_1+i\f_2 $ or more conveniently $\f$ and $\f^*$. Then the Lagrangian is
\begin{equation}
  \lag =(\partial_\m\f )(\partial^\m \f^*)-m^2\f\f^*,
\end{equation}
and the symmetry transformations are
\begin{equation}
  \f\to e^{-i\a }\f,\quad \f^*\to e^{i\a }\f^*.
\end{equation}
We note that 
\begin{equation}
  \pdv{\lag}{\f }=-m^2\f^*,\quad \pdv{\lag}{(\partial_\m\f )}=\partial^\m \f^* \to \partial_\m \left(\pdv{\lag}{(\partial_\m\f )}\right)=\Box \f^* ,
\end{equation}
\begin{equation}
  \pdv{\lag}{\f^* }=-m^2\f,\quad \pdv{\lag}{(\partial_\m\f^* )}=\partial^\m \f \to \partial_\m \left(\pdv{\lag}{(\partial_\m\f^* )}\right)=\Box \f.
\end{equation}
Thus, the equation of motion for this Lagrangian are given by
\begin{equation}
  (\Box +m^2)\f=0,\qquad (\Box +m^2)\f^*=0.
\end{equation}
When there is such a symmetry that depends on some parameter $\a$ that can be taken small (that is, the symmetry is \textit{continuous}), we find, that
\begin{equation}\label{3.22}
  0=\frac{\d \lag}{\d\a }=\sum_n\left\{\left[\pdv{\lag}{\f_n}-\partial_\m \pdv{\lag}{(\partial_\m \f_n )}\right]\frac{\d\f_n}{\d\a }+\partial_\m \left[\pdv{\lag}{(\partial_\m \f_n)}\frac{\d\f_n}{\d\a }\right]\right\},
\end{equation}
where $\f_n$ may be whatever set of fields the Lagrangian depends on. In constrast with \eqref{3.13}, this equation holds even for fields configurations $\f_n $ for which the action is not extremal (i.e. for $\f_n$ that do not satisfy the EOM), since the variation corresponds to a symmetry.

When the EOM \textit{are} satisfied, the \eqref{3.22} reduces to $\partial_\m J^\m=0$, where
\begin{equation}\label{3.23}
	J^\m =\sum_n\pdv{\lag}{(\partial_\m\f_n )}\frac{\d\f_n }{\d\a }.
\end{equation}
This is known as a \textit{Noether current}.












































% Bibliography

%% [A] Recommended: using JHEP.bst file
%% \bibliographystyle{JHEP}
%% \bibliography{biblio.bib}

%% or
%% [B] Manual formatting (see below)
%% (i) We suggest to always provide author, title and journal data or doi:
%% in short all the informations that clearly identify a document.
%% (ii) please avoid comments such as "For a review'', "For some examples",
%% "and references therein" or move them in the text. In general, please leave only references in the bibliography and move all
%% accessory text in footnotes.
%% (iii) Also, please have only one work for each \bibitem.



\newpage
\bibliographystyle{JHEP}
\bibliography{biblio.bib}
\end{document}
