\chapter{First order formalism}
Let us consider the metric ansatz constructed as the $U(1)$ fibration of Einstein-K\"ahler manifolds
\begin{equation}
   \dd{s^2}=g_{\mu\nu}\dd x^\mu \dd x^\nu = \frac{\dd{r^2}}{f(r)}+f(r)h(r)(\dd{\tau}+\B)^2+N(r)\dd{\Sigma}^2\,,
\end{equation}
where $\B=\B_i\dd \bar{x}^{i}$ denotes the K\"ahler potential $1$-form that defines the symplectic structure $\Omega=\dd\B $ associated to the $(2m-2)$-dimensional Einstein-K\"ahler trnasverse with line element
\begin{equation}
  \dd\Sigma^2 = g_{ij}(\bar{x})\dd \bar{x}^{i}\dd \bar{x}^j
\end{equation}
Here, barred coordinates with Latin indices represent those that belong to the transverse section. Thus, coordinates with greek indices can be $x^\mu=\{r,t,\bar{x}^i\}$.

It is straightforward to see that the non-zero metric components are given by
\begin{align}
  g_{rr}=\frac{1}{f},\quad g_{tt}=fh,\quad g_{ti}=fh\B_i,\quad g_{ij}=fh\B_i\B_j+N\bar{g}_{ij}
\end{align}
In order to find the components of the inverse metric, we use the fact that $g^{\m\n }g_{\n\lambda }=\d^\m_\lambda$, resulting in the following system of equations
\begin{equation}
\left\{
\begin{array}{lll}
  1&=g^{rr}g_{rr}\\
  1&=fh(g^{tt}+g^{ti}\B_i)\\
  \d^{i}_j&=fh\B_j(g^{ti}+g^{ik}\B_k)+Ng^{ik}\bar{g}_{kj}\\
  0&=fh\B_i(g^{tt}+g^{tk}\B_k)+Ng^{tk}\bar{g}_{ki}
\end{array}
\right.
\end{equation}
whose solution is given by
\begin{align}
  g^{rr}&=1\\
  g^{tt}&=\frac{\B_i\B_j}{N}\bar{g}^{ij}+\frac{1}{fh}\\
  g^{ti}&=-\frac{1}{N}\B_j\bar{g}^{ij}\\
  g^{ij}&=\frac{1}{N}\bar{g}^{ij}
\end{align}
We define the orthonormal noncoordinate frame basis of the metric () as
\begin{align}
    e^0&=\frac{\dd r}{\sqrt{f(r)}}\,,& e^1&=\sqrt{f(r)h(r)}(\dd\tau+\B)\,,& e^{A}&=\sqrt{N(r)}\,\bar{e}^{A}\,,
\end{align}
\begin{align}
    \omega^{01}&=-\frac{(fh)'}{2h\sqrt{f}}e^1\,, & 
    \omega^{0i}&=-\frac{\sqrt{f}N'}{2N}e^{i}\,, &
    \omega^{1i}&=\frac{\sqrt{fh}}{2N}\Omega^{i}{}_je^j\,, &
    \omega^{ij}&=\bar{\omega}^{ij}-\frac{\sqrt{fh}}{2N}\Omega^{ij}e^1\,.
\end{align}

\begin{align}
  E_0&=\sqrt{f}\partial_r\\
  E_1&=\frac{1}{\sqrt{fh}}\partial_t\\
  E_A&=-\frac{1}{\sqrt{N}}B_j\bar{E}^j_A\partial_t +\frac{1}{\sqrt{N}}\bar{E}^j_A\partial_j
\end{align}

\begin{align}
  R^{rt}_{rt}&=-\frac{1}{2\sqrt{h}}\left[\frac{(fh)'}{\sqrt{h}}\right]'\\
  	R^{ij}_{rt}&=-\frac{1}{2N}\left[\frac{fh}{N}\right]'\Omega^{ij}\\
  	R^{rt}_{ij}&=-\frac{N}{2h}\left[\frac{fh}{N}\right]'\Omega_{ij}\\
  R^{ri}_{rj}&=-\frac{1}{2}\sqrt{\frac{f}{N}}\left[\frac{\sqrt{f}N'}{\sqrt{N}}\right]'\d^i_j\\
  R^{\tau i}_{\tau j}&=\frac{fh}{4N^2}\Omega_{k}^{~i}\Omega^k_{~j}-\frac{N'(fh)'}{4Nh}\delta^{i}_j\\
  R^{ri}_{tj}&=-\frac{f}{4}\left[\frac{fh}{N}\right]'\Omega^{i}_{~j}\\
  R^{\tau i}_{rj}&=\frac{1}{4fh}\left[\frac{fh}{N}\right]'\Omega^{i}_{~j}\\
  R^{ij}_{kl}&=\frac{1}{N}\bar{R}^{ij}_{kl}-\frac{fN'^2}{2N^2}\delta^{i}_{[k}\delta^j_{l]} -\frac{fh}{4N^2}\left(\Omega^{ij}\Omega_{kl}+4\Omega^{[i}_{~[k}\Omega^{j]}_{~l]}\right)\\
     R^{ij}_{\tau k}&\,\rc{\stackrel{?}{=}0}\\
     R^{ri}_{r\tau}&=0
\end{align}

\section{Connection components}
\begin{align}
  \G^r_{~\t\t }&=-\frac{1}{2}f(fh)'\,\checkmark\\
  \G^r_{~rr}&=-\frac{f'}{2f}\,\checkmark\\
  \G^{r}_{~ij }&=-\frac{f(fh)'}{2}\B_i\B_j-\frac{fN'}{2}\bar{g}_{ij}\\
  \G^{r}_{\tau i}&=-\frac{f}{2}B_i(fh)'\\
  \G^\t_{~\t r}&=\frac{1}{2}\frac{(fh)'}{fh}\\
  \G^\tau_{~\t i}&=-\frac{fh}{2N}B^k\Omega_{ik}\\
    \G^\tau_{~ri}&=\frac{B_i}{2}\left[\frac{(fh)'}{fh}-\frac{N'}{N}\right]=\frac{B_i}{2}\dv{r}\left[\ln(\frac{fh}{N})\right]\\
   \G^\tau_{ij}&=\partial_{(i}B_{j)}-B_k\bar{\G}^k_{~ij}+\frac{fh}{N}\Omega_{k(i}\B_{j)}\B^k\\
  \G^{i}_{~\t j}&=-\frac{fh}{2N}\Omega^{i}_{~j}\\
  \G^{i}_{~rj}&=\frac{1}{2}\frac{N'}{N}\d^{i}_j\\
  \G^{i}_{~jk}&=\bar{\G}^{i}_{~jk}+\frac{fh}{N}\B_{(j}\Omega_{k)}^{~i}
\end{align}

where $\Omega_{ij}=2\partial_{[i}B_{j]}$.

\section{Curvature components}
\begin{align}
  R^r_{~trt}&=\partial_r\left[-\frac{1}{2}f(fh)'\right]+\frac{f'(fh)'}{4}+\frac{(fh)'^2}{4h}
\end{align}


\section{Invariant of the transverse section}
We consider
\begin{equation}
  \bar{\mathcal{L}}^{(1)}=\bar{R}
\end{equation}
For the case of $2$-sphere products, we have
\begin{align}
  (\mathbb{S}^2)^1&:\mathcal{L}^{(1)}=2\\
  (\mathbb{S}^2)^2&:\mathcal{L}^{(1)}=4\\
  (\mathbb{S}^2)^3&:\mathcal{L}^{(1)}=6\\
  (\mathbb{S}^2)^4&:\mathcal{L}^{(1)}=8\\
  \vdots\\
  (\mathbb{S}^2)^k&:\mathcal{L}^{(1)}=2k
\end{align}
For the case of $\mathbb{CP}^k$, we have
\begin{align}
  \mathbb{CP}^1&:\mathcal{L}^{(1)}=2\\
  \mathbb{CP}^2&:\mathcal{L}^{(1)}=4\\
  \mathbb{CP}^3&:\mathcal{L}^{(1)}=6\\
  \vdots\\
  \mathbb{CP}^k&:\mathcal{L}^{(1)}=2k
\end{align}

For the case of $(\mathbb{H}^2)^k$ we obtain $\bar{R}=-2k$. 

Now, we consider $$\bar{\mathcal{L}}^{(2)}=\bar{R}^{\m\n }_{\lambda\rho}\bar{R}_{\m\n }^{\lambda\r }-4\bar{R}^\m_\n \bar{R}^\n_\m +\bar{R}^2$$ and we obtain 
For the case of $2$-sphere products, we have
\begin{align}
  (\mathbb{S}^2)^1&:\mathcal{L}^{(2)}=0\\
  (\mathbb{S}^2)^2&:\mathcal{L}^{(2)}=8\\
  (\mathbb{S}^2)^3&:\mathcal{L}^{(2)}=24\\
  (\mathbb{S}^2)^4&:\mathcal{L}^{(2)}=48\\
  (\mathbb{S}^2)^5&:\mathcal{L}^{(2)}=80\\
  \vdots\\
  (\mathbb{S}^2)^k \text{ y } (\mathbb{H}^2)^k&:\mathcal{L}^{(2)}=4k(k-1)
\end{align}

For the case of $\mathbb{CP}^k$, we have
\begin{align}
  \mathbb{CP}^1&:\mathcal{L}^{(2)}=0\\
  \mathbb{CP}^2&:\mathcal{L}^{(2)}=\frac{16}{3}\\
  \mathbb{CP}^3&:\mathcal{L}^{(2)}=18=\frac{72}{4}\\
  \mathbb{CP}^4&:\mathcal{L}^{(2)}=\frac{192}{5}\\
  \mathbb{CP}^5&:\mathcal{L}^{(2)}=\frac{200}{3}=\frac{400}{6}\\
  \mathbb{CP}^6&:\mathcal{L}^{(2)}=\frac{200}{3}=\frac{720}{7}\\
  \vdots\\
  \mathbb{CP}^k \text{ y } \mathbb{CH}^k&:\mathcal{L}^{(2)}=\frac{4k^2(k-1)}{k+1}
\end{align}


Now, we consider $$\bar{\mathcal{L}}^{(3)}$$
For the case of $2$-sphere products, we have
\begin{align}
  (\mathbb{S}^2)^1&:\mathcal{L}^{(1)}=0\\
  (\mathbb{S}^2)^2&:\mathcal{L}^{(1)}=0\\
  (\mathbb{S}^2)^3&:\mathcal{L}^{(1)}=48\\
  (\mathbb{S}^2)^4&:\mathcal{L}^{(1)}=192\\
  (\mathbb{S}^2)^5&:\mathcal{L}^{(1)}=480\\
  (\mathbb{S}^2)^6&:\mathcal{L}^{(1)}=960\\
  \vdots\\
  (\mathbb{S}^2)^k&:\mathcal{L}^{(3)}=8k(k-1)(k-2)=8(k-1)[(k-1)^2-1]
\end{align}

For the case of $2$-hyperboloids products, we have
\begin{align}
  (\mathbb{H}^2)^1&:\mathcal{L}^{(1)}=0\\
  (\mathbb{H}^2)^2&:\mathcal{L}^{(1)}=0\\
  (\mathbb{H}^2)^3&:\mathcal{L}^{(1)}=-48\\
  (\mathbb{H}^2)^4&:\mathcal{L}^{(1)}=-192\\
  \vdots\\
  (\mathbb{H}^2)^k&:\mathcal{L}^{(3)}=-8k(k-1)(k-2)=-8(k-1)[(k-1)^2-1]
\end{align}

For the case of $\mathbb{CP}^k$ products, we have
\begin{align}
  \mathbb{CP}^1&:\mathcal{L}^{(3)}=0\\
\mathbb{CP}^2&:\mathcal{L}^{(3)}=0\\
\mathbb{CP}^3&:\mathcal{L}^{(3)}=18=\frac{288}{16}\\
\mathbb{CP}^4&:\mathcal{L}^{(3)}=\frac{2304}{25}\\
\mathbb{CP}^5&:\mathcal{L}^{(3)}=\frac{800}{3}=\frac{9600}{36}\\
\mathbb{CP}^6&:\mathcal{L}^{(3)}=\frac{28800}{49}\\
  \vdots\\
  \mathbb{CP}^k&:\mathcal{L}^{(3)}=\frac{8k^2(k-1)^2(k-2)}{(k+1)^2}
\end{align}

For the case of $\mathbb{CH}^k$ products, we have
\begin{equation}
    \mathbb{CH}^k:\mathcal{L}^{(3)}=-\frac{8k^2(k-1)^2(k-2)}{(k+1)^2}
\end{equation}





\section{Kheler forms squared}
For $(\mathbb{S}^2)^k$ and $(\mathbb{H}^2)^k$, $\Omega^2=2k$.

For $\mathbb{CP}^k$ and $\mathbb{CH}^k$, $\Omega^2=2k$.

\section{Base manifold Pontryagin}
Let be $\Omega=\dd \mathcal{B}$ the simplectic $2$-form of the Einstein-Khaler manifold. For the case the $\mathbb{CP}^2$ we have
\begin{equation}
	\tilde{\Omega}_{\m\n}\Omega^{\m\n }=4
\end{equation}
where $\tilde{\Omega}_{\m\n}=\frac{1}{2}\epsilon_{\m\n \a\b}\Omega^{\a\b }$ is the dual of $\Omega_{\m\n }$.
































