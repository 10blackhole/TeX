\newpage
\chapter{Eguchi-Hanson}
\section{Eguchi-Hanson Einstein gravity}
Los polinomios de Wheeler encontrados son los siguientes:

En $4 D$ para base manifold $(\mathbb{T}^2)$,$(\mathbb{S}^2)$, $(\mathbb{H}^2)$, $\mathbb{CP}$ y $\mathbb{CH}$ respectivamente son
\begin{equation*}
    r^4f(r)+\frac{\Lambda}{r}r^6+a^4=0
\end{equation*}
\begin{equation*}
    r^4f(r)+\frac{\Lambda}{6}r^6-r^4+a^4=0
\end{equation*}
\begin{equation*}
    r^4f(r)+\frac{\Lambda}{6}r^6+r^4+a^4=0
\end{equation*}
\begin{equation*}
    r^4f(r)+\frac{\Lambda}{6}r^6-r^4+a^4=0
\end{equation*}
\begin{equation*}
    r^4f(r)+\frac{\Lambda}{6}r^6+r^4+a^4=0
\end{equation*}

En $6 D$ para base manifold $(\mathbb{T}^2)^2$,$(\mathbb{S}^2)^2$, $(\mathbb{H}^2)^2$, $\mathbb{CP}^2$ y $\mathbb{CH}^2$ respectivamente son
\begin{equation*}
    r^6f(r)+\frac{\Lambda}{16}r^8+a^6=0
\end{equation*}
\begin{equation*}
    r^6f(r)+\frac{\Lambda}{16}r^8-\frac{2}{3}r^6+a^6=0
\end{equation*}
\begin{equation*}
    r^6f(r)+\frac{\Lambda}{16}r^8+\frac{2}{3}r^6+a^6=0
\end{equation*}
\begin{equation*}
    r^6f(r)+\frac{\Lambda}{16}r^8-\frac{2}{3}r^6+a^6=0
\end{equation*}
\begin{equation*}
    r^6f(r)+\frac{\Lambda}{16}r^8+\frac{2}{3}r^6+a^6=0
\end{equation*}

En $8 D$ para base manifold $(\mathbb{T}^2)^3$,$(\mathbb{S}^2)^3$, $(\mathbb{H}^2)^3$, $\mathbb{CP}^3$ y $\mathbb{CH}^3$ respectivamente son
\begin{equation*}
    r^8f(r)+\frac{\Lambda}{30}r^{10}+a^8=0
\end{equation*}
\begin{equation*}
    r^8f(r)+\frac{\Lambda}{30}r^{10}-\frac{1}{2}r^8+a^8=0
\end{equation*}
\begin{equation*}
    r^8f(r)+\frac{\Lambda}{30}r^{10}+\frac{1}{2}r^8+a^8=0
\end{equation*}
\begin{equation*}
    r^8f(r)+\frac{\Lambda}{30}r^{10}-\frac{1}{2}r^8+a^8=0
\end{equation*}
\begin{equation*}
    r^8f(r)+\frac{\Lambda}{30}r^{10}+\frac{1}{2}r^8+a^8=0
\end{equation*}

En $10 D$ para base manifold $(\mathbb{T}^2)^4$,$(\mathbb{S}^2)^4$, $(\mathbb{H}^2)^4$, $\mathbb{CP}^4$ y $\mathbb{CH}^4$ respectivamente son
\begin{equation*}
    r^{10}f(r)+\frac{\Lambda}{48}r^{12}+a^8=0
\end{equation*}
\begin{equation*}
    r^{10}f(r)+\frac{\Lambda}{48}r^{12}-\frac{2}{5}r^{10}+a^8=0
\end{equation*}
\begin{equation*}
    r^{10}f(r)+\frac{\Lambda}{48}r^{12}+\frac{2}{5}r^{10}+a^8=0
\end{equation*}
\begin{equation*}
    r^{10}f(r)+\frac{\Lambda}{48}r^{12}-\frac{2}{5}r^{10}+a^8=0
\end{equation*}
\begin{equation*}
    r^{10}f(r)+\frac{\Lambda}{48}r^{12}+\frac{2}{5}r^{10}+a^8=0
\end{equation*}

En $12 D$ para base manifold $(\mathbb{S}^2)^5$
\begin{equation*}
    r^{12}f(r)+\frac{\Lambda}{70}r^{14}-\frac{1}{3}r^{12}+a^{12}=0
\end{equation*}

Para $d$ dimensiones el polinomio de Wheeler asociado es
\begin{tcolorbox}
    \begin{equation*}
        r^df(r)+\frac{2\Lambda}{d^2-4}r^{d+2}-\frac{4}{d}\gamma r^d+a^d=0
    \end{equation*}
        \begin{equation*}
        \textcolor{purple}{r^{2m}f(r)+\frac{\Lambda r^{2(m+1)}}{2(m^2-1)}-\frac{2\gamma r^{2m}}{m}+a^{2m}}=0
    \end{equation*}
    \begin{equation*}
        \left(f(r)-\frac{4\gamma}{d}\right)r^d+\frac{2\Lambda}{d^2-4}r^{d+2}+a^d=0
    \end{equation*}
    \begin{equation*}
        \left(f(r)-\frac{2\gamma}{m}\right)r^{2m}+\frac{\Lambda}{2(m^2-1)}r^{2(m+1)}+a^{2m}=0
    \end{equation*}
\end{tcolorbox}

















La función métrica que resuelve
\begin{equation}
    G_{\mu\nu}+\Lambda g_{\mu\nu}=0
\end{equation}
para dimensión $d$ es
\begin{equation}
    f(r)=-\frac{2\Lambda r^2}{d^2-4}-\left(\frac{a}{r}\right)^d+\frac{4\gamma}{d}
\end{equation}
donde $\gamma=\pm 1,0$  dependiendo de la curvatura constante del base manifold.