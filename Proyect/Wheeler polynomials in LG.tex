\documentclass[a4paper,11pt]{report}
\usepackage{jheppub} % for details on the use of the package, please see the JINST-author-manual
\usepackage{lineno}
\usepackage{amsmath,amsthm,amsfonts,amssymb,amscd,physics,cancel,mathtools}
\usepackage{tcolorbox}
\usepackage{marginnote,tensor,tikz}
%~~~~~~~~~ Document setup
\usepackage[spanish]{babel} % English formatting
\usepackage[utf8]{inputenc} % Standard encoding
%\usepackage[a4paper,left=3cm,bottom=3cm]{geometry} % Page formatting
\usepackage{indentfirst} % Indents the first paragraph
\usepackage{amsmath} % Maths type package
\usepackage{bm} % Bold font maths
\usepackage{graphicx} % Advanced graphics package
\usepackage[export]{adjustbox} 
\usepackage{pdflscape} % Make pages landscape
\usepackage{fancyhdr} % Fancy headers
%\usepackage[colorlinks=true,citecolor=blue,urlcolor=blue,linkcolor=black]{hyperref} % Link colours
%\usepackage{natbib} % Bibliography
\usepackage{flafter} % Reference any 'float'
\usepackage[framemethod=tikz]{mdframed} % Box off stuff
\usepackage{color} % Colour support
\usepackage{wrapfig} % Text flowing around figures
\usepackage{lipsum} % Generates meaningless text
% \usepackage{biblatex}
% \addbibresource{sample.bib}

\theoremstyle{definition}
\newtheorem{ej}{Ejemplo}[section]
\newtheorem{prop}{Propiedad}[section]
\newtheorem{teo}{Teorema}[section]

\def\a{\alpha}
\def\b{\beta}
\def\g{\gamma}
\def\G{\Gamma}
\def\d{\delta}
%\def\D{\Delta}
%\def\e{\eta}
\def\la{\lambda}
\def\La{\Lambda}
\def\k{\kappa}
\def\m{\mu}
\def\n{\nu}
\def\r{\rho}
\def\p{\rho}
\def\o{\omega}
\def\s{\sigma}
\def\S{\Sigma}
\def\t{\tau}
\def\p{\pi}
\def\f{\phi}
\def\vf{\varphi}
\def\ep{\epsilon}
\def\th{\theta}
\def\Th{\Theta}
\def\z{\zeta}
\def\id{\mathrm{I}}
\def\M{\mathcal{M}}
\def\E{\mathcal{E}}
\def\tn{\tilde{\nabla}}
\def\TL{\text{TL}}
\def\A{\mathbb{A}}

\newcommand{\e}{\mathrm{e}}
\newcommand{\I}{\mathrm{I}}
\newcommand{\C}{\mathcal{C}}
\newcommand{\D}{\mathrm{D}}
\newcommand{\oo}{\mathring}
\newcommand{\oomega}{\mathring{\omega}}
\newcommand{\eabc}{\epsilon_{abc}}

\newcommand{\B}{\mathcal{B}}

\newcommand{\rc}{\textcolor{red}}
\newcommand{\bc}{\textcolor{blue}}
\newcommand{\cc}{\textcolor{cyan}}
\newcommand{\gc}{\textcolor{green}}
\newcommand{\occ}{\textcolor{orange}}
\newcommand{\pc}{\textcolor{purple}}
% \linenumbers



%\arxivnumber{1234.56789} % if you have one

\title{\boldmath Polinomios de Wheeler en la teoría de Lovelock}

% Collaborations

%% [A] If main author
%% \collaboration{\includegraphics[height=17mm]{collabroation-logo}\\[6pt]
%%  XXX collaboration}

%% or
%% [B] If "on behalf of"
%% \collaboration[c]{on behalf of XXX collaboration}


% Authors
% The "\note" macro will give a warning: "Ignoring empty anchor...", you can safely ignore it.

%% [A] simple case: 2 authors, same institution
%% \author[1]{A. Uthor\note{Corresponding author.}}
%% \author{and A. Nother Author}
%% \affiliation{Institution,\\Address, Country}

%% or, e.g.
%% [B] more complex case: 4 authors, 3 institutions, 2 footnotes
%% \author[a,b]{F. Irst,\note{Now at another university}}
%% \author[c]{S. Econd,}
%% \author[a,2]{T. Hird\note{Also at Some University.}}
%% \author[c,2]{and Fourth}
%% \affiliation[a]{Institution_1,\\Address, Country}
%% \affiliation[b]{Institution_2,\\Address, Country}
%% \affiliation[c]{Institution_3,\\Address, Country}

\author{Borja Diez B.}
\affiliation{Universidad Arturo Pratt, Iquique, Chile}
%\affiliation{Another University,\\
%different-address, Country}

% E-mail addresses: only for the corresponding author
\emailAdd{borjadiez1014@gmail.com}

\abstract{En estas notas iré recopilando los resultados más importantes de mis cálculos de mi tesis de magister. En particular, iré recopilando los polinomios de Wheeler encontrados en diferenter ordenes en la teoría de Lovelock para métricas tipo Eguchi-Hanson con diferentes base manifolds con topología no trivial.}



\begin{document}
\maketitle
\flushbottom
\newpage
\chapter{Eguchi-Hanson}
\section{Eguchi-Hanson Einstein gravity}
Los polinomios de Wheeler encontrados son los siguientes:

En $4 D$ para base manifold $(\mathbb{T}^2)$,$(\mathbb{S}^2)$, $(\mathbb{H}^2)$, $\mathbb{CP}$ y $\mathbb{CH}$ respectivamente son
\begin{equation*}
    r^4f(r)+\frac{\Lambda}{r}r^6+a^4=0
\end{equation*}
\begin{equation*}
    r^4f(r)+\frac{\Lambda}{6}r^6-r^4+a^4=0
\end{equation*}
\begin{equation*}
    r^4f(r)+\frac{\Lambda}{6}r^6+r^4+a^4=0
\end{equation*}
\begin{equation*}
    r^4f(r)+\frac{\Lambda}{6}r^6-r^4+a^4=0
\end{equation*}
\begin{equation*}
    r^4f(r)+\frac{\Lambda}{6}r^6+r^4+a^4=0
\end{equation*}

En $6 D$ para base manifold $(\mathbb{T}^2)^2$,$(\mathbb{S}^2)^2$, $(\mathbb{H}^2)^2$, $\mathbb{CP}^2$ y $\mathbb{CH}^2$ respectivamente son
\begin{equation*}
    r^6f(r)+\frac{\Lambda}{16}r^8+a^6=0
\end{equation*}
\begin{equation*}
    r^6f(r)+\frac{\Lambda}{16}r^8-\frac{2}{3}r^6+a^6=0
\end{equation*}
\begin{equation*}
    r^6f(r)+\frac{\Lambda}{16}r^8+\frac{2}{3}r^6+a^6=0
\end{equation*}
\begin{equation*}
    r^6f(r)+\frac{\Lambda}{16}r^8-\frac{2}{3}r^6+a^6=0
\end{equation*}
\begin{equation*}
    r^6f(r)+\frac{\Lambda}{16}r^8+\frac{2}{3}r^6+a^6=0
\end{equation*}

En $8 D$ para base manifold $(\mathbb{T}^2)^3$,$(\mathbb{S}^2)^3$, $(\mathbb{H}^2)^3$, $\mathbb{CP}^3$ y $\mathbb{CH}^3$ respectivamente son
\begin{equation*}
    r^8f(r)+\frac{\Lambda}{30}r^{10}+a^8=0
\end{equation*}
\begin{equation*}
    r^8f(r)+\frac{\Lambda}{30}r^{10}-\frac{1}{2}r^8+a^8=0
\end{equation*}
\begin{equation*}
    r^8f(r)+\frac{\Lambda}{30}r^{10}+\frac{1}{2}r^8+a^8=0
\end{equation*}
\begin{equation*}
    r^8f(r)+\frac{\Lambda}{30}r^{10}-\frac{1}{2}r^8+a^8=0
\end{equation*}
\begin{equation*}
    r^8f(r)+\frac{\Lambda}{30}r^{10}+\frac{1}{2}r^8+a^8=0
\end{equation*}

En $10 D$ para base manifold $(\mathbb{T}^2)^4$,$(\mathbb{S}^2)^4$, $(\mathbb{H}^2)^4$, $\mathbb{CP}^4$ y $\mathbb{CH}^4$ respectivamente son
\begin{equation*}
    r^{10}f(r)+\frac{\Lambda}{48}r^{12}+a^8=0
\end{equation*}
\begin{equation*}
    r^{10}f(r)+\frac{\Lambda}{48}r^{12}-\frac{2}{5}r^{10}+a^8=0
\end{equation*}
\begin{equation*}
    r^{10}f(r)+\frac{\Lambda}{48}r^{12}+\frac{2}{5}r^{10}+a^8=0
\end{equation*}
\begin{equation*}
    r^{10}f(r)+\frac{\Lambda}{48}r^{12}-\frac{2}{5}r^{10}+a^8=0
\end{equation*}
\begin{equation*}
    r^{10}f(r)+\frac{\Lambda}{48}r^{12}+\frac{2}{5}r^{10}+a^8=0
\end{equation*}

En $12 D$ para base manifold $(\mathbb{S}^2)^5$
\begin{equation*}
    r^{12}f(r)+\frac{\Lambda}{70}r^{14}-\frac{1}{3}r^{12}+a^{12}=0
\end{equation*}

Para $d$ dimensiones el polinomio de Wheeler asociado es
\begin{tcolorbox}
    \begin{equation*}
        r^df(r)+\frac{2\Lambda}{d^2-4}r^{d+2}-\frac{4}{d}\gamma r^d+a^d=0
    \end{equation*}
        \begin{equation*}
        \textcolor{purple}{r^{2m}f(r)+\frac{\Lambda r^{2(m+1)}}{2(m^2-1)}-\frac{2\gamma r^{2m}}{m}+a^{2m}}=0
    \end{equation*}
    \begin{equation*}
        \left(f(r)-\frac{4\gamma}{d}\right)r^d+\frac{2\Lambda}{d^2-4}r^{d+2}+a^d=0
    \end{equation*}
    \begin{equation*}
        \left(f(r)-\frac{2\gamma}{m}\right)r^{2m}+\frac{\Lambda}{2(m^2-1)}r^{2(m+1)}+a^{2m}=0
    \end{equation*}
\end{tcolorbox}

















La función métrica que resuelve
\begin{equation}
    G_{\mu\nu}+\Lambda g_{\mu\nu}=0
\end{equation}
para dimensión $d$ es
\begin{equation}
    f(r)=-\frac{2\Lambda r^2}{d^2-4}-\left(\frac{a}{r}\right)^d+\frac{4\gamma}{d}
\end{equation}
donde $\gamma=\pm 1,0$  dependiendo de la curvatura constante del base manifold.
\newpage
\section{Eguchi-Hanson en Lovelock cuadrático}
Los polinomios de Wheeler encontrados son los siguientes:

En $6 D$ para base manifold $(\mathbb{T}^2)^2$,$(\mathbb{S}^2)^2$, $(\mathbb{H}^2)^2$, $\mathbb{CP}^2$ y $\mathbb{CH}^2$ respectivamente son
\begin{equation*}
    12\alpha f(r)^2r^4-r^6f(r)-\frac{\Lambda r^8}{16}+a^6=0
\end{equation*}
\begin{equation*}
    12\alpha f(r)^2 r^4+(-r^6-16\alpha r^4)f(r)-\frac{\Lambda r^8}{16}+\frac{2}{3}r^6+8\alpha r^4+a^6=0
\end{equation*}
\begin{equation*}
    12\alpha f(r)^2 r^4+(-r^6+16\alpha r^4)f(r)-\frac{\Lambda r^8}{16}-\frac{2}{3}r^6+8\alpha r^4+a^6=0
\end{equation*}
\begin{equation*}
    12\alpha f(r)^2 r^4+(-r^6-16\alpha r^4)f(r)-\frac{\Lambda r^8}{16}+\frac{2}{3}r^6+\frac{16}{3}\alpha r^4+a^6=0
\end{equation*}
\begin{equation*}
    12\alpha f(r)^2 r^4+(-r^6+16\alpha r^4)f(r)-\frac{\Lambda r^8}{16}-\frac{2}{3}r^6+\frac{16}{3}\alpha r^4+a^6=0
\end{equation*}

En $8 D$ para base manifold $(\mathbb{T}^2)^3$,$(\mathbb{S}^2)^3$, $(\mathbb{H}^2)^3$, $\mathbb{CP}^3$ y $\mathbb{CH}^3$ respectivamente son
\begin{equation*}
    32\alpha f(r)^2r^6-r^8f(r)-\frac{\Lambda r^{10}}{30}+a^8=0
\end{equation*}
\begin{equation*}
    32\alpha f(r)^2 r^6+(-r^8-32\alpha r^6)f(r)-\frac{\Lambda r^{10}}{30}+\frac{1}{2}r^8+\frac{32}{3}\alpha r^6+a^8=0
\end{equation*}
\begin{equation*}
    32\alpha f(r)^2 r^6+(-r^8+32\alpha r^6)f(r)-\frac{\Lambda r^{10}}{30}-\frac{1}{2}r^8+\frac{32}{3}\alpha r^6+a^8=0
\end{equation*}
\begin{equation*}
    32\alpha f(r)^2 r^6+(-r^8-32\alpha r^6)f(r)-\frac{\Lambda r^{10}}{30}+\frac{1}{2}r^8+8\alpha r^6+a^8=0
\end{equation*}
\begin{equation*}
    32\alpha f(r)^2 r^6+(-r^8+32\alpha r^6)f(r)-\frac{\Lambda r^{10}}{30}-\frac{1}{2}r^8+8\alpha r^6+a^8=0
\end{equation*}

En $10 D$ para base manifold $(\mathbb{T}^2)^4$,$(\mathbb{S}^2)^4$, $(\mathbb{H}^2)^4$, $\mathbb{CP}^4$ y $\mathbb{CH}^4$ respectivamente son
\begin{equation*}
    60\alpha f(r)^2r^8-f(r)r^{10}-\frac{\Lambda}{48}r^{12}+a^{10}=0
\end{equation*}
\begin{equation*}
    60\alpha f(r)^2r^8+(-r^{10}-48\alpha r^8)f(r)-\frac{\Lambda}{48}+\frac{2}{5}r^{10}+12\alpha r^8+a^{10}=0
\end{equation*}
\begin{equation*}
    60\alpha f(r)^2r^8+(-r^{10}+48\alpha r^8)f(r)-\frac{\Lambda}{48}-\frac{2}{5}r^{10}+12\alpha r^8+a^{10}=0
\end{equation*}
\begin{equation*}
    60\alpha f(r)^2r^8+(-r^{10}-48\alpha r^8)f(r)-\frac{\Lambda}{48}+\frac{2}{5}r^{10}+\frac{48}{5}\alpha r^8+a^{10}=0
\end{equation*}
\begin{equation*}
    60\alpha f(r)^2r^8+(-r^{10}+48\alpha r^8)f(r)-\frac{\Lambda}{48}-\frac{2}{5}r^{10}+\frac{48}{5}\alpha r^8+a^{10}=0
\end{equation*}

En $12 D$ para base manifold $(\mathbb{T}^2)^5$,$(\mathbb{S}^2)^5$, $(\mathbb{H}^2)^5$ respectivamente son
\begin{equation*}
    96\alpha f(r)^2 r^{10}-f(r)r^{12}-\frac{\Lambda}{70}r^{14}+a^{12}=0
\end{equation*}
\begin{equation*}
    96\alpha f(r)^2r^{10}+(-r^{12}-64\alpha r^{10})f(r)-\frac{\Lambda}{70}r^{14}+\frac{1}{3}r^{12}+\frac{64}{5}\alpha r^{10}+a^{12}=0
\end{equation*}
\begin{equation*}
    96\alpha f(r)^2r^{10}+(-r^{12}+64\alpha r^{10})f(r)-\frac{\Lambda}{70}r^{14}+\frac{1}{3}r^{12}+\frac{64}{5}\alpha r^{10}+a^{12}=0
\end{equation*}

En $14 D$ para base manifold $(\mathbb{S}^2)^6$, $(\mathbb{H}^2)^6$ respectivamente son
\begin{equation*}
    140\alpha f(r)^2r^{12}+(-r^{14}-80\alpha r^{12})f(r)-\frac{\Lambda}{96}r^{16}+\frac{2}{7}r^{14}+\frac{40}{2}\alpha r^{12}+a^{14}=0
\end{equation*}
\begin{equation*}
    140\alpha f(r)^2r^{12}+(-r^{14}+80\alpha r^{12})f(r)-\frac{\Lambda}{96}r^{16}-\frac{2}{7}r^{14}+\frac{40}{2}\alpha r^{12}+a^{14}=0
\end{equation*}

En $d$ dimensiones el polinomio de Wheeler asociado es 
\begin{tcolorbox}
    \begin{equation*}
        d(d-4)\alpha f(r)^2r^{d-2}+f(r)[-r^d-8(d-4)\gamma r^{d-2}]-\frac{2\Lambda}{d^2-4}r^{d+2}+\frac{4}{d}\gamma r^d+a^d+C\alpha r^{d-2}=0
    \end{equation*}
\end{tcolorbox}
o de manera equivalente, considerando $d=2m$
    \begin{equation*}
      -4m(m-2)\alpha f(r)^2r^{2(m-1)}+16f(r)\gamma (m-2)r^{2(m-1)}+\textcolor{purple}{r^{2m}f(r)+\frac{\Lambda r^{2(m+1)}}{2(m^2-1)}-\frac{2\gamma r^{2m}}{m}+a^{2m}}-C\alpha r^{2(m-1)}=0
    \end{equation*}

\begin{equation*}
    \left(f(r)-\frac{2\gamma}{m}\right)r^{2m}+\frac{\Lambda}{2(m^2-1)}r^{2(m+1)}-4m(m-2)\alpha\left(\left(f(r)-\frac{2\gamma}{m}\right)^2+\frac{4}{m^2(m-1)}\right)r^{2(m-1)}+a^{2m}=0
\end{equation*}

\begin{equation*}
    F(r)r^{2m}-4m(m-2)\alpha_1\left(F(r)^2+\frac{4}{m^2(m-1)}\right)r^{2(m-1)}+\frac{\Lambda}{2(m^2-1)}r^{2(m+1)}+a^{2m}=0
\end{equation*}
donde $F(r)=f(r)-2\gamma/m$



\newpage
\section{Eguchi-Hanson en Lovelock cúbico}
Los polinomios de Wheeler encontrados son los siguientes:

En $8 D$ para base manifold $(\mathbb{T}^2)^3$,$(\mathbb{S}^2)^3$, $(\mathbb{H}^2)^3$, $\mathbb{CP}^3$ y $\mathbb{CH}^3$ respectivamente son
\begin{equation*}
    384\alpha_3f(r)^3 r^4-32\alpha_2f(r)^2r^6+r^8f(r)+\frac{\Lambda r^{10}}{30}+a^8=0
\end{equation*}
\begin{align*}
    &384\alpha_3f(r)^3 r^4+(-32\alpha_2r^6-576\alpha_3 r^4)f(r)^2+(r^8+32\alpha_2r^6+384\alpha_3r^4)f(r)+\frac{\Lambda r^{10}}{30}-\frac{r^8}{2}-\frac{32\alpha_3 r^6}{3}\\&-128\alpha_3r^4+a^8=0
\end{align*}
\begin{align*}
    &384\alpha_3f(r)^3 r^4+(-32\alpha_2r^6+576\alpha_3 r^4)f(r)^2+(r^8-32\alpha_2r^6+384\alpha_3r^4)f(r)+\frac{\Lambda r^{10}}{30}+\frac{r^8}{2}-\frac{32\alpha_3 r^6}{3}\\&+128\alpha_3r^4+a^8=0
\end{align*}
\begin{equation*}
    384\alpha_3f(r)^3r^4+(-32\alpha_2r^6-576\alpha_3r^4)f(r)^2+(r^8+32\alpha_2r^6+288\alpha r^4)f(r)+\frac{\Lambda r^{10}}{30}-\frac{r^8}{2}-8\alpha_2r^6-48\alpha_3r^4+a^8=0
\end{equation*}
\begin{equation*}
    384\alpha_3f(r)^3r^4+(-32\alpha_2r^6+576\alpha_3r^4)f(r)^2+(r^8-32\alpha_2r^6+288\alpha r^4)f(r)+\frac{\Lambda r^{10}}{30}+\frac{r^8}{2}-8\alpha_2r^6+48\alpha_3r^4+a^8=0
\end{equation*}

En $10 D$ para base manifold $(\mathbb{T}^2)^4$,$(\mathbb{S}^2)^4$, $(\mathbb{H}^2)^4$ respectivamente son
\begin{equation*}
    1920\alpha_3f(r)^3r^6-60\alpha_2f(r)^2r^8+r^{10}f(r)+\frac{\Lambda r^{12}}{48}+a^{10}=0
\end{equation*}
\begin{align*}
    &1920\alpha_3f(r)^3r^6+(-60\alpha_2r^8-2304\alpha_3 r^6)f(r)^2+(r^{10}+48\alpha_2 r^8+1152\alpha_3 r^6)f(r)+\frac{\Lambda r^{12}}{48}-\frac{2r^{10}}{5}\\&-12\alpha_2 r^8-256\alpha_3r^6+a^{10}=0
\end{align*}
\begin{align*}
    &1920\alpha_3f(r)^3r^6+(-60\alpha_2r^8+2304\alpha_3 r^6)f(r)^2+(r^{10}-48\alpha_2 r^8+1152\alpha_3 r^6)f(r)+\frac{\Lambda r^{12}}{48}+\frac{2r^{10}}{5}\\&-12\alpha_2 r^8+256\alpha_3r^6+a^{10}=0
\end{align*}

En $12 D$ para base manifold $(\mathbb{S}^2)^5$ es
\begin{align*}
    &5760\alpha_3f(r)^3r^8+(-96\alpha_2r^{10}-5760\alpha_3r^8)f(r)^2+(r^{12}+64\alpha_2r^{10}+2304\alpha_3r^8)f(r)+\frac{\Lambda r^{14}}{70}-\frac{r^{12}}{3}-\\&\frac{64\alpha_2 r^{10}}{5}-384\alpha_3r^8+a^{12}=0
\end{align*}

El polinomio de Wheeler asociado para $d$-dimensiones donde $d=2m$ va como
\begin{align*}
    &16m(m-1)(m-2)(m-3)\alpha_3f(r)^3 r^{2(m-2)}-\textcolor{blue}{4m(m-2)\alpha_2f(r)^2r^{2(m-1)}}\\&+\textcolor{blue}{f(r)\left(r^{2m}+16\gamma(m-2)\alpha_2 r^{2(m-1)}\right)}\textcolor{blue}{+\frac{\Lambda r^{2(m+1)}}{2(m^2-1)}-\frac{2\gamma r^{2m}}{m}+a^{2m}}=0
\end{align*}

\newpage
\section{Generalizaciones}
El polinomio de Wheeler hasta Lovelock cúbico para $d=2m$ dimensiones, pareciera ser
\begin{equation*}
\begin{split}
    &a^{2m}+\frac{\Lambda r^{2(m+1)}}{2(m^2-1)}+\textcolor{red}{\left(f(r)-\frac{2\gamma}{m}\right)}r^{2m}-4m(m-2)\alpha_2\textcolor{red}{\left(\left(f(r)-\frac{2\gamma}{m}\right)^2+\frac{4\gamma^2}{m^2(m-1)}\right)}r^{2(m-1)}\\
    &+16m(m-1)(m-2)(m-3)\alpha_3\\
    &\times\textcolor{red}{\left(\left(f(r)-\frac{2\gamma}{m}\right)^3-\frac{12f(r)\gamma^2}{m^2}+\frac{8\gamma^3}{m^3}+\frac{12f(r)\gamma}{m(m-1)}-\frac{8\gamma^3}{m(m-1)(m-2)}\right)}^{2(m-2)}=0
\end{split}
\end{equation*}
o de forma más compacta, podemos considerar la siguiente serie
\begin{equation}
    \textcolor{red}{\sum_{k=0}^n(-1)^k\binom{n}{k}\frac{(m-k)!}{m!}f(r)^{n-k}(2\gamma)^k}
\end{equation}
las cuales se convierte en 
\begin{equation*}
    \sum_{l=0}^n\frac{\alpha_l(-4)^{l-1}r^{2(m-l+1)}(m!)^2}{(m-l-1)!(m-1+1)!}\frac{1}{m(m-1)}\sum_{k=0}^n(-1)^k\binom{n}{k}\frac{(m-k)!}{m!}f(r)^{n-k}(2\gamma)^k=a^{2m}
\end{equation*}

El polinomio de Wheeler para orden $n$ en la serie de Lovelock para $d=2m$ dimensiones, pareciera ser
\begin{tcolorbox}
\begin{equation*}
   a^{2m} +\sum_{l=0}^{n}\frac{\alpha_l(-4)^{l-1}m!(m-2)!}{(m-l-1)!(m-l+1)!}F_l(r)r^{2(m-l+1)}=0
\end{equation*}
con
% \begin{equation*}
%     F_l(r)=\sum_{k=0}^l(-1)^k\binom{l}{k}f(r)^{l-k}\left(\frac{2\gamma}{m}\right)^k\left(1-\sigma+\sigma\frac{m^k(m-k)!}{m!}\right)
% \end{equation*}
% o
\begin{equation*}
    F_l(r)=\left(f(r)-\frac{2\gamma}{m}\right)^l+\sigma \sum_{k=0}^l(-1)^k\binom{l}{k}f(r)^{l-k}\left(\frac{2\gamma}{m}\right)^k\left(\frac{m^k(m-k)!}{m!}-1\right)
\end{equation*}
donde $\gamma=0,\pm 1$ para base manifolds productos de $(\mathbb{T}^2)$,$(\mathbb{S}^2)$ y $(\mathbb{H}^2)$ respectivamente.
\end{tcolorbox}

Para el caso cargado, considerando un ansatz de Maxwell alineado en la dirección del fibrado $U(1)$, se tiene
\begin{equation*}
    A=A_\mu\dd x^\mu=\left(\frac{p}{r^{2(m-1)}}+q r^2\right)(\dd t+\mathcal{B})
\end{equation*}

\begin{equation*}
   a^{2m} +\sum_{l=0}^{n}\frac{p(-4)^{l-1}m!(m-2)!}{(m-l-1)!(m-l+1)!}F_l(r)r^{2(m-l+1)}+\frac{4(m-2)}{m-1}\left(\frac{p^2}{r^{2(m-1)}}+\frac{r^{2(m+1)}}{m+1}q^2\right)+16pq r^2=0
\end{equation*}

\chapter{Taub-NUT}
\section{Taub-NUT Einstein gravity}
Los polinomios de Wheeler encontrados son los siguientes:

Para base manifold $(\mathbb{T}^2)$,$(\mathbb{T}^2)^2$,$(\mathbb{T}^2)^3$,$(\mathbb{T}^2)^4$,$(\mathbb{T}^2)^5$ son
\begin{equation*}
    \left(n^{2}-r^{2}\right) f \! \left(r \right)+\Lambda \left(n^{4}+2 n^{2} r^{2}-\frac{1}{3} r^{4}\right)+a^{4} r=0
\end{equation*}
\begin{equation*}
    2 \left(-r +n \right)^{2} \left(n +r \right)^{2} f \! \left(r \right)+\Lambda \left(n^{6}+3 n^{4} r^{2}-n^{2} r^{4}+\frac{1}{5} r^{6}\right)+a^{6} r=0
\end{equation*}
\begin{equation*}
    3 \left(-r +n \right)^{3} \left(n +r \right)^{3} f \! \left(r \right)+\Lambda \left(n^{8}+4 n^{6} r^{2}-2 n^{4} r^{4}+\frac{4}{5} n^{2} r^{6}-\frac{1}{7} r^{8}\right)+a^{8} r=0
\end{equation*}
\begin{equation*}
    4 \left(-r +n \right)^{4} \left(n +r \right)^{4} f \! \left(r \right)+\Lambda \left(n^{10}+5 n^{8} r^{2}-\frac{10}{3} n^{6} r^{4}+2 n^{4} r^{6}-\frac{5}{7} n^{2} r^{8}+\frac{1}{9} r^{10}\right)+a^{10} r=0
\end{equation*}
\begin{equation*}
    5 \left(n -r \right)^{5} \left(n +r \right)^{5} f \! \left(r \right)+\Lambda \left(n^{12}+6 n^{10} r^{2}-5 n^{8} r^{4}+4 n^{6} r^{6}-\frac{15}{7} n^{4} r^{8}+\frac{2}{3} n^{2} r^{10}-\frac{1}{11} r^{12}\right)+a^{10} r=0
\end{equation*}

Una propuesta tentativa es
\begin{equation*}
    a^{2 m} r +\left(m -1\right) \left(n^{2}-r^{2}\right)^{m -1} f \! \left(r \right)+\Lambda (\textcolor{red}{\rm algo})
\end{equation*}
donde ese \textcolor{red}{algo} va como potencias de $(n^2+r^2)$.

El polinomio de Wheeler para la generalización de Taub-NUT en Einstein gravity es
\begin{tcolorbox}
\begin{equation*}
    (m-1)(n^2-r^2)^{m-1}f(r)+\Lambda W_m +\gamma(m-1)W_{m-1}+a^{2m}r=0
\end{equation*}
donde
\begin{equation*}
    W_m\equiv \sum_{i=0}^m(-1)^{m-1}\binom{m}{i}\left(-\frac{n^2}{r^2}\right)^{m-i}\frac{2m-1}{2i-1}r^{2m}
\end{equation*}   
\end{tcolorbox}

La función métrica de Taub-NUT en Einsteing gravity es
\begin{tcolorbox}
\begin{equation*}
   f(r)= \frac{1}{(n^2-r^2)^{m-1}}\left[-\frac{\Lambda}{m-1}W_m-\gamma W_{m-1}+\frac{a^{2m}}{m-1}r\right]
\end{equation*}
donde
\begin{equation*}
    W_m\equiv =(-1)^{m-1}\binom{m}{i}\left(-\frac{n^2}{r^2}\right)^{m-i}\frac{r^{2m}}{2i-1}
\end{equation*}
\end{tcolorbox}



\section{Taub-NUT Einstein-Gauss-Bonnet gravity}
Los polinomios de Wheeler encontrados son los siguientes:

Para base manifold $(\mathbb{T}^2)^2$,$(\mathbb{T}^2)^3$,$(\mathbb{T}^2)^4$,$(\mathbb{T}^2)^5$ son
\begin{equation*}
    \left(n^{6}+3 n^{4} r^{2}-n^{2} r^{4}+\frac{1}{5} r^{6}\right) \Lambda \textcolor{purple}{-12 \alpha  \left(n^{2}+r^{2}\right) f \! \left(r \right)^{2}}+2 \left(n^{2}-r^{2}\right)^{2} f \! \left(r \right)+a^{6} r=0
\end{equation*}
\begin{equation*}
    \left(n^{8}+4 r^{2} n^{6}-2 r^{4} n^{4}+\frac{4}{5} r^{6} n^{2}-\frac{1}{7} r^{8}\right) \Lambda +\textcolor{purple}{\alpha  \left(-36 n^{4}-24 n^{2} r^{2}+60 r^{4}\right) f \! \left(r \right)^{2}}+3 \left(n^{2}-r^{2}\right)^{3} f \! \left(r \right)+a^{8} r=0
\end{equation*}
\begin{align*}
    &\left(n^{10}+5 n^{8} r^{2}-\frac{10}{3} n^{6} r^{4}+2 n^{4} r^{6}-\frac{5}{7} n^{2} r^{8}+\frac{1}{9} r^{10}\right) \Lambda \textcolor{purple}{-72 \left(n^{2}-r^{2}\right)^{2} \left(n^{2}+\frac{7 r^{2}}{3}\right) \alpha  f \! \left(r \right)^{2}}\\&+4 \left(n^{2}-r^{2}\right)^{4} f \! \left(r \right)+a^{10} r=0
\end{align*}
\begin{align*}
    &\left(n^{12}+6 n^{10} r^{2}-5 n^{8} r^{4}+4 n^{6} r^{6}-\frac{15}{7} n^{4} r^{8}+\frac{2}{3} n^{2} r^{10}-\frac{1}{11} r^{12}\right) \Lambda \\&\textcolor{purple}{-120 \alpha  \left(n^{2}+3 r^{2}\right) \left(n^{2}-r^{2}\right)^{3} f \! \left(r \right)^{2}}+5 \left(n^{2}-r^{2}\right)^{5} f \! \left(r \right)+a^{12} r=0
\end{align*}

Debemos encontrar una relación entre los siguientes factores
\begin{align*}
    &12(n^2+r^2)\\
    &12(n^2-r^2)(3n^2+5r^2)\\
    &72(n^2-r^2)^2\\
    &120(n^2+3r^2)(n^2-r^2)^3
\end{align*}

Podemos considerar $\Lambda=0$ y $a=0$ y factorizar los términos restantes y vemos que quedala siguiente estructura
\begin{align*}
   & -2 f \! \left(r \right) \left(6 f \! \left(r \right) \alpha  \,n^{2}+6 f \! \left(r \right) \alpha  \,r^{2}-n^{4}+2 n^{2} r^{2}-r^{4}\right)\\
   &-3 f \! \left(r \right) \left(n -r \right) \left(n +r \right) \left(12 f \! \left(r \right) \alpha  \,n^{2}+20 f \! \left(r \right) \alpha  \,r^{2}-n^{4}+2 n^{2} r^{2}-r^{4}\right)\\
   &-4 f \! \left(r \right) \left(n -r \right)^{2} \left(n +r \right)^{2} \left(18 f \! \left(r \right) \alpha  \,n^{2}+42 f \! \left(r \right) \alpha  \,r^{2}-n^{4}+2 n^{2} r^{2}-r^{4}\right)\\
   &-5 f \! \left(r \right) \left(n -r \right)^{3} \left(n +r \right)^{3} \left(24 f \! \left(r \right) \alpha  \,n^{2}+72 f \! \left(r \right) \alpha  \,r^{2}-n^{4}+2 n^{2} r^{2}-r^{4}\right)
\end{align*}
Es claro ver la relación entre los primeros factores. El desafío ahora es encontrar la relación entre los paréntesis.
\begin{equation*}
    -(m-1)f(r)(n^2-r^2)^{m-3}(\textcolor{blue}{algo}-(n^2-r^2)^2)
\end{equation*}

The Weeler polynomial for plane geometries ($\gamma=0$) in Einstein-Gauss-Bonnet gravity is

\begin{equation*}
   a^{2m}r+\Lambda W_m +(m-1)(n^2-r^2)^{m-1}f(r)-\frac{12(m-1)(m-2)}{2}(n^2-r^2)^{m-3}\left(n^2+\frac{(2m-3)}{3}r^2\right)f(r)^2=0
\end{equation*}

The Wheeler polynomial for $\sigma=0$ in Einstein-Gauss-Bonnet gravity is
\begin{align*}
    &a^{2m}r+\Lambda W_m +\alpha_1(m-1)\left[(n^2-r^2)^{m-1}f(r)+\gamma W_{m-1}\right]\\&-6\alpha_2(m-1)(m-2)\left[(n^2-r^2)^{m-3}\left[n^2+\frac{(2m-3)}{3}r^2\right]f(r)^2+\gamma\left(\frac{2}{3}(n^2-r^2)^{m-2}f(r)+\gamma\frac{(m-1)}{3m}W_{m-2}\right)\right]=0
\end{align*}

The Wheeler polynomial for $\sigma=1$ in Einstein-Gauss-Bonnet gravity is
\begin{align*}
    &a^{2m}r+\Lambda W_m +\alpha_1(m-1)\left[(n^2-r^2)^{m-1}f(r)+\gamma W_{m-1}\right]
    \\&-6\alpha_2(m-1)(m-2)\left[(n^2-r^2)^{m-3}\left[n^2+\frac{(2m-3)}{3}r^2\right]f(r)^2+\gamma\left(\frac{2}{3}(n^2-r^2)^{m-2}f(r)+\gamma\frac{1}{3}W_{m-2}\right)\right]=0
\end{align*}

For the $\mathbb{CP}$'s case, the contribution of the third order is
\begin{align*}
    &12(m-1)(m-2)(m-3)\alpha_3\left((n^2-r^2)^{m-5}[5n^4+(4m-10)n^2r^2+\left(\frac{4(m-2)^2}{3}-\frac{1}{3}\right)r^4]f(r)^3\right.\\&+\left.\gamma(3n^2+(2m-5)r^2)(n^2-r^2)^{m-4}f(r)^2+\frac{m-1}{m}(n^2-r^2)^{m-3}f(r)+\frac{(m-1)(m-2)}{3m^2}W_{m-3}\right)
\end{align*}

For the $\mathbb{S}$'s case, the contribution of the third order is
\begin{align*}
    &12(m-1)(m-2)(m-3)\alpha_3\left((n^2-r^2)^{m-5}[5n^4+(4m-10)n^2r^2+\left(\frac{4(m-2)^2}{3}-\frac{1}{3}\right)r^4]f(r)^3\right.\\&+\left.\gamma(3n^2+(2m-5)r^2)(n^2-r^2)^{m-4}f(r)^2+(n^2-r^2)^{m-3}f(r)+\frac{1}{3(2m-7)}W_{m-3}\right)
\end{align*}

\section{Wheeler polynomial}
\begin{equation}
  a^{2m}r+\sum_{p=0}^n\frac{(-1)^{p-1}}{2}\frac{(2p)!}{p!}\frac{(m-1)!}{(m-1-p)!}F_p(r)\alpha_p=0
\end{equation}
where
\begin{align}
  F_p(r)&\equiv \sum_{i=0}^p\frac{2p-1-(p-1)i}{2p-1}(n^2-r^2)^{m-2p+1+i}V_{p-i}f(r)^{p-i}\gamma^{i}\\
  V_p&\equiv \sum_{i=0}^p\frac{\binom{p-1}{i}\left(\dfrac{n^2}{r^2}\right)^{p-i-1}r^{2(p-1)}(2m-1-2p+2p)!(m-p-1)!}{\binom{i+2(p-1)}{2(p-1)}(2m-1-2p)!(m-1-p+i)!2^{i}}\\
  V_0&\equiv W_m=\sum_{i=0}^m(-1)^{m-1}\binom{m}{i}\left(-\frac{n^2}{r^2}\right)^{m-i}\frac{1}{2i-1}r^{2m}
\end{align}
































% \chapter{CP's self duality Lovelock}

% The $\mathbb{CP}^k$ are solutions of the equations of Lovelock in $D=2k$ dimensions up to a polynomial constraint over the Lovelock couplings $\alpha_p$. For intance, the following are the constraints to order $1,2$ and $3$ in the Lovelock series for $\mathbb{CP}^2,\mathbb{CP}^3$ and $\mathbb{CP}^4$ respectively
% % \begin{align*}
% %     \alpha_0+2\alpha_1&=0\\
% %     \alpha_0+4\alpha_1+6\alpha_2&=0\\
% %     25\alpha_0+15\alpha_1+480\alpha_2+576\alpha_3&=0
% % \end{align*}

% \begin{align*}
%     -\frac{\alpha_0}{2}-\alpha_1&=0\\
%     -\frac{\alpha_0}{2}-2\alpha_1-3\alpha_1&=0\\
%     -\frac{\alpha_0}{2}-3\alpha_1-\frac{48}{5}\alpha_2-\frac{288}{25}\alpha_3&=0
% \end{align*}
\section{Wheeler polynomial v2}
\begin{align}
  &\alpha_0W_m+\alpha_1\sum_{k=0}^1\frac{1}{m^{1-k}}\frac{m!}{(m-1+k)!}(n^2-r^2)^{m-1-k+1}f(r)^k\\
  &+\alpha_2\left[\sum_{k=0}^1\frac{1}{m^{2-k}}\frac{m!}{(m-2+k)!}(n^2-r^2)^{m-2-k+1}f(r)^k+H_{2,2}(n^2-r^2)^{m-2-2+1}f(r)^2\right]\\
  &+\alpha_3\left[\sum_{k=0}^1\frac{1}{m^{3-k}}\frac{m!}{(m-3+k)!}(n^2-r^2)^{m-3-k+1}f(r)^k+\sum_{k=2}^3H_{3,k}(n^2-r^2)^{m-3-k+1}f(r)^k\right]
\end{align}
\begin{equation*}
 \mu r+ \alpha_0W_m+\sum_{p=1}^N\alpha_p\left(\sum_{k=0}^1\frac{1}{m^{p-k}}\frac{m!}{(m-p+k)!}(n^2-r^2)^{m-p-k+1}f(r)^k+\sum_{k=2}^pH_{p,k}(n^2-r^2)^{m-p-k+1}f(r)^k\right)
\end{equation*}






\begin{equation}
  H_{p,k}\equiv \sum_{j=1}^k\frac{(2m-1-2j-2(p-k))!!}{(2m-1-2k-2(p-k))!!}r^{2(k-j)}n^{2(j-1)}
\end{equation}
\begin{equation}
  W_m=\sum_{i=0}^m(-1)^{m-1}\binom{m}{i}\left(-\frac{n^2}{r^2}\right)^{m-i}\frac{1}{2i-1}r^{2m}
\end{equation}

\section{Wheeler polynomial v3}
Para $\sigma=0$:
\begin{equation}
\mu  r+ \sum_{p=0}^N\frac{(-1)^{p-1}}{2}\frac{(2p)!}{p!}\frac{(m-1)!}{(m-1-p)!}\alpha_p\left( \sum_{k=0}^1 G_{p,k}(n^2-r^2)^{m-p-k+1}f(r)^k+(1-\sigma)\sum_{k=2}^p H_{p,k}(n^2-r^2)^{m-p-k+1}f(r)^k\right)
\end{equation}
donde
\begin{align}
  H_{p,k}&\equiv \sum_{j=1}^k\frac{[2(m-j-p+k)-1]!!}{[2(m-p)-1]!!}r^{2(k-j)}n^{2(j-1)}\\
  G_{p,k}&\equiv \frac{1}{m^{p-k}}\frac{m!}{(m-p+k)!}W_{(p,m-p)}^{1-k}\\
  W_{p,m}&\equiv \sum_{j=0}^m(-1)^{j-1}\binom{m}{j}n^{2(m-j)}r^{2j}\frac{1}{2j-1}(n^2-r^2)^{p-m-1}
\end{align}

\section{Wheeler polynomial v4}
Para $\sigma=0$:
\begin{equation}
\mu  r+ \sum_{p=0}^N\frac{(-1)^{p-1}}{2}\frac{(2p)!}{p!}\frac{(m-1)!}{(m-1-p)!}\alpha_p\left( (1-\sigma)\sum_{k=0}^1\gamma^{p-k} G_{p,k}+\sum_{k=2}^p \gamma^{p-k}H_{p,k}\right)=0
\end{equation}
donde
\begin{align}
  H_{p,k}&\equiv (n^2-r^2)^{m-p-k+1}f(r)^k\sum_{j=1}^k\frac{[2(m-j-p+k)-1]!!}{[2(m-p)-1]!!}r^{2(k-j)}n^{2(j-1)}\\
  G_{p,k}&\equiv \frac{1}{m^{p-k}}\frac{m!}{(m-p+k)!}W_{(p,m-p)}^{1-k}(n^2-r^2)^{m-p-k+1}f(r)^k\\
  W_{p,m}&\equiv \sum_{j=0}^m(-1)^{j-1}\binom{m}{j}n^{2(m-j)}r^{2j}\frac{1}{2j-1}(n^2-r^2)^{-(m+1)}
\end{align}

\section{Wheeler polynomial v5}
Para $\sigma=0$:
\begin{equation}
  \mu  r+ \sum_{p=0}^N\frac{(-1)^{p-1}}{2}\frac{(2p)!}{p!}\frac{(m-1)!}{(m-1-p)!}\alpha_p\left(\sum_{k=0}^{1}\gamma^{k}H_{p,p-k} + \sum_{k=2}^{p-1}\gamma^{k}G_{p,k}\right)=0
\end{equation}
donde
\begin{align}
   H_{p,k}&\equiv (n^2-r^2)^{m-p-k+1}f(r)^{k}\sum_{j=1}^k\frac{[2(m-j-p+k)-1]!!}{[2(m-p)-1]!!}r^{2(k-j)}n^{2(j-1)}\\
  G_{p,k}&\equiv \frac{m!}{m^k(m-k)!}(n^2-r^2)^{m-p-k+1}f(r)^{k}
\end{align}






\section{Wheeler polynomial v7}
\begin{equation}
  \mu  r+ \sum_{p=0}^N\frac{(-1)^{p-1}}{2}\frac{(2p)!}{p!}\frac{(m-1)!}{(m-1-p)!}\alpha_p  \left(\sum_{k=0}^p \gamma^k H_{p,p-k}F_{p,k}+\sum_{k=2}^p\left\{(1-\sigma)G_{p,k}+\sigma S_{p,k}-H_{p,p-k}\right\}\gamma^kF_{p,k}\right)=0
\end{equation}
o de manera equivalente
\begin{equation}
  \mu  r+ \sum_{p=0}^N\frac{(-1)^{p-1}}{2}\frac{(2p)!}{p!}\frac{(m-1)!}{(m-1-p)!}\alpha_p  \left(\sum_{k=0}^1 \gamma^kH_{p,p-k}F_{p,k}+\sum_{k=2}^p\left\{(1-\sigma)G_{p,k}+\sigma S_{p,k}\right\}\gamma^kF_{p,k}\right)=0
\end{equation}
donde
\begin{align}
  H_{p,k}&:=\sum_{j=1}^k\frac{[2(m-j-p+k)-1]!!}{[2(m-p)-1]!!}r^{2(k-j)}n^{2(j-1)}\\
  G_{p,k}&:=\frac{m!}{m^k(m-k)!}\\
  S_{p,k}&:=\frac{[2(m-k-p)+3]!!}{[2(m-p)-1]!!}\\
  F_{p,k}&:=(1-\delta_{p,k})\left[(n^2-r^2)^{m-2p+k+1}f(r)^{p-k}\right]+\delta_{p,k} W_{m-p}\\
  W_m&:=\sum_{i=0}^m(-1)^{m-1}\binom{m}{i}\left(-\frac{n^2}{r^2}\right)^{m-i}\frac{1}{2i-1}r^{2m}
\end{align}




\section{Wheeler polynomial F's}
\begin{equation}
  \mu  r+ \sum_{p=0}^N\frac{(-1)^{p-1}}{2}\frac{(2p)!}{p!}\frac{(m-1)!}{(m-1-p)!}\alpha_p  F_p(r)=0
\end{equation}
donde
\begin{align*}
  F_0(r)&=W_m\\
  F_1(r)&=(n^2-r^2)^{m-1}f(r)+\gamma W_{m-1}\\
  F_2(r)&=(n^2-r^2)^{m-3}\left[n^2+\frac{(2m-3)}{3}r^2\right]f(r)^2+\gamma\frac{2}{3}(n^2-r^2)^{m-2}f(r)+\left((1-\sigma)\frac{(m-1)}{3m}+\frac{\sigma}{3}\right)W_{m-2}\\
  F_3(3)&=(n^2-r^2)^{m-5}\left[\frac{(2m-3)(2m-5)}{3}r^4+2(2m-5)n^2r^2+5n^4\right]f(r)^3\\ &~~~+ \gamma(n^2-r^2)^{m-4}[3n^2+(2m-5)r^2]f(r)^2+\left(\frac{(1-\sigma)(m-1)}{m}+\sigma\right)(n^2-r^2)^{m-3}f(r)\\&~~~+\gamma \left(\frac{(1-\sigma)(m-1)(m-2)}{3m^2}+\frac{\sigma}{3(2m-7)}\right)W_{m-3}\\
    W_m&:=\sum_{i=0}^m(-1)^{m-1}\binom{m}{i}\left(-\frac{n^2}{r^2}\right)^{m-i}\frac{1}{2i-1}r^{2m}
\end{align*}
























\newpage
\section{Polynomial constraint}
A property of the $\mathbb{CP}^k$ spaces is that they are Lovelock constant, that is, each tensor of the equations of motion is proportional to the metric. Here are some examples:

For $\mathbb{CP}^2$ in Einstein gravity we have
\begin{align*}
    -\frac{\alpha_0}{2}\delta^\mu_\nu&=-\frac{\alpha_0}{2}\delta^\mu_\nu\\
    \alpha_1\left(R^\mu_\nu-\frac{1}{2}\delta^\mu_\nu\right)&=-\alpha_1\delta^\mu_\nu
\end{align*}

For $\mathbb{CP}^3$ in Einstein-Gauss-Bonnet gravity,
\begin{align*}
    -\frac{\alpha_0}{2}\delta^\mu_\nu&=-\frac{\alpha_0}{2}\delta^\mu_\nu\\
    \alpha_1\left(R^\mu_\nu-\frac{1}{2}\delta^\mu_\nu\right)&=-2\alpha_1\delta^\mu_\nu\\
    \alpha_2 H^\mu_\nu&=-3\alpha_2\delta^\mu_\nu
\end{align*}

For $\mathbb{CP}^4$ in cubic Lovelock,
\begin{align*}
    -\frac{\alpha_0}{2}\delta^\mu_\nu&=-\frac{\alpha_0}{2}\delta^\mu_\nu\\
    \alpha_1\left(R^\mu_\nu-\frac{1}{2}\delta^\mu_\nu\right)&=-3\alpha_1\delta^\mu_\nu\\
    \alpha_2 H^\mu_\nu&=-\frac{48}{5}\alpha_2\delta^\mu_\nu\\
    \alpha_3M^\mu_\nu&=-\frac{288}{25}\alpha_3\delta^\mu_\nu
\end{align*}
where
\begin{align*}
    H^\mu_\nu=-\frac{1}{8}\delta^{\mu\mu_1...\mu_4}_{\nu\nu_1...\nu_4}R^{\nu_1\nu_2}_{\  \mu_1\mu_2}R^{\nu_3\nu_3}_{\  \mu_3\mu_4}
\end{align*}
and
\begin{align*}
    M^\mu_\nu=-\frac{1}{16}\delta^{\mu\mu_1...\mu_6}_{\nu\nu_1...\nu_6}R^{\nu_1\nu_2}_{\  \mu_1\mu_2}R^{\nu_3\nu_3}_{\  \mu_3\mu_4}R^{\nu_5\nu_6}_{\  \mu_5\mu_6}
\end{align*}

In this way, the $\mathbb{CP}^k$ spaces are solution of the Lovelock equations in $D=2k$ dimensions up to a polynomial constraint on the Lovelock couplings $\alpha_p$. For intance, the following are the constraints to order $1,2$ and $3$ in the Lovelock series for $\mathbb{CP}^2,\mathbb{CP}^3$, $\mathbb{CP}^4,\mathbb{CP}^5$ and $\mathbb{CP}^6$ respectively
\begin{align*}
    \frac{\alpha_0}{2}+\alpha_1&=0\\
    \frac{\alpha_0}{2}+2\alpha_1+3\alpha_2&=0\\
    \frac{\alpha_0}{2}+3\alpha_1+\frac{48}{5}\alpha_2+\frac{288}{25}\alpha_3&=0\\
    \frac{\alpha_0}{2}+4\alpha_1+20\alpha_2+\frac{160}{3}\alpha_3&=0\\
    \frac{\alpha_0}{2}+5\alpha_1+\frac{240}{7}\alpha_2+\frac{7200}{49}\alpha_3&=0
\end{align*}

In general, for Einstein gravity we have that $\mathbb{CP}^k$ spaces satisfy the the equations of motion up to the following polynomial constraint
\begin{equation*}
    \frac{\alpha_0}{2}+(k-1)\alpha_1=0
\end{equation*}

The polynomial constraint for cubic Lovelock gravity is
\begin{equation*}
    \frac{\alpha_0}{2}+(k-1)\alpha_1+\frac{2k(k-1)(k-2)}{k+1}\alpha_2+\frac{4k(k-1)^2(k-2)(k-3)}{(k+1)^2}\alpha_3=0
\end{equation*}
or equivantly
\begin{equation*}
    \frac{\alpha_0}{2k}\binom{k}{1}+\frac{2\alpha_1}{k}\binom{k}{2}+\frac{12\alpha_2}{k+1}\binom{k}{3}+\frac{96\alpha_3(k-1)}{(k+1)^2}\binom{k}{4}=0
\end{equation*}

\begin{tcolorbox}
In a more compact way, the polynomial constraint so that the $\mathbb{CP}^k (\gamma=1)$ and $\mathbb{CH}^k (\gamma=-1)$ satisfy the Lovelock equations is
\begin{equation*}
    \frac{1}{k}\sum_{p=0}^n\binom{k}{k-p}^2\frac{l!^2(k-p)}{(k-p+1)}\left(\frac{2}{k+1}\right)^{p-1}\alpha_p=0
\end{equation*}
or equivantely
\begin{equation*}
    \frac{k!^2}{k}\sum_{p=0}^n\frac{\alpha_p}{(k-p-1)!(k-p+1)!}\left(\frac{2\gamma}{k+1}\right)^{p-1}=0
\end{equation*}
\end{tcolorbox}

The Lovelock constant property reads as
\begin{equation*}
    -\frac{1}{2^{p+1}}\delta^{\mu\mu_1....\mu_{2p}}_{\nu\nu_1....\nu_{2p}}R^{\nu_1\nu_2}_{\ \mu_1\mu_2}\cdots R^{\nu_{2p-1}\nu_{2p}}_{\ \mu_{2p-1}\mu_{2p}}=\frac{k!^2}{k}\frac{1}{(k-p-1)!(k-p+1)!}\left(\frac{2}{k+1}\right)^{p-1}\delta^\mu_\nu
\end{equation*}

\begin{tcolorbox}
In a more compact way, the polynomial constraint so that the $\mathbb{CP}^k (\gamma=1)$ and $\mathbb{CH}^k (\gamma=-1)$ satisfy the Lovelock equations is
\begin{equation*}
    \frac{k!^2}{k}\sum_{p=0}^n\frac{\alpha_p}{(k-p-1)!(k-p+1)!}\left(\frac{2\gamma}{k+1}\right)^{p-1}=0
\end{equation*}
\end{tcolorbox}

\newpage
The Einstein-Hilbert equations of motion are
\begin{equation*}
    -\sum_{p=0}^1\frac{\alpha_p}{2^{p+1}}\delta^{\mu\mu_1\mu_2}_{\nu\nu_1\nu_2}R^{\nu_1\nu_2}_{\mu_1\mu_2}=-\frac{\alpha_0}{2}\delta^\mu_\nu+\alpha_1\left(R^\mu_\nu-\frac{1}{2}R\delta^\mu_\nu\right)
\end{equation*}
but they can also be written as
\begin{equation*}
    -\frac{1}{4}\delta^{\mu\mu_1\mu_2}_{\nu\nu_1\nu_2}\left(R^{\nu_1\nu_2}_{\mu_1\mu_2}+c_1^{(1)}\delta^{\nu_1\nu_2}_{\mu_1\mu_2}\right)=R^\mu_\nu-\frac{1}{2}R\delta^\mu_\nu-10c_1^{(1)}\delta^\mu_\nu
\end{equation*}


\section{Otro approach}
Consideremos $n=0$, para Einstein grav tenemos
\begin{equation}
  \left(m -1\right) \left(f \! \left(r \right) r^{2 m -2}-\frac{r^{2 m -2}}{2 m -3}\right)
\end{equation}

\section{Lovelock constant}

\newpage
%\section{Rodolfo}
\begin{center}
\begin{tabular}{ |p{7cm} | p{7cm}|} 
 \hline
 \textbf{Estructura métrica} & \textbf{Estructura afín} \\  \hline \hline
 Las propiedades métricas están dadas por el tensor métrico $g_{\mu\nu}(x)$,  $$\dd s^2=g_{\mu\nu }(x)\dd x^\mu \dd x^\nu .$$ Además, cada punto de la variedad posee su propio espacio tangente, tal que $$g_{\mu\nu }=e^{a}_{~\mu }(x)e^{b}_{~\nu }(x)\eta_{a b}.$$ & Las pripiedades afines están determinadas por la conexión $\omega^{ab},$ $$u^r_\parallel (x)=u^r(x+\dd x)+\dd x^\mu \theta ^r _{~s\mu }u^s(x).$$ Donde los coeficientes $\theta ^r _{~s\mu }$ define el paralelismo. \\
 \hline
\end{tabular}
\end{center}

\textbf{Objetivo 1:} Resolver las ecuaciones de campo de la teoría tensor-escalar-Maxwell en el formalismo de primer orden, utilizando un ansatz estático con sección transversa esférica, hiperbólica y plana. Analizar la estructura causal de este espacio-tiempo, es decir, singularidades, horizontes de eventos, entre otros.
\begin{center}
\begin{tabular}{|p{7cm} | p{0.2cm}|p{0.2cm}| p{0.2cm}| p{0.2cm}| p{0.2cm}| p{0.2cm}|p{0.2cm}| p{0.2cm}| p{0.2cm}|p{0.2cm}|p{0.2cm}|p{0.2cm}|} 
 \hline
 \textbf{Meses}&\textbf{S}&\textbf{O}&\textbf{N}&\textbf{D}&\textbf{E}&\textbf{F}&\textbf{M}&\textbf{A}&\textbf{M}&\textbf{J}&\textbf{J}&\textbf{A}\\ \hline
 \textbf{Actividad 1}: Hallar soluciones de agujero negro para las ecuaciones de campo en gravedad Einstein-Cartan-Maxwell, utilizando un ansatz estático de sección transversal esférica, hiperbólica y plana.&\textbf{X}&\textbf{X}&&&&&&&&&&\\ \hline
 \textbf{Actividad 2}: Verificar y analizar la existencia de las singularizades de los campos geométricos.&&&\textbf{X}&&&&&&&&&\\ \hline
\textbf{ Actividad 3}: Escribir los resultados en el borrador del artículo.&\textbf{X}&\textbf{X}&\textbf{X}&&&&&&&&&\\ \hline
\end{tabular}
\end{center}
 %-------------------------
 
\textbf{ Objetivo 2:} Analizar las propiedades físicas del campo escalar y el campo de Maxwell. Estudiar su comportamiento en diferentes regiones del espacio-tiempo y las condiciones de energía asociadas al tensor energía-momentum.
\begin{center}
\begin{tabular}{|p{7cm} | p{0.2cm}|p{0.2cm}| p{0.2cm}| p{0.2cm}| p{0.2cm}| p{0.2cm}|p{0.2cm}| p{0.2cm}| p{0.2cm}|p{0.2cm}|p{0.2cm}|p{0.2cm}|} 
 \hline
 \textbf{Meses}&\textbf{S}&\textbf{O}&\textbf{N}&\textbf{D}&\textbf{E}&\textbf{F}&\textbf{M}&\textbf{A}&\textbf{M}&\textbf{J}&\textbf{J}&\textbf{A}\\ \hline
 \textbf{Actividad 1}: Explorar las propiedades físicas de los campos de materia en diferentes regiones del espacio-tiempo.&&&&\textbf{X}&\textbf{X}&&&&&&&\\ \hline
 \textbf{Actividad 2}: Inspeccionar las condiciones de energía asociadas al tensor de energía-momentum.&&&&\textbf{X}&\textbf{X}&&&&&&&\\ \hline
\textbf{ Actividad 3}: Escribir los resultados en el borrador del artículo.&&&&\textbf{X}&\textbf{X}&&&&&&&\\ \hline
\textbf{ Actividad 4}: Escritura y envío del artículo.&&&&&&\textbf{X}&&&&&&\\ \hline
\end{tabular}
\end{center}

 %-------------------------
 
\textbf{ Objetivo 3:} Encontrar soluciones cosmológicas de la teoría tenso-escalar en el formalismo de primer orden. Analizar las características del campo escalar y cómo incide en la expansión acelarada del Universo.
\begin{center}
\begin{tabular}{|p{7cm} | p{0.2cm}|p{0.2cm}| p{0.2cm}| p{0.2cm}| p{0.2cm}| p{0.2cm}|p{0.2cm}| p{0.2cm}| p{0.2cm}|p{0.2cm}|p{0.2cm}|p{0.2cm}|} 
 \hline
 \textbf{Meses}&\textbf{S}&\textbf{O}&\textbf{N}&\textbf{D}&\textbf{E}&\textbf{F}&\textbf{M}&\textbf{A}&\textbf{M}&\textbf{J}&\textbf{J}&\textbf{A}\\ \hline
 \textbf{Actividad 1}: Hallar soluciones cosmológicas paras las ecuaciones de campo en gravedad Einstein-Cartan.&&&&&&&\textbf{X}&\textbf{X}&&&&\\ \hline
 \textbf{Actividad 2}: Examinar las propiedades de los campos de materia.&&&&&&&&&\textbf{X}&&&\\ \hline
\textbf{ Actividad 3}: Estudiar la incidencia de la torsión en la expansión acelerada del Universo.&&&&&&&&&\textbf{X}&&&\\ \hline
\textbf{ Actividad 4}: Escribir los resultados en el borrador del artículo.&&&&&&&\textbf{X}&\textbf{X}&\textbf{X}&&&\\ \hline
\end{tabular}
\end{center}
 %-------------------------

\textbf{ Objetivo 4:} Comparar las soluciones analíticas respecto a las soluciones de la teoría de la gravedad puramente métrica y constatar si se conectan continuamente al considerar torsión nula. Contrastar esta nueva solución con datos observacionales.
\begin{center}
\begin{tabular}{|p{7cm} | p{0.2cm}|p{0.2cm}| p{0.2cm}| p{0.2cm}| p{0.2cm}| p{0.2cm}|p{0.2cm}| p{0.2cm}| p{0.2cm}|p{0.2cm}|p{0.2cm}|p{0.2cm}|} 
 \hline
 \textbf{Meses}&\textbf{S}&\textbf{O}&\textbf{N}&\textbf{D}&\textbf{E}&\textbf{F}&\textbf{M}&\textbf{A}&\textbf{M}&\textbf{J}&\textbf{J}&\textbf{A}\\ \hline
 \textbf{Actividad 1}: Comparar las soluciones encontradas con los resultados de Relatividad General.&&&&&&&&&&\textbf{X}&&\\ \hline
 \textbf{Actividad 2}:Contrastar los nuevos resultados con los datos observacionales.&&&&&&&&&&\textbf{X}&\textbf{X}&\\ \hline
\textbf{ Actividad 3}: Escribir los resultados en el borrador del artículo.&&&&&&&&&&\textbf{X}&\textbf{X}&\\ \hline
\textbf{ Actividad 4}: Escribir y enviar el artículo.&&&&&&&&&&&&\textbf{X}\\ \hline
\end{tabular}
\end{center}














\chapter{First order formalism}
Let us consider the metric ansatz constructed as the $U(1)$ fibration of Einstein-K\"ahler manifolds
\begin{equation}
   \dd{s^2}=g_{\mu\nu}\dd x^\mu \dd x^\nu = \frac{\dd{r^2}}{f(r)}+f(r)h(r)(\dd{\tau}+\B)^2+N(r)\dd{\Sigma}^2\,,
\end{equation}
where $\B=\B_i\dd \bar{x}^{i}$ denotes the K\"ahler potential $1$-form that defines the symplectic structure $\Omega=\dd\B $ associated to the $(2m-2)$-dimensional Einstein-K\"ahler trnasverse with line element
\begin{equation}
  \dd\Sigma^2 = g_{ij}(\bar{x})\dd \bar{x}^{i}\dd \bar{x}^j
\end{equation}
Here, barred coordinates with Latin indices represent those that belong to the transverse section. Thus, coordinates with greek indices can be $x^\mu=\{r,t,\bar{x}^i\}$.

It is straightforward to see that the non-zero metric components are given by
\begin{align}
  g_{rr}=\frac{1}{f},\quad g_{tt}=fh,\quad g_{ti}=fh\B_i,\quad g_{ij}=fh\B_i\B_j+N\bar{g}_{ij}
\end{align}
In order to find the components of the inverse metric, we use the fact that $g^{\m\n }g_{\n\lambda }=\d^\m_\lambda$, resulting in the following system of equations
\begin{equation}
\left\{
\begin{array}{lll}
  1&=g^{rr}g_{rr}\\
  1&=fh(g^{tt}+g^{ti}\B_i)\\
  \d^{i}_j&=fh\B_j(g^{ti}+g^{ik}\B_k)+Ng^{ik}\bar{g}_{kj}\\
  0&=fh\B_i(g^{tt}+g^{tk}\B_k)+Ng^{tk}\bar{g}_{ki}
\end{array}
\right.
\end{equation}
whose solution is given by
\begin{align}
  g^{rr}&=1\\
  g^{tt}&=\frac{\B_i\B_j}{N}\bar{g}^{ij}+\frac{1}{fh}\\
  g^{ti}&=-\frac{1}{N}\B_j\bar{g}^{ij}\\
  g^{ij}&=\frac{1}{N}\bar{g}^{ij}
\end{align}
We define the orthonormal noncoordinate frame basis of the metric () as
\begin{align}
    e^0&=\frac{\dd r}{\sqrt{f(r)}}\,,& e^1&=\sqrt{f(r)h(r)}(\dd\tau+\B)\,,& e^{A}&=\sqrt{N(r)}\,\bar{e}^{A}\,,
\end{align}
\begin{align}
    \omega^{01}&=-\frac{(fh)'}{2h\sqrt{f}}e^1\,, & 
    \omega^{0i}&=-\frac{\sqrt{f}N'}{2N}e^{i}\,, &
    \omega^{1i}&=\frac{\sqrt{fh}}{2N}\Omega^{i}{}_je^j\,, &
    \omega^{ij}&=\bar{\omega}^{ij}-\frac{\sqrt{fh}}{2N}\Omega^{ij}e^1\,.
\end{align}

\begin{align}
  E_0&=\sqrt{f}\partial_r\\
  E_1&=\frac{1}{\sqrt{fh}}\partial_t\\
  E_A&=-\frac{1}{\sqrt{N}}B_j\bar{E}^j_A\partial_t +\frac{1}{\sqrt{N}}\bar{E}^j_A\partial_j
\end{align}

\begin{align}
  R^{rt}_{rt}&=-\frac{1}{2\sqrt{h}}\left[\frac{(fh)'}{\sqrt{h}}\right]'\\
  	R^{ij}_{rt}&=-\frac{1}{2N}\left[\frac{fh}{N}\right]'\Omega^{ij}\\
  	R^{rt}_{ij}&=-\frac{N}{2h}\left[\frac{fh}{N}\right]'\Omega_{ij}\\
  R^{ri}_{rj}&=-\frac{1}{2}\sqrt{\frac{f}{N}}\left[\frac{\sqrt{f}N'}{\sqrt{N}}\right]'\d^i_j\\
  R^{\tau i}_{\tau j}&=\frac{fh}{4N^2}\Omega_{k}^{~i}\Omega^k_{~j}-\frac{N'(fh)'}{4Nh}\delta^{i}_j\\
  R^{ri}_{tj}&=-\frac{f}{4}\left[\frac{fh}{N}\right]'\Omega^{i}_{~j}\\
  R^{\tau i}_{rj}&=\frac{1}{4fh}\left[\frac{fh}{N}\right]'\Omega^{i}_{~j}\\
  R^{ij}_{kl}&=\frac{1}{N}\bar{R}^{ij}_{kl}-\frac{fN'^2}{2N^2}\delta^{i}_{[k}\delta^j_{l]} -\frac{fh}{4N^2}\left(\Omega^{ij}\Omega_{kl}+4\Omega^{[i}_{~[k}\Omega^{j]}_{~l]}\right)\\
     R^{ij}_{\tau k}&\,\rc{\stackrel{?}{=}0}\\
     R^{ri}_{r\tau}&=0
\end{align}

\section{Connection components}
\begin{align}
  \G^r_{~\t\t }&=-\frac{1}{2}f(fh)'\,\checkmark\\
  \G^r_{~rr}&=-\frac{f'}{2f}\,\checkmark\\
  \G^{r}_{~ij }&=-\frac{f(fh)'}{2}\B_i\B_j-\frac{fN'}{2}\bar{g}_{ij}\\
  \G^{r}_{\tau i}&=-\frac{f}{2}B_i(fh)'\\
  \G^\t_{~\t r}&=\frac{1}{2}\frac{(fh)'}{fh}\\
  \G^\tau_{~\t i}&=-\frac{fh}{2N}B^k\Omega_{ik}\\
    \G^\tau_{~ri}&=\frac{B_i}{2}\left[\frac{(fh)'}{fh}-\frac{N'}{N}\right]=\frac{B_i}{2}\dv{r}\left[\ln(\frac{fh}{N})\right]\\
   \G^\tau_{ij}&=\partial_{(i}B_{j)}-B_k\bar{\G}^k_{~ij}+\frac{fh}{N}\Omega_{k(i}\B_{j)}\B^k\\
  \G^{i}_{~\t j}&=-\frac{fh}{2N}\Omega^{i}_{~j}\\
  \G^{i}_{~rj}&=\frac{1}{2}\frac{N'}{N}\d^{i}_j\\
  \G^{i}_{~jk}&=\bar{\G}^{i}_{~jk}+\frac{fh}{N}\B_{(j}\Omega_{k)}^{~i}
\end{align}

where $\Omega_{ij}=2\partial_{[i}B_{j]}$.

\section{Curvature components}
\begin{align}
  R^r_{~trt}&=\partial_r\left[-\frac{1}{2}f(fh)'\right]+\frac{f'(fh)'}{4}+\frac{(fh)'^2}{4h}
\end{align}


\section{Invariant of the transverse section}
We consider
\begin{equation}
  \bar{\mathcal{L}}^{(1)}=\bar{R}
\end{equation}
For the case of $2$-sphere products, we have
\begin{align}
  (\mathbb{S}^2)^1&:\mathcal{L}^{(1)}=2\\
  (\mathbb{S}^2)^2&:\mathcal{L}^{(1)}=4\\
  (\mathbb{S}^2)^3&:\mathcal{L}^{(1)}=6\\
  (\mathbb{S}^2)^4&:\mathcal{L}^{(1)}=8\\
  \vdots\\
  (\mathbb{S}^2)^k&:\mathcal{L}^{(1)}=2k
\end{align}
For the case of $\mathbb{CP}^k$, we have
\begin{align}
  \mathbb{CP}^1&:\mathcal{L}^{(1)}=2\\
  \mathbb{CP}^2&:\mathcal{L}^{(1)}=4\\
  \mathbb{CP}^3&:\mathcal{L}^{(1)}=6\\
  \vdots\\
  \mathbb{CP}^k&:\mathcal{L}^{(1)}=2k
\end{align}

For the case of $(\mathbb{H}^2)^k$ we obtain $\bar{R}=-2k$. 

Now, we consider $$\bar{\mathcal{L}}^{(2)}=\bar{R}^{\m\n }_{\lambda\rho}\bar{R}_{\m\n }^{\lambda\r }-4\bar{R}^\m_\n \bar{R}^\n_\m +\bar{R}^2$$ and we obtain 
For the case of $2$-sphere products, we have
\begin{align}
  (\mathbb{S}^2)^1&:\mathcal{L}^{(2)}=0\\
  (\mathbb{S}^2)^2&:\mathcal{L}^{(2)}=8\\
  (\mathbb{S}^2)^3&:\mathcal{L}^{(2)}=24\\
  (\mathbb{S}^2)^4&:\mathcal{L}^{(2)}=48\\
  (\mathbb{S}^2)^5&:\mathcal{L}^{(2)}=80\\
  \vdots\\
  (\mathbb{S}^2)^k \text{ y } (\mathbb{H}^2)^k&:\mathcal{L}^{(2)}=4k(k-1)
\end{align}

For the case of $\mathbb{CP}^k$, we have
\begin{align}
  \mathbb{CP}^1&:\mathcal{L}^{(2)}=0\\
  \mathbb{CP}^2&:\mathcal{L}^{(2)}=\frac{16}{3}\\
  \mathbb{CP}^3&:\mathcal{L}^{(2)}=18=\frac{72}{4}\\
  \mathbb{CP}^4&:\mathcal{L}^{(2)}=\frac{192}{5}\\
  \mathbb{CP}^5&:\mathcal{L}^{(2)}=\frac{200}{3}=\frac{400}{6}\\
  \mathbb{CP}^6&:\mathcal{L}^{(2)}=\frac{200}{3}=\frac{720}{7}\\
  \vdots\\
  \mathbb{CP}^k \text{ y } \mathbb{CH}^k&:\mathcal{L}^{(2)}=\frac{4k^2(k-1)}{k+1}
\end{align}


Now, we consider $$\bar{\mathcal{L}}^{(3)}$$
For the case of $2$-sphere products, we have
\begin{align}
  (\mathbb{S}^2)^1&:\mathcal{L}^{(1)}=0\\
  (\mathbb{S}^2)^2&:\mathcal{L}^{(1)}=0\\
  (\mathbb{S}^2)^3&:\mathcal{L}^{(1)}=48\\
  (\mathbb{S}^2)^4&:\mathcal{L}^{(1)}=192\\
  (\mathbb{S}^2)^5&:\mathcal{L}^{(1)}=480\\
  (\mathbb{S}^2)^6&:\mathcal{L}^{(1)}=960\\
  \vdots\\
  (\mathbb{S}^2)^k&:\mathcal{L}^{(3)}=8k(k-1)(k-2)=8(k-1)[(k-1)^2-1]
\end{align}

For the case of $2$-hyperboloids products, we have
\begin{align}
  (\mathbb{H}^2)^1&:\mathcal{L}^{(1)}=0\\
  (\mathbb{H}^2)^2&:\mathcal{L}^{(1)}=0\\
  (\mathbb{H}^2)^3&:\mathcal{L}^{(1)}=-48\\
  (\mathbb{H}^2)^4&:\mathcal{L}^{(1)}=-192\\
  \vdots\\
  (\mathbb{H}^2)^k&:\mathcal{L}^{(3)}=-8k(k-1)(k-2)=-8(k-1)[(k-1)^2-1]
\end{align}

For the case of $\mathbb{CP}^k$ products, we have
\begin{align}
  \mathbb{CP}^1&:\mathcal{L}^{(3)}=0\\
\mathbb{CP}^2&:\mathcal{L}^{(3)}=0\\
\mathbb{CP}^3&:\mathcal{L}^{(3)}=18=\frac{288}{16}\\
\mathbb{CP}^4&:\mathcal{L}^{(3)}=\frac{2304}{25}\\
\mathbb{CP}^5&:\mathcal{L}^{(3)}=\frac{800}{3}=\frac{9600}{36}\\
\mathbb{CP}^6&:\mathcal{L}^{(3)}=\frac{28800}{49}\\
  \vdots\\
  \mathbb{CP}^k&:\mathcal{L}^{(3)}=\frac{8k^2(k-1)^2(k-2)}{(k+1)^2}
\end{align}

For the case of $\mathbb{CH}^k$ products, we have
\begin{equation}
    \mathbb{CH}^k:\mathcal{L}^{(3)}=-\frac{8k^2(k-1)^2(k-2)}{(k+1)^2}
\end{equation}





\section{Kheler forms squared}
For $(\mathbb{S}^2)^k$ and $(\mathbb{H}^2)^k$, $\Omega^2=2k$.

For $\mathbb{CP}^k$ and $\mathbb{CH}^k$, $\Omega^2=2k$.

\section{Base manifold Pontryagin}
Let be $\Omega=\dd \mathcal{B}$ the simplectic $2$-form of the Einstein-Khaler manifold. For the case the $\mathbb{CP}^2$ we have
\begin{equation}
	\tilde{\Omega}_{\m\n}\Omega^{\m\n }=4
\end{equation}
where $\tilde{\Omega}_{\m\n}=\frac{1}{2}\epsilon_{\m\n \a\b}\Omega^{\a\b }$ is the dual of $\Omega_{\m\n }$.




































\end{document}
