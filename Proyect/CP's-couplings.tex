% \chapter{CP's self duality Lovelock}

% The $\mathbb{CP}^k$ are solutions of the equations of Lovelock in $D=2k$ dimensions up to a polynomial constraint over the Lovelock couplings $\alpha_p$. For intance, the following are the constraints to order $1,2$ and $3$ in the Lovelock series for $\mathbb{CP}^2,\mathbb{CP}^3$ and $\mathbb{CP}^4$ respectively
% % \begin{align*}
% %     \alpha_0+2\alpha_1&=0\\
% %     \alpha_0+4\alpha_1+6\alpha_2&=0\\
% %     25\alpha_0+15\alpha_1+480\alpha_2+576\alpha_3&=0
% % \end{align*}

% \begin{align*}
%     -\frac{\alpha_0}{2}-\alpha_1&=0\\
%     -\frac{\alpha_0}{2}-2\alpha_1-3\alpha_1&=0\\
%     -\frac{\alpha_0}{2}-3\alpha_1-\frac{48}{5}\alpha_2-\frac{288}{25}\alpha_3&=0
% \end{align*}
\section{Wheeler polynomial v2}
\begin{align}
  &\alpha_0W_m+\alpha_1\sum_{k=0}^1\frac{1}{m^{1-k}}\frac{m!}{(m-1+k)!}(n^2-r^2)^{m-1-k+1}f(r)^k\\
  &+\alpha_2\left[\sum_{k=0}^1\frac{1}{m^{2-k}}\frac{m!}{(m-2+k)!}(n^2-r^2)^{m-2-k+1}f(r)^k+H_{2,2}(n^2-r^2)^{m-2-2+1}f(r)^2\right]\\
  &+\alpha_3\left[\sum_{k=0}^1\frac{1}{m^{3-k}}\frac{m!}{(m-3+k)!}(n^2-r^2)^{m-3-k+1}f(r)^k+\sum_{k=2}^3H_{3,k}(n^2-r^2)^{m-3-k+1}f(r)^k\right]
\end{align}
\begin{equation*}
 \mu r+ \alpha_0W_m+\sum_{p=1}^N\alpha_p\left(\sum_{k=0}^1\frac{1}{m^{p-k}}\frac{m!}{(m-p+k)!}(n^2-r^2)^{m-p-k+1}f(r)^k+\sum_{k=2}^pH_{p,k}(n^2-r^2)^{m-p-k+1}f(r)^k\right)
\end{equation*}






\begin{equation}
  H_{p,k}\equiv \sum_{j=1}^k\frac{(2m-1-2j-2(p-k))!!}{(2m-1-2k-2(p-k))!!}r^{2(k-j)}n^{2(j-1)}
\end{equation}
\begin{equation}
  W_m=\sum_{i=0}^m(-1)^{m-1}\binom{m}{i}\left(-\frac{n^2}{r^2}\right)^{m-i}\frac{1}{2i-1}r^{2m}
\end{equation}

\section{Wheeler polynomial v3}
Para $\sigma=0$:
\begin{equation}
\mu  r+ \sum_{p=0}^N\frac{(-1)^{p-1}}{2}\frac{(2p)!}{p!}\frac{(m-1)!}{(m-1-p)!}\alpha_p\left( \sum_{k=0}^1 G_{p,k}(n^2-r^2)^{m-p-k+1}f(r)^k+(1-\sigma)\sum_{k=2}^p H_{p,k}(n^2-r^2)^{m-p-k+1}f(r)^k\right)
\end{equation}
donde
\begin{align}
  H_{p,k}&\equiv \sum_{j=1}^k\frac{[2(m-j-p+k)-1]!!}{[2(m-p)-1]!!}r^{2(k-j)}n^{2(j-1)}\\
  G_{p,k}&\equiv \frac{1}{m^{p-k}}\frac{m!}{(m-p+k)!}W_{(p,m-p)}^{1-k}\\
  W_{p,m}&\equiv \sum_{j=0}^m(-1)^{j-1}\binom{m}{j}n^{2(m-j)}r^{2j}\frac{1}{2j-1}(n^2-r^2)^{p-m-1}
\end{align}

\section{Wheeler polynomial v4}
Para $\sigma=0$:
\begin{equation}
\mu  r+ \sum_{p=0}^N\frac{(-1)^{p-1}}{2}\frac{(2p)!}{p!}\frac{(m-1)!}{(m-1-p)!}\alpha_p\left( (1-\sigma)\sum_{k=0}^1\gamma^{p-k} G_{p,k}+\sum_{k=2}^p \gamma^{p-k}H_{p,k}\right)=0
\end{equation}
donde
\begin{align}
  H_{p,k}&\equiv (n^2-r^2)^{m-p-k+1}f(r)^k\sum_{j=1}^k\frac{[2(m-j-p+k)-1]!!}{[2(m-p)-1]!!}r^{2(k-j)}n^{2(j-1)}\\
  G_{p,k}&\equiv \frac{1}{m^{p-k}}\frac{m!}{(m-p+k)!}W_{(p,m-p)}^{1-k}(n^2-r^2)^{m-p-k+1}f(r)^k\\
  W_{p,m}&\equiv \sum_{j=0}^m(-1)^{j-1}\binom{m}{j}n^{2(m-j)}r^{2j}\frac{1}{2j-1}(n^2-r^2)^{-(m+1)}
\end{align}

\section{Wheeler polynomial v5}
Para $\sigma=0$:
\begin{equation}
  \mu  r+ \sum_{p=0}^N\frac{(-1)^{p-1}}{2}\frac{(2p)!}{p!}\frac{(m-1)!}{(m-1-p)!}\alpha_p\left(\sum_{k=0}^{1}\gamma^{k}H_{p,p-k} + \sum_{k=2}^{p-1}\gamma^{k}G_{p,k}\right)=0
\end{equation}
donde
\begin{align}
   H_{p,k}&\equiv (n^2-r^2)^{m-p-k+1}f(r)^{k}\sum_{j=1}^k\frac{[2(m-j-p+k)-1]!!}{[2(m-p)-1]!!}r^{2(k-j)}n^{2(j-1)}\\
  G_{p,k}&\equiv \frac{m!}{m^k(m-k)!}(n^2-r^2)^{m-p-k+1}f(r)^{k}
\end{align}






\section{Wheeler polynomial v7}
\begin{equation}
  \mu  r+ \sum_{p=0}^N\frac{(-1)^{p-1}}{2}\frac{(2p)!}{p!}\frac{(m-1)!}{(m-1-p)!}\alpha_p  \left(\sum_{k=0}^p \gamma^k H_{p,p-k}F_{p,k}+\sum_{k=2}^p\left\{(1-\sigma)G_{p,k}+\sigma S_{p,k}-H_{p,p-k}\right\}\gamma^kF_{p,k}\right)=0
\end{equation}
o de manera equivalente
\begin{equation}
  \mu  r+ \sum_{p=0}^N\frac{(-1)^{p-1}}{2}\frac{(2p)!}{p!}\frac{(m-1)!}{(m-1-p)!}\alpha_p  \left(\sum_{k=0}^1 \gamma^kH_{p,p-k}F_{p,k}+\sum_{k=2}^p\left\{(1-\sigma)G_{p,k}+\sigma S_{p,k}\right\}\gamma^kF_{p,k}\right)=0
\end{equation}
donde
\begin{align}
  H_{p,k}&:=\sum_{j=1}^k\frac{[2(m-j-p+k)-1]!!}{[2(m-p)-1]!!}r^{2(k-j)}n^{2(j-1)}\\
  G_{p,k}&:=\frac{m!}{m^k(m-k)!}\\
  S_{p,k}&:=\frac{[2(m-k-p)+3]!!}{[2(m-p)-1]!!}\\
  F_{p,k}&:=(1-\delta_{p,k})\left[(n^2-r^2)^{m-2p+k+1}f(r)^{p-k}\right]+\delta_{p,k} W_{m-p}\\
  W_m&:=\sum_{i=0}^m(-1)^{m-1}\binom{m}{i}\left(-\frac{n^2}{r^2}\right)^{m-i}\frac{1}{2i-1}r^{2m}
\end{align}




\section{Wheeler polynomial F's}
\begin{equation}
  \mu  r+ \sum_{p=0}^N\frac{(-1)^{p-1}}{2}\frac{(2p)!}{p!}\frac{(m-1)!}{(m-1-p)!}\alpha_p  F_p(r)=0
\end{equation}
donde
\begin{align*}
  F_0(r)&=W_m\\
  F_1(r)&=(n^2-r^2)^{m-1}f(r)+\gamma W_{m-1}\\
  F_2(r)&=(n^2-r^2)^{m-3}\left[n^2+\frac{(2m-3)}{3}r^2\right]f(r)^2+\gamma\frac{2}{3}(n^2-r^2)^{m-2}f(r)+\left((1-\sigma)\frac{(m-1)}{3m}+\frac{\sigma}{3}\right)W_{m-2}\\
  F_3(3)&=(n^2-r^2)^{m-5}\left[\frac{(2m-3)(2m-5)}{3}r^4+2(2m-5)n^2r^2+5n^4\right]f(r)^3\\ &~~~+ \gamma(n^2-r^2)^{m-4}[3n^2+(2m-5)r^2]f(r)^2+\left(\frac{(1-\sigma)(m-1)}{m}+\sigma\right)(n^2-r^2)^{m-3}f(r)\\&~~~+\gamma \left(\frac{(1-\sigma)(m-1)(m-2)}{3m^2}+\frac{\sigma}{3(2m-7)}\right)W_{m-3}\\
    W_m&:=\sum_{i=0}^m(-1)^{m-1}\binom{m}{i}\left(-\frac{n^2}{r^2}\right)^{m-i}\frac{1}{2i-1}r^{2m}
\end{align*}
























\newpage
\section{Polynomial constraint}
A property of the $\mathbb{CP}^k$ spaces is that they are Lovelock constant, that is, each tensor of the equations of motion is proportional to the metric. Here are some examples:

For $\mathbb{CP}^2$ in Einstein gravity we have
\begin{align*}
    -\frac{\alpha_0}{2}\delta^\mu_\nu&=-\frac{\alpha_0}{2}\delta^\mu_\nu\\
    \alpha_1\left(R^\mu_\nu-\frac{1}{2}\delta^\mu_\nu\right)&=-\alpha_1\delta^\mu_\nu
\end{align*}

For $\mathbb{CP}^3$ in Einstein-Gauss-Bonnet gravity,
\begin{align*}
    -\frac{\alpha_0}{2}\delta^\mu_\nu&=-\frac{\alpha_0}{2}\delta^\mu_\nu\\
    \alpha_1\left(R^\mu_\nu-\frac{1}{2}\delta^\mu_\nu\right)&=-2\alpha_1\delta^\mu_\nu\\
    \alpha_2 H^\mu_\nu&=-3\alpha_2\delta^\mu_\nu
\end{align*}

For $\mathbb{CP}^4$ in cubic Lovelock,
\begin{align*}
    -\frac{\alpha_0}{2}\delta^\mu_\nu&=-\frac{\alpha_0}{2}\delta^\mu_\nu\\
    \alpha_1\left(R^\mu_\nu-\frac{1}{2}\delta^\mu_\nu\right)&=-3\alpha_1\delta^\mu_\nu\\
    \alpha_2 H^\mu_\nu&=-\frac{48}{5}\alpha_2\delta^\mu_\nu\\
    \alpha_3M^\mu_\nu&=-\frac{288}{25}\alpha_3\delta^\mu_\nu
\end{align*}
where
\begin{align*}
    H^\mu_\nu=-\frac{1}{8}\delta^{\mu\mu_1...\mu_4}_{\nu\nu_1...\nu_4}R^{\nu_1\nu_2}_{\  \mu_1\mu_2}R^{\nu_3\nu_3}_{\  \mu_3\mu_4}
\end{align*}
and
\begin{align*}
    M^\mu_\nu=-\frac{1}{16}\delta^{\mu\mu_1...\mu_6}_{\nu\nu_1...\nu_6}R^{\nu_1\nu_2}_{\  \mu_1\mu_2}R^{\nu_3\nu_3}_{\  \mu_3\mu_4}R^{\nu_5\nu_6}_{\  \mu_5\mu_6}
\end{align*}

In this way, the $\mathbb{CP}^k$ spaces are solution of the Lovelock equations in $D=2k$ dimensions up to a polynomial constraint on the Lovelock couplings $\alpha_p$. For intance, the following are the constraints to order $1,2$ and $3$ in the Lovelock series for $\mathbb{CP}^2,\mathbb{CP}^3$, $\mathbb{CP}^4,\mathbb{CP}^5$ and $\mathbb{CP}^6$ respectively
\begin{align*}
    \frac{\alpha_0}{2}+\alpha_1&=0\\
    \frac{\alpha_0}{2}+2\alpha_1+3\alpha_2&=0\\
    \frac{\alpha_0}{2}+3\alpha_1+\frac{48}{5}\alpha_2+\frac{288}{25}\alpha_3&=0\\
    \frac{\alpha_0}{2}+4\alpha_1+20\alpha_2+\frac{160}{3}\alpha_3&=0\\
    \frac{\alpha_0}{2}+5\alpha_1+\frac{240}{7}\alpha_2+\frac{7200}{49}\alpha_3&=0
\end{align*}

In general, for Einstein gravity we have that $\mathbb{CP}^k$ spaces satisfy the the equations of motion up to the following polynomial constraint
\begin{equation*}
    \frac{\alpha_0}{2}+(k-1)\alpha_1=0
\end{equation*}

The polynomial constraint for cubic Lovelock gravity is
\begin{equation*}
    \frac{\alpha_0}{2}+(k-1)\alpha_1+\frac{2k(k-1)(k-2)}{k+1}\alpha_2+\frac{4k(k-1)^2(k-2)(k-3)}{(k+1)^2}\alpha_3=0
\end{equation*}
or equivantly
\begin{equation*}
    \frac{\alpha_0}{2k}\binom{k}{1}+\frac{2\alpha_1}{k}\binom{k}{2}+\frac{12\alpha_2}{k+1}\binom{k}{3}+\frac{96\alpha_3(k-1)}{(k+1)^2}\binom{k}{4}=0
\end{equation*}

\begin{tcolorbox}
In a more compact way, the polynomial constraint so that the $\mathbb{CP}^k (\gamma=1)$ and $\mathbb{CH}^k (\gamma=-1)$ satisfy the Lovelock equations is
\begin{equation*}
    \frac{1}{k}\sum_{p=0}^n\binom{k}{k-p}^2\frac{l!^2(k-p)}{(k-p+1)}\left(\frac{2}{k+1}\right)^{p-1}\alpha_p=0
\end{equation*}
or equivantely
\begin{equation*}
    \frac{k!^2}{k}\sum_{p=0}^n\frac{\alpha_p}{(k-p-1)!(k-p+1)!}\left(\frac{2\gamma}{k+1}\right)^{p-1}=0
\end{equation*}
\end{tcolorbox}

The Lovelock constant property reads as
\begin{equation*}
    -\frac{1}{2^{p+1}}\delta^{\mu\mu_1....\mu_{2p}}_{\nu\nu_1....\nu_{2p}}R^{\nu_1\nu_2}_{\ \mu_1\mu_2}\cdots R^{\nu_{2p-1}\nu_{2p}}_{\ \mu_{2p-1}\mu_{2p}}=\frac{k!^2}{k}\frac{1}{(k-p-1)!(k-p+1)!}\left(\frac{2}{k+1}\right)^{p-1}\delta^\mu_\nu
\end{equation*}

\begin{tcolorbox}
In a more compact way, the polynomial constraint so that the $\mathbb{CP}^k (\gamma=1)$ and $\mathbb{CH}^k (\gamma=-1)$ satisfy the Lovelock equations is
\begin{equation*}
    \frac{k!^2}{k}\sum_{p=0}^n\frac{\alpha_p}{(k-p-1)!(k-p+1)!}\left(\frac{2\gamma}{k+1}\right)^{p-1}=0
\end{equation*}
\end{tcolorbox}

\newpage
The Einstein-Hilbert equations of motion are
\begin{equation*}
    -\sum_{p=0}^1\frac{\alpha_p}{2^{p+1}}\delta^{\mu\mu_1\mu_2}_{\nu\nu_1\nu_2}R^{\nu_1\nu_2}_{\mu_1\mu_2}=-\frac{\alpha_0}{2}\delta^\mu_\nu+\alpha_1\left(R^\mu_\nu-\frac{1}{2}R\delta^\mu_\nu\right)
\end{equation*}
but they can also be written as
\begin{equation*}
    -\frac{1}{4}\delta^{\mu\mu_1\mu_2}_{\nu\nu_1\nu_2}\left(R^{\nu_1\nu_2}_{\mu_1\mu_2}+c_1^{(1)}\delta^{\nu_1\nu_2}_{\mu_1\mu_2}\right)=R^\mu_\nu-\frac{1}{2}R\delta^\mu_\nu-10c_1^{(1)}\delta^\mu_\nu
\end{equation*}
