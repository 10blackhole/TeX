\newpage
\section{Generalizaciones}
El polinomio de Wheeler hasta Lovelock cúbico para $d=2m$ dimensiones, pareciera ser
\begin{equation*}
\begin{split}
    &a^{2m}+\frac{\Lambda r^{2(m+1)}}{2(m^2-1)}+\textcolor{red}{\left(f(r)-\frac{2\gamma}{m}\right)}r^{2m}-4m(m-2)\alpha_2\textcolor{red}{\left(\left(f(r)-\frac{2\gamma}{m}\right)^2+\frac{4\gamma^2}{m^2(m-1)}\right)}r^{2(m-1)}\\
    &+16m(m-1)(m-2)(m-3)\alpha_3\\
    &\times\textcolor{red}{\left(\left(f(r)-\frac{2\gamma}{m}\right)^3-\frac{12f(r)\gamma^2}{m^2}+\frac{8\gamma^3}{m^3}+\frac{12f(r)\gamma}{m(m-1)}-\frac{8\gamma^3}{m(m-1)(m-2)}\right)}^{2(m-2)}=0
\end{split}
\end{equation*}
o de forma más compacta, podemos considerar la siguiente serie
\begin{equation}
    \textcolor{red}{\sum_{k=0}^n(-1)^k\binom{n}{k}\frac{(m-k)!}{m!}f(r)^{n-k}(2\gamma)^k}
\end{equation}
las cuales se convierte en 
\begin{equation*}
    \sum_{l=0}^n\frac{\alpha_l(-4)^{l-1}r^{2(m-l+1)}(m!)^2}{(m-l-1)!(m-1+1)!}\frac{1}{m(m-1)}\sum_{k=0}^n(-1)^k\binom{n}{k}\frac{(m-k)!}{m!}f(r)^{n-k}(2\gamma)^k=a^{2m}
\end{equation*}

El polinomio de Wheeler para orden $n$ en la serie de Lovelock para $d=2m$ dimensiones, pareciera ser
\begin{tcolorbox}
\begin{equation*}
   a^{2m} +\sum_{l=0}^{n}\frac{\alpha_l(-4)^{l-1}m!(m-2)!}{(m-l-1)!(m-l+1)!}F_l(r)r^{2(m-l+1)}=0
\end{equation*}
con
% \begin{equation*}
%     F_l(r)=\sum_{k=0}^l(-1)^k\binom{l}{k}f(r)^{l-k}\left(\frac{2\gamma}{m}\right)^k\left(1-\sigma+\sigma\frac{m^k(m-k)!}{m!}\right)
% \end{equation*}
% o
\begin{equation*}
    F_l(r)=\left(f(r)-\frac{2\gamma}{m}\right)^l+\sigma \sum_{k=0}^l(-1)^k\binom{l}{k}f(r)^{l-k}\left(\frac{2\gamma}{m}\right)^k\left(\frac{m^k(m-k)!}{m!}-1\right)
\end{equation*}
donde $\gamma=0,\pm 1$ para base manifolds productos de $(\mathbb{T}^2)$,$(\mathbb{S}^2)$ y $(\mathbb{H}^2)$ respectivamente.
\end{tcolorbox}

Para el caso cargado, considerando un ansatz de Maxwell alineado en la dirección del fibrado $U(1)$, se tiene
\begin{equation*}
    A=A_\mu\dd x^\mu=\left(\frac{p}{r^{2(m-1)}}+q r^2\right)(\dd t+\mathcal{B})
\end{equation*}

\begin{equation*}
   a^{2m} +\sum_{l=0}^{n}\frac{p(-4)^{l-1}m!(m-2)!}{(m-l-1)!(m-l+1)!}F_l(r)r^{2(m-l+1)}+\frac{4(m-2)}{m-1}\left(\frac{p^2}{r^{2(m-1)}}+\frac{r^{2(m+1)}}{m+1}q^2\right)+16pq r^2=0
\end{equation*}
