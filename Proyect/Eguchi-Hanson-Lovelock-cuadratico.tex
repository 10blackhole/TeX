\newpage
\section{Eguchi-Hanson en Lovelock cuadrático}
Los polinomios de Wheeler encontrados son los siguientes:

En $6 D$ para base manifold $(\mathbb{T}^2)^2$,$(\mathbb{S}^2)^2$, $(\mathbb{H}^2)^2$, $\mathbb{CP}^2$ y $\mathbb{CH}^2$ respectivamente son
\begin{equation*}
    12\alpha f(r)^2r^4-r^6f(r)-\frac{\Lambda r^8}{16}+a^6=0
\end{equation*}
\begin{equation*}
    12\alpha f(r)^2 r^4+(-r^6-16\alpha r^4)f(r)-\frac{\Lambda r^8}{16}+\frac{2}{3}r^6+8\alpha r^4+a^6=0
\end{equation*}
\begin{equation*}
    12\alpha f(r)^2 r^4+(-r^6+16\alpha r^4)f(r)-\frac{\Lambda r^8}{16}-\frac{2}{3}r^6+8\alpha r^4+a^6=0
\end{equation*}
\begin{equation*}
    12\alpha f(r)^2 r^4+(-r^6-16\alpha r^4)f(r)-\frac{\Lambda r^8}{16}+\frac{2}{3}r^6+\frac{16}{3}\alpha r^4+a^6=0
\end{equation*}
\begin{equation*}
    12\alpha f(r)^2 r^4+(-r^6+16\alpha r^4)f(r)-\frac{\Lambda r^8}{16}-\frac{2}{3}r^6+\frac{16}{3}\alpha r^4+a^6=0
\end{equation*}

En $8 D$ para base manifold $(\mathbb{T}^2)^3$,$(\mathbb{S}^2)^3$, $(\mathbb{H}^2)^3$, $\mathbb{CP}^3$ y $\mathbb{CH}^3$ respectivamente son
\begin{equation*}
    32\alpha f(r)^2r^6-r^8f(r)-\frac{\Lambda r^{10}}{30}+a^8=0
\end{equation*}
\begin{equation*}
    32\alpha f(r)^2 r^6+(-r^8-32\alpha r^6)f(r)-\frac{\Lambda r^{10}}{30}+\frac{1}{2}r^8+\frac{32}{3}\alpha r^6+a^8=0
\end{equation*}
\begin{equation*}
    32\alpha f(r)^2 r^6+(-r^8+32\alpha r^6)f(r)-\frac{\Lambda r^{10}}{30}-\frac{1}{2}r^8+\frac{32}{3}\alpha r^6+a^8=0
\end{equation*}
\begin{equation*}
    32\alpha f(r)^2 r^6+(-r^8-32\alpha r^6)f(r)-\frac{\Lambda r^{10}}{30}+\frac{1}{2}r^8+8\alpha r^6+a^8=0
\end{equation*}
\begin{equation*}
    32\alpha f(r)^2 r^6+(-r^8+32\alpha r^6)f(r)-\frac{\Lambda r^{10}}{30}-\frac{1}{2}r^8+8\alpha r^6+a^8=0
\end{equation*}

En $10 D$ para base manifold $(\mathbb{T}^2)^4$,$(\mathbb{S}^2)^4$, $(\mathbb{H}^2)^4$, $\mathbb{CP}^4$ y $\mathbb{CH}^4$ respectivamente son
\begin{equation*}
    60\alpha f(r)^2r^8-f(r)r^{10}-\frac{\Lambda}{48}r^{12}+a^{10}=0
\end{equation*}
\begin{equation*}
    60\alpha f(r)^2r^8+(-r^{10}-48\alpha r^8)f(r)-\frac{\Lambda}{48}+\frac{2}{5}r^{10}+12\alpha r^8+a^{10}=0
\end{equation*}
\begin{equation*}
    60\alpha f(r)^2r^8+(-r^{10}+48\alpha r^8)f(r)-\frac{\Lambda}{48}-\frac{2}{5}r^{10}+12\alpha r^8+a^{10}=0
\end{equation*}
\begin{equation*}
    60\alpha f(r)^2r^8+(-r^{10}-48\alpha r^8)f(r)-\frac{\Lambda}{48}+\frac{2}{5}r^{10}+\frac{48}{5}\alpha r^8+a^{10}=0
\end{equation*}
\begin{equation*}
    60\alpha f(r)^2r^8+(-r^{10}+48\alpha r^8)f(r)-\frac{\Lambda}{48}-\frac{2}{5}r^{10}+\frac{48}{5}\alpha r^8+a^{10}=0
\end{equation*}

En $12 D$ para base manifold $(\mathbb{T}^2)^5$,$(\mathbb{S}^2)^5$, $(\mathbb{H}^2)^5$ respectivamente son
\begin{equation*}
    96\alpha f(r)^2 r^{10}-f(r)r^{12}-\frac{\Lambda}{70}r^{14}+a^{12}=0
\end{equation*}
\begin{equation*}
    96\alpha f(r)^2r^{10}+(-r^{12}-64\alpha r^{10})f(r)-\frac{\Lambda}{70}r^{14}+\frac{1}{3}r^{12}+\frac{64}{5}\alpha r^{10}+a^{12}=0
\end{equation*}
\begin{equation*}
    96\alpha f(r)^2r^{10}+(-r^{12}+64\alpha r^{10})f(r)-\frac{\Lambda}{70}r^{14}+\frac{1}{3}r^{12}+\frac{64}{5}\alpha r^{10}+a^{12}=0
\end{equation*}

En $14 D$ para base manifold $(\mathbb{S}^2)^6$, $(\mathbb{H}^2)^6$ respectivamente son
\begin{equation*}
    140\alpha f(r)^2r^{12}+(-r^{14}-80\alpha r^{12})f(r)-\frac{\Lambda}{96}r^{16}+\frac{2}{7}r^{14}+\frac{40}{2}\alpha r^{12}+a^{14}=0
\end{equation*}
\begin{equation*}
    140\alpha f(r)^2r^{12}+(-r^{14}+80\alpha r^{12})f(r)-\frac{\Lambda}{96}r^{16}-\frac{2}{7}r^{14}+\frac{40}{2}\alpha r^{12}+a^{14}=0
\end{equation*}

En $d$ dimensiones el polinomio de Wheeler asociado es 
\begin{tcolorbox}
    \begin{equation*}
        d(d-4)\alpha f(r)^2r^{d-2}+f(r)[-r^d-8(d-4)\gamma r^{d-2}]-\frac{2\Lambda}{d^2-4}r^{d+2}+\frac{4}{d}\gamma r^d+a^d+C\alpha r^{d-2}=0
    \end{equation*}
\end{tcolorbox}
o de manera equivalente, considerando $d=2m$
    \begin{equation*}
      -4m(m-2)\alpha f(r)^2r^{2(m-1)}+16f(r)\gamma (m-2)r^{2(m-1)}+\textcolor{purple}{r^{2m}f(r)+\frac{\Lambda r^{2(m+1)}}{2(m^2-1)}-\frac{2\gamma r^{2m}}{m}+a^{2m}}-C\alpha r^{2(m-1)}=0
    \end{equation*}

\begin{equation*}
    \left(f(r)-\frac{2\gamma}{m}\right)r^{2m}+\frac{\Lambda}{2(m^2-1)}r^{2(m+1)}-4m(m-2)\alpha\left(\left(f(r)-\frac{2\gamma}{m}\right)^2+\frac{4}{m^2(m-1)}\right)r^{2(m-1)}+a^{2m}=0
\end{equation*}

\begin{equation*}
    F(r)r^{2m}-4m(m-2)\alpha_1\left(F(r)^2+\frac{4}{m^2(m-1)}\right)r^{2(m-1)}+\frac{\Lambda}{2(m^2-1)}r^{2(m+1)}+a^{2m}=0
\end{equation*}
donde $F(r)=f(r)-2\gamma/m$


