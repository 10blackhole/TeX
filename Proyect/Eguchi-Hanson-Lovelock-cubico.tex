\newpage
\section{Eguchi-Hanson en Lovelock cúbico}
Los polinomios de Wheeler encontrados son los siguientes:

En $8 D$ para base manifold $(\mathbb{T}^2)^3$,$(\mathbb{S}^2)^3$, $(\mathbb{H}^2)^3$, $\mathbb{CP}^3$ y $\mathbb{CH}^3$ respectivamente son
\begin{equation*}
    384\alpha_3f(r)^3 r^4-32\alpha_2f(r)^2r^6+r^8f(r)+\frac{\Lambda r^{10}}{30}+a^8=0
\end{equation*}
\begin{align*}
    &384\alpha_3f(r)^3 r^4+(-32\alpha_2r^6-576\alpha_3 r^4)f(r)^2+(r^8+32\alpha_2r^6+384\alpha_3r^4)f(r)+\frac{\Lambda r^{10}}{30}-\frac{r^8}{2}-\frac{32\alpha_3 r^6}{3}\\&-128\alpha_3r^4+a^8=0
\end{align*}
\begin{align*}
    &384\alpha_3f(r)^3 r^4+(-32\alpha_2r^6+576\alpha_3 r^4)f(r)^2+(r^8-32\alpha_2r^6+384\alpha_3r^4)f(r)+\frac{\Lambda r^{10}}{30}+\frac{r^8}{2}-\frac{32\alpha_3 r^6}{3}\\&+128\alpha_3r^4+a^8=0
\end{align*}
\begin{equation*}
    384\alpha_3f(r)^3r^4+(-32\alpha_2r^6-576\alpha_3r^4)f(r)^2+(r^8+32\alpha_2r^6+288\alpha r^4)f(r)+\frac{\Lambda r^{10}}{30}-\frac{r^8}{2}-8\alpha_2r^6-48\alpha_3r^4+a^8=0
\end{equation*}
\begin{equation*}
    384\alpha_3f(r)^3r^4+(-32\alpha_2r^6+576\alpha_3r^4)f(r)^2+(r^8-32\alpha_2r^6+288\alpha r^4)f(r)+\frac{\Lambda r^{10}}{30}+\frac{r^8}{2}-8\alpha_2r^6+48\alpha_3r^4+a^8=0
\end{equation*}

En $10 D$ para base manifold $(\mathbb{T}^2)^4$,$(\mathbb{S}^2)^4$, $(\mathbb{H}^2)^4$ respectivamente son
\begin{equation*}
    1920\alpha_3f(r)^3r^6-60\alpha_2f(r)^2r^8+r^{10}f(r)+\frac{\Lambda r^{12}}{48}+a^{10}=0
\end{equation*}
\begin{align*}
    &1920\alpha_3f(r)^3r^6+(-60\alpha_2r^8-2304\alpha_3 r^6)f(r)^2+(r^{10}+48\alpha_2 r^8+1152\alpha_3 r^6)f(r)+\frac{\Lambda r^{12}}{48}-\frac{2r^{10}}{5}\\&-12\alpha_2 r^8-256\alpha_3r^6+a^{10}=0
\end{align*}
\begin{align*}
    &1920\alpha_3f(r)^3r^6+(-60\alpha_2r^8+2304\alpha_3 r^6)f(r)^2+(r^{10}-48\alpha_2 r^8+1152\alpha_3 r^6)f(r)+\frac{\Lambda r^{12}}{48}+\frac{2r^{10}}{5}\\&-12\alpha_2 r^8+256\alpha_3r^6+a^{10}=0
\end{align*}

En $12 D$ para base manifold $(\mathbb{S}^2)^5$ es
\begin{align*}
    &5760\alpha_3f(r)^3r^8+(-96\alpha_2r^{10}-5760\alpha_3r^8)f(r)^2+(r^{12}+64\alpha_2r^{10}+2304\alpha_3r^8)f(r)+\frac{\Lambda r^{14}}{70}-\frac{r^{12}}{3}-\\&\frac{64\alpha_2 r^{10}}{5}-384\alpha_3r^8+a^{12}=0
\end{align*}

El polinomio de Wheeler asociado para $d$-dimensiones donde $d=2m$ va como
\begin{align*}
    &16m(m-1)(m-2)(m-3)\alpha_3f(r)^3 r^{2(m-2)}-\textcolor{blue}{4m(m-2)\alpha_2f(r)^2r^{2(m-1)}}\\&+\textcolor{blue}{f(r)\left(r^{2m}+16\gamma(m-2)\alpha_2 r^{2(m-1)}\right)}\textcolor{blue}{+\frac{\Lambda r^{2(m+1)}}{2(m^2-1)}-\frac{2\gamma r^{2m}}{m}+a^{2m}}=0
\end{align*}
