\section{Rodolfo}
\begin{center}
\begin{tabular}{ |p{7cm} | p{7cm}|} 
 \hline
 \textbf{Estructura métrica} & \textbf{Estructura afín} \\  \hline \hline
 Las propiedades métricas están dadas por el tensor métrico $g_{\mu\nu}(x)$,  $$\dd s^2=g_{\mu\nu }(x)\dd x^\mu \dd x^\nu .$$ Además, cada punto de la variedad posee su propio espacio tangente, tal que $$g_{\mu\nu }=e^{a}_{~\mu }(x)e^{b}_{~\nu }(x)\eta_{a b}.$$ & Las pripiedades afines están determinadas por la conexión $\omega^{ab},$ $$u^r_\parallel (x)=u^r(x+\dd x)+\dd x^\mu \theta ^r _{~s\mu }u^s(x).$$ Donde los coeficientes $\theta ^r _{~s\mu }$ define el paralelismo. \\
 \hline
\end{tabular}
\end{center}

\textbf{Objetivo 1:} Resolver las ecuaciones de campo de la teoría tensor-escalar-Maxwell en el formalismo de primer orden, utilizando un ansatz estático con sección transversa esférica, hiperbólica y plana. Analizar la estructura causal de este espacio-tiempo, es decir, singularidades, horizontes de eventos, entre otros.
\begin{center}
\begin{tabular}{|p{7cm} | p{0.2cm}|p{0.2cm}| p{0.2cm}| p{0.2cm}| p{0.2cm}| p{0.2cm}|p{0.2cm}| p{0.2cm}| p{0.2cm}|p{0.2cm}|p{0.2cm}|p{0.2cm}|} 
 \hline
 \textbf{Meses}&\textbf{S}&\textbf{O}&\textbf{N}&\textbf{D}&\textbf{E}&\textbf{F}&\textbf{M}&\textbf{A}&\textbf{M}&\textbf{J}&\textbf{J}&\textbf{A}\\ \hline
 \textbf{Actividad 1}: Hallar soluciones de agujero negro para las ecuaciones de campo en gravedad Einstein-Cartan-Maxwell, utilizando un ansatz estático de sección transversal esférica, hiperbólica y plana.&\textbf{X}&\textbf{X}&&&&&&&&&&\\ \hline
 \textbf{Actividad 2}: Verificar y analizar la existencia de las singularizades de los campos geométricos.&&&\textbf{X}&&&&&&&&&\\ \hline
\textbf{ Actividad 3}: Escribir los resultados en el borrador del artículo.&\textbf{X}&\textbf{X}&\textbf{X}&&&&&&&&&\\ \hline
\end{tabular}
\end{center}
 %-------------------------
 
\textbf{ Objetivo 2:} Analizar las propiedades físicas del campo escalar y el campo de Maxwell. Estudiar su comportamiento en diferentes regiones del espacio-tiempo y las condiciones de energía asociadas al tensor energía-momentum.
\begin{center}
\begin{tabular}{|p{7cm} | p{0.2cm}|p{0.2cm}| p{0.2cm}| p{0.2cm}| p{0.2cm}| p{0.2cm}|p{0.2cm}| p{0.2cm}| p{0.2cm}|p{0.2cm}|p{0.2cm}|p{0.2cm}|} 
 \hline
 \textbf{Meses}&\textbf{S}&\textbf{O}&\textbf{N}&\textbf{D}&\textbf{E}&\textbf{F}&\textbf{M}&\textbf{A}&\textbf{M}&\textbf{J}&\textbf{J}&\textbf{A}\\ \hline
 \textbf{Actividad 1}: Explorar las propiedades físicas de los campos de materia en diferentes regiones del espacio-tiempo.&&&&\textbf{X}&\textbf{X}&&&&&&&\\ \hline
 \textbf{Actividad 2}: Inspeccionar las condiciones de energía asociadas al tensor de energía-momentum.&&&&\textbf{X}&\textbf{X}&&&&&&&\\ \hline
\textbf{ Actividad 3}: Escribir los resultados en el borrador del artículo.&&&&\textbf{X}&\textbf{X}&&&&&&&\\ \hline
\textbf{ Actividad 4}: Escritura y envío del artículo.&&&&&&\textbf{X}&&&&&&\\ \hline
\end{tabular}
\end{center}

 %-------------------------
 
\textbf{ Objetivo 3:} Encontrar soluciones cosmológicas de la teoría tenso-escalar en el formalismo de primer orden. Analizar las características del campo escalar y cómo incide en la expansión acelarada del Universo.
\begin{center}
\begin{tabular}{|p{7cm} | p{0.2cm}|p{0.2cm}| p{0.2cm}| p{0.2cm}| p{0.2cm}| p{0.2cm}|p{0.2cm}| p{0.2cm}| p{0.2cm}|p{0.2cm}|p{0.2cm}|p{0.2cm}|} 
 \hline
 \textbf{Meses}&\textbf{S}&\textbf{O}&\textbf{N}&\textbf{D}&\textbf{E}&\textbf{F}&\textbf{M}&\textbf{A}&\textbf{M}&\textbf{J}&\textbf{J}&\textbf{A}\\ \hline
 \textbf{Actividad 1}: Hallar soluciones cosmológicas paras las ecuaciones de campo en gravedad Einstein-Cartan.&&&&&&&\textbf{X}&\textbf{X}&&&&\\ \hline
 \textbf{Actividad 2}: Examinar las propiedades de los campos de materia.&&&&&&&&&\textbf{X}&&&\\ \hline
\textbf{ Actividad 3}: Estudiar la incidencia de la torsión en la expansión acelerada del Universo.&&&&&&&&&\textbf{X}&&&\\ \hline
\textbf{ Actividad 4}: Escribir los resultados en el borrador del artículo.&&&&&&&\textbf{X}&\textbf{X}&\textbf{X}&&&\\ \hline
\end{tabular}
\end{center}
 %-------------------------

\textbf{ Objetivo 4:} Comparar las soluciones analíticas respecto a las soluciones de la teoría de la gravedad puramente métrica y constatar si se conectan continuamente al considerar torsión nula. Contrastar esta nueva solución con datos observacionales.
\begin{center}
\begin{tabular}{|p{7cm} | p{0.2cm}|p{0.2cm}| p{0.2cm}| p{0.2cm}| p{0.2cm}| p{0.2cm}|p{0.2cm}| p{0.2cm}| p{0.2cm}|p{0.2cm}|p{0.2cm}|p{0.2cm}|} 
 \hline
 \textbf{Meses}&\textbf{S}&\textbf{O}&\textbf{N}&\textbf{D}&\textbf{E}&\textbf{F}&\textbf{M}&\textbf{A}&\textbf{M}&\textbf{J}&\textbf{J}&\textbf{A}\\ \hline
 \textbf{Actividad 1}: Comparar las soluciones encontradas con los resultados de Relatividad General.&&&&&&&&&&\textbf{X}&&\\ \hline
 \textbf{Actividad 2}:Contrastar los nuevos resultados con los datos observacionales.&&&&&&&&&&\textbf{X}&\textbf{X}&\\ \hline
\textbf{ Actividad 3}: Escribir los resultados en el borrador del artículo.&&&&&&&&&&\textbf{X}&\textbf{X}&\\ \hline
\textbf{ Actividad 4}: Escribir y enviar el artículo.&&&&&&&&&&&&\textbf{X}\\ \hline
\end{tabular}
\end{center}













