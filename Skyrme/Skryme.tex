\documentclass[a4paper,10pt]{article}
\usepackage{jheppub} % for details on the use of the package, please see the JINST-author-manual
\usepackage{lineno}
\usepackage{amsmath,amsthm,amsfonts,amssymb,amscd,physics,cancel,mathtools}
\usepackage{tcolorbox}
\usepackage{marginnote,tensor}
\usepackage{tcolorbox}
%~~~~~~~~~ Document setup
% \usepackage[spanish]{babel} % English formatting
\usepackage[utf8]{inputenc} % Standard encoding
% \usepackage[a4paper,left=3cm,bottom=3cm]{geometry} % Page formatting
\usepackage{indentfirst} % Indents the first paragraph
\usepackage{amsmath} % Maths type package
\usepackage{bm} % Bold font maths
\usepackage{graphicx} % Advanced graphics package
\usepackage[export]{adjustbox} 
\usepackage{pdflscape} % Make pages landscape
\usepackage{fancyhdr} % Fancy headers
% \usepackage[colorlinks=true,citecolor=blue,urlcolor=blue,linkcolor=black]{hyperref} % Link colours
%\usepackage{natbib} % Bibliography
% \usepackage{flafter} % Reference any 'float'
% \usepackage[framemethod=tikz]{mdframed} % Box off stuff
\usepackage{color} % Colour support
\usepackage{wrapfig} % Text flowing around figures
\usepackage{lipsum} % Generates meaningless text
\usepackage{xcolor}
%\usepackage{biblatex}
%\usepackage[backend=bibtex]{biblatex}
%\addbibresource{bibliography.bib}
%\hypersetup{colorlinks=true, linkcolor=blue}

\newtheorem{ej}{Example}[section]
\newtheorem{sol}{Solution}[section]
\newtheorem{dem}{Proof}[section]

\def\a{\alpha}
\def\b{\beta}
\def\g{\gamma}
\def\G{\Gamma}
\def\d{\delta}
%\def\D{\Delta}
%\def\e{\eta}
\def\la{\lambda}
\def\La{\Lambda}
\def\k{\kappa}
\def\m{\mu}
\def\n{\nu}
\def\r{\rho}
\def\p{\rho}
\def\o{\omega}
\def\s{\sigma}
\def\S{\Sigma}
\def\t{\tau}
\def\p{\pi}
\def\f{\phi}
\def\vf{\varphi}
\def\ep{\epsilon}
\def\th{\theta}
\def\Th{\Theta}
\def\z{\zeta}
\def\id{\mathrm{I}}

\newcommand{\e}{\mathrm{e}}
\newcommand{\I}{\mathrm{I}}
\newcommand{\C}{\mathcal{C}}
\newcommand{\D}{\mathrm{D}}
\newcommand{\oo}{\mathring}
\newcommand{\oomega}{\mathring{\omega}}
\newcommand{\eabc}{\epsilon_{abc}}

%-----COLORS LIST ------
\definecolor{azure(colorwheel)}{rgb}{0.0, 0.5, 1.0}
\definecolor{DarkViolet}{RGB}{148,0,211}
\definecolor{myDarkBlue}{rgb}{0,0.1,0.7}
\definecolor{DarkBlue}{RGB}{0,0,153}
\definecolor{amber}{rgb}{1.0, 0.49, 0.0}
\definecolor{amaranth}{rgb}{0.9, 0.17, 0.31}
\definecolor{nicered}{rgb}{0.7,0.1,0.1}
\definecolor{brown}{rgb}{0.5,0.1,0.1}
\definecolor{nicegreen}{rgb}{0.0,0.3,0.0}
\definecolor{tealgreen}{rgb}{0.0, 0.51, 0.5}
\def\red#1{{\color{red} #1}}
\def\green#1{{\color{green} #1}}
\def\blue#1{{\color{blue} #1}}
\def\orange#1{{\color{orange} #1}}
%----------------------
\newcommand{\mycolor}{DarkViolet}
\def\myColor#1{{\color{\mycolor} #1}}
\definecolor{tclr}{RGB}{148,0,211}
%----------------------
\newcommand{\corr}[1]{\textcolor{nicered}{#1}}
\newcommand{\nick}[1]{\textcolor{olive}{#1}}
\newcommand{\teo}[1]{\textcolor{azure(colorwheel)}{#1}}
\newcommand{\chteo}[2]{\corr{\st{#1}} \teo{(#2)}}
\newcommand{\bako}[1]{\textcolor{DarkViolet}{#1}}
\newcommand{\than}[1]{\textcolor{magenta}{#1}}

%----------------------
\usepackage{hyperref}
\hypersetup{colorlinks,bookmarksopen,
	bookmarksnumbered,
	citecolor={nicered},
	linkcolor={myDarkBlue},
	urlcolor={tealgreen},
	pdfstartview=FitH}


% \arxivnumber{1234.56789} % if you have one

\title{\boldmath Skyrme model}

% Collaborations

%% [A] If main author
%% \collaboration{\includegraphics[height=17mm]{collabroation-logo}\\[6pt]
%%  XXX collaboration}

%% or
%% [B] If "on behalf of"
%% \collaboration[c]{on behalf of XXX collaboration}


% Authors
% The "\note" macro will give a warning: "Ignoring empty anchor...", you can safely ignore it.

%% [A] simple case: 2 authors, same institution
%% \author[1]{A. Uthor\note{Corresponding author.}}
%% \author{and A. Nother Author}
%% \affiliation{Institution,\\Address, Country}

%% or, e.g.
%% [B] more complex case: 4 authors, 3 institutions, 2 footnotes
%% \author[a,b]{F. Irst,\note{Now at another university}}
%% \author[c]{S. Econd,}
%% \author[a,2]{T. Hird\note{Also at Some University.}}
%% \author[c,2]{and Fourth}
%% \affiliation[a]{Institution_1,\\Address, Country}
%% \affiliation[b]{Institution_2,\\Address, Country}
%% \affiliation[c]{Institution_3,\\Address, Country}

\author{Borja Diez}
\affiliation{Universidad Arturo Prat}
% \affiliation{Another University,\\
% different-address, Country}

% E-mail addresses: only for the corresponding author
\emailAdd{borjadiez1014@gmail.com}

\abstract{Abstract...}




\begin{document}
\maketitle
%\tableofcontents
%\flushbottom

%\section{Differentiable manifolds}
\subsection{From Topological Spaces to Differentable Manifolds}
%TODO agregar imagen
It is assumed that the reader is acquainted with the notion of a topological space as a structure on which one can define a neighborhood and continuous functions. A \textbf{homeomorphism} between two topological spaces is a 1-1 map $\varphi: X\to Y$ for which both $\varphi$ and its inverse $\varphi^{-1}$ are continuous. If $\varphi$ and $\varphi^{-1}$ are continuously differentiable then $\varphi$ is called a \textbf{diffeomorphism}.

A $D$-dimensional manifold $M^D$ is a topological space that locally has the properties of a $D$-dimensional Euclidean space $\mathbb{R}^D$ : A neighborhood of a point in $M^D$
can continuously be mapped in a one-to-one way to the neighborhood of a point in
$\mathbb{R}^D$. To be more precise, introduce a \textbf{chart} $(U_\alpha,\varphi_\alpha)$ as a homeomorphism $\varphi_\alpha$ from
an open set $U_\alpha\subset M^D$ into an open set $R_\alpha\subset \mathbb{R}^D$. Two charts are compatible if the overlap maps are diffeomorphims $(\varphi_1\cdot \varphi_2\in C^\infty,\varphi_2\cdot \varphi_1^{-1}\in C^\infty)$ unless $U_1\cap U_2=\emptyset$. A set of compatible charts covering $M^D$ is called an atlas. In every chart the manifold can be equipped with a coordinate system: for $x\in M^D$ the coordinates are $x^\mu=\varphi(x)\in \mathbb{R}^D$. The naming makes it clear what one is aiming at. For instance the surface of a sphere, although not being homeomorphic to a plane, locally has enough smoothness to be mapped into an atlas. One chart is not sufficient since there will always be a point on the sphere that cannot be projected to the plane.

In manuscript will only treat finite-dimensional manifolds. One possibility of extending the notion of manifolds to infinite dimensions is to consider Banach manifolds modeled on Banach spaces. It is also assumed that we are dealing with $C^\infty$ manifolds. In certain contexts it might suffice that the charts are $C^k$-related. Also complex manifolds are investigated in mathematics and applied to modern theoretical physics (catchword: Kähler manifolds). In these the transition functions are required to be analytic.

\subsection{Tensor Bundles}
On a manifold one can erect tensor bundles as “superstructures” by starting with defining the tangent and cotangent spaces of a manifold.

\subsubsection{Tangent Bundle and Vector Fields}
We are interested in the notion of vectors on a manifold $M$ (henceforth I will mostly drop the index for the dimension of the manifold and for the Euclidean space). The idea is to introduce these as tangent vectors of curves 'through' $x\in M$: A curve through a point $x$ is a smooth mapping of an interval $I=[0,1]\subset\mathbb{R}$ to the manifold:
\begin{equation}
  C=\mathbb{I}\to M\qquad t\mapsto C(t)\qquad \mbox{with}\qquad C(0)=x
\end{equation}
The coordinates of this curve are $x^\mu(C(t))$, and the tangent vector to this curve is
\begin{equation}
  \dv{t}x^\mu(C(t))
\end{equation}
Since one can have more then one curve with $C(0)=x$, the proper definition is: A \textit{tangent vector} $x\in M$ is an equivalence class of curves in $M$, where the equivalence relation between two curves is that they are tangent at the point $x$. Another-equivalent- definition is to understand a tangent vector as a directional derivative: Consider functions $f\in \mathcal{F}M$, that is $f:M\to \mathbb{R}$. The change of $f$ along a curve is given by
\begin{equation}
  \dv{t}f(C(t)),\qquad \mbox{locally}\qquad \pdv{x^\mu}f\dv{x^\mu(C(t))}{t}
\end{equation}

In defining
\begin{equation}
  X=(X^\mu\partial_\mu)\qquad \mbox{with}\qquad X^\mu =\dv{x^\mu(C(t))}{t}
\end{equation}
we can write $\dv{t}f(C(t))=Xf$. For every point along the curve we take this expression to define the differential operator $X_x$ as the tangent vector to the manifold in $x\in M$. All tangent vectors at a point in the manifold can be shown to build a vector space $\mathfrak{X}_xM$ isomorphic to $\mathbb{R}^D$. The natural basis in $\mathfrak{X}_xM$ is the coordinate or \textbf{holonomic} basis $\{\partial_\mu\}$. But of course any other (\textbf{anholonomic}) basis $\{e_I}
\}$





























\section{Skyrme model}
Un campo de Skyrme es descrito por un modelo sigma no lineal con términos adicionales el cual puede ser escrito convenientemente en términos de un campo escalar $U$ valuado sobre algún grupo, digamos $SU(2)$. El Lagrangiano de Skyrme describe interacciones no-lineales a bajas energías de piones o bariones.

La acción de Einstein-Skyrme viene dada por
\begin{equation}
  S=S_G+S_{\rm Skyrme}
\end{equation}
donde 
\begin{align}
  S_G&=\kappa\int\dd^4x\sqrt{-g}(R-2\Lambda)\\
  S_{\rm Skyrme}&=\frac{K}{2}\int\dd^4x\sqrt{-g}\Tr\left(\frac{1}{2}R^\m R_\m +\frac{\lambda}{16}F_{\m\n }F^{\m\n }\right)
\end{align}
con $\kappa=(16\p G)^{-1}$, donde $G$ es la constante gravitacional de Newton y los parámetros $K$ y $\lambda$ son fijados por el experimento. Aquí $R_\m $ y $F_{\m\n }$ se definen según
\begin{align}
  R_\m :=U^{-1}\nabla_\m U\label{R}\\
  F_{\m\n}:=[R_\m, R_\n ]\label{F}
\end{align}

Las ecuaciones de Einstein resultantes son
\begin{equation}
  G_{\m\n }+\Lambda g_{\m\n }=\frac{1}{2\k }T_{\m\n }
\end{equation}
donde
\begin{equation}\label{Tmn}
  T_{\m\n }=-\frac{K}{2}\Tr\left[\left(R_\m R_\n -\frac{1}{2}g_{\m\n }R^\a R_\a \right)+\frac{\lambda}{4}\left(F_{\m\a }F_\n ^{~\a }-\frac{1}{4}g_{\m\n }F_{\a\b }F^{\a\b}\right)\right]
\end{equation}

Las ecuaciones de Skyrme vienen dadas por
\begin{equation}
  \nabla^\m R_\m +\frac{\lambda}{4}\nabla^\m [R^\n ,F_{\m\n }]=0
\end{equation}

Aquí $R_\m $ es expresado como
\begin{equation}
  R_\m =R_\m ^{i}t_i,\qquad i=1,2,3
\end{equation}
en la base de los generadores de $SU(2)$ $t_i$. Los índices de grupo (índices latinos) son subiso y bajados con la métrica plana $\d_{ij}$. Duchos generadores satisfacen la siguiente relación
\begin{equation}\label{titj}
  t_it_j=-\d_{ij}\id-\epsilon_{ijk}t_k
\end{equation}
donde $\id $ es la matriz identidad de $2\times 2$ y $\epsilon_{ijk}$ es el símbolo de Levi-Civita totalemnte antisimétrico con $\epsilon_{123}=\epsilon^{123}=1$. Los generadores $t_j$ se relacionan con las matrices de Pauli mediante $t_j=-i\sigma_j$.

Usando \eqref{titj} obtenemos la siguiente relación de conmutación,
\begin{align}
  [t_i,t_j]&=t_it_j-t_jt_i\\
  &=-\d_{ij}\id -\epsilon_{ijk}t_k+\d_{ji}\id +\epsilon_{jik}t_k\\
  &=-2\epsilon_{ijk}t_k\label{[titj]}
\end{align}
luego,
\begin{equation}
 \boxed{F_{\m\n }^{i}=[R_\m ,R_\n ]^{i}=-2\epsilon_{ijk}R_\m ^{i}R_\n ^j}
\end{equation}


De auí en adelante usaremos la parametrización estandar para un campo escalar $U(x^\m )$ valuado sobre $SU(2)$ dada por
\begin{equation}\label{U}
  U^{\pm }(x^\m )=Y^0(x^\m )\id \pm Y^{i}(x^\m )t_i
\end{equation}
donde los $Y^0$ y $Y^{i}$ satisfacen
\begin{equation}
  (Y^0)^2+Y^{i}Y_i=1
\end{equation}
Tomándole la derivada covariante a esta expresión es directo ver que
\begin{equation}\label{ydy0}
  Y^0\nabla_\m Y_0+Y^{i}\nabla_\m Y_i=0
\end{equation}

Con el fin de calcular cosas en esta teoría, es conveniente expresar cada uno de los objetos obtenidos a partir de $R_\m^{i}t_i$ a lo largo de los generadores del grupo, y es lo que haremos a partir de ahora.

De \eqref{R} y \eqref{U}
\begin{align}
  R_\m&=U^{-1}\nabla_\m U\\
  &=(Y^0\id-Y^{i}t_i)\nabla_\m (Y^0\id +Y^jt_j)\\
  &=(Y^0\id-Y^{i}t_i)(\nabla_\m Y^0\id +\nabla_\m Y^jt_j)\\
  &=Y^0\nabla_\m Y^0\id+Y^0\nabla Y^jt_j-Y^{i}\nabla Y^0t_i-Y^{i}\nabla_\m Y^jt_it_j\\
  &=Y^0\nabla_\m Y^0\id+Y^0\nabla Y^jt_j-Y^{i}\nabla Y^0t_i+Y^{i}\nabla_\m Y^j(\d_{ij}\id+\epsilon_{ijk}t_k)\\
  &=\underbrace{(Y^0\nabla_\m Y^0+Y^{i}\nabla_\m Y^{i})}_0\id +Y^0\nabla Y^it_i-Y^{i}\nabla Y^0t_i+\epsilon_{ijk}Y^{i}\nabla_\m Y^jt_k\\
  &=Y^0\nabla Y^it_i-Y^{i}\nabla Y^0t_i+\epsilon_{ijk}Y^{i}\nabla_\m Y^jt_k
\end{align}
donde hemos usado \eqref{ydy0}. Así, las componentes de $R_\m $ a lo largo de los generadores de $SU(2)$ vienen dadas por
\begin{equation}
  \boxed{R_\m^k=\epsilon^{ijk}Y_{i}\nabla_\m Y_j+Y^0\nabla_\m  Y^k-Y^{k}\nabla_\m  Y^0}
\end{equation}

Con este objeto, podemos construir todo el resto.

Notemos que
\begin{align}
  \Tr(t_it_j)&=\Tr(-\d_{ij}\id -\epsilon_{ijk}t_k)=-2\d_{ij}
\end{align}

De esta manera,
\begin{equation}
  \Tr(R_\m R_\n )=-2R_\m ^{i}R_\n ^j \d_{ij}
\end{equation}
Definiendo
\begin{equation}
  S_{\m\n }:=R_\m ^{i}R_\n ^j \d_{ij}
\end{equation}
se tiene
\begin{equation}\label{trrr}
  \Tr(R_\m R_\n )=-2S_{\m\n }
\end{equation}

Además,
\begin{align}
  \Tr(F_{\m\a }F_\n ^{~\a })&=\Tr(F_{\m\a }^{i}t_iF_\n ^{j\a }t_j)\\
  &=F_{\m\a }^{i}F_\n ^{j\a }\Tr(t_it_j)\\
  &=-2\d_{ij}F_{\m\a }^{i}F_\n ^{j\a }\\
  &=-2\d_{ij}(-2\epsilon_{ilm}R_\m ^lR_\a ^m)(-2\epsilon_{jpq}R_\n ^pR^{q\a })\\
  &=-8\d_{ij}(\epsilon_{ilm}\epsilon_{jpq}R_\m ^lR_\a ^m R_\n ^p R^{q\a })\\
  &=-8(\d_{lp}\d_{mq}-\d_{lq}\d_{mp})R_\m ^lR_\a ^m R_\n ^p R^{q\a }\\
  &=8S_{\m\a }S_\n ^{~\a }-8S_{\m\n }S
\end{align}
\begin{equation}\label{trff}
   \Tr(F_{\m\a }F_\n ^{~\a })=8S_{\m\a }S_\n ^{~\a }-8S_{\m\n }S
\end{equation}

Veamos ahora como queda \eqref{Tmn}. Primero notemos que
\begin{equation}
  \Tr(R-\m R_\n -\frac{1}{2}g_{\m\n }R^\a R_\a )=-2S_{\m\n }+g_{\m\n }S
\end{equation}
y
\begin{equation}
  \Tr(F_{\m\a }F_\n ^{~\a }-\frac{1}{4}g_{\m\n }F_{\a\b }F^{\a\b })=8S_{\m\a }S_\n ^{~\a }-8S_{\m\n }S-\frac{1}{4}g_{\m\n }(8S_{\a\b }S^{\a\b }-S^2)
\end{equation}
Luego, es directo ver que
\begin{equation}
  \boxed{T_{\m\n }=K\left[S_{\m\n }-\frac{1}{2}g_{\m\n}S+\lambda\left\{S_{\m\n}S-S_{\m\a }S_\n ^{~\a }-\frac{1}{4}g_{\m\n}(S^2-S_{\a\b}S^{\a\b })\right\}\right]}
\end{equation}


























































% Bibliography

%% [A] Recommended: using JHEP.bst file
%% \bibliographystyle{JHEP}
%% \bibliography{biblio.bib}

%% or
%% [B] Manual formatting (see below)
%% (i) We suggest to always provide author, title and journal data or doi:
%% in short all the informations that clearly identify a document.
%% (ii) please avoid comments such as "For a review'', "For some examples",
%% "and references therein" or move them in the text. In general, please leave only references in the bibliography and move all
%% accessory text in footnotes.
%% (iii) Also, please have only one work for each \bibitem.



\newpage
\bibliographystyle{JHEP}
\bibliography{biblio.bib}
\end{document}
