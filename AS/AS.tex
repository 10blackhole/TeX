\documentclass[a4paper,11pt]{article}
\usepackage{jheppub} % for details on the use of the package, please see the JINST-author-manual
\usepackage{lineno}
\usepackage{amsmath,amsthm,amsfonts,amssymb,amscd,physics,cancel,mathtools}
\usepackage{mathrsfs}
\usepackage{tcolorbox}
\usepackage{marginnote,tensor}
\usepackage{tcolorbox}
%~~~~~~~~~ Document setup
% \usepackage[spanish]{babel} % English formatting
\usepackage[utf8]{inputenc} % Standard encoding
% \usepackage[a4paper,left=3cm,bottom=3cm]{geometry} % Page formatting
\usepackage{indentfirst} % Indents the first paragraph
\usepackage{amsmath} % Maths type package
\usepackage{bm} % Bold font maths
\usepackage{graphicx} % Advanced graphics package
\usepackage[export]{adjustbox} 
\usepackage{pdflscape} % Make pages landscape
\usepackage{fancyhdr} % Fancy headers
% \usepackage[colorlinks=true,citecolor=blue,urlcolor=blue,linkcolor=black]{hyperref} % Link colours
%\usepackage{natbib} % Bibliography
% \usepackage{flafter} % Reference any 'float'
% \usepackage[framemethod=tikz]{mdframed} % Box off stuff
\usepackage{color} % Colour support
\usepackage{wrapfig} % Text flowing around figures
\usepackage{lipsum} % Generates meaningless text
\usepackage{xcolor}
%\usepackage{biblatex}
%\usepackage[backend=bibtex]{biblatex}
%\addbibresource{bibliography.bib}
%\hypersetup{colorlinks=true, linkcolor=blue}

\newtheorem{ej}{Example}[section]
\newtheorem{sol}{Solution}[section]
\newtheorem{dem}{Proof}[section]
\newtheorem{prop}{Propiedad}[section]

\def\a{\alpha}
\def\b{\beta}
\def\g{\gamma}
\def\G{\Gamma}
\def\d{\delta}
%\def\D{\Delta}
%\def\e{\eta}
\def\la{\lambda}
\def\La{\Lambda}
\def\k{\kappa}
\def\m{\mu}
\def\n{\nu}
\def\r{\rho}
\def\p{\rho}
\def\o{\omega}
\def\s{\sigma}
\def\S{\Sigma}
\def\t{\tau}
\def\p{\pi}
\def\f{\phi}
\def\vf{\varphi}
\def\ep{\epsilon}
\def\th{\theta}
\def\Th{\Theta}
\def\z{\zeta}
\def\id{\mathrm{I}}
\def\M{\mathcal{M}}
\def\E{\mathcal{E}}
\def\tn{\tilde{\nabla}}
\def\TL{\text{TL}}
\def\A{\mathbb{A}}
\def\i{\mathrm{i}}
\def\M{\mathscr{M}}
\def\LL{\mathscr{L}}


%-----COLORS LIST ------
\definecolor{azure(colorwheel)}{rgb}{0.0, 0.5, 1.0}
\definecolor{DarkViolet}{RGB}{148,0,211}
\definecolor{myDarkBlue}{rgb}{0,0.1,0.7}
\definecolor{DarkBlue}{RGB}{0,0,153}
\definecolor{amber}{rgb}{1.0, 0.49, 0.0}
\definecolor{amaranth}{rgb}{0.9, 0.17, 0.31}
\definecolor{nicered}{rgb}{0.7,0.1,0.1}
\definecolor{brown}{rgb}{0.5,0.1,0.1}
\definecolor{nicegreen}{rgb}{0.0,0.3,0.0}
\definecolor{tealgreen}{rgb}{0.0, 0.51, 0.5}
\def\red#1{{\color{red} #1}}
\def\green#1{{\color{green} #1}}
\def\blue#1{{\color{blue} #1}}
\def\orange#1{{\color{orange} #1}}
%----------------------
\newcommand{\mycolor}{DarkViolet}
\def\myColor#1{{\color{\mycolor} #1}}
\definecolor{tclr}{RGB}{148,0,211}
%----------------------
\newcommand{\corr}[1]{\textcolor{nicered}{#1}}
\newcommand{\nick}[1]{\textcolor{olive}{#1}}
\newcommand{\teo}[1]{\textcolor{azure(colorwheel)}{#1}}
\newcommand{\chteo}[2]{\corr{\st{#1}} \teo{(#2)}}
\newcommand{\bako}[1]{\textcolor{DarkViolet}{#1}}
\newcommand{\than}[1]{\textcolor{magenta}{#1}}

\newcommand{\rc}{\textcolor{red}}
\newcommand{\bc}{\textcolor{blue}}
\newcommand{\cc}{\textcolor{cyan}}
\newcommand{\gc}{\textcolor{green}}
\newcommand{\occ}{\textcolor{orange}}
\newcommand{\pc}{\textcolor{purple}}

%----------------------
\usepackage{hyperref}
\hypersetup{colorlinks,bookmarksopen,
	bookmarksnumbered,
	citecolor={nicered},
	linkcolor={myDarkBlue},
	urlcolor={blue},
	pdfstartview=FitH}


% \arxivnumber{1234.56789} % if you have one

\title{\boldmath Notes on Asymptotic Symmetries}

% Collaborations

%% [A] If main author
%% \collaboration{\includegraphics[height=17mm]{collabroation-logo}\\[6pt]
%%  XXX collaboration}

%% or
%% [B] If "on behalf of"
%% \collaboration[c]{on behalf of XXX collaboration}


% Authors
% The "\note" macro will give a warning: "Ignoring empty anchor...", you can safely ignore it.

%% [A] simple case: 2 authors, same institution
%% \author[1]{A. Uthor\note{Corresponding author.}}
%% \author{and A. Nother Author}
%% \affiliation{Institution,\\Address, Country}

%% or, e.g.
%% [B] more complex case: 4 authors, 3 institutions, 2 footnotes
%% \author[a,b]{F. Irst,\note{Now at another university}}
%% \author[c]{S. Econd,}
%% \author[a,2]{T. Hird\note{Also at Some University.}}
%% \author[c,2]{and Fourth}
%% \affiliation[a]{Institution_1,\\Address, Country}
%% \affiliation[b]{Institution_2,\\Address, Country}
%% \affiliation[c]{Institution_3,\\Address, Country}

\author{Borja Diez}
\affiliation{Universidad Arturo Prat}
% \affiliation{Another University,\\
% different-address, Country}

% E-mail addresses: only for the corresponding author
\emailAdd{borjadiez1014@gmail.com}

\abstract{These notes are based on \cite{Ciambelli:2022vot} and are for personal study purposes only.}




\begin{document}
\maketitle
%\tableofcontents
%\flushbottom

%\section{Differentiable manifolds}
\subsection{From Topological Spaces to Differentable Manifolds}
%TODO agregar imagen
It is assumed that the reader is acquainted with the notion of a topological space as a structure on which one can define a neighborhood and continuous functions. A \textbf{homeomorphism} between two topological spaces is a 1-1 map $\varphi: X\to Y$ for which both $\varphi$ and its inverse $\varphi^{-1}$ are continuous. If $\varphi$ and $\varphi^{-1}$ are continuously differentiable then $\varphi$ is called a \textbf{diffeomorphism}.

A $D$-dimensional manifold $M^D$ is a topological space that locally has the properties of a $D$-dimensional Euclidean space $\mathbb{R}^D$ : A neighborhood of a point in $M^D$
can continuously be mapped in a one-to-one way to the neighborhood of a point in
$\mathbb{R}^D$. To be more precise, introduce a \textbf{chart} $(U_\alpha,\varphi_\alpha)$ as a homeomorphism $\varphi_\alpha$ from
an open set $U_\alpha\subset M^D$ into an open set $R_\alpha\subset \mathbb{R}^D$. Two charts are compatible if the overlap maps are diffeomorphims $(\varphi_1\cdot \varphi_2\in C^\infty,\varphi_2\cdot \varphi_1^{-1}\in C^\infty)$ unless $U_1\cap U_2=\emptyset$. A set of compatible charts covering $M^D$ is called an atlas. In every chart the manifold can be equipped with a coordinate system: for $x\in M^D$ the coordinates are $x^\mu=\varphi(x)\in \mathbb{R}^D$. The naming makes it clear what one is aiming at. For instance the surface of a sphere, although not being homeomorphic to a plane, locally has enough smoothness to be mapped into an atlas. One chart is not sufficient since there will always be a point on the sphere that cannot be projected to the plane.

In manuscript will only treat finite-dimensional manifolds. One possibility of extending the notion of manifolds to infinite dimensions is to consider Banach manifolds modeled on Banach spaces. It is also assumed that we are dealing with $C^\infty$ manifolds. In certain contexts it might suffice that the charts are $C^k$-related. Also complex manifolds are investigated in mathematics and applied to modern theoretical physics (catchword: Kähler manifolds). In these the transition functions are required to be analytic.

\subsection{Tensor Bundles}
On a manifold one can erect tensor bundles as “superstructures” by starting with defining the tangent and cotangent spaces of a manifold.

\subsubsection{Tangent Bundle and Vector Fields}
We are interested in the notion of vectors on a manifold $M$ (henceforth I will mostly drop the index for the dimension of the manifold and for the Euclidean space). The idea is to introduce these as tangent vectors of curves 'through' $x\in M$: A curve through a point $x$ is a smooth mapping of an interval $I=[0,1]\subset\mathbb{R}$ to the manifold:
\begin{equation}
  C=\mathbb{I}\to M\qquad t\mapsto C(t)\qquad \mbox{with}\qquad C(0)=x
\end{equation}
The coordinates of this curve are $x^\mu(C(t))$, and the tangent vector to this curve is
\begin{equation}
  \dv{t}x^\mu(C(t))
\end{equation}
Since one can have more then one curve with $C(0)=x$, the proper definition is: A \textit{tangent vector} $x\in M$ is an equivalence class of curves in $M$, where the equivalence relation between two curves is that they are tangent at the point $x$. Another-equivalent- definition is to understand a tangent vector as a directional derivative: Consider functions $f\in \mathcal{F}M$, that is $f:M\to \mathbb{R}$. The change of $f$ along a curve is given by
\begin{equation}
  \dv{t}f(C(t)),\qquad \mbox{locally}\qquad \pdv{x^\mu}f\dv{x^\mu(C(t))}{t}
\end{equation}

In defining
\begin{equation}
  X=(X^\mu\partial_\mu)\qquad \mbox{with}\qquad X^\mu =\dv{x^\mu(C(t))}{t}
\end{equation}
we can write $\dv{t}f(C(t))=Xf$. For every point along the curve we take this expression to define the differential operator $X_x$ as the tangent vector to the manifold in $x\in M$. All tangent vectors at a point in the manifold can be shown to build a vector space $\mathfrak{X}_xM$ isomorphic to $\mathbb{R}^D$. The natural basis in $\mathfrak{X}_xM$ is the coordinate or \textbf{holonomic} basis $\{\partial_\mu\}$. But of course any other (\textbf{anholonomic}) basis $\{e_I}
\}$





























\section{Symmetries}
 We will focus on global and local symmetries firstly, and later on local symmetries only, as they are the main actor in the theory of asymptotic symmetries and the corner proposal.
 
 The naive textbook definition of global and local symmetries is that the latter are generated by spacetime-dependent parameters while the former are generated by constant-in-spacetime ones. Then, global symmetries are physical, acting non-trivially on the system, whereas local symmetries are redundancies, i.e. gauge symmetries, simply expressing our ignorance in defining the physical variables of said system. As we will discuss in detail, this textbook definition is too naive, because \textit{there are local symmetries that become physical in the presence of boundaries}. The method one should use to distinguish trivial symmetries from physical ones is not whether they are generated by constant parameters or not, but rather whether they have vanishing associated charges or not. The tools to address this question, the celebrated Noether’s theorems, are ubiquitous in theoretical and mathematical physics.
 
 There are two main frameworks that we can use to study Noether's theorems, each coming with some advantages and drawbacks.
 
 \begin{itemize}
 	\item \textit{Hamiltonian approach}: in this framework, time and space are split, and one studies the evolution of trajectories in phase space. The pros are: it is a robust, unambiguous formulation, and it is the closest approach of classical physics to quantum physics. The cons are mainly two. First it can become quickly a technical and heavy machinery. Second, and more importantly for the discussion here, it explicitly breaks spacetime covariance.
 	\item \textit{Lagrangian approach}: instead of an evolutionary problem, the phase space here is thought of as the set of solutions to the equations of motion of the theory. The latter can be shown to be in one-to-one correspondence with the space of evolution of trajectories. The pros are that it is easier to manipulate and retains spacetime covariance. The cons are that it is more ambiguous, and sometimes, it is harder to gather physical intuition with it.
 \end{itemize}
 
 It is in the Lagrangian approach that we will express Noether's theorems. In this approach, the so-called covariant phase space formalism applies, which is the framework we will use for the rest of these lectures. It stems from Noether's works, but it reached its final form thanks to the works of Wald et al. \cite{Lee:1990nz,Wald:1993nt,Iyer:1994ys,Iyer:1995kg,Wald:1999wa}.
 
 \subsection{Covariant Phase Space}
 The idea is to develop the calculus both in spacetime and in field space.
 
\subsubsection*{Spacetime Calculus} Consider a differentiable manifold $M$. Prior to a metric structure, one can introduce the de Rham calculus on the space of forms. Given a vector field $\xi\in TM$, a one form is an application from $TM$ to $C^\infty(M)$, that is, an element of the dual bundle $T^*M$. A generic $p$-form is an element of $\wedge^pT^*M$, where the symbol $\wedge$ stands for the wedge (skew-symmetric) product. The space of all forms constitutes the \textit{de Rham complex}
\begin{equation}
	\Omega^\bullet(M,\mathbb{R}):=\bigoplus_{n=0}^d\wedge^n T^*M,
\end{equation}
where $0$-forms are scalars, and the maximal degree of a form is the dimension of the manifold, $d =$dim($M$). We will denote by $\dd $ and $\i$ the exterior derivative and interior product on this complex, respectively. The exterior derivative increases the form degree by one, while the interior product decreases it by one. The exterior derivative is assumed to be a co-boundary on the complex, which means that it is a nilpotent operation $\dd^2=0$. The Lie derivative compares these two operations when applied in different orders,
\begin{equation}
	\mathcal{L}_\xi:=\dd i_\xi + i_\xi\dd ,
\end{equation}
and thus it does not change the form degree. This formula is sometimes referred to as \textit{Cartan’s magic formula}. In the following, we will repeatedly use that two interior products anti-commute, that $i_\xi $ of a scalar is zero, and $\dd $ of a spacetime top-form vanishes.

\subsubsection*{Variational calculus}
One can reproduce these results on the field space. A field space $\G $ is defined as the space of all possible field configurations. It is assumed to be a differentiable manifold. One can thus introduce a calculus on the space of forms on such manifold, called variational calculus. Given a vector field $V\in T\G $, a one form is an application from $T\G $ to the space of functionals, $F=C^\infty (\G )$. A generic $p$-form is an element of $\wedge^pT^*\G $, and the space of all forms constitutes the \textit{variational complex}
\begin{equation}
	\Omega^\bullet(\G ,\mathbb{R}):=\bigoplus_{n=0}^{\dim \G }\wedge^n T^*\G ,
\end{equation}
where $0$-forms are now functionals. We will denote by $\d$ and $I$ the exterior derivative and interior product on this complex, respectively, with $\d^2=0$. The exterior derivative increases the form degree by one, and it is what we commonly refer to as a field variation, while the interior product decreases it by one, and we usually think of it as a field contraction. Cartan’s magica formula then gives the Lie derivative
\begin{equation}
	\mathfrak{L}_V:=\d I_V+I_V\d .
\end{equation}
One can put together spacetime and field-space forms to form the variational bicomplex, defined on $(M,\Gamma)$. We will use the notation ($p, q$)-form to refer to a form on the bicomplex which is a spacetime $p$-form and a field-space $q$-form. On the bicomplex, one can construct an exterior derivative which is $\dd +\d $. The latter is nilpotent if $\dd \d=-\d \dd $. In the following, we will repeatedly use that $I_V$ of a functional is zero, and that two interior products anti-commute. Unless stated otherwise, we will also assume that vector fields in $TM$ are field-independent, such that $\d\xi=0$, for all $\xi\in TM$.

\subsubsection*{Lagrangian theory}
Let us apply this formalism to a Lagrangian theory. The action reads
\begin{equation}
	S=\int_ML.
\end{equation}

The Lagrangian $L$ is thus a top form in spacetime, and does not contain field variations, so it is a functional of the fields $\varphi\in \G$. Given that $d=\dim M$, the Lagrangian is a $(d, 0)$-form, in the language just established.

Under an arbitrary field variation $\varphi\to\varphi+\d\varphi$, it is always possible to rewrite the Lagrangian as a term linear in the variations, and a total derivative
\begin{equation}\label{6}
	\d L=(\text{EOM})\d\varphi+\dd\theta.
\end{equation}
This is an important equality. The first term on the right-hand side of \eqref{6} gives the equations of motion of the theory, while $\theta$ is the local pre-symplectic potential (we will explain shortly the "pre"). It contains a field variation and appears with a total derivative in spacetime, so it is a $(d − 1, 1)$-form. Suppose $M$ has a boundary $\partial M$. Denoting by $\approx$ an equality that holds only on shell of the equations of motion, we then have
\begin{equation}\label{7}
	\d S=\int_M\d L=\int_M((\text{EOM})\d\varphi+\dd\theta)\approx\int_{\partial M}\theta.
\end{equation}
The local pre-symplectic form is defined as $\omega:=\d \theta$, which is a $(d-1,2)$-form. This local expression can be integrated on a Cauchy slice $\S $ to give the so-called pre-symplectic $2$-form
\begin{equation}\label{8}
	\Omega:=\int_\S\omega,
\end{equation}
 which is thus a $(0,2)$-form. This quantity is a crucial ingredient of the theory, because it carries the Poisson bracket, as we will see.
 
 Consider a vector field $V\in T\G$ which is an isometry of $\omega$, that is, $\mathfrak{L}_V\omega=0$. Using that $\d\omega=\d^2\theta=0$, we get
 \begin{equation}\label{9}
 	\mathfrak{L}_V\omega=(I_V\d+\d I_V)	\omega=\d(I_V\omega)=0.
 \end{equation}
 Assuming trivial cohomology in the space of $1$-forms on $\G $ (or else assuming $\mathfrak{L}_V\theta=0$), we derive
 \begin{equation}\label{10}
 	\d I_V\omega=0\implies I_V\omega=-\d J_V,
 \end{equation}
 for a $(d-1,0)$-form $J_V$. Notice that we will disregard embeddings in this first part, to connect to older parts of the literature, and thus we will perform manipulations such as
 \begin{equation}\label{11}
 	\d\Omega=\d\int_\S\omega=\int_\S\d\omega=0.
 \end{equation}
This kind of manipulations is not harmless, and it is by understanding it carefully that the theory of embeddings in the covariant phase space emerged.
%TODO add a phrase.
 Assuming \eqref{11} holds, one defines the global functional $H_V$ via
 \begin{equation}\label{12}
 	I_V\Omega:=-\d H_V,\qquad H_V:=\int_\S J_V.
 \end{equation}
 A vector field $V\in T\G $ that satisfies $\mathfrak{L}_V\Omega=0$ is called a \textit{symplectomorphism}. A vector field satisfying \eqref{12} is called a \textit{Hamiltonian vector field}. For trivial cohomology, no distinction is required, for all symplectomorphisms are Hamiltonian.
 
 
We have introduced all the ingredients necessary to formulate Noether’s theorems. We reiterate that we are here considering a setup where there are no issues coming from symplectic or spacetime fluxes, and the charges we will construct are assumed to be integrable, conserved, and finite, even without a careful treatment of embeddings (assuming \eqref{11}). In this setup, charges come directly from contracting the pre-symplectic $2$-form. These are called \textit{canonical charges}, and are unanimously defined in the literature to be the physical charges. We will review later what happens when these assumptions fail, and the possible resolutions. In particular, the assumptions made here are too stringent to describe a general gravitational system (or more precisely to describe all asymptotic symmetries in such system), where charges cannot be constructed canonically from the pre-symplectic $2$-form, due to fluxes. There are many proposals addressing this problem, as we will review in the following.

\subsection{Noether's Theorems and Charge Algebra}
The quantity $J_V$ introduced above is called the \textit{local Noether current}. Its expression depends on the nature of the symmetry.

\subsubsection*{Internal symmetry} We directly compute
\begin{equation}\label{13}
	-\d J_V=I_V\omega=\mathfrak{L}_V\theta-\d I_V\theta.
\end{equation}
If one assumes that $\mathfrak{L}_V\theta=0$, then
\begin{equation}\label{14}
	J_V=I_V\theta,
\end{equation}
modulo $\d$-exact terms. Note that we have not imposed the equations of motion at this stage.

\subsubsection*{Spacetime symmetry}
Spacetime symmetry To distinguish this case from the previous one, we refer to the vector field in field space associated to a spacetime vector field $\xi$ as $V_\xi $ . We compute as before
\begin{equation}\label{15}
	-\d J_{V_\xi }=I_{V_\xi }\omega=\mathfrak{L}_{V_\xi }\theta-\d I_{V_\xi }\theta.
\end{equation}
Now, however, one has that $\mathfrak{L}_{V_\xi }\theta=\mathcal{L}_\xi \theta$, because we are dealing with spacetime transformations
(this is background independence, in the usual language, $\d_{V_\xi}=\mathcal{L}_\xi $ ). Then we get
\begin{equation}\label{16}
	-\d J_{V_\xi }=\mathcal{L}_\xi \theta-\d I_{V_\xi }\theta=i_\xi \dd \theta+\dd\i_\xi \theta-\d I_{V_\xi }\theta.
\end{equation}
If we assume that $\dd i_\xi \theta$ pulls back to zero at the boundary under scrutiny\footnote{It is this assumption that is problematic in general, with fluxes, and led to different appreciations of the Noether current in the literature} and we go on-shell, we obtain
\begin{equation}\label{17}
	J_{V_\xi }\approx I_{V_\xi }\theta-i_\xi L,
\end{equation}
again modulo $\d $-exact terms. In contrast to the previous derivation, we here assumed the equations of motion. This current is called the \textit{local weakly-vanishing Noether current} because, as we show hereafter, it is identically zero on-shell, modulo $\d $-exact terms.

\subsubsection*{Noether's first theorem}
This theorem states that given a global symmetry, there is an associated codimension-$1$ conserved quantity. To show this, we evaluate the field-space Lie derivative of the action under the symmetry in question
\begin{equation}\label{18}
	\mathfrak{L}_VS=\int_M(I_V\d +\d I_V)L=\int_M I_V\d L=\int_M(I_V(\text{EOM}\d\varphi)+\dd I_V\theta).
\end{equation}
If $V$ is a global symmetry, then $\mathfrak{L}_VS=0$. Using \eqref{14}, we get
\begin{equation}\label{19}
	\dd J_V=-I_V(\text{EOM}\dd\varphi)\approx 0.
\end{equation}
So the local Noether current is conserved on-shell. This implies that the global functional
\begin{equation}\label{20}
	H_V:=\int_\S J_V,
\end{equation}
is the integral of a codimension-$1$ conserved global charge on-shell, and it is called the \textit{Noether charge}.

\subsubsection*{Noether's second theorem}
This theorem states that given a gauge symmetry, there is an associ-
ated codimension-$2$ conserved quantity. This means that \footnote{We generically refer to a vector field in field space as $V$. When some properties apply only to a $V$ associated to a spacetime diffeomorphism, we use $V_\xi $ . We will call the charge $H_\xi $ instead of $H_{V_\xi }$ , simply to be compatible with the common notation in the literature.}
\begin{equation}\label{21}
	J_V\approx \dd Q_V.
\end{equation}
Conservation then follows straightforwardly from $\dd^2=0$. Let us show that this current is at best a $d$-exact term on-shell. To show this, we evaluate the field-space Lie derivative of the action under the symmetry in question
\begin{equation}\label{22}
	\mathfrak{L}_VS=\int_M(I_V\d +\d I_V)L=\int_M(I_V(\text{EOM}\d\varphi)+\dd I_V\theta)=\int_MI_V(\text{EOM}\d\varphi)+\int_{\partial M}I_V\theta.
\end{equation}
If $V$ is an internal gauge symmetry, then we have $\mathfrak{L}_VS=0$ and we proceed as before. If, on the other hand, $V=V_\xi $ is associated to a spacetime symmetry, then we have $\mathfrak{L}_{V_\xi }S=\mathcal{L}_\xi S$. So we compute
\begin{equation}\label{23}
	\mathcal{L}_\xi S=\int_M\mathcal{L}_\xi L=\int_M\dd i_\xi L=\int_{\partial M}i_\xi L.
\end{equation}
Equating \eqref{22} with \eqref{23} we gather
\begin{equation}\label{24}
	\int_{\partial M}I_{V_\xi }\theta\approx\int_{\partial M}i_\xi L\implies J_{V_\xi }\approx \dd Q_\xi .
\end{equation}
Proceeding as before, we define the global functional
\begin{equation}\label{25}
	H_\xi :=\int_\S J_{V_\xi }.
\end{equation}
However, this becomes a codimension-$2$ conserved charge on-shell. Indeed, if the hypersurface $\S $ has a codimension-$2$ surface $S=\partial\S $ at its boundary, we obtain
\begin{equation}\label{26}
	H_\xi \approx \int_\S \dd Q_\xi =\int_S Q_\xi .
\end{equation}

This is again called the \textit{Noether charge}, but for gauge symmetries. Given its codimension-$2$ nature, it is sometimes referred to as \textit{surface charge}. The surface $S$ is generically called a $corner$, and so this charge is also called a $corner charge$, in the recent literature. It is this theorem that determines whether a gauge symmetry, in the presence of a boundary, is still a pure gauge transformation or becomes physical. In fact, this distinction is encoded in whether the associated Noether charge vanishes or not, respectively.

\subsubsection*{Poisson bracket}
A symplectic $2$-form has two properties, it is closed in field space, and it is non-degenerate:
\begin{equation}
	\d\Omega=0,\qquad I_V\Omega=0\Leftrightarrow V=0.
\end{equation}
The reason why we used the expression "pre-"symplectic so far is that non-degeneracy is not guaranteed, and indeed there are in general zero modes, that is, non-vanishing symmetries that annihilate $\Omega$. Once these modes are removed (by quotienting these symmetries out), the symplectic $2$-form carries a Poisson bracket representation of the charge algebra. Consider $V,W\in T\G $, then we define the Poisson bracket between symplectomorphisms as
\begin{equation}\label{28}
	\{H_V,H_W\}=\mathfrak{L}_W H_V.
\end{equation}
Using this definition and the fact that $V,W$ are symplectomorphisms, one easily proves that the bracket is skew symmetric
\begin{equation}\label{29}
	\{H_V,H_W\}=I_W\d H_V=-I_WI_V\Omega=I_VI_W\Omega=-I_V\d H_W=-\{H_W,H_V\}.
\end{equation}
The Poisson bracket is a fundamental ingredient in the theory of asymptotic symmetries, as it gives rise to the symmetry principle organising observables of a given theory. Along the way, we have also shown that
\begin{equation}\label{30}
	\{H_V,H_W\}=I_VI_W\Omega,
\end{equation}
 which demonstrates that the symplectic $2$-form determines the Poisson bracket of the theory.
 
 Consider now spacetime symmetries, where the definition \eqref{28}, given two spacetime vector fields $\xi, \z  \in T\G $, reads
 \begin{equation}\label{31}
 	\{H_\xi,H_\z \}=\mathfrak{L}_{V_\z  }H_\xi .
 \end{equation}
 On the other hand, we also have in this case the Lie bracket of vector fields in $TM$
 \begin{equation}\label{32}
 	[\xi,\z ]=\mathcal{L}_\xi \z .
 \end{equation}
 How is the Poisson bracket of charges (and thus the charge algebra) related to the Lie bracket of vector fields (and thus the symmetry algebra)? To answer this question, we need to demonstrate the following result. Given two symplectomorphisms $V_\xi $ and $V_\z $ , one has
 \begin{equation}\label{33}
 	I_{V_{[\xi ,\z ]}}\Omega=\d \{H_\xi ,H_\z \}.
 \end{equation}
 Let us prove this. Using $I_{[V_\xi ,V_\z ]}=I_{V_{[\xi,\z ]}}$ , and recalling the calculus identity
 \begin{equation}\label{34}
 	I_{[V_\xi ,V_\z ]}=[\mathfrak{L}_{V_\xi },I_{V_\z }],
 \end{equation}
 we get
 \begin{equation}\label{35}
 	I_{V_{[\xi,\z ]}}\Omega=[\mathfrak{L}_{V_\xi },I_{V_\z }]\Omega=\mathfrak{L}_{V_\xi }I_{V_\z }\Omega-I_{V_\z }\mathfrak{L}_{V_\xi }\Omega.
 \end{equation}
 The last term vanishes, for $V_\xi $ is a symplectomorphism, and thus
 \begin{equation}\label{36}
 	I_{V_{[\xi,\z ]}}\Omega=\d I_{V_\xi }I_{V_\z }\Omega+I_{V_\xi }\d I_{V_\z }\Omega.
 \end{equation}
 Using $\d\Omega=0$, we can rewrite $I_{V_\xi }\d I_{V_\z }\Omega=I_{V_\xi }\mathfrak{L}_{V_\z }\Omega=0$, because also $V_\z $ is a symplectomorphisms, such that we eventually gather
\begin{equation}\label{37}
	I_{V_{[\xi,\z ]}}\Omega=\d I_{V_\xi }I_{V_\z }\Omega=\d \{H_\xi,H_\z \},
\end{equation}
which is exactly what we wanted to prove, \eqref{33}. Using similar manipulations, we leave as an exercise to prove that the Lie bracket Jacobi identity induces the Poisson bracket Jacobi identity.

We wish to establish the relationship between the charges and the vector fields algebras. First we observe that the identity $I_{V_\xi }\Omega=-\d H_\xi $ sets a correspondence between the field-space vector field $V_\xi $ and the functional $H_\xi $. This correspondence is \textit{cohomological}: two charges $H_\xi $ and $H_\xi +\k $ with $\d\k=0$ correspond to the same vector field $V_\xi $. From \eqref{33}, it similarly follows that $\{H_\xi ,H_\z \}$ and $\{H_\xi ,H_\z \}+\k_{\xi,\z }$ with $\d\k _{\xi,\z }=0$ correspond to the same vector field $V_{[\xi,\z ]}$. Consider thus
\begin{equation}\label{38}
	-\d H_{[\xi,\z ]}=I_{V_{[\xi,\z ]}}\Omega=\d\{H_\xi,H_\z \}.
\end{equation}
From the cohomological argument above, we then deduce
\begin{equation}\label{39}
	\{H_\xi ,H_\z \}=-H_{[\xi,\z ]}+\k_{\xi,\z },
\end{equation}
as long as $\d\k _{\xi\z }=0$. What we have just shown is that \textit{the Poisson bracket of charges represents the Lie bracket of symmetries projectively}. A constant in field space (like $\k $) appearing in the Poisson bracket is known as a \textit{central extension}, where the word "central" means that it commutes with all generators of the algebra ($\d\k =0$). So another way to restate this result is that the charge algebra represents the symmetry algebra modulo central extensions. This is a general and important result. We will later show an example of this in AdS$_3$ gravity, where the central extension will be proportional to the well-known Brown-Henneaux central charge.

Before presenting the general theory of asymptotic symmetries, we familiarize with the con- struction of conserved quantities for gauge symmetries in a simple example.
 
 

 
 
 

 
 
 

 
 
 

 
 
 

 
 
 

 
 
 

 
 
 






% Bibliography

%% [A] Recommended: using JHEP.bst file
%% \bibliographystyle{JHEP}
%% \bibliography{biblio.bib}

%% or
%% [B] Manual formatting (see below)
%% (i) We suggest to always provide author, title and journal data or doi:
%% in short all the informations that clearly identify a document.
%% (ii) please avoid comments such as "For a review'', "For some examples",
%% "and references therein" or move them in the text. In general, please leave only references in the bibliography and move all
%% accessory text in footnotes.
%% (iii) Also, please have only one work for each \bibitem.



\newpage
\bibliographystyle{JHEP}
\bibliography{biblio.bib}
\end{document}
