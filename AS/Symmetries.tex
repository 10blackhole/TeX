\section{Symmetries}
 We will focus on global and local symmetries firstly, and later on local symmetries only, as they are the main actor in the theory of asymptotic symmetries and the corner proposal.
 
 The naive textbook definition of global and local symmetries is that the latter are generated by spacetime-dependent parameters while the former are generated by constant-in-spacetime ones. Then, global symmetries are physical, acting non-trivially on the system, whereas local symmetries are redundancies, i.e. gauge symmetries, simply expressing our ignorance in defining the physical variables of said system. As we will discuss in detail, this textbook definition is too naive, because \textit{there are local symmetries that become physical in the presence of boundaries}. The method one should use to distinguish trivial symmetries from physical ones is not whether they are generated by constant parameters or not, but rather whether they have vanishing associated charges or not. The tools to address this question, the celebrated Noether’s theorems, are ubiquitous in theoretical and mathematical physics.
 
 There are two main frameworks that we can use to study Noether’s theorems, each coming with some advantages and drawbacks.
 
 \begin{itemize}
 	\item \textit{Hamiltonian approach}: in this framework, time and space are split, and one studies the evolution of trajectories in phase space. The pros are: it is a robust, unambiguous formulation, and it is the closest approach of classical physics to quantum physics. The cons are mainly two. First it can become quickly a technical and heavy machinery. Second, and more importantly for the discussion here, it explicitly breaks spacetime covariance.
 	\item \textit{Lagrangian approach}: instead of an evolutionary problem, the phase space here is thought of as the set of solutions to the equations of motion of the theory. The latter can be shown to be in one-to-one correspondence with the space of evolution of trajectories. The pros are that it is easier to manipulate and retains spacetime covariance. The cons are that it is more ambiguous, and sometimes, it is harder to gather physical intuition with it.
 \end{itemize}
 
 It is in the Lagrangian approach that we will express Noether’s theorems. In this approach, the so-called covariant phase space formalism applies, which is the framework we will use for the rest of these lectures. It stems from Noether’s works, but it reached its final form thanks to the works of Wald et al. \cite{Lee:1990nz,Wald:1993nt,Iyer:1994ys,Iyer:1995kg,Wald:1999wa}.
 
 \subsection{Covariant Phase Space}
 The idea is to develop the calculus both in spacetime and in field space.
 
\subsubsection*{Spacetime Calculus} Consider a differentiable manifold $M$. Prior to a metric structure, one can introduce the de Rham calculus on the space of forms. Given a vector field $\xi\in TM$, a one form is an application from $TM$ to $C^\infty(M)$, that is, an element of the dual bundle $T^*M$. A generic $p$-form is an element of $\wedge^pT^*M$, where the symbol $\wedge$ stands for the wedge (skew-symmetric) product. The space of all forms constitutes the \textit{de Rham complex}
\begin{equation}
	\Omega^\bullet(M,\mathbb{R}):=\bigoplus_{n=0}^d\wedge^n T^*M,
\end{equation}
where $0$-forms are scalars, and the maximal degree of a form is the dimension of the manifold, $d =$dim($M$). We will denote by $\dd $ and $\i$ the exterior derivative and interior product on this complex, respectively. The exterior derivative increases the form degree by one, while the interior product decreases it by one. The exterior derivative is assumed to be a co-boundary on the complex, which means that it is a nilpotent operation $\dd^2=0$. The Lie derivative compares these two operations when applied in different orders,
\begin{equation}
	\mathcal{L}_\xi:=\dd i_\xi + i_\xi\dd ,
\end{equation}
and thus it does not change the form degree. This formula is sometimes referred to as \textit{Cartan’s magic formula}. In the following, we will repeatedly use that two interior products anti-commute, that $i_\xi $ of a scalar is zero, and $\dd $ of a spacetime top-form vanishes.

\subsubsection*{Variational calculus}
One can reproduce these results on the field space. A field space $\G $ is defined as the space of all possible field configurations. It is assumed to be a differentiable manifold. One can thus introduce a calculus on the space of forms on such manifold, called variational calculus. Given a vector field $V\in T\G $, a one form is an application from $T\G $ to the space of functionals, $F=C^\infty (\G )$. A generic $p$-form is an element of $\wedge^pT^*\G $, and the space of all forms constitutes the \textit{variational complex}
\begin{equation}
	\Omega^\bullet(\G ,\mathbb{R}):=\bigoplus_{n=0}^{\dim \G }\wedge^n T^*\G ,
\end{equation}
where $0$-forms are now functionals. We will denote by $\d$ and $I$ the exterior derivative and interior product on this complex, respectively, with $\d^2=0$. The exterior derivative increases the form degree by one, and it is what we commonly refer to as a field variation, while the interior product decreases it by one, and we usually think of it as a field contraction. Cartan’s magica formula then gives the Lie derivative
\begin{equation}
	\mathfrak{L}_V:=\d I_V+I_V\d .
\end{equation}
One can put together spacetime and field-space forms to form the variational bicomplex, defined on $(M,\Gamma)$. We will use the notation “($p, q$)-form” to refer to a form on the bicomplex which is a spacetime $p$-form and a field-space $q$-form. On the bicomplex, one can construct an exterior derivative which is $\dd +\d $. The latter is nilpotent if $\dd \d=-\d \dd $. In the following, we will repeatedly use that $I_V$ of a functional is zero, and that two interior products anti-commute. Unless stated otherwise, we will also assume that vector fields in $TM$ are field-independent, such that $\d\xi=0$, for all $\xi\in TM$.

\subsubsection*{Lagrangian theory}
Let us apply this formalism to a Lagrangian theory. The action reads
\begin{equation}
	S=\int_ML.
\end{equation}
The Lagrangian $L$ is thus a top form in spacetime, and does not contain field variations, so it is a functional of the fields $\varphi\in \G$. Given that $d=\dim M$, the Lagrangian is a $(d, 0)$-form, in the language just established.
 
 
 
 

 
 
 

 
 
 

 
 
 

 
 
 

 
 
 

 
 
 

 
 
 

 
 
 
