\documentclass[10pt]{article}
\usepackage[margin=1in]{geometry}
\usepackage{jheppub} % for details on the use of the package, please see the JINST-author-manual
% page formatting
\usepackage{fancyhdr}
\pagestyle{fancy}

\renewcommand{\sectionmark}[1]{\markright{\textsf{#1}}}
\renewcommand{\subsectionmark}[1]{}
\lhead{\textbf{\thepage} \ \ \nouppercase{\rightmark}}
\chead{}
\rhead{}
\lfoot{}
\cfoot{}
\rfoot{}
\setlength{\headheight}{14pt}

\linespread{1.03} % give a little extra room
\setlength{\parindent}{0.2in} % reduce paragraph indent a bit
\setcounter{secnumdepth}{2} % no numbered subsubsections
\setcounter{tocdepth}{2} % no subsubsections in ToC
\usepackage{lineno}
\usepackage{amsmath,amsthm,amsfonts,amssymb,amscd,physics,cancel,mathtools,fontspec}
\usepackage{tcolorbox}
\usepackage{marginnote,tensor}
%\usepackage[spanish]{babel}
%~~~~~~~~~ Document setup
% \usepackage[spanish]{babel} % English formatting
\usepackage[utf8]{inputenc} % Standard encoding
% \usepackage[a4paper,left=3cm,bottom=3cm]{geometry} % Page formatting
\usepackage{indentfirst} % Indents the first paragraph
\usepackage{amsmath} % Maths type package
\usepackage{bm} % Bold font maths
\usepackage{graphicx} % Advanced graphics package
\usepackage[export]{adjustbox} 
\usepackage{pdflscape} % Make pages landscape
\usepackage{fancyhdr} % Fancy headers
% \usepackage[colorlinks=true,citecolor=blue,urlcolor=blue,linkcolor=black]{hyperref} % Link colours
%\usepackage{natbib} % Bibliography
% \usepackage{flafter} % Reference any 'float'
% \usepackage[framemethod=tikz]{mdframed} % Box off stuff
\usepackage{color} % Colour support
\usepackage{wrapfig} % Text flowing around figures
\usepackage{lipsum} % Generates meaningless text
\usepackage{xcolor}
%\usepackage{biblatex}
%\usepackage[backend=bibtex]{biblatex}
%\addbibresource{bibliography.bib}
%\hypersetup{colorlinks=true, linkcolor=blue}


\theoremstyle{definition}
\newtheorem{ej}{Ejemplo}[section]
\newtheorem{teor}{Teorema}[section]
\newtheorem{sol}{Solución}[section]
\newtheorem{dem}{Demostración}[section]
\newtheorem{cor}{Corolario}[section]
\newtheorem{post}{Postulado}
\newtheorem{prop}{Propiedad}[section]
\newtheorem{prueba}{Prueba}[section]

\def\a{\alpha}
\def\b{\beta}
\def\g{\gamma}
\def\G{\Gamma}
\def\d{\delta}
%\def\D{\Delta}
%\def\e{\eta}
\def\la{\lambda}
\def\La{\Lambda}
\def\k{\kappa}
\def\m{\mu}
\def\n{\nu}
\def\r{\rho}
\def\p{\rho}
\def\o{\omega}
\def\s{\sigma}
\def\S{\Sigma}
\def\t{\tau}
\def\p{\pi}
\def\f{\phi}
\def\vf{\varphi}
\def\ep{\epsilon}
\def\th{\theta}
\def\Th{\Theta}
\def\z{\zeta}

\def\I{\mathcal{I}}



%-----COLORS LIST ------
\definecolor{azure(colorwheel)}{rgb}{0.0, 0.5, 1.0}
\definecolor{DarkViolet}{RGB}{148,0,211}
\definecolor{myDarkBlue}{rgb}{0,0.1,0.7}
\definecolor{DarkBlue}{RGB}{0,0,153}
\definecolor{amber}{rgb}{1.0, 0.49, 0.0}
\definecolor{amaranth}{rgb}{0.9, 0.17, 0.31}
\definecolor{nicered}{rgb}{0.7,0.1,0.1}
\definecolor{brown}{rgb}{0.5,0.1,0.1}
\definecolor{nicegreen}{rgb}{0.0,0.3,0.0}
\definecolor{tealgreen}{rgb}{0.0, 0.51, 0.5}
\def\red#1{{\color{red} #1}}
\def\green#1{{\color{green} #1}}
\def\blue#1{{\color{blue} #1}}
\def\orange#1{{\color{orange} #1}}
%----------------------
\newcommand{\mycolor}{DarkViolet}
\def\myColor#1{{\color{\mycolor} #1}}
\definecolor{tclr}{RGB}{148,0,211}
%----------------------
\newcommand{\corr}[1]{\textcolor{nicered}{#1}}
\newcommand{\nick}[1]{\textcolor{olive}{#1}}
\newcommand{\teo}[1]{\textcolor{azure(colorwheel)}{#1}}
\newcommand{\chteo}[2]{\corr{\st{#1}} \teo{(#2)}}
\newcommand{\bako}[1]{\textcolor{DarkViolet}{#1}}
\newcommand{\than}[1]{\textcolor{magenta}{#1}}
%----------------------
\usepackage{hyperref}
\hypersetup{colorlinks,bookmarksopen,
	bookmarksnumbered,
	citecolor={nicered},
	linkcolor={myDarkBlue},
	urlcolor={tealgreen},
	pdfstartview=FitH}





% \arxivnumber{1234.56789} % if you have one

%\title{\boldmath Mecánica Estadística}

% Collaborations

%% [A] If main author
%% \collaboration{\includegraphics[height=17mm]{collabroation-logo}\\[6pt]
%%  XXX collaboration}

%% or
%% [B] If "on behalf of"
%% \collaboration[c]{on behalf of XXX collaboration}


% Authors
% The "\note" macro will give a warning: "Ignoring empty anchor...", you can safely ignore it.

%% [A] simple case: 2 authors, same institution
%% \author[1]{A. Uthor\note{Corresponding author.}}
%% \author{and A. Nother Author}
%% \affiliation{Institution,\\Address, Country}

%% or, e.g.
%% [B] more complex case: 4 authors, 3 institutions, 2 footnotes
%% \author[a,b]{F. Irst,\note{Now at another university}}
%% \author[c]{S. Econd,}
%% \author[a,2]{T. Hird\note{Also at Some University.}}
%% \author[c,2]{and Fourth}
%% \affiliation[a]{Institution_1,\\Address, Country}
%% \affiliation[b]{Institution_2,\\Address, Country}
%% \affiliation[c]{Institution_3,\\Address, Country}

\author{Borja Diez}
\affiliation{Universidad Arturo Prat}
% \affiliation{Another University,\\
% different-address, Country}

% E-mail addresses: only for the corresponding author
\emailAdd{borjadiez1014@gmail.com}

\abstract{Notas sobre Mecánica Estadistica }



\begin{document}
% make title page
\thispagestyle{empty}
\bigskip \
\vspace{0.1cm}

\begin{center}
{\fontsize{22}{22} \selectfont Notes on}
\vskip 16pt
{\fontsize{36}{36} \selectfont \bf \sffamily The asymptotic structure of spacetime: a Hamiltonian perspective}
\vskip 24pt
{\fontsize{18}{18} \selectfont \rmfamily Borja Diez} 
\vskip 6pt
{\fontsize{14}{14} \selectfont \ttfamily borjadiez1014@gmail.com}
\vskip 6pt
{\fontsize{14}{14} \selectfont  \sffamily \today}
\vskip 24pt
\end{center}
%
%Estas notas de clase están basadas en el curso dictado por el \href{https://inspirehep.net/authors/1318424?ui-citation-summary=true}{Dr. Ignacio Araya} durante el primer semestre del año 2024 en la Universidad Arturo Prat y han sido escritas con propósito de estudio personal.
%
%Las notas están divididas por clase. Adicionalmente han sido complementadas con desarrollos de cálculo personal y comentarios sacados principalmente de \textcolor{blue}{Lecture notes on statistical mechanics} de \href{https://inspirehep.net/authors/992816}{Scott Pratt}. 




% make table of contents
\newpage
\tableofcontents
\newpage

\section{Brief introduction: BMS, matching conditions and spatial infinity}
\begin{itemize}
	\item Boundary conditions originally given $D=4$ din not exhibit the BMS group bot only Poincaré at $i^0$ \cite{Regge:1974zd}.
	\item BMS diffeos preserve boundary conditions at $\mathcal{I}$ (exact symmetries of General Relativity). The should appear independently of the description (incluiding slcings adapted to $i^0$).
	\item Invariance of the gravitational $S$-matrix under BMS is based on the assumption of \textit{antipodal matching conditions} of th fields and charges between $\I_-^+$ and $\I_+^-$ (clearly involves $i^0$).
	\item Connecting $i^0$ with $\I_-^+$ and $\I_+^-$ is a non-trivial and subtle question. Evolution of reasonable Cauchy data make null infinity not so smooth. Metric and Weyl tensor develop logarithmic singularities
\end{itemize} 

BMS symetry emerges at $i^0$ in $D=4$ through the reconsideration of the parity conditions \cite{Henneaux:2018cst}. 
\begin{itemize}
	\item The central ingredients are finiteness and invariance of the off-shell action: boundary conditions that make the kinetic term finite (well-defines symplectic structure).
	\item  Symmetries are canonical: we can assoaciate to any symmetry a charge-generator.
	\item Matching conditions imposeed by Strominger are a consequence of the boundary conditions imposed at $i^0$ for having a well-defines action printiple.
\end{itemize}

The symmary of this lecture will be the following:
\begin{itemize}
	\item Review of the asymptotic analysis on spacelike hypersurfaces that are asymptotically flat through the Hamiltonian approach. Based on: \cite{Regge:1974zd}, \cite{Henneaux:2018cst}, \cite{Henneaux:2019yax}.
	\item Logarithmic supertranslations and supertranslation-invariant Lorentz charges. Based on: \cite{Fuentealba:2022xsz}, \cite{Fuentealba:2023syb}.
\end{itemize}


\section{BMS symmetry at spatial infinity}
\subsection{Einstein gravity in Hamiltonian form}
In the ADM decomposition one decompose the spacetime in space-like hypersurfaces at constant time. This hypersurfaces are separated by a $N\dd t$ distance where $N$ is knwon as the \textit{lapse} function and measure the separetaion between two slides. In the other hand, if we have a point in a initial hypersurface,  we can measure if this point move along the surface a distance $N^{i}\dd t$ which is captured by the \textit{shift} function $N^{i}$. Now we can compute how the line element change form one surface to the other one, 
\begin{equation}
  \dd s^2=-N^2\dd t^2+g_{ij}(\dd x^{i}+N^{i}\dd t)(\dd x^{j}+N^{j}\dd t).
\end{equation}
From here we can read how is the ADM decomposition of the metric,
\begin{equation}
  g_{\m\n }=\mqty(N^{i}N^{i}-N^2&&& N^j\\\\
  N^{i}&&& g_{ij})
\end{equation}
and its inverse
\begin{equation}
  g^{\m\n }=\mqty(-\dfrac{1}{N^2}&&& \dfrac{N^{i}}{N^2}\\\\
  \dfrac{N^j}{N^2}&&& g^{ij}-\dfrac{N^{i}N^j}{N^2})
\end{equation}

The Einstein-Hilbert action in Hamiltonian form, using the ADM decomposition is given by
\begin{equation}
  S[g_{ij},\p^{ij},N^\perp,N^{i}]=\int\dd t\left[\int\dd^3x\left(\p^{ij}\dot{g}_{ij}-N^\perp\mathcal{H}_\perp-N^{i}\mathcal{H}_i\right)+B_\infty\right]
\end{equation}
where $g_{ij}$ corresponds to the spatial component of the metric and $B_\infty$ is a boundary term at infinity which is going to be fixed by the fall-off of the fields ($g_{ij},\p^{ij}$) and the boundary conditions that we impose under the following criteria are
\begin{itemize}
	\item incluiding as many solutions as possible,
	\item making the action finite and,
	\item yielding finite/integrable canonical generators
\end{itemize}
and the fall-off of the constraints are
\begin{equation}
  \mathcal{H}_\perp=\frac{1}{\sqrt{g}}\left(\p^{ij}\p_{ij}-\frac{\p^2}{2}\right)-\sqrt{g}R,\qquad \mathcal{H}_i=-2\nabla^j\p_{ij}
\end{equation}
where $\sqrt{g}$ is the square root of the determinant of the spatial part of the metric which is related with de square root of the determinant of the metric manifold as $\sqrt{-g}=N\sqrt{g}$. 

\subsection{Diffeomorphisms in Hamiltonian description}
Symmetries? Diffeomorphisms that leave the action invariant up to surface integrals at the time boundarie: $(g_{ij},\p^{ij})\to (g'_{ij},\p'^{ij})$. In the Hamiltonian description this can be found by takin thecanonical  Poisson bracket
\begin{equation}
  \d_{\xi,\xi^{i}}\Phi=\left\{\Phi,\int\dd^3x(\xi^\perp\mathcal{H}_\perp+\xi^{i}\mathcal{H}_i)\right\}
\end{equation}
with
\begin{equation}
  \{g_{ij}(x),\p^{kl}(x')\}=\d^{(k}_{(i}\d^{l)}_{j)}\d^{(3)}(x-x')
\end{equation}
Then, we will obtain the following transformation for the fields
\begin{align}
  \d_{\xi,\xi^{i}}g_{ij}&=\frac{2\xi }{\sqrt{g}}\left(\p_{ij}-\frac{1}{2}g_{ij}\p \right)+\mathcal{L}_\xi g_{ij}\\
  \d_{\xi,\xi^{i}}\p^{ij}&=-\xi\sqrt{g}\left(R^{ij}-\frac{1}{2}g^{ij}R\right)+\frac{\xi}{2\sqrt{g}}g^{ij}\left(\p^{mn}\p_{mn}-\frac{\p^2}{2}\right)\\&~~~~-\frac{2\xi}{\sqrt{g}}\left(\p^{im}\p^j_m -\frac{1}{2}\p^{ij}\p \right) +\sqrt{g}\left(\xi^{|ij}-g^{ij}\xi^{|m}_{~~|m}\right)+\mathcal{L}_\xi \p^{ij}
\end{align}

\subsection{Asymptotic analysis: Regge-Teitelboim boundary conditions}
Let us consider the satandar Regge-Teitelmboim (RT) boundary conditions:
\begin{equation}
  g_{ij}=\d_{ij}+\frac{\bar{h}_{ij}}{r}+\mathcal{O}(r^{-2}),\qquad \p^{i}=\frac{\bar{\p }^{ij}}{r^2}+\mathcal{O}(r^{-3})
\end{equation}
where
\begin{equation}
  \bar{h}_{ij}(-n^{i})=\bar{h}_{ij}(n^{i})=\text{even}\qquad \bar{\p }_{ij}(-n^{i})=-\bar{\p }_{ij}(n^{i})=\text{odd}
\end{equation}


























% Bibliography

%% [A] Recommended: using JHEP.bst file
%% \bibliographystyle{JHEP}
%% \bibliography{biblio.bib}

%% or
%% [B] Manual formatting (see below)
%% (i) We suggest to always provide author, title and journal data or doi:
%% in short all the informations that clearly identify a document.
%% (ii) please avoid comments such as "For a review'', "For some examples",
%% "and references therein" or move them in the text. In general, please leave only references in the bibliography and move all
%% accessory text in footnotes.
%% (iii) Also, please have only one work for each \bibitem.

\newpage
\bibliographystyle{JHEP}
\bibliography{biblio.bib}
\end{document}
