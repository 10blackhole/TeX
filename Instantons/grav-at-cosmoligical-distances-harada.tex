\section{Construcción}
Harada buscaba construir una teoría de gravitación donde la constante comológica $\Lambda$ emerga como una constante de inegración y no desde un comienzo en las ecuaciones de campo y donde a ley de conservación $\nabla_\m T^\m _{~\n} =0$ no se asuma desde un comienzo, sino que sea derivada de las ecuaciones de campo.

Gravedad conforme y gravedad de Cotton satisfacen estas condiciones, sin embargo, en estas teorías, cualquier métrica conformalmente plana sirve como solución de vacío, lo cual es in problema ya que las teorías podrían admitir soluciones no físicas. Por ejemplo, la métrica de FLRW e solución de vacío en estas teorías, para $a(t)$ arbitraria.

Harada examinó dos posibles dos objetos totalmente simétricos construidos a partir de derivadas de la curvatura:
\begin{enumerate}
	\item $\nabla_\r R_{\m\n }+\nabla_\m  R_{\n \r  }+\nabla_\n  R_{\r \m }$
	\item $(g_{\m\n }\partial_\r +g_{\n \r  }\partial_\m  +g_{\r \m }\partial_\n )R $
\end{enumerate}
Estas cantidades son lineamente independientes, luego, el lado izquerdo de las ecuaciones de campo se puede construir como una combinación lineal de ellos. El lado derecho de las ecuaciones de movimiento pueden ser construidos de manera análoga usando
\begin{enumerate}
	\item $\nabla_\r T_{\m\n }+\nabla_\m  T_{\n \r  }+\nabla_\n  T_{\r \m }$
	\item $(g_{\m\n }\partial_\r +g_{\n \r  }\partial_\m  +g_{\r \m }\partial_\n )T $
\end{enumerate}
Así, las ecuaciones de campo pueden ser ecritas como
\begin{align*}
  a(\nabla_\r R_{\m\n }+\nabla_\m  R_{\n \r  }+\nabla_\n  R_{\r \m })+b(g_{\m\n }\partial_\r +g_{\n \r  }\partial_\m  +g_{\r \m }\partial_\n )R &=c(\nabla_\r T_{\m\n }+\nabla_\m  T_{\n \r  }+\nabla_\n  T_{\r \m })\\
  &~~~~+d(g_{\m\n }\partial_\r +g_{\n \r  }\partial_\m  +g_{\r \m }\partial_\n )R
\end{align*}
multiplicando a ambos lados por $g^{\n\r }$
\begin{align}\label{1}
  a(2\nabla_\n R^\n _\m +\nabla_\m R)+6b\nabla_\m R=c(2\nabla_\n T^\n _\m +\nabla_\m T)+6d\nabla_\m T
\end{align}
\begin{prop}
	\begin{equation}\label{prop1}
  2\nabla_\m R^\m _\n =\nabla_\n R
\end{equation}
\end{prop}
Usando \eqref{prop1} en \eqref{1} se tiene
\begin{equation}
  2(a +3b)\nabla_\m R=2c\nabla_\lambda T^\lambda_{~\m }+(c+6d)\nabla_\m T
\end{equation}
Para asegurar la ley de conservación $\nabla_\m T^\m _{~\n }=0$, se debe cumplir que
\begin{equation}
  a+3b=0,\qquad c+6d=0
\end{equation}
Además, imponemos la condición de que cada solución de las ecuaciones de Einstein, satisfaga \eqref{1},



