\section{Nonlinear Electrodynamics}
As is defined in the Plebanski book \cite{Plebanski:1970zz}, nonlinear electrodynamics are described by the following action principle
\begin{equation}\label{1}
  S_{\rm E}[g,A,P]=-\frac{1}{4\p }\int\dd^4 x\sqrt{-g}\left(\frac{1}{2}F_{\m\n }P^{\m\n }-\mathscr{H}(\mathscr{P},\mathscr{Q})\right)
\end{equation}
which depends on the metric $g_{\m\n }$, the gauge potential $A_\m$, and the antisymmetric tensor $P_{\m\n }$. Here the structural function $\mathscr{H}$ describes the precise nonlinear electrodynamics and depends, in general, on the two Lorentz scalars that can be constructed with $P^{\m\n }$ [XX]. As usual, the field strength is related to the gauge potential as $F=\dd A$, ensuring the \textit{Faraday equations}
\begin{equation}\label{2}
  \dd F=0.
\end{equation}
On the other hand, the variation of action \eqref{1} with respect to the gauge potential leads to the \textit{Maxwell equations}
\begin{equation}\label{3}
	\dd \star P=0,
\end{equation}
where $\star$ stands for the Hodge dual, whereas varying \eqref{1} with respect to the antisymmetric tensor $P^{\m\n }$ yields the constitutive relations
\begin{equation}\label{4}
  F_{\m\n }=\pdv{\mathscr{H}}{\mathscr{P}}P_{\m\n }+\pdv{\mathscr{H}}{\mathscr{Q}}\star P_{\m\n }.
\end{equation}
Notice that Maxwell electrodynamics is recovered for $\mathscr{H}=\mathscr{P}$, giving linear constitutive relations. Lastly, the corresponding energy-momentum tensor reads
\begin{equation}\label{5}
  4\p T_{\m\n }^{\rm E}=F_{\m\a }P_{\n }^{~\a }-g_{\m\n }\left(\frac{1}{2}F_{\a\b }P^{\a\b }-\mathscr{H}\right). 
\end{equation}
The main motivation for the action principle \eqref{SE} is that now Maxwell equations \eqref{3} remain linear as the Fara- day ones \eqref{2} while the nonlinearity is encoded into the constitutive relations \eqref{4}. Consequently, Maxwell equations \eqref{3} can be now understood just like the Faraday ones \eqref{2}, i.e., implying the local existence of a vector potential $\star P=\dd A^*$. Therefore, from the point of view of the action principle \eqref{1}, a solution to nonlinear electrodynamics can be understood as a pair of vector potentials $A$ and $A^*$ compatible with the constitutive relations \eqref{4}. Additionally, since in four dimensions both Faraday \eqref{2} and Maxwell \eqref{3} equations define conservation laws, there are conserved quantities related to them defined by the following integrals
\begin{equation}
  p=\frac{1}{4\p }\int_{\partial\Sigma}F,\qquad q=\frac{1}{4\p }\int_{\partial\Sigma}\star P, 
\end{equation}
where the integration is taken at the boundary of con-
stant time hypersurfaces $\Sigma $; obviously, these are nothing other than the magnetic and electric charges, respectively.

After this brief and useful introduction, and in order to prepare for what follows, we review a strategy that has proved to be fruitful when nonlinear electrodynamics is considered in General Relativity [XX]. This strategy simply consists of working in a null tetrad of the spacetime metric
\begin{equation}\label{7}
  g=2e^1\otimes_s e^2+2e^3\otimes_s e^4, 
\end{equation}
aligned along the common eigenvectors of the electro-
magnetic fields, i.e.
\begin{equation}\label{8}
  F+\i  \star P=(D+\i B)e^1\wedge e^2+(E+\i H)e^3\wedge e^4\,  .
\end{equation}
Here, the first pair of the tetrad is composed of complex conjugates one-forms, while the last pair is real. Additionally, it has been implicitly assumed that the electromagnetic configuration is algebraically general; namely, the real invariants $E$, $B$, $D$, and $H$ which are related to the eigenvalues are not all zero at the same time. We also remark that the scalars $E$ and $D$ are associated with the intensity of the electric field and electric induction, respectively, as perceived in the null frame, while $H$ and $B$ are their magnetic counterparts. In terms of the aligned tetrad invariants \eqref{8}, the standard invariants take the form
\begin{subequations}\label{9}
	\begin{align}
  \mathscr{F}+\i \mathscr{G}&\equiv \frac{1}{4}F_{\m\n }F^{\m\n }+\frac{\i }{4}F_{\m\n}\star F^{\m\n }=-\frac{1}{2}(E+\i B)^2,\\
  \mathscr{P}+\i \mathscr{Q}&\equiv \frac{1}{4}P_{\m\n }P^{\m\n }+\frac{\i }{4}P_{\m\n}\star P^{\m\n }=-\frac{1}{2}(D+\i H)^2,
\end{align}
\end{subequations}
resulting in a parabolic relation between them. Therefore, the structural function is reparameterized as $\mathscr{H}(\mathscr{P},\mathscr{Q})=\mathscr{H}(D,H)$, which leads to a simpler version of the constitutive relations \eqref{4} that now reads
\begin{equation}\label{10}
  E+\i B=(-\partial_D+\i \partial_H)\mathscr{H}.
\end{equation}
This is not the only advantage of choosing an aligned tetrad, it also results in a diagonalization of the energy-momentum tensor allowing only two independent components. The latter are better expressed through the trace, $\tr T^{\rm E}$, of \eqref{5} together with its traceless part, $\hat{T}^{\rm E}\equiv T^{\rm E}-\frac{1}{4} g\tr T^{\rm E}$, according to
\begin{subequations}
	\begin{align}\label{11}
  2\pi \tr T^{\rm E}&=DE-BH+2\mathscr{H},\\
  4\pi \hat{T}^{\rm E}&=(DE+BH)(e^1\otimes_s e^2-e^3\otimes_s e^4).
\end{align}
\end{subequations}
Here $E$ and $B$ must be determined from the constitutive relations \eqref{10}. This approach has been employed since the seminal work of Plebanski \cite{Plebanski:1970zz} to the formulations of nonlinear electrodynamics pioneered in [XX] which also is reviewed below. The power of this approach is such that it has been instrumental in the derivation of the first genuine example of a spinning charged black hole \cite{Garcia-Diaz:2021bao, DiazGarcia:2022jpc, Ayon-Beato:2022dwg}.

The action principle \eqref{1} is concretely obtained as a Legendre transform from a Lagrangian, $-\frac{1}{4\p }\mathscr{L}(\mathscr{F},\mathscr{G})$, which prompts dubbing the action structural function $\mathscr{H}(\mathscr{P},\mathscr{Q})$ as the \textit{Hamiltonian}. This total Legendre transform is concretely given in terms of the new variables by
\begin{equation}\label{12}
  \mathscr{L}(E,B)=BH-DE-\mathscr{H}(D,H).
\end{equation}
This highlights that in the more known Lagrangian formulation the fundamental variables are instead $E$ and $B$, where the other must be determined from the constitutive relations. Using \eqref{10} as $\dd\mathscr{H}=-E\dd D+B\dd H$ in the differential of \eqref{12}, the constitutive relations acquire now the alternative simple form
\begin{equation}\label{13}
  D+\i H=(-\partial_E+\i \partial_B)\mathscr{L}.
\end{equation}
Furthermore, the total Legendre transform \eqref{12} motivates the definition of the following two partial Legendre transforms
\begin{subequations}
	\begin{align}
  \mathscr{M}^+(D,B)&=BH-\mathscr{M}(D,H),\\
  \mathscr{M}^-(E,H)&=DE+\mathscr{M}(D,H),
\end{align}
\end{subequations}
which were first introduced in \cite{Salazar:1987ap} and led to two alternative dual descriptions of nonlinear electrodynamics, using purely inductions or intensities as independent variables. As was first pointed out and exploited in \cite{Salazar:1987ap}, these dual formulations are precisely the ideal ones to transparently describe theories invariant under duality rotations. \bc{maybe add more text}. Additionally, they are indispensable to determine the electrodynamics supporting the spinning nonlinearly charged black holes. Correspondingly, the constitutive relations in the mixed representations are written as
\begin{subequations}\label{15}
	\begin{align}
  E+\i H&=(\partial_D +\i \partial_B)\mathscr{M}^+,\\
  D+\i B&=(\partial_E+\i \partial_H)\mathscr{M}^-,
\end{align}
\end{subequations}

























