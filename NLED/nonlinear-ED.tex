\section{Nonlinear Electrodynamics}
As is defined in the Pleban ́ski book [33], nonlinear elec- trodynamics are described by the following action prin- ciple
\begin{equation}
  S_{\rm E}[g,A,P]=-\frac{1}{4\p }\int\dd^4 x\sqrt{-g}\left(\frac{1}{2}F_{\m\n }P^{\m\n }-\mathscr{H}(\mathscr{P},\mathscr{Q})\right)
\end{equation}
which depends on the metric gμν, the gauge potential Aμ and the antisymmetric tensor Pμν. Here the struc- tural function H describes the precise nonlinear electro- dynamics and depends, in general, on the two Lorentz scalars that can be constructed with P μν [5]; see the first equality of Eq. (9b). As usual, the field strength is re- lated to the gauge potential as F = dA, ensuring the Faraday equations
\begin{equation}
  \dd F=0
\end{equation}
On the other hand, the variation of action (1) with re- spect to the gauge potential leads to the Maxwell equa- tions
\begin{equation}
	\dd \star P=0
\end{equation}
where ⋆ stands for the Hodge dual, whereas varying (1) with respect to the antisymmetric tensor Pμν yields the constitutive relations



