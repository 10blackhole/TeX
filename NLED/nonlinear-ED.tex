\section{Nonlinear Electrodynamics}
As is defined in the Plebanski book [XX], nonlinear electrodynamics are described by the following action principle
\begin{equation}\label{SE}
  S_{\rm E}[g,A,P]=-\frac{1}{4\p }\int\dd^4 x\sqrt{-g}\left(\frac{1}{2}F_{\m\n }P^{\m\n }-\mathscr{H}(\mathscr{P},\mathscr{Q})\right)
\end{equation}
which depends on the metric $g_{\m\n }$, the gauge potential $A_\m$, and the antisymmetric tensor $P_{\m\n }$. Here the structural function $\mathscr{H}$ describes the precise nonlinear electrodynamics and depends, in general, on the two Lorentz scalars that can be constructed with $P^{\m\n }$ [XX]. As usual, the field strength is related to the gauge potential as $F=\dd A$, ensuring the Faraday equations
\begin{equation}
  \dd F=0.
\end{equation}
On the other hand, the variation of action \eqref{SE} with respect to the gauge potential leads to the Maxwell equations
\begin{equation}
	\dd \star P=0,
\end{equation}
where $\star$ stands for the Hodge dual, whereas varying \eqref{SE} with respect to the antisymmetric tensor $P^{\m\n }$ yields the constitutive relations
\begin{equation}
  F_{\m\n }=\pdv{\mathscr{H}}{\mathscr{P}}P_{\m\n }+\pdv{\mathscr{H}}{\mathscr{Q}}\star P_{\m\n }.
\end{equation}
Notice that Maxwell electrodynamics is recovered for $\mathscr{H}=\mathscr{P}$, giving linear constitutive relations. Lastly, the corresponding energy-momentum tensor reads
\begin{equation}
  4\p T_{\m\n }^{\rm E}=F_{\m\a }P_{\n }^{~\a }-g_{\m\n }\left(\frac{1}{2}F_{\a\b }P^{\a\b }-\mathscr{H}\right). 
\end{equation}
The main motivation for the action principle (1) is that now Maxwell equations (3) remain linear as the Fara- day ones (2) while the nonlinearity is encoded into the constitutive relations (4). Consequently, Maxwell equa- tions (3) can be now understood just like the Faraday ones (2), i.e., implying the local existence of a vector po- tential ⋆P = dA∗. Therefore, from the point of view of the action principle (1), a solution to nonlinear electro- dynamics can be understood as a pair of vector poten- tials A and A∗ compatible with the constitutive relations (4). Additionally, since in four dimensions both Faraday (2) and Maxwell (3) equations define conservation laws,





