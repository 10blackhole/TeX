\section{Symmetries}
Let us consider the Lagrangian for a complex scalar field
\begin{equation}
  \LL(\f )=\frac{1}{2}(\partial_\m \f )(\partial^\m \bar{\f })-V(\f\bar{\f }).
\end{equation}
It is invariant under $U(1)$ transformations, 
\begin{equation}
  \f\to \f' =e^{i\a }\f .
\end{equation}
Indeed,
\begin{align}
  \LL(\f ')&=\frac{1}{2}(\partial_\m \f' )(\partial^\m \bar{\f' })-V(\f'\bar{\f' })\\
  &=\frac{1}{2}e^{i\a }(\partial_\m \f )e^{-i\a }(\partial^\m \bar{\f })-V(\f\bar{\f })\\
  &=\frac{1}{2}(\partial_\m \f )(\partial^\m \bar{\f })-V(\f\bar{\f })\\
  &=\LL (\f ).
\end{align}
Then, we say that the Lagrangian possesses a continuous symmetry which gives rise to a conserved current, that is, $\partial_\m j^\m =0$ which in turn gives rise to a conserved charge defined in the usual way
\begin{equation}
  Q:=\int j^0 \dd^3x.
\end{equation}
In general, all continuous transformation which leaves the Lagrangian invariant, defines a symmetry from which we can find a conserved charge.




