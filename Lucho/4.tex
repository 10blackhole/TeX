\section{Clase 4}
%\subsection{* Interludio de la clase pasada}
De la clase pasada habíamos afirmado que el conjunto $\f^r=0$ es regular si y sólo si el Jacobiano $J_i^r\equiv \eval{\pdv{\f^r}{z^{i}}}_\G $ tiene rango máximo, es decir, $\text{Rango}(J)=R$. 

\begin{teor}	\textbf{(Teorema del rango constante)}
\begin{equation}
u=f(x)=\left\{\begin{array}{ll}
  u_1&=f_1(x_1,..,x_n)\\
  &\vdots\\
  u_m&=f_m(x_1,..,x_n)
\end{array}
\right.
\end{equation}
\begin{equation}
  J(f(x))=\left(\pdv{f}{x}\right)
\end{equation}
La matriz Jacobiana $Jf(x)$ tenga un rango máximo igual a $r$, con $0<r\leq m$.
\begin{itemize}
	\item Todos los menores de orden  mayor que $r$
	\item Para cada $r\in D$, existe un menor de orden $r$ en donde $Jf$ es nulo.
\end{itemize}
Entonces:
\begin{enumerate}
	\item Existen $r$ funciones $f_i$ independientes,
	\item Cada una de las $m-r$ funciones restantes dependen de las funciones independientes.
\end{enumerate}
\end{teor}

Al final de la clase pasada hicimos dos afirmaciones:
\begin{Propo}
	Sobre el subespacio $\G_p$ definidos $\Phi_\a (p,q)=0$, el Hamiltoniano canónico sólo depende de de $\{q_i\}$, $\{p_j\}$ independientes.
\end{Propo}
\begin{Propo}
	Las ecuaciones de Hamilton serán 
	\begin{equation}
\begin{split}
  &\qd^n =\pdv{H_p}{p_n }=\pdv{H_c}{p_n }+u^\a \pdv{\f_\a }{p_n }\\
  &\dot{p}=-\pdv{H_p}{q^n }=-\pdv{H_c}{q^n}-u^\a \pdv{\f_\a }{q^n }\\
 &\f_\a =0
\end{split}
\end{equation}
\end{Propo}

Para el caso b) aparecían nuevos vínculos ($X$) a los cuales nuevamente imponemos la preservación en el tiempo
\begin{equation}
  [X_a,H_c]+u^m[X_a,\f_m]=0
\end{equation}
Una vez finalizado el proceso, nos quedaremos con una serie de vínculos secundarios, que denotaremos
\begin{equation}
  \f_a=0;\qquad a=\underbrace{(N-R_w)}_s+1+\cdots + s+k
\end{equation}
donde $k$ es el número de vínculos secundarios y por lo tanto, el número total de vínculos será
\begin{equation}
  \Phi_j=0,\qquad j=1,...,s+k
\end{equation}

\subsection{Ecuaciones débiles y fuertes}
\begin{defi}
Vamos a introducir el símbolo de igualdad débil para las ecuaciones de los vínculos
\begin{equation}
  \f_j\approx 0
\end{equation}
Con esto enfatizamos que esta cantidad está numéricamente restringida a ser cero pero que no es idénticamente cero a lo largo de todo el espacio de fase. Esto significa en particular que los corchetes de Poisson con funciones dependientes de las variables canónicas no se anulan.

Más generalmente, dos funciones $F$ y $G$ que coinciden en la superficie de los vínculos, se dicen débilmente iguales $F\approx G$
\begin{equation}
  F\approx G\implies F-G=g^j\f_j
\end{equation}
\end{defi}

\subsection{Restricciones en los multiplicadores de Lagrange}
Suponemos que ya hemos encontrado todos los vínculos, entonces podemos pasar a estudiar las resticciones sobre los multiplicadores de Lagrange, las cuales serán de la forma
\begin{equation}\label{4.2}
  [\f_j,H_c]+u^m[\f_j,\f_m]\approx 0
\end{equation}
donde $m=1,...,s$ y $j=1,...,s+k$ y $s$ es el número de vínculos primarios. Podemos considerar a \eqref{4.2} como un conjunto de $k+s$ ecuaciones lineales con $s\leq s+k$ para $u^m$ con coeficientes que son funciones de $p$ y $q$.

La solución más general para \eqref{4.2} es 
\begin{equation}
  u^m=U^m+\n^m
\end{equation}
donde $U^m$ es una solución particular de \eqref{4.2} y $\n^m $ es la solución homogénea de \eqref{4.2}, es decir,
\begin{equation}
  \n^\m [\f_j,\f_m]\approx 0.
\end{equation}














